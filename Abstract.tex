\section{Abstract} 
We describe a novel approach for identifying microbial contributions to soil C-cycling dynamics using nucleic acid stable isotope probing coupled with next generation sequencing. In a series of parallel soil microcosms we amended soils with a complex mixture of model carbon (C) substrates and inorganic nutrients common to plant biomass, where a single C constituent is substituted for its \textsuperscript{13}C-labeled equivalent. Using this approach we assessed incorportation of \textsuperscript{13}C-xylose or \textsuperscript{13}C-cellulose label as proxies for labile soluble C and polymeric insoluble C utilization, respectively. Using CsCl gradient fractionation, incorporation of \textsuperscript{13}C into DNA was measured over a period of 30 days. The 16S rRNA gene sequences from all CsCl gradient fractions were characterized by 454 pyrosequencing and classified into Operational Taxonomic Units (OTU). We describe specific patterns of C-assimilation by discrete OTUs as a function of substrate, time, and level of isotope incorporation. Incorporation of \textsuperscript{13}C from xylose into OTUs was observed at days 3, 7, and 14, while notable incorporation of \textsuperscript{13}C from cellulose was observed only after day 14. Of over 6,000 OTUs detected, a total of 43 and 35 unique OTUs significantly assimilated \textsuperscript{13}C from xylose and cellulose, respectively, spanning throughout 7 phyla.  

