\section{Abstract}
We describe an approach for identifying microbial contributions
to soil C-cycling dynamics using nucleic acid stable isotope probing coupled
with next generation sequencing (HR-SIP). We amended three series of soil microcosms
with a complex mixture of model carbon (C) substrates and inorganic nutrients
similar to plant biomass. A single C constituent in the C substrate mixture was
substituted for its $^{13}$C-labeled equivalent in two microcosm series.
Specifically, in separate microcosms we substituted $^{13}$C-xylose or
$^{13}$C-cellulose for their unlabeled equivalents. Xylose and cellulose were
chosen to represent labile soluble C and polymeric insoluble C, respectively.
Microcosm DNA was interrogated for $^{13}$C incorporation at days 1, 3, 7, 14 and 
30. Incorporation of $^{13}$C from xylose into microcosm DNA
was observed at days 1, 3, and 7, while incorporation of $^{13}$C from
cellulose was peaked at day 14 and was maintained through day 30. Of over 6,000
OTUs detected, a total of 49 and 63 unique OTUs assimilated $^{13}$C from
xylose and cellulose into DNA, respectively. Xylose assimilating OTUs were more
abundant in the microcosm community than cellulose assimilating OTUs, while
cellulose OTUs demonstrated a greater substrate specificity than xylose OTUs.
