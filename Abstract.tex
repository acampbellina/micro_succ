\section{Abstract} We explored the dynamics of microbial carbon (C)
decomposition in soil by coupling DNA Stable Isotope Probing (SIP) and high
throughput sequencing. Our experiment evaluated the degradative succession
hypothesis, described dynamics of C metabolism during organic matter
degradation, and characterized bacteria that metabolize labile and structural
C in soils. We added to soil a complex amendment representing plant derived
organic matter. We added the same amendment to each soil but the identity of
$^{13}$C labeled component (either xylose or cellulose) in each experimental
treatment was varied. We assessed $^{13}$C assimilation into DNA from xylose or
cellulose for SSU rRNA gene OTUs finding evidence of $^{13}$C-incorporation
from $^{13}$C-xylose and $^{13}$C-cellulose in 49 and 63 OTUs, respectively.
Microorganisms assimilated xylose-C into DNA on days 1, 3, and 7 and
cellulose-C into DNA on days 14 and 30. The types of microorganisms that
assimilated of xylose-C into DNA changed over time initially dominated by
\textit{Firmicutes} at day 1 followed by \textit{Bacteroidetes} at day 3 and
then \textit{Actinbacteria} at day 7. This dynamic activity suggests $^{13}$C
was exchanged between microbes. Microbes that metabolized cellulose-C belong to
cosmopolitan soil lineages that remain poorly characterized physiologically and
uncultured, including: \textit{Spartobacteria}, \textit{Chloroflexi} and
\textit{Planctomycetes}. Our study interrogates microbial contributions to soil
C-cycling and determining how microbial community composition influences
biogeocemical processes such as C-cycling will require detailed knowledge of
microbial ecophysiology.
