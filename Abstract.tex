\section{Introductory Paragraph} We explored the microbial contributions to
decomposition using a sophisticated approach to DNA Stable Isotope Probing
(SIP). Our experiment evaluated the dynamics and ecological characteristics of
functionally defined microbial groups that metabolize labile and structural C
in soils. We added to soil a complex amendment representing plant derived
organic matter substituted with either  $^{13}$C-xylose or $^{13}$C-cellulose
to represent labile and structural C pools derived from abundant components of
plant biomass. We found evidence for $^{13}$C-incorporation into DNA from
$^{13}$C-xylose and $^{13}$C-cellulose in~49 and~63 operational taxonomic
units (OTUs), respectively. The types of microorganisms that assimilated
$^{13}$C in the $^{13}$C-xylose treatment changed over time being predominantly
\textit{Firmicutes} at day~1 followed by \textit{Bacteroidetes} at day~3 and
then \textit{Actinobacteria} at day~7. These $^{13}$C-labeling dynamics 
suggest labile C traveled through different trophic levels. In contrast,
microorganisms generally metabolized cellulose-C after 14 days and did not
change to the same extent in phylogenetic composition over time. Microorganisms
that metabolized cellulose-C belonged to poorly characterized but cosmopolitan
soil lineages including \textit{Verrucomicrobia}, \textit{Chloroflexi} and
\textit{Planctomycetes}. We show that microbial life history traits are likely to
constrain the diversity of microorganisms that participate in the soil C-cycle.
