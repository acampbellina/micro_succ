\section{Abstract} We identified microorganisms participating in xylose and/or
cellulose decomposition in soil microcosms using nucleic acid stable isotope
probing (SIP) coupled to next generation sequencing. 49 and 63
OTUs assimilated $^{13}$C from xylose and cellulose into DNA, respectively.
Microorganisms assimilated xylose-C at days 1, 3, and 7. Cellulose-C
assimilation peaked at day 14 and was maintained at day 30. Many SIP-identified
cellulose degraders are members of cosmopolitan but physiologically
uncharacterized soil microbial lineages including \textit{Spartobacteria},
\textit{Chloroflexi} and \textit{Planctomycetes}. $^{13}$C from Xylose was
initially assimilated by \textit{Firmicutes} followed by \textit{Bacteroidetes}
and \textit{Actionbacteria}. Trophic interactions may have caused this temporal
pattern of incorporation. Soil C cycling models, however, often disregard
bacterial trophic interactions. Fast growing substrate generalists assimilated
xylose-C and slow growing substrate specialists assimilated cellulose-C.
Xylose-C assimilators within time points clustered phylogenetically, and
cellulose-C assimilators clustered phylogenetically overall. Knowledge of soil
C-cycling functional guild diversity, membership and activity will improve the
predictive power of terrestrial C flux models. 
