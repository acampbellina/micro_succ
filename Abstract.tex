\section{Abstract} We explored the dynamics of microbial contributions to
decomposition in soil by coupling DNA Stable Isotope Probing (SIP) and high
throughput DNA sequencing. Our experiment evaluated the degradative succession
hypothesis, described dynamics of carbon (C) metabolism during organic matter
degradation, and characterized bacteria that metabolize labile and structural
C in soils. We added a complex amendment representing plant derived organic
matter to soil substituting $^{13}$C-xylose or $^{13}$C-cellulose for unlabeled
equivalents in two experimental treatments which were monitored for 30 days.
Xylose and cellulose are abundant components in plant biomass and represent
labile and structural C pools, respectively. We characterized 5,940 SSU rRNA
gene operational taxonomic units (OTUs) finding evidence for
$^{13}$C-incorporation into DNA from $^{13}$C-xylose and $^{13}$C-cellulose
in~49 and~63 OTUs, respectively. In the $^{13}$C-xylose treatment the types of
microorganisms that incorporated $^{13}$C into DNA changed over time dominated
by \textit{Firmicutes} at day~1 followed by \textit{Bacteroidetes} at day~3 and
then \textit{Actinobacteria} at day~7. These dynamics of $^{13}$C-labeling
suggest labile C traveled through different trophic levels within the soil
bacterial community. The microorganisms that metabolized cellulose-C increased
in relative abundance over the course of the experiment with the highest number
of OTUs exhibiting evidence for $^{13}$C-assimilation after 14 to 30 days.
Microbes that metabolized cellulose C belonged to cosmopolitan soil lineages
that remain uncharacterized including \textit{Spartobacteria},
\textit{Chloroflexi} and \textit{Planctomycetes}. Using an approach that
reveals the C assimilation dynamics of specific microbial lineages we describe
the ecological properties of functionally defined microbial groups that
contribute to decomposition in soil.
