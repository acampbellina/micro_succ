\section{Abstract} We explored the dynamics of microbial carbon (C)
decomposition in soil by coupling DNA Stable Isotope Probing (SIP) and high
throughput sequencing. Our experiment evaluated the degradative succession
hypothesis, described dynamics of C metabolism during organic matter
degradation, and characterized bacteria that metabolize labile and structural
C in soils. We added to soil a complex amendment representing plant derived
organic matter. We added the same amendment to each soil but the identity of
$^{13}$C labeled component in each experimental treatment was varied. In two
treatments unlabeled xylose or cellulose were substituted for $^{13}$C
equivalents and in one treatment all components were unlabeled. Xylose and
cellulose are abundant components of plant biomass and represent both labile
and structural C pools, respectively. We assessed $^{13}$C assimilation into
DNA from xylose or cellulose for SSU rRNA gene OTUs finding evidence of
$^{13}$C-incorporation from $^{13}$C-xylose and $^{13}$C-cellulose in 49 and 63
OTUs, respectively. Microorganisms assimilated xylose-C into DNA on days 1, 3,
and 7 and cellulose-C into DNA on days 14 and 30. The types of microorganisms
that assimilated of xylose-C into DNA changed over time initially dominated by
\textit{Firmicutes} at day 1 followed by \textit{Bacteroidetes} at day 3 and
then \textit{Actinbacteria} at day 7. The dynamic patterns of $^{13}$C
incorporation into DNA in response to adding $^{13}$C-xylose along with
relative abundance fluctuations of OTUs in the microcosms suggest $^{13}$C was
exchanged between microbes in different trophic levels. In contrast, the suite
microorganisms that were $^{13}$C labeled in response to $^{13}$C-cellulose did
not change to the same extent over time. Microbes that metabolized cellulose-C
belong to cosmopolitan soil lineages that remain poorly characterized
physiologically and uncultured, including: \textit{Spartobacteria},
\textit{Chloroflexi} and \textit{Planctomycetes}. Determining how microbial
community composition impacts influences biogeocemical processes such as
C-cycling requires knowledge of microbial ecophysiology.
