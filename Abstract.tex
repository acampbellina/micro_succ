\section{Introductory Paragraph} We explored the dynamics of microbial contributions to
decomposition in soil by coupling DNA Stable Isotope Probing (SIP) and high
throughput DNA sequencing. Our experiment evaluated the degradative succession
hypothesis, described dynamics of carbon (C) metabolism during organic matter
degradation, and characterized bacteria that metabolize labile and structural
C in soils. We added a complex amendment representing plant derived organic
matter to soil. We substituted $^{13}$C-xylose or $^{13}$C-cellulose for their
unlabeled equivalents in two experimental treatments which were monitored for
30 days.  Xylose and cellulose are abundant components in plant biomass and
represent labile and structural C pools, respectively. We found evidence for
$^{13}$C-incorporation into DNA from $^{13}$C-xylose and $^{13}$C-cellulose
in~49 and~63 operational taxonomic unites (OTUs), respectively. The types of
microorganisms that appeared $^{13}$C-labeled in the $^{13}$C-xylose treatment
changed over time being predominantly \textit{Firmicutes} at day~1 followed by
\textit{Bacteroidetes} at day~3 and then \textit{Actinobacteria} at day~7.
These dynamics of $^{13}$C-labeling suggest labile C traveled through different
trophic levels within the soil bacterial community. In contrast, the
microorganisms that metabolized cellulose-C increased in relative abundance
later (after 14 days) with the highest number of OTUs exhibiting evidence for
$^{13}$C-assimilation after 14 days. Microorganisms that metabolized
cellulose-C belonged to cosmopolitan soil lineages that remain uncharacterized
including \textit{Spartobacteria}, \textit{Chloroflexi} and
\textit{Planctomycetes}.  Using an approach that reveals the C assimilation
dynamics of specific microbial lineages we describe the ecological properties
of functionally defined microbial groups that contribute to decomposition in
soil.
