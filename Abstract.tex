\section{Abstract} We explored the dynamics of microbial carbon (C)
decomposition in soil by coupling DNA Stable Isotope Probing (SIP) and high
throughput sequencing. Our experiment evaluated the degradative succession
hypothesis, described the dynamics of C metabolism during organic matter
degradation, and characterized bacteria that metabolize labile and structural
C in soils. Soil microcosms received a nutrient and resource mixture
representing plant derived organic matter that included xylose and cellulose.
In two treatments we substituted xylose or cellulose for $^{13}$C-xylose or
$^{13}$C-cellulose, respectively, and we included a control treatment where both
xylose and cellulose were unlabeled. Xylose and cellulose represent labile and
structural C pools abundant in plant biomass. We distributed SSU rRNA gene
sequences into a total of 5,940 operational taxonomic units (OTU) and
characterized these OTUs for evidence of $^{13}$C-incorporation into DNA from
$^{13}$C-xylose and $^{13}$C-cellulose. Forty-nine and 63 OTUs appeared to assimilate $^{13}$C
into DNA from $^{13}$C-xylose and $^{13}$C-cellulose, respectively. Microorganisms
assimilated xylose-C into DNA on days 1, 3, and
7 and cellulose-C into DNA on days 14 and 30. The types of microorganisms that
assimilated of xylose-C into DNA changed over time initially dominated by
\textit{Firmicutes} at day 1 followed by \textit{Bacteroidetes} at day 3 and
then \textit{Actinbacteria} at day 7. The dynamic patterns of heavy isotope
incorporation from added $^{13}$C-xylose along with relative abundance fluctuations
of OTUs in the microcosms suggest trophic C exchange between these members of
the soil microbial community. In contrast, the microorganisms that assimilated
cellulose-C did not change over time. Microbes that metabolized cellulose-C
belong to cosmopolitan soil lineages that remain poorly characterized
physiologically and uncultured, including: \textit{Spartobacteria},
\textit{Chloroflexi} and \textit{Planctomycetes}. Determining how microbial
community structure impacts major C transformations in the terrestrial C-cycle
requires knowledge of microbial ecophysiology.
