\section{Abstract} We explored the dynamics of microbial carbon (C)
decomposition in soil by coupling DNA Stable Isotope Probing (SIP) and high
throughput sequencing. Our experiment evaluated the degradative succession
hypothesis, described dynamics of C metabolism during organic matter
degradation, and characterized bacteria that metabolize labile and
structural C in soils. We added a complex amendment representing plant
derived organic matter to soil substituting $^{13}$C-xylose
or $^{13}$C-cellulose for unlabeled equivalents in two experimental
treatments. Xylose and cellulose are abundant components in plant biomass
and represent labile and structural C pools, respectively. We assessed
$^{13}$C assimilation into DNA for SSU rRNA gene OTUs finding evidence of
$^{13}$C-incorporation from $^{13}$C-xylose and $^{13}$C-cellulose in 49
and 63 OTUs, respectively. Microorganisms primarily assimilated xylose-C
into DNA on days~1,~3, and~7 and cellulose-C on days~14 and~30. The types
of microorganisms that assimilated xylose-C changed with time initially
dominated by \textit{Firmicutes} at day~1 followed by
\textit{Bacteroidetes} at day~3 and then \textit{Actinbacteria} at day~7.
Temporal dynamics of $^{13}$C-labeling suggests $^{13}$C microorganisms at
different trophic levels exchanged C. Microbes that metabolized
cellulose-C belonged to cosmopolitan soil lineages that remain
uncharacterized physiologically, including: \textit{Spartobacteria},
\textit{Chloroflexi} and \textit{Planctomycetes}. Our study unearths links
microorganisms to specific soil C processes revealing ecological
properties of specific microorganisms within complex communities.
