\section{Abstract}
We describe a novel approach for identifying microbial contributions to soil C-cycling dynamics using nucleic acid stable isotope probing coupled with next generation sequencing (SIP-NGS). In a series of parallel soil microcosms we amended soils with a complex mixture of model carbon (C) substrates and inorganic nutrients common to plant biomass, where a single C constituent is substituted for its \textsuperscript{13}C-labeled equivalent. Using this approach we assessed incorportation of \textsuperscript{13}C-xylose or \textsuperscript{13}C-cellulose as proxies for labile soluble C and polymeric insoluble C utilization, respectively. Using CsCl gradient fractionation, incorporation of \textsuperscript{13}C into DNA was measured over 30 days. The 16S rRNA gene sequences from CsCl gradient fractions were characterized by 454 pyrosequencing and classified into Operational Taxonomic Units (OTU). We describe specific patterns of C-assimilation by discrete OTUs as a function of substrate, time, and level of isotope incorporation. Incorporation of \textsuperscript{13}C from xylose into OTUs was observed at days 1, 3, and 7, while notable incorporation of \textsuperscript{13}C from cellulose was observed only after day 14. Of over 6,000 OTUs detected, a total of 43 and 35 unique OTUs significantly assimilated \textsuperscript{13}C from xylose and cellulose, respectively. We did not observe consistent C utilization at the phylum level although both xylose and cellulose utilization were observed across 7 phyla each revealing a high diversity of bacteria able to utilize these substrates. OTUs that assimilate xylose and those that assimilate cellulose are largely mutually exclusive. Xylose assimilating OTUs are more abundant in the microbial community than cellulose assimilating OTUs, while cellulose OTUs demonstrate a greater substrate specificity than xylose OTUs. Furthermore, the increased depth provided by SIP-NGS allowed us to identify several novel cellulose utilizing bacteria.        

