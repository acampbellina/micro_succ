\subsection{Cellulose degraders exhibit higher substrate specificty than xylose
utilizers} Cellulose responders exhibited a greater shift in BD than xylose
responders in response to isotope incorporation (Figure~\ref{fig:shift},
p-value 1.86e$^{-06}$). $^{13}$C-cellulose responders shifted on average 0.0163
g/mL (sd 0.0094) whereas xylose responders shifted on average 0.0097 (sd
0.0094). For reference, 100\% $^{13}$C DNA shifts X.XX g/mL relative to the BD
of its $^{12}$C counterpart. DNA BD increases as its ratio of $^{13}$C to
$^{12}$C increases. An organism that only assimilates C into DNA from a
$^{13}$C isotopically labeled source, will have a greater $^{13}$C:$^{12}$C
ratio in its DNA than an organism utilizing a mixture of isotopically labeled
and unlabeled C sources. Upon labeling, DNA from an organism that incorporates
exclusively $^{13}$C will increase in buoyant density more than DNA from an
organism that does not exclusively utilize isotopically labeled C. Therefore
the magnitude DNA buoyant density shifts indicate substrate specificity given
our experimental design as only one substrate was labeled in each amendment. We
measured density shift as the change in an OTU's density profile center of mass
between corresponding contol and labeled gradients. Density shifts, however,
should not be evaluated on an individual OTU basis as a small number of density
shifts are observed for each OTU and the variance of the density shift metric
at the level of individual OTUs is unknown. It is therefore more informative to
compare density shifts among substrate responder groups. Further, density
shifts are based on relative abundance profiles and would be theoretically
muted in comparison to density shifts based on absolute abundance profiles and
should be interpreted with this transformation in mind. It should also be noted
that there was overlap in observed density shifts between $^{13}$C-cellulose
and $^{13}$C-xylose responder groups suggesting that although cellulose
degraders are generally more substrate specific than xylose utilizers, some
cellulose degraders show less substrate specificity for cellulose than some
xylose utilizers for xylose (Figure~\ref{fig:shift}), and, each responder group
exhibits a range of substrate specificites (Figure~\ref{fig:shift}).
