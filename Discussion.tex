\section{Discussion} 
% Fakesubsubsection:Pure culture based studies have historically driven soil 
Pure culture based studies have historically driven soil microbial ecology
research but culturing has not captured \textit{in situ} numerically abundant
soil genera \citep{Janssen2006}. DNA-SIP can characterize functional roles for
thousands of phylotypes in a single experiment without cultivation. We
identified 104 OTUs in an agricultural soil that incorporated $^{13}$C from
xylose and/or cellulose into biomass and characterized substrate specificity
and C-cycling dynamics for these OTUs. $^{13}$C-xylose and $^{13}$C-cellulose
responsive OTUs included members of cosmopolitan yet functionally
uncharacterized soil phylogenetic groups such as \textit{Verrucomicrobia},
\textit{Planctomycetes} and \textit{Chloroflexi}.

\subsection{Microbial response to isotopic labels}
% Fakesubsubsection: We propose that microbial decomposition
We propose that C added to soil microcosms in this experiment took the
following path through the microbial food web (Figure~\ref{fig:foodweb}):
Labile C such as xylose was first assimilated by fast-growing opportunistic
\textit{Firmicutes} aerobic spore formers. The remaining labile C biomass
C from growing \textit{Firmicutes} was assimilated in succession by 
\textit{Bacteroidetes}, \textit{Actinobacteria} and \textit{Proteobacteria}
phylotypes that were tuned to lower C substrate concentrations, were
predatory bacteria (e.g. \textit{Agromyces}), and/or were specialized for
consuming viral lysate. C from polymeric substrates was decomposed by bacteria
after 14 days. Canonical cellulose degrading bacteria such as
\textit{Cellvibrio} degraded cellulose but uncharacterized lineages in the
\textit{Chloroflexi}, \textit{Planctomycetes} and \textit{Verrucomicrobia},
specifically the \textit{Spartobacteria}, were also significant contributors to
cellulose decomposition.

\subsection{Ecological strategies of soil microorganisms participating in the
decomposition of organic matter}
% Fakesubsubsection:We assessed the ecology of
We assessed the ecology of $^{13}$C-responsive OTUs by estimating each OTU's
the \textit{rrn} gene copy number and the BD shift upon $^{13}$C-labeling.
\textit{rrn} gene copy number correlates positively with growth rate
\citep{11125085} and BD shift is indicative of substrate specificity (see
results). Ecological metrics predict $^{13}$C-cellulose responsive OTUs grow
slower (Figure~\ref{fig:shift}, Figure~\ref{fig:copy}), have greater substrate
specificity (Figure~\ref{fig:shift}), and are generally lower abundance members
of the bulk community than $^{13}$C-xylose responsive OTUs
(Figure~\ref{fig:shift}). The higher abundance of xylose responders may also be
in part due to higher \textit{rrn} gene copy number. $^{13}$C-xylose responsive
OTUs that responded beginning at day 3 had greater \textit{rrn} gene copy
number than OTUs that responded to $^{13}$C-xylose later
(Figure~\ref{fig:shift}, Figure~\ref{fig:copy}) suggesting fast-growing
microbes assimilated $^{13}$C from xylose before slow growers. 

% Fakesubsubsection:NRI values have been used to assess
NRI values are useful metrics for assessing phylogenetic clustering
\citep{Webb2000} and have recently been used to assess clustering of soil OTUs
categorized by response to wet up \citep{Evans2014a,Placella2012}. To our
knowledge assessing the phylogenetic clustering of OTUs found to incorporate
heavy isotopes into biomass during SIP incubations has not been attempted. We
found that cellulose and xylose responders are clustered and overdispersed,
respectively. This suggests that the ability to degrade cellulose is
phylogenetically conserved possibly reflecting the complexity of cellulose
degradation biochemistry. The positive relationship with a physiological
trait's phylogenetic depth and complexity has been noted previously
\citep{Martiny2013a} and the $^{13}$C-cellulose response trait depth observed
in this study (X.XX 16S rRNA gene sequence divergence) is on the same order as
that observed for glycoside hydrolases which are diagnostic enzymes in
cellulose degradation \citep{Berlemont2013}. Overdispersion, as we saw for the
$^{13}$C-xylose responsive OTUs, may be indicative of a trait that can be
transferred between species via horizontal gene transfer and/or a trait that is
broadly distributed phylogenetically. It's not clear, though, if all
$^{13}$C-xylose responsive organisms were labeled as a result of primary xylose
assimilation (see below), and therefore it's not clear if $^{13}$C-xylose
responsive OTUs in this experiment constitute a single ecologically meaningful
group or multiple ecological groups. 

\subsection{Implications for soil C cycling models}
% Fakesubsubsection:Land management, climate, pollution and
Land management, climate, pollution and disturbance can all influence soil
community composition \citep{McGuire2010} which in turn can influence soil
biogeochemical process rates (e.g. \citep{Berlemont2014a}). Assessing
functional group diversity and establishing identities of functional group
members is necessary to predict how biogeochemical process rates can change
with community composition \citep{Schimel_1995,McGuire2010}. 
%We highlight three key results in our study with implications for models of
%soil C cycling: \begin{itemize} \item labile and polymeric C decomposers
%functional guilds are similar in diversity, \item labile C cycling may include
%significant trophic interactions, and \item soil biomass is a heterogeneous
%mixture of ecological groups that respond as temporally cohesive units
%\end{itemize}
Biogeochemical processes carried out by few taxa or ``narrow'' guilds are
influenced more significantly by changes in community composition than
processes carried out by greater numbers of taxa
\citep{Schimel_1995,McGuire2010}. Labile and recalcitrant C decomposition are
considered to be carried out by ``broad'' and ``narrow'' functional guilds,
respectively \citep{Schimel_1995,McGuire2010}. However, the diversity of active
labile C and insoluble, polymeric C decomposers in soil has not been directly
quantified. Notably, we found more OTUs responded to $^{13}$C-cellulose, 63,
than $^{13}$C-xylose, 49. Also, it is possible that many $^{13}$C-xylose
responders are predatory bacteria as opposed to primary labile C degraders (see
below). The cellulose and xylose decomposer functional guilds were
non-overlapping in membership and represented a small fraction of total soil
community diversity (Figure~\ref{fig:genspec}); of 104 $^{13}$C-responders only
8 responded to both cellulose and xylose. Interestingly, while xylose use is
undoubtedly more widely distributed among global microorganisms than the
ability to degrade cellulose, the number of active xylose utilizers in our
microcosms was not greater than the number of cellulose decomposers.

% Fakesubsubsection:Both $^{13}$C-cellulose and $^{13}$C-xylose
Both $^{13}$C-cellulose and $^{13}$C-xylose responders were largely clustered
near the tips of the phylogenetic tree (NTI $>$ 0) at taxonomic levels broader
than the OTUs established in this study (Figure~\ref{fig:tiledtree}). Therefore
$^{13}$C-responders can be distributed into even fewer clades than number of
OTUs (Figure~\ref{fig:tiledtree}).\textit{Active} cellulose and xylose
responder groups were ``narrow'' in that few lineages relative to total
observed lineages were active participants in cellulose or xylose decomposition
but there is not working quantitative definition of what constitutes ``narrow''
versus ``broad'' in the literature. 

% Fakesubsubsection:We propose two scenarios for how changes
We propose two scenarios for how changes in community composition can affect
C cycling based on our study. For cellulose decomposition, our results suggest
that cellulose degradation is a conserved and narrowly distributed trait
\textit{in situ}. Therefore, changes in community composition could greatly
influence the number of microorganisms capable of cellulose degradation per
unit soil and change cellulose decomposition process rates. For xylose
decomposition, on the other hand, community shifts might not change the number
of xylose decomposers as it is likely many microorganisms are able to use
xylose. Rather, since we observed phylogenetically and ecologically coherent
groups respond to $^{13}$C-xylose within a time point, changes in soil
community composition might shift primary xylose decomposers from one
phylogenetic group to another. Process rates may not change if the number of
xylose decomposers is rate limiting, but, different and phylogenetically
coherent groups of xylose decomposers may differently allocate C resources and
precipitate a different trophic cascade thus influencing soil C fate if not
dynamics. 

% Fakesubsubsection:It's not clear whether the observed activity
The activity succession from \textit{Firmicutes} to \textit{Bacteroidetes} and
finally \textit{Actinobacteria} in response to $^{13}$C-xylose addition is
a trophic cascade and/or the manifestation of the activity of functional
groups tuned to different resource concentrations. \textit{Actinobacteria}
(e.g. \textit{Agromyces}) and \textit{Bacteroidetes} have been previously
implicated as predatory soil bacteria \citep{Lueders2006,16346402}, however,
and the activity peak of \textit{Bacteroidetes} and \textit{Actinobacteria}
occurred with a corresponding decrease in \textit{Firmicutes} $^{13}$C-xylose
responder relative abundance. Considering that \textit{Agromyces} and certain
\textit{Bacteroidetes} types are likely soil predators  one parsimonious
hypothesis for $^{13}$C-labelling of \textit{Bacteroidetes} and
\textit{Actinobacteria} with a corresponding decrease in abundance of
$^{13}$C-labeled \textit{Firmicutes} is that the \textit{Bacteroidetes} and
\textit{Actinobacteria} fed on $^{13}$C-labeled \textit{Firmicutes}. If the
temporal dynamics of $^{13}$C-xylose incorporation are due to trophic
interactions, many, if not most, fast-growing labile C degraders were consumed
by predatory bacteria. Hence, predatory interactions between soil bacteria may
be of importance for modelling soil C turnover though intra-bacteria trophic
interactions in soil C cycling are rarely considered (e.g.
\citep{Moore1988}).

\subsection{Conclusion} 
% Fakesubsubsection:Microorganisms sequester atmospheric C and respire
Microorganisms sequester atmospheric C and respire SOM influencing climate
change on a global scale but we do not know which microorganisms carry out
specific soil C transformations. In this experiment microbes from
physiologically uncharacterized but ubiquitous soil lineages participated in
cellulose decomposition. Cellulose C degraders included members of the
\textit{Verrucomicrobia} (\textit{Spartobacteria}), \textit{Chloroflexi},
\textit{Bacteroidetes} and \textit{Planctomycetes}. \textit{Spartobacteria} in
particular are globally cosmopolitan soil microorganisms and are often the most
abundant \textit{Verrucomicrobia} order in soil \citep{Bergmann_2011}. Our
results also suggest that members of the \textit{Bacteroidetes} and
\textit{Actinobacteria} act in the cascade of labile, soluble C through soil
trophic levels possibly as predators. NEEDS A FINAL SENTENCE.
