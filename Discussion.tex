\section{Discussion} 
% Fakesubsubsection:Pure culture based studies have historically driven soil 
Pure culture based studies have historically driven soil microbial ecology
research but cultured isolates have not captured \textit{in situ} numerically
abundant genera \citep{Janssen2006}. DNA-SIP can
characterize functional roles for thousands of phylotypes in a single
experiment without cultivation. We identified 104 OTUs in an agricultural soil
that incorporated $^{13}$C from xylose and/or cellulose into biomass and
characterized substrate specificity and C-cycling dynamics for these soluble
and polymeric C degraders. Included in $^{13}$C-xylose and
$^{13}$C-cellulose responsive OTUs were members of cosmopolitan yet
functionally uncharacterized soil phylogenetic groups such as
\textit{Verrucomicrobia}, \textit{Planctomycetes} and \textit{Chloroflexi}.

\subsection{Microbial response to isotopic labels}
% Fakesubsubsection: We propose that microbial decomposition
We propose that C added to soil microcosms in this experiment took the
following path through the microbial food web (Figure~\ref{fig:foodweb}):
First, labile C such as xylose was assimilated by fast-growing opportunistic
\textit{Firmicutes} spore formers. The remaining labile C and new biomass C was
assimilated in succession by slower growing \textit{Bacteroidetes},
\textit{Actinobacteria} and \textit{Proteobacteria} phylotypes that were either
tuned to lower C substrate concentrations, were predatory bacteria (e.g.
\textit{Agromyces}), and/or were specialized for consuming viral lysate. C from
polymeric substrates entered the bacterial community after 14 days. Canonical
cellulose degrading bacteria such as \textit{Cellvibrio} degraded cellulose
but uncharacterized lineages in the \textit{Chloroflexi},
\textit{Planctomycetes} and \textit{Verrucomicrobia}, specifically the
\textit{Spartobacteria}, were also significant contributors to cellulose
decomposition.

\subsection{Ecological strategies of soil microorganisms participating in the
decomposition of organic matter}
% Fakesubsubsection:We assessed the ecology of
We assessed the ecology of $^{13}$C-responsive OTUs by estimating the
\textit{rrn} gene copy number and the BD shift upon labeling for each OTU.
\textit{rrn} gene copy number correlates positively with growth rate
\citep{11125085} and BD shift is indicative of substrate specificity (see
results). Ecological metrics suggest $^{13}$C-cellulose responsive OTUs grow
slower (Figure~\ref{fig:shift}, Figure~\ref{fig:copy}), have greater substrate
specificity (Figure~\ref{fig:shift}), and are generally lower abundance than
$^{13}$C-xylose responsive OTUs (Figure~\ref{fig:shift}). The higher abundance
of xylose responders may also be in part due to higher \textit{rrn} gene copy
number. $^{13}$C-xylose responsive OTU \textit{rrn} gene copy number correlated
inversely with the time at which the OTU was first found to incorporate
$^{13}$C into DNA (Figure~\ref{fig:shift}, Figure~\ref{fig:copy}) suggesting
fast-growing microbes assimilated $^{13}$C from xylose before slow
growers.  

% Fakesubsubsection: Labile C responder ecological strategies were more varied
Labile C responder ecological strategies were more varied than polymeric
ecological strategies perhaps because $^{13}$C labeled microorganisms did
not primarily assimilate xylose but became labeled via predatory interactions
and/or are saprophytes. It's not clear whether the observed activity succession
from \textit{Firmicutes} to \textit{Bacteroidetes} and finally
\textit{Actinobacteria} in response to $^{13}$C-xylose addition marks a trophic
cascade or functional groups tuned to different resource concentrations or
both. Notably, each temporally defined response group clustered
phylogenetically suggesting a uniform ecological strategy
(Figure~\ref{fig:trees}). It's also clear that some of the
non-\textit{Firmicutes} $^{13}$C-xylose responders are closely related to known
predators (e.g. \textit{Agromyces}) and \textit{Bacteroidetes} have been
previously implicated as predatory soil bacteria \citep{Lueders2006}. If
the temporal dynamics of $^{13}$C-xylose incorporation are due to trophic
interactions, our results suggest that there are many predatory soil bacteria
that consume fast-growing, aerobic spore-formers. Hence, predatory interactions
between soil bacteria may be of importance for modelling soil C turnover.

\subsection{Implications for soil C cycling models}
% Fakesubsubsection:How -- or if -- phylogenetic
Land management, climate, pollution and disturbance can all influence soil
community composition CITE which in turn can influence biogeochemical process
rates CITE. Assessing functional group diversity and establishing identities of
functional group members is an important step in predicting how biogeochemical
process rates can change with community composition CITE. We highlight three
key results in our study with implications for soil C cycling models: (1)
labile and polymeric C decomposers may have similar within group phylogenetic
breadth, (2) soil biomass does not respond \textit{en masse} to C additions, and
(3) trophic interactions among soil bacteria may greatly influence soil C fate.

It is assumed that transformations carried out by functional guilds of
''narrow'' phylogenetic breadth can be influenced more significantly by
compositional change than processes carried out by a large number of taxa (CITE
Tresseder, Schimel). ''Narrow'' processes are thought to include nitrogen
transformations such as nitrogen fixation and denitrification, and recalcitrant
C decomposition (CITE). Labile C is considered a ''broad'' process
comparatively (CITE), however, the phylogenetic breadth of active labile C and
recalcitrant C decomposers in soil has not been observed directly. Our results
are in conflict with the established paradigm that labile C degradation is
a ''broader'' process than recalcitrant C degradation. In fact, we found more
OTUs responded to $^{13}$C-cellulose, 63, than $^{13}$C-xylose, 49. Further,
both $^{13}$C-cellulose and $^{13}$C-xylose responders were largely clustered
at taxonomic levels broader than the OTUs established in this study
(Figure~\ref{fig:trees}) with \textit{Proteobacteria} $^{13}$C-responders being
a possible exception for cellulose and xylose. The phylogenetic clustering of
substrate response suggests that both cellulose and xylose decomposition are
''narrow'' processes as opposed to ''broad'' and if follows that labile
C dynamics in soil may be influenced similarly in magnitude to recalcitrant
C dynamics in response to changes in community composition. 

"However, most process-based models that incorporate microbes are based on
simple microbial parameters or functions"

Biogeochemical process models are the specific mechanism by which process rates
are predicted. Biogeochemical models of soil C cycling, however, often treat
soil biomass with simple and few parameters CITE.  

Trophic interactions in soil C cycling are often modelled at the taxonomic
level of domain CITE. That is, models have considered C cycling between
bacteria and protists (CITE), for example, but not C cycling between among
bacteria. We observed a cascade in activity with xylose additions that
suggests significant trophic interactions among soil bacteria. Although we
cannot be sure the temporal nature of $^{13}$C-xylose response observed in our
study can be attributed to predatory interactions, we note that the activity
peak of \textit{Bacteroidetes} and \textit{Actinobacteria} occurred with
a concomitant decrease in \textit{Firmicutes} $^{13}$C-xylose responder
relative abundance. Considering that \textit{Agromyces} and certain
\textit{Bacteroidetes} types are likely soil predators (CITE) one parsimonious
hypothesis for $^{13}$C-labelling of \textit{Bacteroidetes} and
\textit{Actinobacteria} with a corresponding decrease in abundance of
$^{13}$C-labeled \textit{Firmicutes} is that the \textit{Bacteroidetes} and
\textit{Actinobacteria} fed on the $^{13}$C-labeled \textit{Firmicutes}. 

"We conclude that predicting the influence of declining species diversity on
trophic-level dynamics and ecosystem processes is difficult, at least in food
webs with a small initial number of species, unless the characteristics of
species and the nature of their interactions are known."

"Research questions of the highest urgency are the assignment of species
to functional groups and determining the redundancy of species within
functional groups"

"These priorities follow from the need to address the extent of any loss
of functioning in soils, associated with intensive agriculture, forest
disturbance, pollution of the environment, and global environmental
change."

"The soil biota considered at present to be most at risk are species-poor
functional groups among macrofaunal shredders of organic matter,
bioturbators of soil, specialized bacteria like nitrifiers and nitrogen
fixers, and fungiforming mycorrhizas" 

Key results:

Verrucomicrobia, Chloroflexi, Firmicutes, for instance

narrow versus broad processes, labile versus recalcitrant carbon

trophic levels are inter-domain or implied as such

(1) diversity of functional groups -- species poor, narrow processes 

(2) predators

(3) different types for different substrates, not en masse

Three findings are particularly notable: (1) soil microbes that use cellulose
are largely different than those that use xylose (Figure~\ref{fig:ord}), (2)
C from labile sources passes through phylogenetically coherent groups in
succession (Figure~XX), and, (3) a small subset of the total suite of soil
microbes used cellulose or xylose in our microcosms. Our findings imply that
environmental perturbation can affect decomposition of C substrates in
different ways, phylogenetically coherent groups of predators may feed on
opportunistic microbes that grow on labile C, and that the diversity of soil
microorganisms does not respond \textit{en masse} to C input. Predicting how
environmental changes will influence ecosystem processes such as cellulose
decomposition will require studies to assess what mechanisms -- biotic or
otherwise -- are rate limiting, how different microbes allocate C resources,
which microorganisms participate in specific C transformations, and whether
a few key taxa or complementary, diverse assemblages participate in the
process. Here, we focus on the diversity and identities within functional
guilds of C-cycling soil microbes.

\subsection{Conclusion} 
% Fakesubsubsection:Microorganisms sequester atmospheric C and respire
Microorganisms sequester atmospheric C and respire soil organic
matter (SOM) influencing climate change on a global scale but we do not
know which microorganisms carry out specific soil C transformations. In
this experiment microbes from uncharacterized yet ubiquitous soil lineages
participated in cellulose decomposition. Cellulose C degraders included
members of the \textit{Verrucomicrobia} (\textit{Spartobacteria}),
\textit{Chloroflexi}, \textit{Bacteroidetes} and \textit{Planctomycetes}.
\textit{Spartobacteria} in particular are globally cosmopolitan soil
microorganisms and are often the most abundant \textit{Verrucomicrobia}
order in soil CITE. Our results also suggest that members of the
\textit{Bacteroidetes} and \textit{Actinobacteria} act in the cascade of
labile, soluble C through soil trophic levels possibly as predators. Both
points illustrate the complexity of soil C dynamics and fate. The largely
phylogenetically coherent ecological groups observed in this study suggest that
soil C dynamics are tied to phylogenetic composition.
