\section{Discussion}
% Fakesubsubsection:Early soil microbial ecology was informed
Early soil microbial ecology was informed by studying pure cultures.  As
reviewed by \citet{Janssen2006} historically important pure cultures from soil
included nine genera \textit{Agrobacterium}, \textit{Alcaligenes},
\textit{Arthrobacter}, \textit{Bacillus}, \textit{Flavobacterium},
\textit{Micromonospora}, \textit{Nocardia}, \textit{Pseudomonas}, and
\textit{Streptomyces} but culture-independent surveys of soil microbial
diversity revealed soil may harbor as many as 5,000 OTUs per half gram
\citep{Schloss2006}. We observed almost 6,000 OTUs in this study. Although
culturing techniques are able to produce isolates from diverse soil lineages
\citep{Janssen2002}, most numerically dominant soil microorganisms are still
uncultured and we know little of their ecophysiology \citep{Janssen2006}. Soil
nutrient flux models often simplify and perhaps oversimplify microbial growth
efficiency by assuming single term for all microbial biomass
\citep{Manzoni2012a} , but we need to characterize the functional roles of soil
microorganisms before constructing more sophisticated models of microbial
influence on nutrient cycling. Culturing has classically revealed microbial
physiology but even if microbiologists could mimic environmental conditions in
the lab such that all soil microbes could be isolated in pure culture,
comprehensively establishing functional characteristics of soil microorganisms
would be logistically limited by the scale of soil microbial diversity.
DNA-SIP, in contrast, can characterize the functional roles of thousands of
phylogenetic types in a single experiment. For example, we assayed 5,940 OTUs
for the ability to incorporate C from xylose or cellulose into biomass and
elucidated functional characteristics for 104 OTUs. Included in the number of
$^{13}$C-xylose and $^{13}$C-cellulose responders were members of many
numerically dominant yet functionally uncharacterized soil phylogenetic groups
such as \textit{Verrucomicrobia}, \textit{Planctomycetes} and
\textit{Chloroflexi}. We also show how DNA-SIP can be used to assay substrate
specificity and temporal dynamics of C-cycling including microbial community
succession during decomposition.

\subsection{Phylogenetic affiliation of $^{13}$C-cellulose and $^{13}$C-xylose
    -responsive microorganisms}
% Fakesubsubsection:\textit{Verrucomicrobia} are ubiquitous in soil
\textit{Verrucomicrobia}, cosmopolitan soil microbes
\citep{Bergmann_2011}, can comprise up to 23\% of 16S rRNA gene sequences in
high-throughput DNA sequencing surveys of SSU rRNA genes in soil
\citep{Bergmann_2011} and represent as high as 9.8\% of soil 16S rRNA
\citep{Buckley_2001}. Many \textit{Verrucomicrobia} were first isolated in the
last decade \cite{Wertz_2011} but only one of the 15 most abundant
verrucomicrobial phylotypes in a global soil sample collection shared greater
than 93\% sequence identity with an cultured isolate \citep{Bergmann_2011}.
Genomic analyses and physiological profiling of \textit{Verrucomicrobia}
isolates revealed \textit{Verrucomicrobia} are capable of methanotrophy and
diazotrophy \citep{Wertz_2011} and reviewed by \citet{Wertz_2011}.
\textit{Verrucomicrobia} cultivars degrade cellulose in culture and select
\textit{Verrucomicrobia} genomes possess the genes necessary to degrade
cellulose \citep{Otsuka_2012, Wertz_2011}.  The function and or global
significance of soil \textit{Verrucomicrobia} in global C-cycling is unknown.
For example, only two of the ten putative verrucomicrobial cellulose degraders
identified in this experiment shares at least 95\% sequence identity with
a isolate ("OTU.83" and "OTU.627", Table~\ref{tab:cell}). Seven of 10
$^{13}$C-cellulose responding verrucomicrobial OTUs were classified as
\textit{Spartobacteria}. \textit{Spartobacteria} were the numerically dominant
family of \textit{Verrucomicrobia} in SSU rRNA gene surveys of 181 globally
distributed soil samples \citep{Bergmann_2011}. \textit{Verrucomicrobia}
lineages particularly \textit{Spartobacteria}, given their ubiquity and
abundance in soil as well as their demonstrated incorporation of $^{13}$C from
$^{13}$C-cellulose, may be contributing to cellulose decomposition on a global
scale.

% Fakesubsubsection:Soil \textit{Chloroflexi} have been found to assimilate
Soil \textit{Chloroflexi} have been found to assimilate cellulose before in
DNA-SIP studies with $^{13}$C-cellulose \citep{Schellenberger_2010}. Previously
identified \textit{Chloroflexi} $^{13}$C-cellulose-responders were
underrepresented in culture collections \citep{Schellenberger_2010}.  The
\textit{Chloroflexi} identified as cellulose degraders in this study are also
only distantly related to isolates \ref{tab:cell}.  Chloroflexi are among the
six most abundant soil phyla commonly recovered from soil microbial diversity
surveys \citep{Janssen2006}. Chloroflexi are typically not as as abundant as
\textit{Verrucomicrobia} but are roughly as abundant as \textit{Bacteroidetes}
and \textit{Planctomycetes} types \citep{Janssen2006}.  4 of 5
$^{13}$C-cellulose responsive \textit{Chloroflexi} identified in this study are
annotated as belonging to the \textit{Herpetosiphon} although the
\textit{Herpetosiphon} SSU rRNA gene sequences from this study from 
$^{13}$C-cellulose responsive OTUs all share less
than 95\% sequence identity with their
closest cultured relative in the \textit{Herpetosiphon} genus (\textit{H.
    geysericola}). \textit{H. geysericola} is a predatory bacterium shown to
prey upon \textit{Aerobacter} in culture and can also digest cellulose
\citep{Lewin1970}. In our study, "Herpetosiphon" $^{13}$C-cellulose responders
did not show a delayed response to $^{13}$C-cellulose as compared to other
responders but nonetheless could have
become labeled by feeding on primary $^{13}$C-cellulose degraders. The prey
specificity of predatory bacteria is not well established especially \textit{in
    situ}. $^{13}$C-labeling would be positively correlated with prey
specificity. If the predator specifically preyed upon one population then it
could take on the same labeling percent as that population. Preying on multiple
types would produce a mixed and potentially diluted labeling signature if some
of the prey populations were not isotopically labeled.

% Fakesubsubsection:We also observed $^{13}$C-incorporation
We also observed $^{13}$C-incorporation from cellulose from
\textit{Proteobacteria}, \textit{Planctomycetes} and \textit{Bacteroidetes}. 
Strains in Proteobacteria, Planctomycetes and Bacteroidetes have all been
previously implicated in cellulose degradation. Planctomycetes is the
least studied of the three phyla and only one Planctomycetes isolate can
grow on cellulose. None of the seven \textit{Planctomycetes} cellulose degraders
identified in  this experiment are closely related to isolates. Soil
\textit{Planctomycetes} may be significant contributors to bacterial cellulose 
degradation.\textit{Acidobacteria} did not appear to incorporate $^{13}$C
from cellulose into DNA in
our microcosms though \textit{Acidobacteria} have been found to degrade
cellulose in culture CITE and are a numerically significant soil phylum CITE.
The \textit{Acidobacteria} in our microcosms were mainly annotated as belonging
to the "XX" group. NEED TO DEVELOP THIS FURTHER.

%Fakesubsubsection:Xylose responders
PASS

\subsection{Ecological strategies of soil microorganisms participating in the
    decomposition of organic matter}
% Fakesubsubsection:Ecological strategies of soil microorganisms
Ecological strategies of soil microorganisms have been observed by
cultivation-dependent studies CITE and include characterization of growth
rates, \ldots  We observed ecological strategies among substrate
responders although as boundaries between groups were not well defined the
observed ecological strategy space appeared more like a continuum than discrete
points. In general, $^{13}$C-cellulose responders were slow-growers with low
\textit{rrn} gene copy number. $^{13}$C-cellulose responders were also
typically low abundance members of the microbial community and exhibited higher
substrate specificity than $^{13}$C-xylose responders (Figure~\ref{fig:shift}).
Within the $^{13}$C-cellulose responders there was faint 
ecological strategy \textit{sub}-groups. For instance,
\textit{Verrucomicrobia}, \textit{Chloroflexi} and \textit{Planctomycetes} had
distinct patterns of substrate specificity, \textit{rrn} gene copy number and
temporal dynamics in the bulk community (Figure~XX). MORE...

% Fakesubsubsection:HR-SIP allows us to assess C substrate specificity
HR-SIP allows us to assess C substrate specificity and temporal dynamics of
$^{13}$C-incorporation. C substrate specificity is assessed by measuring the BD
shift of OTU DNA upon $^{13}$C incorporation. OTUs that incorporate more
$^{13}$C per unit DNA have greater specify for the labeled substrate than OTUs
that incorporate less $^{13}$C per unit DNA. $^{13}$C-cellulose incorporating
OTUs as a group displayed greater substrate specificity than $^{13}$C-xylose
incorporating OTUs. This suggests that polymeric C-degraders tend to be
specialists tuned to particular C-substrates such as cellulose or lignin
whereas labile C-degraders are generalists able to assimilate C from many
different labile sources. Although we observed a succession of $^{13}$C-xylose
responders (Figure~\ref{fig:l2fc} and \ref{fig:xyl_count}), there was no
discernible difference in substrate specificity between $^{13}$C-xylose
responders that first responded at days 1, 3 or 7. There was, however, a strong
positive relationship between $^{13}$C-xylose responders that first responded
at days 1, 3 or 7 in \textit{rrn} copy number. $^{13}$C-xylose responders that
first responded at day 1 had higher estimated \textit{rrn} copy number than
responders that first responded at day 3 which had higher \textit{rrn} copy
number than responders that first responded at day 7.  Therefore, OTUs that
grow faster assimilate C from xylose faster intuitively.  However, fast growers
are replaced with respect to xylose C assimilation with slower growers, saprophytes
and/or predators as xylose diminishes. There was a succession in activity
with time from fast growing spore-formers to \textit{Bacteroidetes} types and
finally \textit{Actinobacteria} in our microcosms. The succession hypothesis of
decomposition groups ecological units by substrate CITE, however, our results
suggest there is a succession of microbial activity for even a single
substrate. Hence, in soil, there is an ecological hierarchy coarsely defined by
the ability to assimilate C from labile or polymeric sources but within the
labile substrate degraders there are ecological subunits tuned to specific
substrate concentrations and/or trophic cascades on the temporal order of
days..

% Fakesubsubsection: We propose that microbial decomposition
We propose that microbial decomposition of organic matter in soil follows the
following path. Labile C such as xylose is assimilated by fast-growing
opportunistic \textit{Firmicutes} spore formers and \textit{Proteobacteria}.
Fast growing types generally have high growth efficiency hence much of the
earliest assimilated C will be converted into biomass. Leftover labile C and
new biomass C is assimilated by slower growing \textit{Actinobacteria},
\textit{Proteobacteria} and \textit{Bacteroidetes} types that are tuned to
lower C substrate concentrations, are predatory bacteria (e.g. \textit{Agromyces}),
and/or are specialized for consuming viral lysate. Polymeric C is likely
assimilated by fungal cellulose degraders before bacterial degraders but the
differences in fungal and bacterial soil C assimilation are outside the scope
of this study. Polymeric C enters the bacterial community after several weeks.
Canonical cellulose degrading bacteria such as \textit{Cellvibrio} are major
cellulose degraders but uncharacterized lineages in the \textit{Chloroflexi}
and \textit{Verrucomicrobia}, specifically the \textit{Spartobacteria}, are
significant contributors to soil cellulose decomposition. C originally from
polymeric substrates may also enter the bacterial community via predatory
actions of lineages that specifically prey upon cellulose degraders that are
perhaps members of the \textit{Herpetosiphon} genus although it is not known at
this time the relative contibution, if any, of this predatory action. The assimilation of
C from polymeric substrates precipitates a smaller trophic cascade because
polymeric C degraders exhibit lower growth efficiency than their labile C
degrading counterparts and thus less assimilated C is converted into biomass that
can move higher tropic levels.

\subsection{DNA-SIP methods considerations}
% Fakesubsubsection:We found that control gradients yield amplifiable
We found that control gradients yield amplifiable DNA well into the heavy
fractions. In fact, we observe XX OTUs in the heavy fractions of control
gradients. Therefore, presence in heavy fractions of $^{13}$C gradients alone
is not evidence of $^{13}$C incorporation into DNA. Only enrichment in
$^{13}$C heavy fractions relative to control gradient fractions demonstrates
incorporation even if heavy fractions are shown to be different from controls
by fingerprinting techniques CITE Neufeld 2010. The concentration of uniform
buoyant density macromolecules in density gradients is Gaussian shaped at
equilibrium CITE and therefore the presence of DNA in heavy fractions is not
simply a function of label incorporation and/or high G+C content but is also
influenced by the abundance. Disentangling abundance and high G+C influence
from label incorporation is reliably accomplished with experimental design that
incorporates appropriate control microcosms.

Xylose responders change over time. Implications for DNA-SIP. Succession within
succession.

Response not consistent across phyla.

% Fakesubsubsection:Nucleic-acid SIP coupled to microbiome fingerprinting
Nucleic-acid SIP coupled to microbiome fingerprinting techniques progressed
from simple proof-of-concept experiments with pure cultures
\citep{radajewski2000stable} to DGGE, ARISA and/or tRFLP-enabled studies of
microcosm microbial assemblages \citep{Haichar_2007}. Recently large
experiments employed multiple labeled substrates and high-throughput amplicon
and/or shotgun DNA sequencing \citep{Verastegui_2014} revealing the relative
contributions of sampling location and DNA-SIP gradient density on phylogenetic
profile variance from DNA-SIP experiments (CITE Conrad and Neufeld). Although
density gradient position can account for approximately XX\% of phylogenetic
profile variance, soil type appears to represent greater variance than can be
established across density gradients from soil DNA. Association of phylogenetic
types with "heavy" DNA-SIP density gradient fractions in labeled gradients and
not "heavy" fractions in unlabeled control gradients suggests label
incorporation in to biomass for the associated phylogenetic type. Our study
shows that DNA-SIP can also characterize C use in three additional 
dimensions 1) temporally, isotopic labels can demonstrate C substrate use
dynamics on the scale of days both between substrates and for a single
substrate, 2) profiling DNA-SIP density gradients along the full gradient
length can demonstrate patterns in substrate use specificity, and 3) $^{13}$C
incorporation can be established at high resolution taxonomic groups such
as 97\% sequence identity OTUs.

fraction-level $^{13}$C assimilation dynamics and membership differences}
% Fakesubsubsection:Techniques to identify $^{13}$C-DNA in DNA-SIP
Techniques to identify $^{13}$C-DNA in DNA-SIP studies (reviewed by CITE)
include adding $^{13}$C carrier DNA to density gradients, visualizing
density gradient DNA with DNA intercalating dyes CITE, and screening
density gradient fractions with microbial community fingerprinting
techniques or qPCR CITE. $^{13}$C DNA and high G+C $^{12}$C-DNA occur at
the same buoyant density positions in the density gradient
\citep{Buckley_2007}. Comparing control gradients with $^{13}$C labeled
gradients differentiates $^{13}$C DNA from high G+C $^{12}$C DNA. High
throughput DNA sequencing allows microbial ecologists to survey the
microbial composition of thousands of samples CITE EMP. We surveyed the
16S rRNA gene profile of entire CsCl gradients using high throughput DNA
sequencing technology.

% Fakesubsubsection:Each CsCl gradient fraction possesses a distinct
Each CsCl gradient fraction possesses a distinct array of SSU rRNA gene
phylogenetic types. DNA buoyant density (BD) drives differences in CsCl
gradient fraction SSU rRNA gene composition (Figure~ref{fig:ord}). Low G+C
DNA occurs in greater abundance at lighter densities and vice versa. We
fed microcosms identical C substrate mixtures save for the identity of
$^{13}$C labeled substrates, and by design, the phylogenetic composition
of bulk DNA from microcosms harvested at the same time point is similar
regardless of whether the microcosm received a $^{13}$C substrate. Hence,
SSU rRNA gene profile differences between gradients harvested at the same
time reflect $^{13}$C incorporation into bulk microcosm DNA. $^{13}$C-DNA
BD is 'heavier' than its $^{12}$C counterpart causing heavy fractions in
gradients receiving $^{13}$C-DNA to differ in phylogenetic content from
corresponding heavy fractions from gradients receiving $^{12}$C-DNA of the
same bulk phylogenetic composition.

% Fakesubsubsection:Microcosms incorporated $^{13}$C from both
Microcosms incorporated $^{13}$C from both $^{13}$C-xylose and
$^{13}$C-cellulose into biomass because gradients from both $^{13}$C-xylose and
$^{13}$C-cellulose fed microcosm differ from controls (Figure~\ref{fig:ord}).
Heavy in contrast to light density gradient fractions present more pronounced
differences between control and labeled gradients (Figure~\ref{fig:ord}).
Demonstrating isotope incorporation requires careful comparisons between
control and labeled gradients over the same buoyant density range. By
sequencing CsCl gradient fractions from both control and labeled gradients
across the full density gradient with DNA harvested from microcosms at multiple
time points, we can observe where in the density gradient the strongest
$^{13}$C isotope incorporation signal presents and when $^{13}$C isotope
incorporation begins (Figure~\ref{fig:ord}). $^{13}$C incorporation from xylose
and cellulose is strongest at days 1/3/7 and days 14/30, respectively
(Figure~\ref{fig:ord}). Additionally, labeled gradient fraction phylogenetic
profiles diverge from controls at heavy buoyant densities
(Figure~\ref{fig:ord}). $^{13}$C-DNA from $^{13}$C-xylose microcosms differs in
phylogenetic composition from $^{13}$C-cellulose microcosm $^{13}$C-DNA
indicating distinct microbial community members assimilated xylose and
cellulose (Figure~\ref{fig:ord}). Lastly, ordination indicates $^{13}$C xylose
labeled microorganisms changed in phylogenetic type over incubation days 1, 3
and 7 (Figure~\ref{fig:ord}).
\subsection{Conclusion}
% Fakesubsubsection:SOM represents more C than the marine and atmospheric
SOM represents more C than the marine and atmospheric C reservoirs
combined and approximately 80\% of SOM flux is mediated by microorganisms
CITE. SOM decomposition is more sensitive to temperature changes than
primary productivity CITE. Climate change can affect terrestrial microbial
communities on a global scale CITE Garcia-Pichel. Therefore, climate
change has the potential to influence SOM flux and storage. Knowing how
temperature changes the rates at which C decomposes in soil is essential
to predict how climate change will affect SOM flux and storage.
Additionally we need to observe how climate change alters the abundance
and activity of key microbial players. Bur first, we must identify which
microbial phylogenetic types decompose different SOM C components.

We found that $^{13}$C substrate responders changed as much as X-fold in
relative abundance over time (Figure~XX). This is in contrast to
a previous study CITE which suggested cellulose decomposers were found to
be consistent in relative abundance with time. Although presence in heavy
fractions can indicate $^{13}$C labeling, not all DNA in heavy fractions
is $^{13}$C-labeled. Some DNA is heavy due to high G+C.
With lower resolution fingerprinting techniques the banding pattern of SSU
rRNA gene sequences can look similar across the entire density gradient
CITE, however, high throughput sequencing of density gradient fractions
shows light and heavy fractions are statistically different even when
input DNA is entirely unlabeled (Figure~\ref{fig:time_class}, and CITE).
Hence, DNA-SIP studies that do not incorporate controls wherein amendments
contain only $^{12}$C substrates, may confuse high G+C organisms with
organisms that incorporated $^{13}$C into biomass. 

The succession hypothesis of decomposition predicts a succession from
microbial types that use labile C to those that use recalcitrant polymeric
C over time CITE. Cellulose degraders succeeded labile C degraders as
predicted. But, in response to $^{13}$C-xylose,  \textit{Firmicutes}
phylotypes were succeeded by \textit{Bacteroidetes} which were then
succeeded by \textit{Actinobacteria} representing a nested succession
(Figure~XX). 

Soluble C more robust to temperature changes because or redundancy? But we
found similar numbers of xylose and cellulose degrader
Microorganisms sequester atmospheric carbon and respire soil organic matter
(SOM) influencing climate change on a global scale but which microbial lineages
transform different soil C components are not established. Molecular tools will
unravel the soil microbial food web and reveal how specific microbial lineages
impact soil C flux. We present a cultivation independent, molecular, DNA-SIP
method to chart C use into microbial lineages. Our results show physiologically
undefined cosmopolitan microbial lineages decompose cellulose. We also show
phylogenetic groups rise and fall and are supplanted by others in activity over
7 days in response to labile C addition.  

Models of SOM turnover incorporate estimates of the carbon use efficiency
(CUE) for the microbial community where CUE is the ratio of biomass
production to biomass production plus combined respiration for growth and
maintenance CITE.  CUE estimates from soil... 
