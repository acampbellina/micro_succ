\section{Discussion} 
% Fakesubsubsection:Pure culture based studies have historically driven soil 
Pure culture based studies have historically driven soil microbial ecology
research but cultured isolates have not captured \textit{in situ} numerically
abundant genera \citep{Janssen2006}. DNA-SIP can
characterize functional roles for thousands of phylotypes in a single
experiment without cultivation. We identified 104 OTUs in an agricultural soil
that incorporated $^{13}$C from xylose and/or cellulose into biomass and
characterized substrate specificity and C-cycling dynamics for these soluble
and polymeric C degraders. Included in $^{13}$C-xylose and
$^{13}$C-cellulose responsive OTUs were members of cosmopolitan yet
functionally uncharacterized soil phylogenetic groups such as
\textit{Verrucomicrobia}, \textit{Planctomycetes} and \textit{Chloroflexi}.

\subsection{Microbial response to isotopic labels}
% Fakesubsubsection: We propose that microbial decomposition
We propose that C added to soil microcosms in this experiment took the
following path through the microbial food web (Figure~\ref{fig:foodweb}):
First, labile C such as xylose was assimilated by fast-growing opportunistic
\textit{Firmicutes} spore formers. The remaining labile C and new biomass C was
assimilated in succession by slower growing \textit{Bacteroidetes},
\textit{Actinobacteria} and \textit{Proteobacteria} phylotypes that were either
tuned to lower C substrate concentrations, were predatory bacteria (e.g.
\textit{Agromyces}), and/or were specialized for consuming viral lysate. C from
polymeric substrates entered the bacterial community after 14 days. Canonical
cellulose degrading bacteria such as \textit{Cellvibrio} degraded cellulose
but uncharacterized lineages in the \textit{Chloroflexi},
\textit{Planctomycetes} and \textit{Verrucomicrobia}, specifically the
\textit{Spartobacteria}, were also significant contributors to cellulose
decomposition.

\subsection{Ecological strategies of soil microorganisms participating in the
decomposition of organic matter}
% Fakesubsubsection:We assessed the ecology of
We assessed the ecology of $^{13}$C-responsive OTUs by estimating the
\textit{rrn} gene copy number and the BD shift upon labeling for each OTU.
\textit{rrn} gene copy number correlates positively with growth rate
\citep{11125085} and BD shift is indicative of substrate specificity (see
results). Ecological metrics suggest $^{13}$C-cellulose responsive OTUs grow
slower (Figure~\ref{fig:shift}, Figure~\ref{fig:copy}), have greater substrate
specificity (Figure~\ref{fig:shift}), and are generally lower abundance than
$^{13}$C-xylose responsive OTUs (Figure~\ref{fig:shift}). The higher abundance
of xylose responders may also be in part due to higher \textit{rrn} gene copy
number. $^{13}$C-xylose responsive OTU \textit{rrn} gene copy number correlated
inversely with the time at which the OTU was first found to incorporate
$^{13}$C into DNA (Figure~\ref{fig:shift}, Figure~\ref{fig:copy}) suggesting
fast-growing microbes assimilated $^{13}$C from xylose before slow growers.
Labile C responder ecological strategies were more varied than polymeric
ecological strategies perhaps because $^{13}$C labeled microorganisms did not
primarily assimilate xylose but became labeled via predatory interactions
and/or are saprophytes. 

\subsection{Implications for soil C cycling models}
% Fakesubsubsection:Land management, climate, pollution and
Land management, climate, pollution and disturbance can all influence soil
community composition \citep{McGuire2010}which in turn can influence soil
biogeochemical process rates (e.g. \citep{Berlemont2014a}). Assessing
functional group diversity and establishing identities of functional group
members is an important step in predicting how biogeochemical process rates can
change with community composition \citep{Schimel_1995,McGuire2010}. We
highlight three key results in our study with implications for models of soil
C cycling: \begin{itemize} \item labile and polymeric C decomposers are similar
in phylogenetic breadth, \item labile C cycling may include significant trophic
interactions, and \item soil biomass does not respond \textit{en masse} to
C additions.\end{itemize}

% Fakesubsubsection:Transformations carried out
Transformations carried out by functional guilds of "narrow" phylogenetic
breadth are influenced more significantly by changes in community composition
than processes carried out by greater numbers of taxa
\citep{Schimel_1995,McGuire2010}. Labile and recalcitrant C decomposition are
considered to be carried out by "broad" and "narrow" functional
guilds,respectively \citep{Schimel_1995,McGuire2010}. However, the phylogenetic
breadth of active labile C and recalcitrant C decomposers in soil has not been
directly quantified. Our results conflict with the established
paradigm of decreasing diversity in guilds responsible transforming more
recalcitrant C substrates. We found more OTUs responded to $^{13}$C-cellulose,
63, than $^{13}$C-xylose, 49, and it is possible that many $^{13}$C-xylose
responders are predatory bacteria as opposed to labile C primary degraders (see
below). Both $^{13}$C-cellulose and $^{13}$C-xylose responders were largely
clustered at taxonomic levels broader than the OTUs established in this study
(Figure~\ref{fig:trees}) with \textit{Proteobacteria} $^{13}$C-responders being
an exception for both cellulose and xylose responders. When grouped at greater
phylogenetic depth $^{13}$C-responders can be distributed into a few clades.
This suggests both cellulose and xylose decomposition are "narrow" processes as
opposed to "broad". It follows, therefore, that labile C dynamics in soil may
be influenced similarly to recalcitrant C dynamics in response to changes in
soil community composition. 

% Fakesubsubsection:It's not clear whether the observed activity
The activity succession from \textit{Firmicutes} to \textit{Bacteroidetes} and
finally \textit{Actinobacteria} in response to $^{13}$C-xylose addition marks
a trophic cascade and/or functional groups tuned to different resource
concentrations. \textit{Actinobacteria} (e.g. \textit{Agromyces}) and
\textit{Bacteroidetes} have been previously implicated as predatory soil
bacteria \citep{Lueders2006}. Further, the activity peak of
\textit{Bacteroidetes} and \textit{Actinobacteria} occurred with a concomitant
decrease in \textit{Firmicutes} $^{13}$C-xylose responder relative abundance.
Considering that \textit{Agromyces} and certain \textit{Bacteroidetes} types
are likely soil predators \citep{Lueders2006,16346402} one parsimonious
hypothesis for $^{13}$C-labelling of \textit{Bacteroidetes} and
\textit{Actinobacteria} with a corresponding decrease in abundance of
$^{13}$C-labeled \textit{Firmicutes} is that the \textit{Bacteroidetes} and
\textit{Actinobacteria} fed on the $^{13}$C-labeled \textit{Firmicutes}. If the
temporal dynamics of $^{13}$C-xylose incorporation are due to trophic
interactions, our results suggest that many if not most fast-growing labile
C degraders are consumed by predatory bacteria. Hence, predatory interactions
between soil bacteria may be of importance for modelling soil C turnover.
Intra-bacteria trophic interactions in soil C cycling are often not
considered (e.g. \citealt{Moore1988}).

% Fakesubsubsection:Biogeochemical process models are the specific mechanism
It has been proposed that incorporating soil microbial community structure
into biogeochemical process models will improve predictions of global
C fluxes \citep{McGuire2010} but first we need to understand the
functional roles of soil microorganisms and the diversity of functional
guilds. In this study we demonstrate the soil functional guilds that
participated xylose and cellulose decomposition in our microcosms. These
functional guilds were non-overlapping in membership and represented a small
fraction of total soil community diversity (Figure~\ref{fig:genspec}). Of
104 $^{13}$C-responders only 8 responded to both cellulose and xylose.
Also, while we observed nearly 6,000 OTUs the number of conclusively
active OTUs in a given C transformation was two orders of magnitude fewer than
the total number of OTUs. Our results suggest that ecosystem C cycling models
may need to assess parameters independently for C-cycling functional guilds as
functional guilds are largely non-overlapping with respect to microbial
membership and each functional guild represents a small fraction of overall
soil microbial diversity.

\subsection{Conclusion} 
% Fakesubsubsection:Microorganisms sequester atmospheric C and respire
Microorganisms sequester atmospheric C and respire soil organic
matter (SOM) influencing climate change on a global scale but we do not
know which microorganisms carry out specific soil C transformations. In
this experiment microbes from uncharacterized yet ubiquitous soil lineages
participated in cellulose decomposition. Cellulose C degraders included
members of the \textit{Verrucomicrobia} (\textit{Spartobacteria}),
\textit{Chloroflexi}, \textit{Bacteroidetes} and \textit{Planctomycetes}.
\textit{Spartobacteria} in particular are globally cosmopolitan soil
microorganisms and are often the most abundant \textit{Verrucomicrobia}
order in soil \citep{Bergmann_2011}. Our results also suggest that
members of the \textit{Bacteroidetes} and \textit{Actinobacteria} act in
the cascade of labile, soluble C through soil trophic levels possibly as
predators. Both points illustrate the complexity of soil C dynamics and
fate. The largely phylogenetically coherent ecological groups observed in
this study suggest that soil C dynamics are tied to phylogenetic
composition.
