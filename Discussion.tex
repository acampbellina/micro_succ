\section{Discussion} 
% Fakesubsubsection: We identified microorganisms
We identified microorganisms participating in soil C cycling using a nucleic
acid SIP approach. Specifically, we observed assimilation of $^{13}$C from
either $^{13}$C-xylose or $^{13}$C-cellulose into DNA for 104 OTUs in an 
agricultural soil. We found $^{13}$C from $^{13}$C-xylose appeared to
move into and then out of groups of related OTUs over time. By coupling nucleic
acid SIP to high throughput sequencing we could diagnose OTU activity even when
OTUs were at low relative abundance in non-fractionated DNA (e.g. on three
occasions we did not detect $^{13}$C-responders in the non-fractionated DNA).
Our results support the degradative succession hypothesis, elucidate
ecophysiological properties of soil microorganisms, reveal activity of
widespread uncultured soil bacteria, and begin to piece together the microbial
food web in soils. 

% Fakesubsubsection: The degradative succession hypothesis
The degradative succession hypothesis predicts an ecological transition in
activity during the decomposition of labile and structural plant organic
matter. Our results concur with the degradative succession hypothesis.
Microorganisms rapidly metabolized xylose-C while cellulose-C metabolization proceeded
slowly. Xylose is a major constituent of hemicellulose and is a labile
component of fresh plant biomass. The phylogenetic composition of xylose
responders changed between days~1,~3 and~7 and few OTUs appeared
$^{13}$C-labeled in the $^{13}$C-xylose treatment after day~7. In the
$^{13}$C-cellulose treatment, $^{13}$C-labeled OTUs were few in the
beginning of the experiment and most abundant on day~14 and~30. Finally, few
(8~of~104) OTUs appeared to metabolize both xylose and cellulose indicating
most $^{13}$C-responders had distinct activity and that cellulose responders
grew in succession to xylose responders.

% Fakesubsubsection: Correlations between community composition
The ecological characteristics of microorganisms are often inferred from
correlations between community composition and environmental characteristics
\citep{Fierer2007}. In this experiment, we identified ecological
groups as a function of \textit{in situ} metabolism and inferred the ecological
properties of these groups through temporal dynamics of $^{13}$C-assimilation,
the extent of OTU $^{13}$C-labeling, and phylogenetic affiliation.
Xylose responders grew faster than cellulose responders and appeared to
assimilate C from multiple sources. Xylose responders assimilated xylose-C into
DNA within~24 hours and had low $\Delta\hat{BD}$ relative to cellulose
responders suggesting xylose was not the sole C source used for growth. Xylose
represented 15\% of the amendment and~3.5\% of total soil C. Xylose responders
often included the most abundant OTUs within the non-fractionated DNA and had
high estimated \textit{rrn} copy number relative to cellulose responders.
However, to some degree, high \textit{rrn} gene copy number may inflate
observed xylose responder relative abundance. Notably, the majority of xylose
responder SSU rRNA genes (86\%) matched SSU rRNA genes from cultured isolates
at high sequence identity ($>$ 97\%). 

% Fakesubsubsection: In contrast, the results
Cellulose responders, on the other hand, incorporated $^{13}$C into DNA after
xylose responders and appeared to specialize in using cellulose as a C source.
Cellulose responders grew over a span of weeks and had high $\Delta\hat{BD}$
indicating cellulose remained their dominant C source even though multiple
C sources were present (cellulose represented 6\% of total C present in
soil at the start of the experiment). Cellulose responders were also lower in
relative abundance on average within the non-fractionated DNA and had lower
estimated \textit{rrn} copy number than xylose responders. The majority of
cellulose responders were not close relatives of cultured isolates although
a number of cellulose responders shared high SSU rRNA gene sequence identity
with cultured \textit{Proteobacteria} (e.g. \textit{Cellvibrio}), . We
identified cellulose responders among phyla such as \textit{Verrucomicrobia},
\textit{Chloroflexi}, and \textit{Planctomycetes} -- common soil phyla whose
functions within soil communities remain unknown.

% Fakesubsubsection: Verrucomicrobia comprise
\textit{Verrucomicrobia} represented 16\% of the cellulose responders.
\textit{Verrucomicrobia} are cosmopolitan soil microbes \citep{Bergmann_2011}
that can make up to 23\% of SSU rRNA gene sequences in soils
\citep{Bergmann_2011} and 9.8\% of soil SSU rRNA \citep{Buckley_2001}. Genomic
analyses and laboratory experiments show that various isolates within the
\textit{Verrucomicrobia} are capable of methanotrophy, diazotrophy, and
cellulose degradation \citep{Wertz_2011,Otsuka_2012}. Moreover,
\textit{Verrucomicrobia} have been hypothesized to degrade polysaccharides in
many environments \citep{Fierer_2013,10543821,Herlemann_2013}. However, only
one of the 15 most abundant verrucomicrobial phylotypes in globally distributed
soil samples shared $>$ 93\% SSU rRNA gene sequence identity with a cultured
isolate \citep{Bergmann_2011} and hence the role of soil
\textit{Verrucomicrobia} in global C-cycling remains unknown. The majority of
verrucomicrobial cellulose responders belonged to two clades that fall within
the \textit{Spartobacteria} (Figure~\ref{fig:tiledtree}).
\textit{Spartobacteria} outnumbered all other \textit{Verrucomicrobia}
phylotypes in SSU rRNA gene surveys of
181 globally distributed soil samples \citep{Bergmann_2011}. Given their ubiquity and abundance
in soil as well as their demonstrated incorporation of $^{13}$C from
$^{13}$C-cellulose, \textit{Verrucomicrobia} lineages, particularly
\textit{Spartobacteria}, may be important contributors to cellulose
decomposition on a global scale. 

% Fakesubsubsection: Other notable cellulose responders
Other notable cellulose responders include OTUs in the \textit{Planctomycetes}
and \textit{Chloroflexi} both of which have previously been shown to
assimilate $^{13}$C from $^{13}$C-cellulose added to soil
\citep{Schellenberger_2010}. \textit{Planctomycetes} are common in soil
\citep{Janssen2006}, comprising 4 to 7\% of bacterial cells in many soils
\citep{Zarda_1997,Chatzinotas_1998} and 7\% $\pm$ 5\% of SSU rRNA
\citep{buckley_2003}. Although soil \textit{Planctomycetes} are widespread,
their activities in soil remain uncharacterized. \textit{Plantomycetes}
represented 16\% of cellulose responders and shared $<$ 92\% SSU rRNA gene
sequence identity to their most closely related cultured isolates.
\textit{Chloroflexi} are known for metabolically diverse lifestyles ranging
from anoxygenic phototrophy to organohalide respiration \citep{Hug_2013} and
are among the six most abundant bacterial phyla in soil \citep{Janssen2006}.
Recent studies have focused on \textit{Chloroflexi} roles in C cycling
\citep{Hug_2013,Goldfarb_2011,Cole_2013} and several \textit{Chloroflexi}
isolates use cellulose \citep{Hug_2013,Goldfarb_2011,Cole_2013}. Four of the
five \textit{Chloroflexi} cellulose responders belong to a single clade within
the \textit{Herpetosiphonales} (Figure~\ref{fig:tiledtree}). 

% Fakesubsubsection: Finally, a single
Finally, a single cellulose responder belonged to the \textit{Melainabacteria}
phylum (95\% shared SSU rRNA gene sequence identity with \textit{Vampirovibrio
chlorellavorus}). The phylogenetic position of \textit{Melainabacteria} is
debated but \textit{Melainabacteria} have been proposed to be
a non-phototrophic sister phylum to \textit{Cyanobacteria}. An analysis of
a \textit{Melainabacteria} genome \citep{Di_Rienzi_2013} suggests the
genomic capacity to degrade polysaccharides though \textit{Vampirovibrio
chlorellavorus} is an obligate predator of green alga \citep{gromov_1972}.

% Fakesubsubsection: Responders did not necessarily
Responders did not necessarily assimilate $^{13}$C directly
from $^{13}$C-xylose or $^{13}$C-cellulose. In many ways, knowledge of
secondary C degradation and/or microbial biomass turnover may be more
interesting with respect to the soil C-cycle than knowledge of primary
degradation. The response to xylose suggests xylose-C moved through different
trophic levels within the soil bacterial food web. The \textit{Bacilli}
degraded xylose first (65\% of the xylose-C had been respired by day~1)
representing 84\% of day~1 xylose responders. \textit{Bacilli} also comprised
about 6\% of SSU rRNA genes present in non-fractionated DNA on day~1. However,
few \textit{Bacilli} remained $^{13}$C-labeled by day~3 and their abundance
declined reaching about 2\% of soil SSU rRNA genes by day~30. Members of the
\textit{Bacillus} \citep{Cleveland2007} and \textit{Paenibacillus} in
particular \citep{Verastegui_2014} have been previously implicated as labile
C decomposers. The decline in relative abundance of \textit{Bacilli} could be
attributed to mortality and/or sporulation coupled to mother cell lysis.
\textit{Bacteroidetes} OTUs appeared $^{13}$C-labeled at day~3 concomitant with
the decline in relative abundance and loss of $^{13}$C-label for
\textit{Bacilli}. Finally, \textit{Actinobacteria} appeared $^{13}$C-labeled at
day~7 as \textit{Bacteroidetes} xylose responders declined in relative
abundance and became unlabeled. Hence, it seems reasonable to propose that
\textit{Bacteroidetes} and \textit{Actinobacteria} xylose responders became
labeled via the consumption of $^{13}$C derived from $^{13}$C-labeled microbial
biomass as opposed to primary degradation of $^{13}$C-xylose. 

% Fakesubsubsection: Trophic transfer could be due
The inferred physiology of \textit{Actinobacteria} and \textit{Bacteroidetes}
xylose responders provides further evidence for C transfer by
saprotrophy and/or predation. Most of the \textit{Actinobacteria} xylose
responders that appeared $^{13}$C-labeled at day~7 were members of the
\textit{Micrococcales} (Figure~\ref{fig:tiledtree}) and the most abundant
$^{13}$C-labeled \textit{Micrococcales} OTU at day~7 (“OTU.4”,
Table~\ref{tab:xyl}) is annotated as belonging in the \textit{Agromyces}.
\textit{Agromyces} are facultative predators that feed on the gram-positive
\textit{Luteobacter} in culture \citep{16346402}. Additionally, certain types
of \textit{Bacteroidetes} can assimilate $^{13}$C from $^{13}$C-labeled
\textit{Escherichia coli} added to soil \citep{Lueders2006}.
Alternatively, it is possible that \textit{Bacilli},
\textit{Bacteroidetes}, and \textit{Actinobacteria} are adapted to use
xylose at different concentrations and that the observed activity dynamics
resulted from changes in xylose concentration over time and/or that
\textit{Actinobacteria} and \textit{Bacteroidetes} xylose responders
consumed waste products generated by primary xylose metabolism (e.g.
organic acids produced during xylose metabolism). These latter two
hypotheses cannot explain the sequential loss of $^{13}$C-label, however.
If trophic transfer caused the activity dynamics, at least three different
ecological groups exchanged C in~7 days. Models of the soil C cycle often
exclude trophic interactions between soil bacteria (e.g.
\citep{Moore1988}), yet when soil C models do account for predators and/or
saprophytes, trophic interactions are predicted to have significant
effects on the fate of soil C \citep{Kaiser2014a}. 

\subsection{Implications for soil C cycling models}
% Fakesubsubsection: Models of soil
Functional niche characterization for soil microorganisms is necessary to
predict whether and how biogeochemical processes vary with microbial community
composition. Functional niches are defined by soil microbiologists and have
been successfully incorporated into biogeochemical process models (E.g.
\citep{wieder_2014a,Kaiser2014a}). In some C models ecological strategies such
as growth rate and substrate specificity are parameters for functional niche
behavior \citep{Kaiser2014a}. The phylogenetic breadth of a functionally
defined group is often inferred from the distribution of diagnostic genes
across genomes \citep{Berlemont2013} or from the physiology of isolates
cultured on laboratory media \citep{Martiny2013}. For instance, the wide
distribution of the glycolysis operon in microbial genomes is interpreted
as evidence that many soil microorganisms participate in glucose turnover
\citep{McGuire2010}. However, the functional niche may depend less on the
distribution of diagnostic genes across genomes and more on life history
traits that allow organisms to compete for a given substrate as it occurs
in the soil. For instance, fast growth and rapid resuscitation allow
microorganisms to compete for labile C which may often be transient in
soil. Hence, life history traits may constrain the diversity of microbes
that metabolize a given C source in the soil under a given set of
conditions.

% Fakesubsubsection: Biogeochemical processes
Biogeochemical processes mediated by a broad array of taxa are assumed
insensitive to community change relative to
processes mediated by a narrow suite of microorganisms
\citep{Schimel_1995,McGuire2010}. In addition, the diversity of
a functionally defined group engaged in a specific C transformation is
expected to correlate positively with C lability \citep{McGuire2010}.
However, the diversity of labile C and structural C decomposers in soil
has not been quantified directly. We found comparable numbers of OTUs
responded to $^{13}$C-cellulose and $^{13}$C-xylose (63 and~49,
respectively). Cellulose responders were phylogenetically clustered
suggesting that the ability to degrade cellulose is phylogenetically
conserved. The clade depth of cellulose responders, 0.028 SSU rRNA gene
sequence dissimilarity, is on the same order as that observed for
glycoside hydrolases which are diagnostic enzymes for cellulose
degradation \citep{Berlemont2013}. Xylose responders clustered in terminal
branches indicating groups of closely related taxa metabolized xylose but
xylose responders also clustered phylogenetically with respect to time of
response (Figure~\ref{fig:tiledtree}, Figure~\ref{fig:xyl_count}).
For example, xylose responders on day~1 are dominated by members of
\textit{Paenibacillus}. Thus, microorganisms that degraded labile C and
structural C were both limited in diversity. Although the genes for xylose
metabolism are likely widespread in the soil community, it's possible only
a limited diversity of organisms had the ecological characteristics
required to degrade xylose under experimental conditions. Therefore it's
possible that only a limited number of taxa actually participate in the
metabolism of labile C-sources under a given set of conditions, and hence
changes in community composition may alter the dynamics of structural
\textit{and} labile C-transformations in soil.

% Fakesubsubsection: Broadly, we observed
Broadly, we observed labile C use by fast growing generalists and structural
C use by slow growing specialists. These results agree with the MIMICS model
which simulates leaf litter decomposition by modeling microbial decomposers
as two functionally defined groups, copiotrophs or oligotrophs
\citep{wieder_2014a}. Including these functional types improved predictions
of C storage in response to environmental change. We identified
microbial lineages engaged in labile and structural C decomposition that
can be defined as copiotrophs or oligotrophs, respectively. Our results suggest
greater and or faster turnover for copiotroph biomass relative to oligotroph 
biomass, and that the copiotroph-oligotroph dichotomy leaves out guilds that may
play important roles in soil-C cycling.  That is, soil-C
may travel through multiple bacterial trophic levels where each C transfer
represents an opportunity for C stabilization in association with soil minerals
or C loss by respiration. Our understanding of soil C dynamics will likely
improve as we develop a more granular understanding of the ecological diversity
of microorganisms that mediate C transformations in soil.

\subsection{Conclusion} 
% Fakesubsubsection: Microorganisms sequester atmospheric C
Microorganisms govern\\ C-transformations in soil influencing climate change on
a global scale but we do not know the identities of microorganisms that carry
out specific transformations. In this experiment microbes from physiologically
uncharacterized but cosmopolitan soil lineages participated in cellulose
decomposition. Cellulose responders included members of the
\textit{Verrucomicrobia} (\textit{Spartobacteria}), \textit{Chloroflexi},
\textit{Bacteroidetes} and \textit{Planctomycetes}. \textit{Spartobacteria} in
particular are globally cosmopolitan soil microorganisms and are often the most
abundant \textit{Verrucomicrobia} order in soil \citep{Bergmann_2011}.
Fast-growing aerobic spore formers from \textit{Firmicutes} assimilated labile
C in the form of xylose. Xylose responders within the \textit{Bacteroidetes}
and \textit{Actinobacteria} likely became labeled by consuming $^{13}$C-labeled
constituents of microbial biomass either by saprotrophy or predation. Our
results suggest that cosmopolitan \textit{Spartobacteria} may degrade cellulose
on a global scale, decomposition of plant C may initiate trophic transfer 
within the bacterial food web, and life history traits may act
as a filter constraining the diversity of active microorganisms relative to
those with the genomic potential for a given metabolism.
