\section{Discussion} 
% Fakesubsubsection:
We highlight two key results with implications for understanding structure-function
relationships in soils, and for applying DNA-SIP in future studies of the soil-C
cycle. First, cellulose responders were members of physiologically undescribed
taxonomic groups with few exceptions. This suggests that we have much to learn
about the diversity of structural-C decomposers in soil before we can begin to
assess how they are affected by climate change and land management. Second, the
response to xylose was characterized by a succession in activity from
\textit{Paenibacillus} OTUs (day 1) to \textit{Bacteroidetes} (day 3) and finally
\textit{Micrococcales} (day 7). This activity succession was mirrored by relative
abundance profiles and may mark trophic-C exchange between these groups and/or
adaptation to different substrate concentrations. Trophic interactions will
critically influence how the global soil-C reservoir will respond to climate
change CITE but we know little of biological interactions among soil bacteria.
Often bacteria are cast as a single trophic level CITE but it may be
appropriate to investigate the soil food web at greater granularity.
Additionally, our results show that DNA-SIP results can change dramatically
over time suggesting that multiple time points are necessary to rigorously and
comprehensively describe which microorganisms consume $^{13}$C-labeled
substrates in nucleic acid SIP incubations.

% Fakesubsubsection:
Microorganisms that consumed $^{13}$C-cellulose were not closely related to any
physiologically characterized cultured isolates but were members of
cosmopolitan phylogenetic groups in soil including \textit{Spartobacteria},
\textit{Planctomycetes}, and \textit{Chloroflexi}. Often cellulose responders
were less than XX\% related to their closest cultured relative showing that we
can infer little, if anything at all, of their physiology from culture-based
studies. Notably, many \textit{Spartobacteria} were among the cellulose responder OTUs.
This is particularly interesting as \textit{Spartobacteria} are globally
dispersed and found in a variety of soil types CITE. These lineages may play
important roles in global cellulose turnover and would be interesting to track
in soil warming experiments to project how climate change might influence
cellulose process rates in soil (please see SI note 1 for further discussion of the
phylogenetic affiliation of cellulose responders). It should also be noted that
we amended our soil with bacterial cellulose which differs in structure than
plant biomass cellulose. These structural differences might select for different 
decomposers and caution should be taken when extrapolating our results.

% Fakesubsubsection:
In addition to taxonomic identity, we quantified several ecological
properties of microorganisms that were actively engaged in labile and
structural C decomposition in an agricultural soil over time. Labile C was consumed
before structural C by different microorganisms. This was expected and is
consistent with the degradative succession hypothesis. Consumers of labile C
had higher estimated \textit{rrn} gene copy number than structural C consumers.
\textit{rrn} copy number is positively correlated with the ability to
resuscitate quickly in response to nutrient influx CITE. Both xylose and cellulose
responders were terminally clustered phylogenetically suggesting habitat filtering
for labile and structural C consumers. In contrast to
structural C consumers, labile C consumers showed evidence for lower substrate
specificity. We assessed substrate specificity by measuring buoyant density
shift in response to $^{13}$C-labeling (see $\Delta|hat{BD}$). This technique
could be used to  assess relative substrate specificity for microorganisms that
consume a variety of substrates across a range of lability.  Although labile C
consumption is generally considered to be a ``broad'' process, we found that
xylose responders at day 1 were mainly members of one genus,
\textit{Paenibacillus}.  It should be noted, however, the types of
$^{13}$C-labeled microorganisms changed over time in the $^{13}$C-xylose
treatment, and these microorganisms were overall more diverse than those
labeled in the $^{13}$C-cellulose treatment. It is not clear, though,  if
$^{13}$C-labeled microorganisms in the $^{13}$C-xylose treatment beyond day 1
can be considered $^{13}$C-xylose consumers as they may have consumed
$^{13}$C-labeled metabolic byproducts or $^{13}$C-labeled biomass (see below).
Regardless, most xylose-C decomposition happened by day 1 and therefore this
experiment suggests that life-history traits such as the ability to resuscitate
quickly and/or maximum growth rate may constrain the diversity of
microorganisms involved in the consumption of the lion's share of labile C. 
Hence labile C decomposition may not be as ``broad'' a process as generally
thought based on studies of the genomic potential and ability in culture for
labile C consumption. The phylogenetic breadth of microbial guilds is important
to measure as the diversity of microorganisms that participate in an ecosystem
function is assumed to be positively correlated with how robust the function is
to changes in community composition CITE (see SI note 2 for further discussion
with respect to soil-C modelling).

% Fakesubsubsection:
We propose that the temporal fluctuations in $^{13}$C-labeling in the
$^{13}$C-xylose treatment are due to trophic exchange of $^{13}$C.
Alternatively, the temporal dynamics could be caused by microorganisms tuned to
different substrate concentrations and/or cross-feeding. Only trophic exchange,
however, can account for the precipitous drop in abundance of
\textit{Paenibacillus} with subsequent $^{13}$C-labeling of \textit{Bacteroidetes} at
day 3 followed by Micrococcales at day 7.  Furthermore, \textit{Bacteroidetes} types
have been shown to become $^{13}$C labeled after the addition of live
$^{13}$C-labeled \textit{Escherichia coli} to soil CITE and a member of the
\textit{Micrococcales}, \textit{Agromyces ramosus} (with which one of the $^{13}$C-xylose
responders shared 100\% SSU rRNA gene sequence identity), is a known predator
that preys on many microorganisms including yeast and \textit{Micrococcus
luteus} CITE. \textit{Agromyces} are abundant microorganisms in many soils and
the most abundant xylose responder in our experiment -- the fourth most abundant OTU in our pooled
dataset -- shares 100\% SSU rRNA gene sequence identity to \textit{Agromyces
ramosus}. Climate change is expected to enhance the availability of
labile C in soil and soil predators may mitigate the response of microorganisms
to increased labile C CITE though the extent of bacterial predatory 
activity in soil is unknown. Elucidating the identities of bacterial
predator in soil will assist in assessing the implications of climate change
on global soil-C storage.

\subsection{Implications for soil C cycling models}
% Fakesubsubsection: Models of soil
Functional niche characterization for soil microorganisms is necessary to
predict whether and how biogeochemical processes vary with microbial community
composition. Functional niches are defined by soil microbiologists and have
been successfully incorporated into biogeochemical process models (E.g.
\citep{wieder_2014a,Kaiser2014a}). In some C models ecological strategies such
as growth rate and substrate specificity are parameters for functional niche
behavior \citep{Kaiser2014a}. The phylogenetic breadth of a functionally
defined group is often inferred from the distribution of diagnostic genes
across genomes \citep{Berlemont2013} or from the physiology of isolates
cultured on laboratory media \citep{Martiny2013}. For instance, the wide
distribution of the glycolysis operon in microbial genomes is interpreted
as evidence that many soil microorganisms participate in glucose turnover
\citep{McGuire2010}. However, the functional niche may depend less on the
distribution of diagnostic genes across genomes and more on life history
traits that allow organisms to compete for a given substrate as it occurs
in the soil. For instance, fast growth and rapid resuscitation allow
microorganisms to compete for labile C which may often be transient in
soil. Hence, life history traits may constrain the diversity of microorganisms
that metabolize a given C source in the soil under a given set of
conditions.

% Fakesubsubsection: Biogeochemical processes
Biogeochemical processes mediated by a broad array of taxa are assumed
insensitive to community change relative to
processes mediated by a narrow suite of microorganisms
\citep{Schimel_1995,McGuire2010}. In addition, the diversity of
a functionally defined group engaged in a specific C transformation is
expected to correlate positively with C lability \citep{McGuire2010}.
However, the diversity of labile C and structural C decomposers in soil
has not been quantified directly. We found comparable numbers of OTUs
responded to $^{13}$C-cellulose and $^{13}$C-xylose (63 and~49,
respectively). Cellulose responders were phylogenetically clustered
suggesting that the ability to degrade cellulose is phylogenetically
conserved. The clade depth of cellulose responders, 0.028 SSU rRNA gene
sequence dissimilarity, is on the same order as that observed for
glycoside hydrolases which are diagnostic enzymes for cellulose
degradation \citep{Berlemont2013}. Xylose responders clustered in terminal
branches indicating groups of closely related taxa metabolized xylose but
xylose responders also clustered phylogenetically with respect to time of
response (Figure~\ref{fig:tiledtree}, Figure~\ref{fig:xyl_count}).
For example, xylose responders on day~1 are dominated by members of
\textit{Paenibacillus}. Thus, microorganisms that degraded labile C and
structural C were both limited in diversity. Although the genes for xylose
metabolism are likely widespread in the soil community, it's possible only
a limited diversity of organisms had the ecological characteristics
required to degrade xylose under experimental conditions. Therefore it's
possible that only a limited number of taxa actually participate in the
metabolism of labile C-sources under a given set of conditions, and hence
changes in community composition may alter the dynamics of structural
\textit{and} labile C-transformations in soil.

% Fakesubsubsection: Broadly, we observed
Broadly, we observed labile C use by fast growing generalists and structural
C use by slow growing specialists. These results agree with the MIMICS model
which simulates leaf litter decomposition by modeling microbial decomposers
as two functionally defined groups, copiotrophs or oligotrophs
\citep{wieder_2014a}. Including these functional types improved predictions
of C storage in response to environmental change. We identified
microbial lineages engaged in labile and structural C decomposition that
can be defined as copiotrophs or oligotrophs, respectively. We highlight two
additional considerations for soil-C process models based on our results.
First, soil-C may travel through multiple trophic levels within the bacterial
community where each C transfer represents an opportunity for C stabilization
in association with soil minerals or C loss by respiration. And second,
although labile C consumption is generally considered to be a broad process in
terms of microbial participants, we observed that only a small number of
related OTUs conclusively consumed xylose-C (see SI for additional discussion) and
that fast growth, as opposed the ability to use xylose, may constrain the diversity
of microorganisms that process labile-C \textit{in situ} which may often be
pulse delivered and transient. The diversity of microbial participants in a
biogeochemical process is thought to determine how robust process rates are to
changes in community composition. Our understanding of soil C dynamics will
likely improve as we develop a more granular understanding of the ecological
diversity of microorganisms that mediate C transformations in soil.

\subsection{Conclusion} 
% Fakesubsubsection: Microorganisms sequester atmospheric C
Microorganisms govern C-transformations in soil influencing climate change on
a global scale but we do not know the identities of microorganisms that carry
out each C transformation. In this experiment microorganisms from physiologically
uncharacterized but cosmopolitan soil lineages participated in cellulose
decomposition. Cellulose responders included members of the
\textit{Verrucomicrobia} (\textit{Spartobacteria}), \textit{Chloroflexi},
\textit{Bacteroidetes} and \textit{Planctomycetes}. \textit{Spartobacteria} in
particular are globally cosmopolitan soil microorganisms and are often the most
abundant \textit{Verrucomicrobia} order in soil \citep{Bergmann_2011}.
Fast-growing aerobic spore formers from \textit{Firmicutes} assimilated labile
C in the form of xylose. Xylose responders within the \textit{Bacteroidetes}
and \textit{Actinobacteria} likely became labeled by consuming $^{13}$C-labeled
constituents of microbial biomass either by saprotrophy or predation. Our
results suggest that cosmopolitan \textit{Spartobacteria} may degrade cellulose
on a global scale, decomposition of labile plant C may initiate trophic transfer 
within the bacterial food web, and life history traits may act
as a filter constraining the diversity of active microorganisms relative to
those with the genomic potential for a given metabolism.
