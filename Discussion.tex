\section{Discussion}
% Fakesubsubsection:Pure culture based studies drove early
Pure culture based studies drove early soil microbial ecology research.
Historically important soil isolates included nine genera:
\textit{Agrobacterium}, \textit{Alcaligenes}, \textit{Arthrobacter},
\textit{Bacillus}, \textit{Flavobacterium}, \textit{Micromonospora},
\textit{Nocardia}, \textit{Pseudomonas}, and \textit{Streptomyces}
(\citep{Alexander1977} and reviewed by \citep{Janssen2006}) but
culture-independent surveys of soil microbial diversity revealed soil can
harbor 5,000 OTUs per half gram of soil \citep{Schloss2006} and that cultured
isolates did not represent \textit{in situ} numerically abundant genera. We
recovered almost 6,000 OTUs in this study. Although culturing techniques can
produce isolates from diverse soil phylogenetic lineages \citep{Janssen2002},
numerically dominant soil microorganisms are still uncultured and we know
little of their ecophysiology \citep{Janssen2006}. In contrast, DNA-SIP can
characterize functional roles for thousands of phylotypes in a single
experiment. We found 104 OTUs in an agricultural soil that can incorporate
from xylose and/or cellulose into biomass. We also used DNA-SIP to assay
substrate specificity and temporal dynamics of C-cycling or soluble and
polymeric C degraders. Included in the $^{13}$C-xylose and $^{13}$C-cellulose
responsive OTUs were members of numerically dominant yet functionally
uncharacterized soil phylogenetic groups such as \textit{Verrucomicrobia},
\textit{Planctomycetes} and \textit{Chloroflexi}.

\subsection{Microbial response to isotopic labels}
% Fakesubsubsection: We propose that microbial decomposition
We propose that C added to soil microcosms in this experiment took the
following path through the microbial food web (Figure~\ref{fig:foodweb}):
First, labile C such as xylose was assimilated by fast-growing opportunistic
\textit{Firmicutes} spore formers. The remaining labile C and new biomass C was
assimilated in succession by slower growing \textit{Bacteroidetes},
\textit{Actinobacteria} and \textit{Proteobacteria} phylotypes that were either
tuned to lower C substrate concentrations, were predatory bacteria (e.g.
\textit{Agromyces}), and/or were specialized for consuming viral lysate. C from
polymeric substrates entered the bacterial community after 14 days. Canonical
cellulose degrading bacteria such as \textit{Cellvibrio} degraded cellulose
but uncharacterized lineages in the \textit{Chloroflexi},
\textit{Planctomycetes} and \textit{Verrucomicrobia}, specifically the
\textit{Spartobacteria}, were also significant contributors to cellulose
decomposition. 

\subsection{Ecological strategies of soil microorganisms participating in the
decomposition of organic matter}
% Fakesubsubsection:We assessed the ecology of
We assessed the ecology of $^{13}$C-responsive OTUs by estimating the
\textit{rrn} gene copy number and the BD shift upon labeling for each OTU.
\textit{rrn} gene copy number correlates positively with growth rate
\citep{11125085} and BD shift is indicative of substrate specificity (see
results). We also observed how $^{13}$C-substrate responsive OTUs changed in
relative abundance with time in the microcosms and the abundance rank of
$^{13}$C-substrate responsive OTUs in the bulk DNA. Ecological metrics show
$^{13}$C-cellulose responsive OTUs grow slower (Figure~\ref{fig:copy}), have
greater substrate specificity (Figure~\ref{fig:shift}), and are generally lower
abundance than $^{13}$C-xylose responsive OTUs (Figure~\ref{fig:shift}). The
higher abundance of xylose responders may also be in part due to of their high
\textit{rrn} gene copy number resulting in inflated relative abundance per
genome. There are only faint ecological differences
within the $^{13}$C-cellulose responsive OTUs but the combination of
\textit{rrn} gene copy number, BD shift, abundance rank and relative abundance
change over time is consistent with phylum membership (Figure~RADVIZ).
$^{13}$C-xylose responsive OTU \textit{rrn} gene copy number correlated
inversely with the time at which the OTU was first found to incorporate
$^{13}$C into DNA (Figure~\ref{fig:copy}) suggesting that fast-growing microbes
assimilated $^{13}$C from xylose before slow growers.  

% Fakesubsubsection:Ecological metrics suggest
Ecological metrics suggest cellulose degraders are substrate specialists that
grow slow and are in low bulk abundance. Labile C responder ecological
strategies were more varied perhaps because some $^{13}$C labeled
microorganisms did not primarily assimilate xylose but became labeled via
predatory interactions and/or are saprophytes. $^{13}$C-xylose responsive OTUs
are generalists, grow faster and are more abundant when compared to
$^{13}$C-cellulose responders. $^{13}$C-xylose responders vary in growth rate
and while generally higher abundance than $^{13}$C-cellulose responders can
also be low abundance microorganisms. It's not clear whether the observed
activity succession from \textit{Firmicutes} to \textit{Bacteroidetes} and
finally \textit{Actinobacteria} in response to $^{13}$C-xylose addition marks
a trophic cascade or functional groups tuned to different resource
concentrations or both. Notably, each temporally defined response group
clustered phylogenetically suggesting a uniform ecological strategy
(Figure~\ref{fig:trees}). It's also clear that some of the
non-\textit{Firmicutes} $^{13}$C-xylose responders are closely related to known
predators (\textit{Agromyces}) and many marine predatory bacteria are
members of the \textit{Bacteroidetes} (\citep{Banning2010a}). If the
temporal dynamics of $^{13}$C-xylose incorporation are due to trophic
interactions, our results suggest that there are many predatory soil
bacteria that consume fast-growing, opportunistic, primary labile
C assimilating, gram-positive spore-formers. Hence, trophic interactions
among soil bacteria may be of importance in soil C turnover models.

% Fakesubsubsection:How -- or if -- phylogenetic
How -- or if -- phylogenetic composition affects SOM dynamics is an open
question CITE. Phylogenetic composition could affect SOM dynamics if SOM
transformations were not functionally redundant traits and if biology is rate
limiting for key C transformations CITE. Alternatively, even with functional
redundancy resource allocation at the cell level can influence SOM fate CITE.
It is likely that the ability to carry out soil C transformations are redundant
within and between soil microbial communities and that in the mineral soil
abiotic factors are rate limiting CITE. Therefore phylogenetic composition in
mineral soil likely influences soil C fate as opposed to turnover. We
demonstrate phylogenetically coherent response to soluble C additions. For
instance, most of the initial response to xylose can be attributed to aerobic
spore formers. Assuming cellular resource allocation is consistent with
phylogeny, it follows then that phylogenetic composition can significantly
influence SOM fate. Aerobic spore-formers, for example, are found in different
proportions across soil biomes CITE and even within regional agricultural soils
CITE Berthrong. If present and abundant, aerobic spore-formers may be primary
soluble C decomposers and allocate C in specific quantities into intra and
extracellular C components. Further, aerobic spore-formers may have
a phylogenetically coherent resistance to predation which would affect soil
C fate from secondary decomposition. Although, not demonstrated in this study,
the allocation of C from soluble, labile pools in a soil without or under
conditions not suitable for aerobic spore-formers may be significantly
different. Polymeric C, on the other hand, did not show the same phylogenetic
coherence as soluble C decomposition in this study. This suggests that resource
allocation among cellulose degraders would not have a single phylogenetic
signal and the fate of polymeric C would not be tied to phylogenetic
composition. Though cellulose degraders as a whole likely allocate
C differently than labile C degraders.

\subsection{Conclusion} 
% Fakesubsubsection:SOM represents more C than the
Microorganisms sequester atmospheric C and respire soil organic matter (SOM)
influencing climate change on a global scale but we do not know which
microorganisms carry out specific soil C transformations. In this experiment
microbes from uncharacterized yet ubiquitous and often abundant soil lineages
participated in cellulose decomposition. Cellulose C degraders included members
of the \textit{Verrucomicrobia} (\textit{Spartobacteria}),
\textit{Chloroflexi}, \textit{Bacteroidetes} and \textit{Planctomycetes}.
\textit{Spartobacteria} in particular are abundant microorganisms in many soil
biomes and often the most abundant \textit{Verrucomicrobia} order in soil. Our
results also suggest that members of the \textit{Bacteroidetes} and
\textit{Actinobacteria} act in the cascade of labile, soluble C through soil
trophic levels possibly as predators. Both points illustrate the complexity of
soil C dynamics.
