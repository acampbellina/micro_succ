\section{Discussion}
% Fakesubsubsection:Early soil microbial ecology was informed
Pure cultures drove early soil microbial ecology. Historically important pure
cultures from soil included nine genera \textit{Agrobacterium},
\textit{Alcaligenes}, \textit{Arthrobacter}, \textit{Bacillus},
\textit{Flavobacterium}, \textit{Micromonospora}, \textit{Nocardia},
\textit{Pseudomonas}, and \textit{Streptomyces} (CITE Anderson, reviewed by
\citet{Janssen2006}) but culture-independent surveys of soil microbial
diversity revealed soil can harbors 5,000 OTUs per half gram
\citep{Schloss2006}. We recovered almost 6,000 OTUs in this study. Although
culturing techniques can produce isolates from diverse soil lineages
\citep{Janssen2002}, numerically dominant soil microorganisms are still
uncultured and we know little of their ecophysiology \citep{Janssen2006}.
DNA-SIP can characterize functional roles for thousands of phylotypes in
a single experiment. We found 104 OTUs in an agricultural soil that can
incorporate C from xylose or cellulose into biomass. Included in the
$^{13}$C-xylose and $^{13}$C-cellulose responsive OTUs were members of 
numerically dominant yet functionally uncharacterized soil phylogenetic groups
such as \textit{Verrucomicrobia}, \textit{Planctomycetes} and
\textit{Chloroflexi}. We also show how DNA-SIP can be used to assay substrate
specificity and temporal dynamics of C-cycling including microbial community
succession during decomposition.

\subsection{Microbial response to isotopic labels}
% Fakesubsubsection: We propose that microbial decomposition
We propose that organic matter in soil follows the
following path (Figure~XX): Labile C such as xylose is assimilated by
fast-growing opportunistic \textit{Firmicutes} spore formers. The remaining
labile C and new biomass C is assimilated in succession by slower
growing \textit{Bacteroidetes}, \textit{Actinobacteria} and
\textit{Proteobacteria} phylotypes that are either tuned to lower C substrate
concentrations, are predatory bacteria (e.g. \textit{Agromyces}), and/or are
specialized for consuming viral lysate. Polymeric C is likely assimilated by
fungal cellulose degraders before bacterial degraders CITE but the differences
in fungal and bacterial soil C assimilation are outside the scope of this
study. Polymeric C enters the bacterial community after several weeks.
Canonical cellulose degrading bacteria such as \textit{Cellvibrio} are major
cellulose degraders but uncharacterized lineages in the \textit{Chloroflexi},
\textit{Planctomycetes} and \textit{Verrucomicrobia}, specifically the
\textit{Spartobacteria}, are significant contributors to soil cellulose
decomposition as well. 

\subsection{Phylogenetic affiliation of $^{13}$C-cellulose
    $^{13}$C-xylose responsive microorganisms}

% Fakesubsubsection:\textit{Verrucomicrobia} are ubiquitous in soil
\textit{Verrucomicrobia}, cosmopolitan soil microbes
\citep{Bergmann_2011}, can comprise up to 23\% of 16S rRNA gene sequences in
high-throughput DNA sequencing surveys of SSU rRNA genes in soil
\citep{Bergmann_2011} and can account for up to 9.8\% of
soil 16S rRNA \citep{Buckley_2001}. Many \textit{Verrucomicrobia} were first
isolated in the last decade \cite{Wertz_2011} but only one of the 15 most
abundant verrucomicrobial phylotypes in a global soil sample collection shared
greater than 93\% sequence identity with a cultured isolate
\citep{Bergmann_2011}. Genomic analyses and physiological profiling of
\textit{Verrucomicrobia} isolates revealed \textit{Verrucomicrobia} are capable
of methanotrophy, diazotrophy, and cellulose degradation \citep{Otsuka_2012,
Wertz_2011}. The function of soil \textit{Verrucomicrobia} in global C-cycling
remains unknown. Only two of the ten putative cellulose degrading
\textit{Verrucomicrobia} identified in this experiment shares at least 95\%
sequence identity with an isolate ("OTU.83" and "OTU.627",
Table~\ref{tab:cell}). Seven of ten $^{13}$C-cellulose responding
verrucomicrobial OTUs were classified as \textit{Spartobacteria} which are
a numerically dominant family of \textit{Verrucomicrobia} in SSU rRNA gene
surveys of 181 globally distributed soil samples \citep{Bergmann_2011}. Given
their ubiquity and abundance in soil as well as their demonstrated
incorporation of $^{13}$C from $^{13}$C-cellulose, \textit{Verrucomicrobia}
lineages, particularly \textit{Spartobacteria}, may be important contributors
to cellulose decomposition on a global scale.

% Fakesubsubsection:Soil \textit{Chloroflexi} have been found to assimilate
Cellulose degrading soil \textit{Chloroflexi} have previously been identified
in DNA-SIP studies \citep{Schellenberger_2010}. The cellulose degrading
\textit{Chloroflexi} in this study are only distantly related to isolates
\ref{tab:cell}. Chloroflexi are among the six most abundant soil phyla commonly
recovered soil microbial diversity surveys \citep{Janssen2006}.
Chloroflexi are typically not as as abundant as \textit{Verrucomicrobia} but
are roughly as abundant as \textit{Bacteroidetes} and \textit{Planctomycetes}
\citep{Janssen2006}.  Four of five $^{13}$C-cellulose responsive
\textit{Chloroflexi} identified in this study are annotated as belonging to the
\textit{Herpetosiphon} although the \textit{Herpetosiphon} SSU rRNA gene
sequences from this study from $^{13}$C-cellulose responsive OTUs all share
less than 95\% sequence identity with their closest cultured relative in the
\textit{Herpetosiphon} genus (\textit{H. geysericola}). \textit{H. geysericola}
is a predatory bacterium shown to prey upon \textit{Aerobacter} in culture and
can also digest cellulose \citep{Lewin1970}. In our study, "Herpetosiphon"
$^{13}$C-cellulose responders did not show a delayed response to
$^{13}$C-cellulose as compared to other responders but nonetheless could have
become labeled by feeding on primary $^{13}$C-cellulose degraders. The prey
specificity of predatory bacteria is not well established especially \textit{in
situ}. $^{13}$C-labeling would be positively correlated with prey specificity.
If the predator specifically preyed upon one population then it could take on
the same labeling percent as that population. Preying on multiple types would
produce a mixed and dilute labeling signature if some of the prey
were not isotopically labeled.

% Fakesubsubsection:We also observed $^{13}$C-incorporation
We also observed $^{13}$C-incorporation from cellulose by
\textit{Proteobacteria}, \textit{Planctomycetes} and \textit{Bacteroidetes}. 
Strains in Proteobacteria, Planctomycetes and Bacteroidetes have all been
previously implicated in cellulose degradation. Planctomycetes is the
least studied of the three phyla and only one Planctomycetes isolate can
grow on cellulose. None of the seven \textit{Planctomycetes} cellulose degraders
identified in this experiment are closely related to isolates.
\textit{Acidobacteria} did not pass or operational criteria for assessing
$^{13}$C incorporation from cellulose into DNA in our microcosms. 
\textit{Acidobacteria} have been found to degrade cellulose in culture CITE and
are a numerically significant soil phylum CITE. \textit{Acidobacteria} have
been shown to dominate at low nutrient availability (CITE: cederlund 2014),
which may explain why were not active in this study's  nutrient replete
microcosm conditions. The \textit{Acidobacteria} in our microcosms were mainly
annotated as belonging to the "XX" group. 

\subsection{Ecological strategies of soil microorganisms participating in the
decomposition of organic matter}
% Fakesubsubsection:Ecological strategies of soil microorganisms
We observed ecological strategies among substrate responders although
boundaries between groups were not well defined. The observed ecological
strategy space appeared more like a continuum than discrete points. In general,
$^{13}$C-cellulose responders were slow-growers with low \textit{rrn} gene copy
number. $^{13}$C-cellulose responders were also typically low abundance members
of the microbial community and exhibited higher substrate specificity than
$^{13}$C-xylose responders (Figure~\ref{fig:shift}). 

%Fakesubsubsection:There was, however, as strong positive relationship
The \textit{Verrucomicrobia},
\textit{Chloroflexi} and \textit{Planctomycetes} $^{13}$C-cellulose responders exhibit 
phylum specific patterns of
substrate specificity, \textit{rrn} gene copy number, and growth over time (Figure~XX). There was a strong positive relationship between $^{13}$C-xylose
responders that responded at days 1,
3 or 7 in \textit{rrn} copy number. The estimated \textit{rrn} copy number of $^{13}$C-xylose responders declined over time, with the highest estimates at day 1. Therefore, OTUs that
grow faster assimilate C from xylose faster. With time fast growers
are replaced, with respect to xylose C assimilation, by slower growers (i.e. saprophytes and/or predators) as xylose diminishes. There was a succession in
activity with time from fast growing spore-formers to \textit{Bacteroidetes} and, finally, \textit{Actinobacteria} in our microcosms. The succession
hypothesis of decomposition groups ecological units by substrate CITE, however,
our results suggest there is a succession of microbial activity within
a single substrate. Hence, in soil, there is an ecological hierarchy coarsely
defined by the ability to assimilate C from labile or polymeric sources but
within the labile substrate degraders there are ecological subunits tuned to
specific substrate concentrations and/or trophic cascades on the temporal order
of days.

% Fakesubsubsection:HR-SIP allows us to assess C substrate specificity
C substrate specificity can be assessed by measuring the BD shift of OTU DNA
upon $^{13}$C incorporation. OTUs that incorporate more $^{13}$C per unit DNA
have greater specificity for the labeled substrate than OTUs that incorporate
less $^{13}$C per unit DNA. $^{13}$C-cellulose incorporating OTUs as a group
displayed greater substrate specificity than $^{13}$C-xylose incorporating
OTUs. This suggests that polymeric C-degraders tend to be specialists tuned to
particular C-substrates such as cellulose or lignin whereas labile C-degraders
are generalists able to assimilate C from many different labile sources.
Although we observed a succession of $^{13}$C-xylose responders
(Figure~\ref{fig:l2fc} and \ref{fig:xyl_count}), there was no discernible
difference in substrate specificity between $^{13}$C-xylose responders at days 1, 3 or 7. 

\subsection{Conclusion} 
% Fakesubsubsection:SOM represents more C than the

Microorganisms sequester atmospheric carbon and respire soil organic matter
(SOM) influencing climate change on a global scale but which microbial lineages
transform different soil C components are not established. Molecular tools will
unravel the soil microbial food web and reveal how specific microbial lineages
impact soil C flux. We present a high resolution DNA-SIP 
method to chart fine-scale C use into microbial lineages. This approach allows us to resolve discrete OTUs that would otherwise
be missed using fingerprinting techniques or bulk community sequencing efforts. Our results show physiologically
undefined cosmopolitan microbial lineages decompose cellulose. We also show
phylogenetic groups rise and fall and are supplanted by others in activity over
7 days in response to labile C addition. OTUs that assimilate xylose and those that assimilate cellulose are largely mutually exclusive.

The succession hypothesis of decomposition predicts a succession from
microbial types that use labile C to those that use recalcitrant polymeric
C over time CITE. Cellulose degraders succeeded labile C degraders as
predicted. But, in response to $^{13}$C-xylose,  \textit{Firmicutes}
phylotypes were succeeded by \textit{Bacteroidetes} which were then
succeeded by \textit{Actinobacteria} representing a nested succession
(Figure~XX). We found that $^{13}$C substrate responders changed as much as X-fold in
relative abundance over time (Figure~XX). This is in contrast to
a previous study CITE which suggested cellulose decomposers were found to
be consistent in relative abundance with time.
 
The xylose responders demonstrate a smaller change in BD than the
cellulose responders suggesting that xylose responders assimilate multiple C
sources (labeled and unlabeled) consistent with a generalist response, while
cellulose responders are more heavily labeled suggesting that cellulose is
their main source of C, a response more consistent with a specialist lifestyle.
Xylose responders include many taxa, such as spore-fomers, known for the
ability to respond rapidly to an influx of new nutrients while cellulose
responders include many OTUs that are common uncultivated soil organisms.
 
We did not observe consistent C utilization within a phylum although both
xylose and cellulose utilization were observed across 7 phyla each revealing a
high diversity of bacteria able to utilize these substrates. The high taxonomic
diversity may enable substrate metabolism under a broad range of environmental
conditions \citep{Goldfarb_2011}. Other studies of microbial communities have
observed a positive correlation with taxonomic or phylogenetic diversity and
functional diversity
\citep{Fierer_2012,Fierer_2013,Philippot_2010,Tringe_2005,Gilbert_2010,Bryant_2012}.
The data presented here supports that specific functional attributes can be
shared among diverse, yet distinct, taxa while closely related taxa may have
very different physiologies \citep{Fierer_2012,Philippot_2010}. This
information adds to the growing collection of data suggesting that community
membership is important to biogeochemical processes. Furthermore, it highlights
a need to examine substrate utilization by discrete microbial taxa within a
whole community context to better understand how specific community members
function within the whole. 

The sensitivity of HR-SIP provides a means to elucidate substrate utilization
by discrete microbial taxa with the hope that we can begin to construct a
belowground C food web. We obtained enough information to conclusively
determine isotope incorporation for 61\% of the more than 6,000 OTUs detected.
For those OTUs with enough information (n = 3,825), approximately 2\% (n = 72)
significantly assimilated $^{13}$C from either xylose or cellulose. In the
future deeper sequencing will enable us to increase coverage and assess C use
by more community members. Using the informations we gain from SIP-NGS, we can
expand our knowledge of specific C-cycling OTUs by taking a targeted
metagenomic approach in the nucleic acid pools of 'heavy' fractions.
Furthermore, we can now expand our knowledge of soil C use dynamics to a wide
array of C substrates and increase our grasp on specific community member
contributions. Illuminating these microbial contributions associated with
decomposition in soil are important because as environments change, there are
measurable and functional changes in soil C \citep{Grandy_2008} which could
cumulatively have large impacts at a global scale.
