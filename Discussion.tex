\section{Discussion}
Discuss interpretations of plots
    -compare to literature of what we already knew about cellulose (or xylose) degradation, microbial succession, or the application of realistic C substrate additions to soil and how that gives us a more natural view of what we might expect to see in nature.  
Boast about what this technique can do/ how it's added to our field.
Discuss the short-comings


While we are currently able to resolve highly responsive OTUs which has provided greater taxonomic resolution than previous techniques (cloning, TRFLP), the next step is to find better ways to resolve taxa that are partially responsive resulting in a less dramatic effect from the microbial community than highly responsive taxa.  

PCoA demonstrates microbial succession and based on xylose and cellulose falling out separately, it indicates different microbial community members are responsible for C degradation.  Using X ordination we can tease out members that cause the greatest shift in treatment versus control.  We then generate C utilization charts to demonstration discrete OTUs in control versus treatment.  Bulk community shifts over time. 

unlabeled substrate inherent in the system

Discuss interpretations of plots -compare to literature of what we already knew about cellulose (or xylose) degradation, microbial succession, or the application of realistic C substrate additions to soil and how that gives us a more natural view of what we might expect to see in nature. Boast about what this technique can do/ how it’s added to our field. Discuss the short-comings
While we are currently able to resolve highly responsive OTUs which has provided greater taxonomic resolution than previous techniques (cloning, TRFLP), the next step is to find better ways to resolve taxa that are partially responsive resulting in a less dramatic effect from the microbial community than highly responsive taxa.
Our study is consistent with carbon degradative succession has previously been demonstrated (ref). We demonstrate a rapid decrease in the labile carbon, xylose, confirmed by it’s 13C label incorporation into the microbial community DNA during the first 7 days of the experiment, after which, the label is not detectable in the DNA. Subsequently our data demonstrates a slow degradation of the more recalcitrant, polymeric carbon demonstrated by 13C-cellulose label incorporation into the microbial community DNA at 14 and 30 days.  We did not observe the 13C-cellulose signal leave the DNA within the time limits, 30 days, of our experiment.  This degradation succession is also confirmed by isotopic analysis of the soil from the microcosms.  PCoA demonstrates microbial succession and based on xylose and cellulose falling out separately, it indicates different microbial community members are responsible for C degradation. Using X ordination we can tease out members that cause the greatest shift in treatment versus control. We then generate C utilization charts to demonstration discrete OTUs in control versus treatment. Bulk community shifts over time.

It is likely that the slow degradation of cellulose can be attributed to the energy-taxing process of synthesizing cellulolytic enzymes and exporting them, as cellulose is broken down externally (Schimel & Schaeffer 2012, Lynd et al 2002).  As a result, microorganisms responsible for the synthesis of cellulases preferentially shuttle energy towards enzyme synthesis rather than biomass until cellulose hydrolysis begins (Schimel & Schaeffer 2012).  This accounts for the delay in growth and ultimately the slow decomposition of cellulose (Perez et al 2002, Schimel & Schaeffer 2012).

Traditionally, SIP is performed with a single time point in an attempt to minimize or eliminate cross-feeding of the substrate, as cross-feeding could cause false positives in the data set.  However, in a temporal food web mapping study, this cross-feeding is an exciting advantage of this system because it will enable us to track substrates of interest (i.e. cellulose) through many trophic cascades.  Observed organisms will exhibit a range of 13C labeling, 0-100\%, with primary consumers being the most enriched and subsequent trophic levels being less enriched (Morris 2002, McDonald 2005, Ziegler 2005).  This relationship of trophic level consumption to dilution of label will facilitate tracking C as it moves through various operational taxonomic units (OTUs) in the soil.  


With the rapid advancement and declining costs of high throughput sequencing, it has become increasingly easier to probe microbial communities.  In this study, we couple stable-isotope probing with 454 pyrosequencing in order to better understand organic matter decomposition dynamics as a function of soil microbial community C utilization. Overall, changes in microbial community composition over time is consistent with C decomposition being accompanied by a microbial community succession. 
