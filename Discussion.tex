\section{Discussion} 
\subsection{Microbial response to isotopic labels}
% Fakesubsubsection: We propose that microbial decomposition
DNA-SIP can establish functional roles for thousands of phylotypes in a single
experiment without cultivation. We identified 104 soil OTUs that incorporated
$^{13}$C from xylose and/or cellulose into biomass over time. With this
information we can build a conceptual model for the soil food web with respect
to xylose and cellulose in our microcosms. We propose xylose and
cellulose C added to soil microcosms took the following path through the
microbial food web (Figure~\ref{fig:foodweb}): fast-growing \textit{Firmicutes}
spore formers first assimilated xylose C within 24 hours. Over the next
6 days, biomass from early-responding \textit{Firmicutes} and the remaining
xylose-C and  was consumed by \textit{Bacteroidetes},
\textit{Actinobacteria}, and \textit{Proteobacteria} phylotypes. Canonical
cellulose degrading bacteria like \textit{Cellvibrio} and members of
cosmopolitan yet functionally uncharacterized soil phylogenetic groups like
\textit{Chloroflexi}, \textit{Planctomycetes} and \textit{Verrucomicrobia},
specifically the \textit{Spartobacteria}, decomposed cellulose. Cellulose
C incorporation into microbial biomass peaked at day 14 and extended through
day 30.

\subsection{Ecological strategies of soil microorganisms participating in the
decomposition of organic matter}
% Fakesubsubsection:Models of soil C cycling rely on
Models of soil C cycling rely on functional niches defined by ecologists and
ecological strategies such as growth rate and substrate specificity are
parameters for functional niche behavior in soil C models \citep{Kaiser2014a}.
Functional niches are commonly discovered by observing how community structure
changes with changing conditions \cite{Fierer2007}. In this experiment, DNA-SIP
revealed functional niche membership. We also used DNA-SIP data to quantify
substrate specificity which is related to the magnitude of DNA BD shift upon
$^{13}$C labeling (see Results). Moreover, we assessed growth rate of
functional niche members by estimating niche member \textit{rrn} gene copy
number, a genomic feature reliably extrapolated from phylogeny that is
indicative of how fast a microorganism grows \citep{11125085,Kembel_2012}. We
found that $^{13}$C-cellulose responsive OTUs are slow growing substrate
specialists relative to $^{13}$C-xylose responders (Figure~\ref{fig:shift},
Figure~\ref{fig:copy}). We also found that $^{13}$C-xylose responsive OTUs that
incorporated $^{13}$C into biomass at day one grow faster than OTUs that
responded later (Figure~\ref{fig:shift}, Figure~\ref{fig:copy}). If OTUs that
first responded after day 1 consumed labeled OTUs that responded at day 1, this
suggests that predators and/or saprophytes grow slower than microorganisms
directly assimilating labile C and has implications for modelling trophic
niches. We also note that $^{13}$C-cellulose responders are generally lower
abundance members of the bulk community than $^{13}$C-xylose responsive OTUs
(Figure~\ref{fig:shift}), however, High \textit{rrn} gene copy number may
inflate $^{13}$C-xylose-responder abundance.

% Fakesubsubsection:NRI values have been used to assess
NRI values quantify phylogenetic clustering \citep{Webb2000} and have 
been used to assess clustering of soil OTUs that responded similarly to soil
wet up \citep{Evans2014a,Placella2012}. To our knowledge, assessing
phylogenetic clustering of OTUs found to incorporate heavy isotopes into
biomass during SIP incubations has not been attempted. We found that $^{13}$C-
cellulose and xylose responders are clustered and overdispersed, respectively.
This suggests that the ability to degrade cellulose is phylogenetically
conserved possibly reflecting the complexity of cellulose degradation
biochemistry. The positive relationship between a physiological trait's
phylogenetic depth and complexity has been noted previously
\citep{Martiny2013a} and the clade depth of $^{13}$C-cellulose responders,
0.028 16S rRNA gene sequence dissimilartiy, is on the same order as that
observed for glycoside hydrolases which are diagnostic enzymes for cellulose
degradation \citep{Berlemont2013}. Overdispersion, as we saw for the
$^{13}$C-xylose responsive OTUs, may be indicative of a readily horizontally
transferred trait and/or a trait that is broadly distributed phylogenetically,
or, in this case may indicate that the overdispersed group includes more than
one trait. It's not clear which $^{13}$C-xylose responsive organisms were
labeled as a result of primary xylose assimilation (see below), and therefore
it's not clear if $^{13}$C-xylose responsive OTUs in this experiment constitute
a single ecologically meaningful group or multiple ecological groups.
Temporally defined $^{13}$C-xylose responder groups, however, are
phylogenetically coherent (Figure~\ref{fig:tiledtree},
Figure~\ref{fig:xyl_count}). For example, most day
1 $^{13}$C-xylose responders are members of the \textit{Paenibacillus} (see
Supplemental~Note~XX). Notably, \textit{Paenibacillus} have been previously
implicated as labile C decomposers \citep{Verastegui_2014}.

% Fakesubsubsection:Intuitively C cycling trait diversity is inferred
Intuitively we infer C cycling functional guild diversity from the distribution
of diagnostic genes across genomes \citep{Berlemont2013} or from screening
culture collections for a particular trait \citep{Martiny2013}. For
instance, the wide distribution of the glycolysis operon in microbial
genomes is interpreted as evidence that many soil microorganisms
participate in glucose turnover \citep{McGuire2010}. \textit{In situ}
functional guild diversity, however, can vary significantly from diversity
assessed by functionally screening isolates and/or genomes. Xylose use in soil,
for instance, may be less a function of catabolic pathway distribution across
genomes and more a function of lifestyle. Phenomena such as seasonal change
\citep{Schmidt2007}, and rainfall \citep{Placella2012} pulse deliver nutrients
and resources to soil. Therefore, fast growth and/or rapid resuscitation upon
wet up \citep{Placella2012} allow microorganisms to favorably compete for
labile C resources. Life history may limit the diversity of labile
C assimilators as life history determines growth rate and dessication
resistance even though the ability to use labile C is phylogenetically
dispersed. DNA-SIP is useful for establishing \textit{in situ} phylogenetic
clustering and diversity of functional guilds because DNA-SIP can account for
life history strategies by targeting active microorganisms. Additionally,
snapshot estimates of community composition commonly inform soil
structure-function-relationship studies \citep{Fierer2007} but labile
C decomposition might not be linked to snapshot community structure.
Alternatively labile C decomposition might be linked specifically to community
structure \textit{dynamics}. That is, fast growing spore formers would not need
to maintain high abundance to significantly mediate cycling of pulse delivered
resources. This accentuates the usefulness of DNA-SIP for describing soil
ecology as DNA-SIP assesses activity which can be decoupled from snapshot
abundance.

\subsection{Implications for soil C cycling models}
% Fakesubsubsection:Land management, climate, pollution and
Land management, climate, pollution and disturbance can influence soil
community composition \citep{McGuire2010} which in turn influences soil
biogeochemical process rates (e.g. \citep{Berlemont2014a}). Assessing
functional group diversity and establishing identities of functional group
members is necessary to predict how biogeochemical process rates will change
with community composition \citep{Schimel_1995,McGuire2010}. Aggregate
biogeochemical processes that are the sum of many subprocesses involve a broad
array of taxa and are assumed to be less influenced by community change than
narrow processes that involve a single, specific chemical transformation by
a smaller suite of microbial participants \citep{Schimel_1995,McGuire2010}.
Within an aggregate process such as C decomposition, subprocesses can be
further classified as broad or narrow \citep{McGuire2010}. In theory,
``broad'' and ``narrow'' functional guilds decompose labile and
recalcitrant C, respectively \citep{McGuire2010}. However, the diversity
of active labile C and insoluble, polymeric C decomposers in soil has not
been directly quantified. Notably, we found more OTUs responded to
$^{13}$C-cellulose, 63, than $^{13}$C-xylose, 49. Also, it is possible that
many $^{13}$C-xylose responders are predatory bacteria or saprophytes as
opposed to primary labile C degraders (see below). Cellulose and xylose
decomposer functional guilds were non-overlapping in membership -- of
104 $^{13}$C-responders only 8 responded to both cellulose and xylose -- and
represented a small fraction of total soil community diversity
(Figure~\ref{fig:genspec}). While xylose use is undoubtedly more widely
distributed among microbial genomes than the ability to degrade cellulose, the
number of unique active cellulose decomposers OTUs outnumbered the number of
unique active xylose utilizers OTUs.

% Fakesubsubsection:Both $^{13}$C-cellulose and $^{13}$C-xylose
Both $^{13}$C-cellulose and $^{13}$C-xylose responders largely clustered
near the tips of the phylogenetic tree (NTI $>$ 0) at taxonomic levels broader
than the OTUs established in this study (Figure~\ref{fig:tiledtree}).
Therefore, $^{13}$C-responders distribute into fewer clades than OTUs
(Figure~\ref{fig:tiledtree}). 

% Fakesubsubsection:It's not clear whether the observed activity
Trophic interactions and/or functional groups tuned to different resource
concentrations caused the activity succession from \textit{Firmicutes} to
\textit{Bacteroidetes} and finally \textit{Actinobacteria} in response to
xylose addition. \textit{Actinobacteria} (e.g. \textit{Agromyces}) and
\textit{Bacteroidetes} have been previously implicated as predatory soil
bacteria \citep{Lueders2006,16346402}, and, \textit{Bacteroidetes} and
\textit{Actinobacteria} activity peaked while \textit{Firmicutes}
$^{13}$C-xylose responder relative abundance plummeted in our microcosms.
Considering \textit{Agromyces} and \textit{Bacteroidetes} phylotypes are likely
soil predators, one parsimonious hypothesis for $^{13}$C-labelling of
\textit{Bacteroidetes} and \textit{Actinobacteria} with a corresponding
decrease $^{13}$C-labeled \textit{Firmicutes} abundance is that
\textit{Bacteroidetes} and \textit{Actinobacteria} fed on $^{13}$C-labeled
\textit{Firmicutes}. Besides predation, mother cell lysis could be the
mechanism for transferring C from spore formers to \textit{Bacteroidetes} and
\textit{Actinobacteria}. If the temporal dynamics of $^{13}$C-xylose
incorporation are due to trophic interactions, predatory bacteria or
saprophytes consumed many, if not most, fast-growing labile C degraders.
Hence, soil C cycling models should include trophic interactions between soil
bacteria but rarely do (e.g. \citep{Moore1988}). When soil C models do account
for predators/saprophytes, trophic interactions are predicted to significantly
influence c:N ratios of soil DOM relative to litter C:N, and, cause significant
amounts of microbial biomass to be recycled \citep{Kaiser2014a}.

% Fakesubsubsection:We propose two scenarios whereby community
We propose two scenarios in the context of our results whereby community
composition could affect C dynamics and fate. Genomic evidence shows cellulose
degradation is a phylogenetically conserved trait \citep{Berlemont2013}. Our
study evaluates the phylogenetic conservation of soil cellulose degradation in
active microorganisms via DNA-SIP and genomic evidence concurs with our
results. A decrease in cellulose degrader abundance would diminish cellulose
decomposition process rates as few soil microorganisms can fill the
phylogenetically conserved cellulose degradation niche. Dispersed cellulose
decomposers could renew ecosystem function, however. For labile
C decomposition, the absence of fast growing spore formers would allow other
microbes to assimilate labile C provided dispersal does not enable rapid
recolonization. Primary labile C degraders in this study grow fast, and
form spores and these distinct ecological strategies might indicate distinct
C use dynamics and/or resource allocation. New labile C degraders may
metabolize and allocate labile C differently thus changing labile C dynamics
and fate. Further, labile C degrader substitution could affect biomass
C turnover by predatory bacteria or saprophytes that feed on fast growing,
spore forming labile C decomposers. On the other hand, spore formation enables
dispersal \citep{Nicholson2000} which would allow fast growing spore formers to
continuously occupy the labile C decomposition niche. One proposed mechanism
for similar decomposition rates of labile C across soils varying in community
composition is that labile C can be used widely by microorganisms
\citep{McGuire2010}. An alternative hypothesis for consistent labile C process
rates across different soils is that labile C degraders disperse readily.
Notably, other lineages implicated in rapid labile C turnover include members
of the \textit{Actinobacteria} \citep{Placella2012} and many soil
\textit{Actinobacteria} form hyphae that facilitate dispersal
\citep{KILLHAM2007}. The two hypotheses are not mutually exclusive, but our
results and previous studies suggest that environmental conditions unfavorable
to fast-growing spore-formers and/or quickly resuscitated, hyphal
\textit{Actinobacteria} may impact labile C dynamics and fate.

\subsection{Conclusion} 
% Fakesubsubsection:Microorganisms sequester atmospheric C and respire
Microorganisms sequester atmospheric C and respire SOM influencing climate
change on a global scale but we do not know which microorganisms carry out
specific soil C transformations. In this experiment microbes from
physiologically uncharacterized but ubiquitous soil lineages participated in
cellulose decomposition. Cellulose C degraders included members of the
\textit{Verrucomicrobia} (\textit{Spartobacteria}), \textit{Chloroflexi},
\textit{Bacteroidetes} and \textit{Planctomycetes}. \textit{Spartobacteria} in
particular are globally cosmopolitan soil microorganisms and are often the most
abundant \textit{Verrucomicrobia} order in soil \citep{Bergmann_2011}.
Fast-growing \textit{Firmicutes} spore formers assimilated labile C in our
microcosms. \textit{Bacteroidetes} and \textit{Actinobacteria} phylotypes,
previously implicated as predators, may have fed on the fast growing
\textit{Firmicutes}. Our results suggest that, cosmopolitan
\textit{Spartobacteria} may degrade cellulose on a global scale, bacterial
tropic interactions can significantly impact soil C cycling, and life history
ecological strategies such as fast growth  constrain functional guild
diversity for labile C decomposition.
