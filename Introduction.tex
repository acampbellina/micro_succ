\section{Introduction}
% Fakesubsubsection:Excluding plant biomass, there are 2,300 Pg of
Excluding plant biomass, there are 2,300 Pg of carbon (C) stored in soils
worldwide which accounts for $\sim$80\% of the global terrestrial C pool
\citep{Amundson_2001,BATJES_1996}. When organic C from plants reaches soil it
is degraded by fungi, archaea, and bacteria. This C is returned to the
atmosphere as CO$_{2}$ or remains in the soil as humic substances that can
persist up to 2000 years \citep{yanagita1990natural}. The majority of plant
biomass C in soil is respired and produces 10 times more CO$_{2}$ than
anthropogenic emissions on an annual basis \citep{chapin2002principles}.
Global changes in atmospheric CO$_{2}$, temperature, and ecosystem nitrogen
inputs are expected to impact primary production and C inputs to soils
\citep{Groenigen_2006} but it remains difficult to predict the response of soil
processes to anthropogenic change \citep{DAVIDSON_2006}. Current climate change
models concur on atmospheric and oceanic but not terrestrial C predictions
\citep{Friedlingstein_2006}. Contrasting terrestrial C model predictions
reflect how little is known about soil C cycling.  Inconsistencies in terrestial
modeling could be improved by elucidating the relationship between dissolved
organic carbon and microbial communities in soils \citep{Neff_2001}.

% Fakesubsubsection:An estimated 80-90\% of C cycling in soil
An estimated 80-90\% of C cycling in soil is mediated by microorganisms
\citep{ColemanCrossley_1996,Nannipieri_2003}. Understanding microbial
processing of nutrients in soils presents a special challenge due to the
hetergeneous nature of soil ecosystems and methods limitations. Soils are
biologically, chemically, and physically complex which affects microbial
community composition, diversity, and structure \citep{Nannipieri_2003}.
Confounding factors such as physical protection/aggregation, moisture content,
pH, temperature, frequency and type of land disturbance, soil history,
mineralogy, N quality and availability, and litter quality all affect the
ability of the soil microbial community to access and metabolize C substrates
\citep{Sollins_Homann_Caldwell_1996,Kalbitz_2000}. Further, rates of metabolism
are often measured without knowing the identity of the microbial species
involved \citep{ndi_Pietramellara_Renella_2003} leaving the importance of
community membership towards maintaining ecosystem functions unknown
\citep{Allison_2008,ndi_Pietramellara_Renella_2003,Schimel_2012}. Litter bag
experiments have shown that the community composition of soils can have
quantitative and qualitative impacts on the breakdown of plant materials
\citep{Schimel_1995}. Reciprocal exchange of litter type and microbial inocula
under controlled environmental conditions reveals that differences in community
composition can account for 85\% of the variation in litter carbon
mineralization \citep{Strickland_2009}. In addition, assembled communities of
cellulose degraders reveal that the composition of the community has
significant impacts on the rate of cellulose degradation \citep{Wohl_2004}. 

% Fakesubsubsection:An important step in understanding soil C cycling
An important step in understanding soil C cycling dynamics is identifing
contributions from specific microbial lineages and investigating the
relationship between genetic diversity and community structure with function
\citep{O_Donnell_2002}. The vast majority of microorganisms continue to resist
cultivation in the laboratory, and even when cultivation is achieved, the
traits expressed by a microorganism in culture may not be representative of
those expressed when in its natural habitat. Stable-isotope probing (SIP)
provides a unique opportunity to link microbial identity to activity and has
been utilized to expand our knowledge of biogeochemical processes
\citep{Chen_Murrell_2010}. The most successful applications of this technique
have identified organisms which mediate processes performed by a narrow set of
functional guilds such as methanogens \citep{Lu_2005}. The technique has been
less applicable to the study of soil C cycling because of limitations in
resolving power as a result of simultaneous labeling of many different
organisms in the community. Additionally, molecular applications such as tRFLP,
DGGE and cloning that are frequently used in conjunction with SIP provide
insufficient resolution of taxon identity and depth of coverage. We have
developed an approach called \textbf{H}igh \textbf{R}esolution SIP (HR-SIP)
that employs a complex mixture of substrates added to soil at a low
concentration relative to soil organic matter pools along with high throughput
DNA sequencing. This greatly expands the ability of nucleic acid SIP to explore
complex patterns of C-cycling in microbial communities with increased
resolution.

% Fakesubsubsection:A temporal cascade occurs in natural microbial
A temporal cascade occurs in natural microbial communities during the plant
biomass degradation in which labile C degradation precedes polymeric C
degradation \citep{Hu_1997,Rui_2009}.  The aim of this study is to track the
temporal dynamics of C assimilation through discrete individuals of the soil
microbial community to provide greater insight into soil C-cycling.  Our
experimental approach includes addition of a soil organic matter (SOM) simulant
(a complex mixture of model carbon sources and inorganic nutrients common to
plant biomass), where a single C constituent is substituted for its
$^{13}$C-labeled equivalent, to soil.  Parallel incubations of soils amended
with this complex C mixture allows us to test how different C substrates
cascade through discrete taxa within the soil microbial community.  In this
study we use $^{13}$C-xylose and $^{13}$C-cellulose as a proxy for labile and
polymeric C, respectively.  We couple nucleic acid stable isotope probing with
high throughput DNA sequencing to identify soil microbial community members
responsible for specific C transformations.  Amplicon sequencing of 16S rRNA
gene fragments from many gradient fractions and multiple gradients make it
possible to track C assimilation by hundreds of taxa.
