\section{Introduction}
% Fakesubsubsection:Excluding plant biomass, there are 2,300 Pg of
Excluding plant biomass, there are 2,300 Pg of carbon (C) stored in soils
worldwide which accounts for $\sim$80\% of the global terrestrial C pool
\citep{Amundson_2001,BATJES_1996}. When organic C from plants reaches soil it
is degraded by fungi, archaea, and bacteria.  The majority of plant biomass
C in soil is respired and produces 10 times more CO$_{2}$ annually than
anthropogenic emissions \citep{chapin2002principles}. Global
changes in atmospheric CO$_{2}$, temperature, and ecosystem nitrogen inputs are
expected to impact soil C input \citep{Groenigen_2006}. Current climate change
models concur on atmospheric and oceanic but not terrestrial C predictions
\citep{Friedlingstein_2006}. Contrasting terrestrial C model predictions
reflect how little is known about soil C cycling. Inconsistencies in
terrestial modeling could be improved by elucidating the relationship between
dissolved organic C and soil microbial community composition \citep{Neff_2001}.


% Fakesubsubsection:An important step in understanding soil C cycling
An important step in understanding soil C cycling dynamics is identifying the
\textit{in situ} activity of specific microbial lineages to establish the
relationship between community structure and function \citep{O_Donnell_2002}.
An estimated 80-90\% of soil C cycling is mediated by microorganisms
\citep{ColemanCrossley_1996,Nannipieri_2003} but understanding microbial
processing of soil nutrients is challenging due to soil's hetergeneous nature
and methods limitations. The vast majority of microorganisms resist cultivation
in the laboratory. Stable-isotope probing (SIP) links microbial identity and
activity without cultivation and has expanded our knowledge of microbial
contributions to biogeochemical processes \citep{Chen_Murrell_2010}. The most
successful applications of SIP have identified organisms which mediate
processes performed by a narrow set of functional guilds such as methanogens
\citep{Lu_2005}. SIP has been less applicable to the study of soil C cycling
because of limitations in resolving power as a result of simultaneous labeling
of many different organisms in the community. Additionally, molecular
applications such as tRFLP, DGGE and cloning that are frequently used in
conjunction with SIP provide insufficient resolution of taxon identity and/or
depth of coverage. We developed an approach called \textbf{H}igh
\textbf{R}esolution-SIP (HR-SIP) that employs a complex mixture of substrates
added to soil at a low concentration relative to soil organic matter pools
along with high throughput DNA sequencing. This greatly expands the ability of
nucleic acid SIP to explore complex patterns of C-cycling in microbial
communities with increased resolution.

% Fakesubsubsection:A temporal cascade occurs in natural microbial
A temporal activity cascade occurs in natural microbial communities during
plant biomass degradation in which labile C is degraded before polymeric
C \citep{Hu_1997,Rui_2009}.  The aim of this study was to observe temporal
dynamics of C assimilation through discrete soil community members. Our
experimental approach included the addition of a soil organic matter (SOM)
simulant (a complex mixture of model carbon sources and inorganic nutrients
common to plant biomass) to soil microcosms where a single C component is
substituted for its $^{13}$C-labeled equivalent. Parallel incubations of soils
amended with this complex C mixture allows us to observer how different
C substrates move through the soil microbial community. In this study we used
$^{13}$C-xylose and $^{13}$C-cellulose as a proxy for labile and polymeric C,
respectively, and coupled nucleic acid stable isotope probing with high
throughput DNA sequencing. Amplicon sequencing of 16S rRNA
gene fragments from many gradient fractions and multiple gradients make it
possible to track C associated activities for hundreds of soil taxa.
