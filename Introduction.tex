\section{Introduction}
% Fakesubsubsection:Excluding plant biomass, there are 2,300 Pg of
Soils worldwide contain 2,300 Pg of carbon (C) which accounts for nearly~80\%
of the C present in the terrestrial biosphere
\citep{Amundson_2001,BATJES_1996}. C respiration by soil microorganisms
produces annually tenfold more CO$_{2}$ than fossil fuel emissions
\citep{chapin2002principles}. Despite the contribution of microorganisms to
global C flux, many global C models ignore microbial physiological diversity
and its impacts on microbial activity in soils.
\citep{Allison2010,Six2006,Treseder2011}. Further, predictions of climate
change feedbacks on soil C flux improve when biogeochemical models explicitly
represent microbial physiology \citep{Wieder2013}. However, we still know
little about the ecophysiology of soil microorganisms, and such knowledge
should assist the development and refinement of global C models
\citep{Bradford2008,Neff_2001,McGuire2010}.

%Fakesubsubsection:The decomposition
Cellulose comprises most plant C (30-50\%) followed by hemicellulose (20-40\%),
and lignin (15-25\%) \citep{Lynd2002}. Hemicellulose, being the most soluble,
degrades in the early stages of decomposition. Xylans are often an abundant
component of hemicellulose, and xylans themselves include differing amounts of
xylose, glucose, arabinose, galactose, mannose, and rhamnose \citep{Saha2003}.
Xylose is often the most abundant sugar in hemicellulose, comprising as much as
60-90\% of xylan in some plants (e.g hardwoods) \citep{Spiridon2008}, wheat
\citep{Sun2005}, and switchgrass \citep{Bunnell2013}. Microbes that respire
sugars proliferate during the initial stages of decomposition
\citep{Garrett1951,Alexander1964}, and metabolize as much as 75\% of sugar
C during the first 5 days of decomposition \citep{Engelking2007}. In contrast,
cellulose decomposition proceeds more slowly with rates increasing for
approximately 15 days while degradation continues for 30-90 days
\citep{Hu1997,Engelking2007}. It is hypothesized that different microbial
guilds mediate the decomposition of different plant biomass components
\citep{Hu1997,Rui2009,AnneliseHKjoller2002,Bastian_2009}. The degradative
succession hypothesis posits that fast growing organisms proliferate in
response to the labile fraction of plant biomass such as sugars
\citep{Garrett1963,Bremer1994} followed by slow growing organisms targeting
structural C such as cellulose \citep{Garrett1963}. Evidence to support the
degradative succession hypothesis comes from observing soil respiration
dynamics and characterizing microbes cultured at different stages of
decomposition. The degree to which the succession hypothesis presents an
accurate model of litter decomposition has been called into question
\citep{AnneliseHKjoller2002,Frankland_1998,Osono_2005} and it's clear that
we need new approaches to dissect microbial contributions to
C transformations in soils.

% Fakesubsubsection:Though microorganisms mediate
Though microorganisms mediate 80-90\% of the soil C-cycle
\citep{ColemanCrossley_1996,Nannipieri_2003}, and microbial community
composition can account for significant variation in C mineralization
\citep{Strickland_2009} , terrestrial C-cycle models rarely consider the
community composition of soils \citep{Zak2006,Reed2007}. We measure rates of
soil C transformations without knowledge of the organisms that
mediate these reactions \citep{Nannipieri_2003} leaving undefined the
importance of community membership towards maintaining ecosystem function
\citep{Nannipieri_2003,Schimel_2012,Allison_2008}. Variation in microbial
community composition can be linked effectively to rates of soil processes when
diagnostic genes for specific functions are available (e.g. denitrification
\citep{Cavigelli2000}, nitrification \citep{Carney2004,Hawkes2005,Webster2005},
methanotrophy \citep{Gulledge1997}, and nitrogen fixation \citep{Hsu2009}).
However, the complexity of soil C transformations and the lack of diagnostic
genes for describing these transformations has limited progress in
characterizing the contributions of individual microbes to the soil C-cycle.
Remarkably, we still lack basic information on the physiology and ecology of
the majority of organisms that live in soils. For example, contributions to
soil processes remain uncharacterized for entire and cosmopolitan bacterial
phyla in soil  such as \textit{Acidobacteria}, \textit{Chloroflexi},
\textit{Planctomycetes}, and \textit{Verrucomicrobia}. These phyla combined can
comprise 32\% of soil microbial communities (based on surveys of the SSU rRNA
genes in soil) \citep{Janssen2006,Buckley2002}. 

% Fakesubsubsection:Functional guild membership
Characterizing the functions of microbial taxa has relied historically on
culturing microorganisms and subsequently characterizing physiology in the
laboratory and on environmental surveys of genes diagnostic for specific
processes. However, most microorganisms are difficult to grow in
culture \citep{Janssen2006} and many processes lack suitable diagnostic genes.
Nucleic acid stable-isotope probing (SIP) links genetic identity and activity
without the need to grow microorganisms in culture and has expanded our
knowledge of microbial contributions to biogeochemical processes
\citep{Chen_Murrell_2010}. However, nucleic acid SIP has notable complications
including the need to add large amounts of labeled substrate
\citep{radajewski2000stable}, label dilution resulting in partial labelling of
nucleic acids \citep{radajewski2000stable,Manefield_2002,McDonald_2005}, the
potential for cross-feeding and secondary label incorporation
\citep{Morris_2002,Hutchens2004,14686943,DeRito2005,McDonald_2005,Ziegler_2005},
and variation in genome G$+$C content
\citep{Buckley_2007,9780408708036,Holben1995,Nusslein1999}. As a result, most
applications of SIP have targeted specialized microorganisms such as
methanotrophs \citep{radajewski2000stable}, methanogens \citep{lu2005},
syntrophs \citep{lueders2004}, or microbes that target pollutants
\citep{derito2005}. SIP has proved less useful for exploring the soil C-cycle
because it has lacked the resolution necessary to manage effectively the signal
complexity that results from adding components of plant biomass to microbial
communities in soil. High throughput DNA sequencing technology, however,
improves the resolving power of SIP \citep{Aoyagi2015}. 

% Fakesubsubsection:High throughput sequencing
Coupling SIP with high throughput DNA sequencing now enables exploration of
microbial C-cycling in soils. SSU rRNA amplicons can be sequenced
from numerous density gradient fractions across multiple samples thereby
increasing the resolution of a typical nucleic acid SIP experiment
\citep{Verastegui_2014}. It is now possible to use far less isotopically
labeled substrate resulting in more environmentally realistic experimental
conditions \citep{Aoyagi2015}. We have employed such a high resolution DNA
stable isotope probing approach to explore the assimilation of $^{13}$C labeled
xylose and/or cellulose into bacterial DNA in an agricultural soil. 

Specifically, we added to soil a complex amendment that simulated organic
matter derived from fresh plant biomass. All treatments received the same
amendment but the identity of the isotopically labeled substrate was varied
between treatments. We set up a control treatment where all components were
unlabeled, a treatment with $^{13}$C-xylose, and a treatment with
$^{13}$C-cellulose. Soil was sampled at days 1, 3, 7, 14, and 30 and we
identified which microorganisms had assimilated $^{13}$C into DNA at each point
in time. The experiment was designed to provide a test of the
degradative succession hypothesis in the context of soil bacteria, to identify
soil bacteria that metabolize xylose and cellulose, and to characterize
temporal dynamics of xylose and cellulose metabolism in soil. 
