\section{Introduction}
% Fakesubsubsection:Excluding plant biomass, there are 2,300 Pg of
Soils worldwide contain 2,300 Pg of carbon (C) which accounts for nearly 80\%
of the C present in the terrestrial biosphere
\citep{Amundson_2001,BATJES_1996}. C respiration by soil microorganisms
produces annually tenfold more CO$_{2}$ than fossil fuel emissions
\citep{chapin2002principles}. Despite the contribution of microorganisms to
global C flux, many global C models ignore microbial
physiological diversity and its impacts on microbial activity in soils.
\citep{Allison2010,Six2006,Treseder2011} and predictions of climate change
feedbacks on soil C flux improve when biogeochemical models explicitly
represent microbial physiology \citep{Wieder2013}. However, we still know
little about the ecophysiology of soil microorganisms. Efforts that describe
the ecophysiology of microorganisms engages in soil C-cycling 
should assist the development and refinement of terrestrial
C biogeochemical models \citep{Bradford2008,Neff_2001,McGuire2010}.

%Fakesubsubsection:The decomposition
Cellulose comprises most plant C (30-50\%) followed by hemicellulose (20-40\%),
and lignin (15-25\%) \citep{Lynd2002}. Hemicellulose, being the most soluble,
degrades most easily as compared to cellulose and lignin, and is targeted in
the early stages of decomposition. Hemicellulose composition varies
considerably with xylans being most abundant, themselves
composed of differing amounts of xylose, glucose, arabinose, galactose,
mannose, and rhamnose \citep{Saha2003}. Xylose is often the most abundant sugar
in hemicellulose, comprising as much as 60-90\% of xylan in some plants (e.g
hardwoods) \citep{Spiridon2008}, wheat \citep{Sun2005}, and switchgrass
\citep{Bunnell2013}. Microbes that respire sugars proliferate during the
initial stages of decomposition \citep{Garrett1951,Alexander1964}, and
metabolize as much as 75\% of sugar C during the first 5 days of decomposition
\citep{Engelking2007}. In contrast, cellulose decomposition proceeds more
slowly with rates increasing for approximately 15 days while 
degradation continues for 30-90 days \citep{Hu1997,Engelking2007}. It is
hypothesized that distinct microbial functional guilds mediate fresh,
plant-derived organic matter decomposition and that these guilds proliferate as
they decompose compounds of increasing lability over time
\citep{Hu1997,Rui2009,AnneliseHKjoller2002,Bastian2009}. For instance, this
degradative succession hypothesis posits that rapidly growing plant sugar
decomposers proliferate first \citep{Garrett1963,Bremer1994} followed by slow
growing degraders of structural C \citep{Garrett1963}. Evidence to support the
degradative succession hypothesis comes from observing soil respiration
dynamics and characterizing microbes cultured at different stages of
decomposition. The degree to which the succession hypothesis presents an
accurate model of litter decomposition has been called into question
\citep{AnneliseHKjoller2002,Frankland1998,Osono_2005} and it's clear that new
approaches are needed to dissect microbial contributions to C transformations
in soils.

% Fakesubsubsection:Though microorganisms mediate
Though microorganisms mediate 80-90\% of the soil C-cycle
\citep{ColemanCrossley_1996,Nannipieri_2003}, and microbial community
composition can account for significant variation in C mineralization
\citep{Strickland_2009} , terrestrial C-cycle models rarely consider the
community composition of soils \citep{Zak2006,Reed2007}. Metabolic rates of
soil C transformations are measured without knowledge of the organisms that
mediate these reactions \citep{Nannipieri_2003}, leaving undefined the
importance of community membership towards maintaining ecosystem function
\citep{Nannipieri_2003,Schimel_2012,Allison_2008}. Variation in microbial
community composition can be linked most directly to rates of soil processes
when diagnostic genes for specific functions are available to represent those
processes (e.g. denitrification \citep{Cavigelli2000}, nitrification
\citep{Carney2004,Hawkes2005,Webster2005}, methanotrophy \citep{Gulledge1997},
and nitrogen fixation \citep{Hsu2009}). However, the complexity of soil
C transformations and the lack of convenient functional genes for describing
these transformations has limited progress in characterizing the contributions
of individual microbes to the soil C-cycle. Remarkably, we still lack basic
information on the physiology and ecology of the majority of organisms that
live in soils. For example, contributions to soil processes remain
uncharacterized for entire bacterial phyla such as Acidobacteria, Chloroflexi,
Planctomycetes, and Verrucomicrobia. These phyla combined can comprise 32\% of
soil microbial communities (based on surveys of the SSU rRNA genes in soil)
\citep{Janssen2006,Buckley2002} and they are nearly ubiquitous in soil. 

% Fakesubsubsection:Functional guild membership
Functional guild membership and diversity define connections between microbial
community structure and function. Linking genetic identity to function in
microbial ecosystems has relied historically on culturing microorganisms and
surveying diagnostic genes for functions of interest. However, most
microorganisms are difficult to grow in culture \citep{Janssen2006} 
and many processes lack suitable diagnostic genes. Nucleic acid stable-isotope
probing (SIP) links genetic identity and activity without the need to grow
microorganisms in culture and has expanded our knowledge of microbial
contributions to biogeochemical processes \citep{Chen_Murrell_2010}. However,
nucleic acid SIP has notable complications including the need to add large
amounts of labeled substrate \citep{radajewski2000stable}, label dilution
resulting in partial labelling of nucleic acids
\citep{radajewski2000stable,Manefield_2002,McDonald_2005}, the potential for
cross-feeding and secondary label incorporation
\citep{Morris_2002,Hutchens2004,14686943,DeRito2005,McDonald_2005,Ziegler_2005},
and variation in genome G$+$C content
\citep{Buckley_2007,9780408708036,Holben1995,Nusslein1999}. As a result, most
applications of SIP have targeted specialized microbial functional guilds of
limited diversity (e.g. methanotrophs \citep{radajewski2000stable}). SIP has
generally proved less useful in analysis of the overall soil C-cycle because it
has lacked the resolution necessary to manage effectively the signal complexity
that results from adding components of plant biomass to microbial communities
in soil. High throughput DNA sequencing technology, however, improves the
resolving power of SIP. 

% Fakesubsubsection:High throughput sequencing
Coupling SIP with high throughput DNA sequencing now enables exploration of
microbial C-cycling in soils. SSU rRNA amplicons can be are readily sequenced
from numerous density gradient fractions across multiple samples thereby
increasing the resolution of a typical nucleic acid SIP experiment
\citep{Verastegui_2014}. As a consequence of this increased resolution, it is
also possible to use far less isotopically labeled substrate resulting in
conditions that are more environmentally realistic \citep{Aoyagi2015}. We have
employed such a high resolution DNA stable isotope probing approach to explore
the assimilation of $^{13}$C labeled xylose and/or cellulose into bacterial DNA
in an agricultural soil. 


Specifically, we added to soil microcosms a mixture of nutrient and resource
mixture that simulated organic matter derived from fresh plant biomass. All
microcosms received the same nutrient and resource mixture but the identity of
the isotopically labeled substrate was varied between treatments. We set up
a control treatment where all components were unlabeled, a treatment with
$^{13}$C-xylose, and a treatment with $^{13}$C-cellulose. Soil in microcosms
were samples at days 1, 3, 7, 14, and 30 and we assessed which microorganisms
had assimilated $^{13}$C into DNA at each sampling point. The experiment was
designed to provide a test of the degradative succession hypothesis as applied
to soil bacteria, to identify the bacteria that metabolize xylose and cellulose
in soils, and to characterize the temporal dynamics of xylose and cellulose
metabolism during the degradation of fresh organic matter. 
