\section{Introduction}
% Fakesubsubsection:Excluding plant biomass, there are 2,300 Pg of
Excluding plant biomass, there are 2,300 Pg of carbon (C) in soils
worldwide which accounts for $\sim$80\% of the global terrestrial C pool
\citep{Amundson_2001,BATJES_1996}. Fungi, archaea, and bacteria degrade plant
biomass that reaches soil respiring the majority of plant biomass C producing
10 times more CO$_{2}$ annually than anthropogenic emissions
\citep{chapin2002principles}. Rising atmospheric CO$_{2}$ may stimulate plant
growth and in turn increase plant biomass C input to soil
\citep{Groenigen_2006}. Current climate change models concur on atmospheric and
oceanic, but not terrestrial, global C flux predictions
\citep{Friedlingstein_2006}. Contrasting terrestrial C model predictions
reflect how little is known about soil C cycling. We need to establish the
roles and diversity of soil microbial community members involved with soil
C cycling to reconcile inconsistencies in terrestial C models
\citep{Neff_2001,McGuire2010a}.

% Fakesubsubsection:An important step in understanding soil C cycling
Functional guild membership and diversity establish the connections between
soil functions and community structure \citep{O_Donnell_2002}. Microorganisms
mediate an estimated 80-90\% of soil C cycling
\citep{ColemanCrossley_1996,Nannipieri_2003} but the complexity of soil
obfuscates microbial contributions to soil C cycling and the majority of
microorganisms resist cultivation in the laboratory. Stable-isotope probing
(SIP), however, links genetic identity and activity without cultivation and has
expanded our knowledge of microbial contributions to biogeochemical processes
\citep{Chen_Murrell_2010}. Successful applications of SIP have identified
organisms which mediate processes performed by functionally specialized
microorganisms of limited diversity such as methanogens \citep{Lu_2005} but SIP
has been less applicable in soil C cycling studies because simultaneous
labeling of many different organisms has necessitated prohibitory resolving
power. High throughput DNA sequencing technology, however, improves the
resolving power of SIP enabling exploration of complex soil C-cycling
processes.

% Fakesubsubsection:A temporal cascade occurs in natural microbial
This study aimed to observe labile C versus polymeric C assimilation dynamics
in the soil microbial community. To soil microcosms we added a mixture of
nutrients and C substrates that simulated the composition of plant biomass. All
microcosms received the same C substrate mixture where the only difference
between treatments was the identity of the isotopically labeled substrate.
Specifically, we set up a series of microcosms with three treatments: in one
treatment xylose was substituted for its $^{13}$C-equivalent, in another
cellulose was substituted for its $^{13}$C-equivalent, and in the third
treatment all substrates in the mixture were unlabeled. We harvested microcosms
from each treatment at days 3, 7, 14 and 30 and additionally harvested
microcosms receiving $^{13}$C-xylose and unlabeled substrates on day 1. We
chose to label xylose and cellulose to contrast labile C and polymeric
C decomposition, respectively. Post incubation, we sequenced 16S rRNA genes
from SIP density fractions with high throughput DNA sequencing technology. Our
experimental design allowed us to observe the soil microbial community members
that assimilated xylose-C and cellulose-C over time.
