\section{Introduction}
% Fakesubsubsection:Excluding plant biomass, there are 2,300 Pg of
Excluding plant biomass, there are 2,300 Pg of carbon (C) stored in soils
worldwide which accounts for $\sim$80\% of the global terrestrial C pool
\citep{Amundson_2001,BATJES_1996}. When organic C from plants reaches soil it
is degraded by fungi, archaea, and bacteria.  The majority of plant biomass
C in soil is respired and produces 10 times more CO$_{2}$ annually than
anthropogenic emissions \citep{chapin2002principles}. Global
changes in atmospheric CO$_{2}$, temperature, and ecosystem nitrogen inputs are
expected to impact soil C input \citep{Groenigen_2006}. Current climate change
models concur on atmospheric and oceanic, but not terrestrial, C predictions
\citep{Friedlingstein_2006}. Contrasting terrestrial C model predictions
reflect how little is known about soil C cycling. Inconsistencies in
terrestial modeling could be improved by elucidating the relationship between
dissolved organic C and soil microbial community composition \citep{Neff_2001}.

% Fakesubsubsection:An important step in understanding soil C cycling
To establish the relationship between community structure and soil function we
must identify the \textit{in situ} activity of specific soil microbes 
\citep{O_Donnell_2002}. An estimated 80-90\% of soil C cycling is
mediated by microorganisms \citep{ColemanCrossley_1996,Nannipieri_2003} but
exploring soil C processing is challenging due to soil's hetergeneous nature.
The majority of microorganisms resist cultivation in the laboratory.
Stable-isotope probing (SIP) links genetic identity and activity without
cultivation and has been used to expand our knowledge of microbial
contributions to biogeochemical processes \citep{Chen_Murrell_2010}. Successful
applications of SIP have identified organisms which mediate processes performed
by a narrow set of functional guilds such as methanogens \citep{Lu_2005} but
SIP has been less applicable in soil C cycling studies due to limitations in
resolving power as a result of simultaneous labeling of many different
organisms. High throughput DNA sequencing technology, however, has enabled
exploration of complex soil C-cycling patterns via SIP.

% Fakesubsubsection:A temporal cascade occurs in natural microbial
A temporal activity cascade occurs in natural microbial communities during
plant biomass degradation in which labile C is degraded before polymeric
C \citep{Hu_1997,Rui_2009}. This study's aim was to observe C assimilation
dynamics in the soil microbial community. Our experimental approach
included the addition of a soil organic matter (SOM) simulant to soil
microcosms where a single C component was substituted for its $^{13}$C
equivalent. Parallel incubations of soils amended with this C mixture allowed
us to observe how different C substrates move through the soil microbial
community. In this study we used $^{13}$C-xylose and $^{13}$C-cellulose as
a proxy for labile and polymeric C, respectively, and coupled nucleic acid SIP
with high throughput DNA sequencing. Amplicon sequencing of 16S rRNA genes from
gradient fractions of multiple density gradients made it possible to track
C assimilation by hundreds of soil taxa.
