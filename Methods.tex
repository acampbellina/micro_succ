\section{Methods}
Additional information on sample collection and analytical methods is provided
in Supplemental Materials and Methods. 

Twelve soil cores (5 cm diameter x 10 cm depth) were collected from six 
sampling locations within an organically managed agricultural field in Penn
Yan, New York. Soils were sieved (2 mm), homogenized, distributed into flasks
(10 g in each 250 ml flask, n = 36) and equilibrated for 2 weeks. Soils were
amended with a mixture containing 5.3 mg C g-1 soil dry weight (d.w.) and
brought to 50\% water holding capacity. The mixture contained 38\% cellulose,
23\% lignin, 20\% xylose, 3\% arabinose, 1\% galactose, 1\% glucose, and 0.5\%
mannose by mass, with the remaining 13.5\% mass composed of an amino acid
(in-house made replica of Teknova C0705) and macronutrient mixture (Murashige
and Skoog, Sigma Aldrich M5524).  This mixture approximates the molecular
composition of switchgrass biomass with hemicellulose replaced by its
constituent monomers (Reference 87). Three parallel treatments were performed
which varied the isotopic composition of the mixture: (1) unlabeled control,
(2) 13C-cellulose (synthesized as described in Supplemental Methods), (3)
13C-xylose (98 atom\% 13C, Sigma Aldrich). A total of 12 microcosms were
established per treatment. Other details relating to substrate addition can be
found in Supplemental Methods. Microcosms were sampled destructively at days
1 (control and xylose only), 3, 7, 14, and 30 and soils were stored at
-80 $^{\circ}$C until nucleic acid extraction. The abbreviation “13CXPS” refers
to the 13C-xylose treatment (13C Xylose Plant Simulant), “13CCPS” refers to the
13C-cellulose treatment and “12CCPS” refers to the unlabeled control. 

Nucleic acids were extracted using a modified Griffiths protocol (Reference 88)
and DNA was prepared for isopycnic centrifugation as previously described
(Reference 55). A total of 5 μg of DNA was added to each 4.7 ml CsCl density
gradient composed of 1.69 g ml-1 CsCl in15 mM Tris-HCl, 15 mM EDTA, 15 mM KCl.
Centrifugation was performed in a TLA-110 rotor at 55,000 rpm and 20°C for 66
hr. Fractions of 100 μL were collected by displacement at 3.3 μL s-1
(Reference~48) into Acroprep$^{TM}$ 96 filter plates (Pall Life Sciences 5035).
The refractive index of each fraction was measured immediately using a Reichart
AR200 digital refractometer as previously described (Reference 55). The DNA was
desalted by washing (3 x 200 μL) with TE in the Acroprep filter wells followed
by resuspension in 50 μL TE. SSU rRNA genes were amplified from gradient
fractions (n = 20 per gradient) and from non-fractionated DNA.
Barcoded primers consisted of: 454-specific adapter B, a 10 bp barcode
(Reference 90), a 2 bp linker (5’-CA-3’), and 806R primer for reverse primer
(BA806R); and 454-specific adapter A, a 2 bp linker (5’-TC-3’), and 515F primer
for forward primer (BA515F). Each PCR contained 1.25 U μl-1 AmpliTaq Gold (Life
Technologies, Grand Island, NY; N8080243), 1X Buffer II (100 mM Tris-HCl, 500
mM KCl, pH~8.3), 2.5 mM MgCl2, 200 μM of each dNTP, 0.5 mg ml-1 BSA, 0.2 μM
BA515F, 0.2 μM BA806R, and 10 μL of 1:30 DNA template in 25 μl total volume).
Reactions were performed in triplicate and pooled. Amplified DNA was gel
purified using the Wizard SV gel and PCR clean-up system (Promega, Madison, WI;
A9281) per manufacturer’s protocol. Samples were normalized by SequalPrep
normalization plates (Invitrogen, Carlsbad, CA; A10510) and pooled in equimolar
concentration. Amplicons were sequenced on Roche 454 FLX system using titanium
chemistry at Selah Genomics (Columbia, SC). 

SSU rRNA gene sequences were initially screened by maximum expected
errors at a specific read length threshold (17) and aligned to the Silva
reference using the Mothur NAST aligner (21, 22). Anomalous reads were
discarded and 87\% of original reads remained. Remaining reads were annotated
using the “UClust” taxonomic annotation framework in QIIME (18, 19) with 97\%
cluster seeds from the Silva SSU rRNA database as reference (provided at QIIME
website) (20). Sequences were distributed into OTUs using the UPARSE
methodology (17). OTU centroids were established at a threshold of 97\%
sequence identity. For phylogenetic reconstruction, alignment was performed
with SSU-Align (23, 24). Columns in the alignment that were aligned with poor
confidence (< 95\% of characters had posterior probability alignment scores
> 95\%) were masked. FastTree (25) was used with default parameters to build
the phylogeny. NMDS ordination was performed on weighted Unifrac (32)
distances. The Phyloseq (33) wrapper for Vegan (34) (both R packages) was used
to compute sample values along NMDS axes. The 'adonis' function in Vegan was
used to perform Adonis tests (default parameters) (36). 

We used DESeq2, an RNA-Seq differential expression statistical framework (29),
to identify OTUs that were enriched in  high density gradient fractions from
$^{13}$C-treatments relative to corresponding density fractions from control
treatments (for review of RNA-Seq differential expression statistics applied to
microbiome OTU count data see (30)). We define "high density gradient
fractions" as gradient fractions whose density falls between 1.7125 - 1.755
g ml-1. DESeq2 was used to calculate moderated $log_{2}$ fold change and
corresponding standard error in OTU relative abundance between $^{13}$C
treatments and control across high density gradient fractions. A one-sided Wald
test was used to statistically assess fold change values (using corresponding
standard errors). The null hypothesis was that $log_{2} Fold Change$ was less
than one standard deviation above the mean of all $log_{2} Fold Change$ values.
P-values were corrected for multiple comparisons by using the Benjamini and
Hochberg (BH) method (31). Independent filtering was performed on the basis of
sparsity prior to correcting P-values for multiple comparisons. Briefly, OTUs
were eliminated if they failed to appear in at least XX\% of high density
gradient fractions for a given $^{13}$C and control treatment pair, these OTUs
are unlikely to have sufficient data to allow for the determination of
statistical significance. Following sparsity filtering X,XXX OTUs were analyzed
for enrichment in $^{13}$C treatment heavy gradient fractions relative to
control.

Twelve soil cores (5 cm diameter x 10 cm depth) were collected from six 
sampling locations within an organically managed agricultural field in Penn
Yan, New York. Soils were sieved (2 mm), homogenizing sieved soil, and
pre-incubating 10 g of dry soil weight in flasks for 2 weeks. Soils were
amended with a 5.3 mg g soil\textsuperscript{-1} carbon mixture; representative
of natural concentrations \cite{Schneckenberger_2008}. Mixture contained 38\%
cellulose, 23\% lignin, 20\% xylose, 3\% arabinose, 1\% galactose, 1\% glucose,
and 0.5\% mannose by mass, with the remaining 13.5\% mass composed of an amino
acid (in-house made replica of Teknova C0705) and basal salt mixture (Murashige
and Skoog, Sigma Aldrich M5524). Three parallel treatments were performed; (1)
unlabeled control,(2)\textsuperscript{13}C-cellulose,
(3)\textsuperscript{13}C-xylose (98 atom\% \textsuperscript{13}C, Sigma
Aldrich). Each treatment had 2 replicates per time point (n = 4) except day 30
which had 4 replicates; total microcosms per treatment n = 12, except
\textsuperscript{13}C-cellulose which was not sampled at day 1, n = 10. Other
details relating to substrate addition can be found in SI. Microcosms were
sampled destructively (stored at -80$^{\circ}$C until nucleic acid processing)
at days 1 (control and xylose only), 3, 7, 14, and 30. Microcosm treatments are
identified in figures by the following code: ``13CXPS'' refers to the amendment
with $^{13}$C-xylose ($^{13}$\textbf{C} \textbf{X}ylose \textbf{P}lant
\textbf{S}imulant), ``13CCPS'' refers to the $^{13}$C-cellulose amendment and
``12CCPS'' refers to the amendment that only contained $^{12}$C (i.e. control).

Nucleic acids were extracted using a modified Griffiths procotol
\cite{Griffiths_2000}. To prepare nucleic acid extracts for isopycnic
centrifugation as previously described \cite{Buckley_2007}, DNA was size
selected (\textgreater 4kb) using 1\% low melt agarose gel and $\beta$-agarase
I enzyme extraction per manufacturers protocol (New England Biolab, M0392S).
For each time point in the series isopycnic gradients were setup using
a modified protocol \cite{Neufeld_2007} for a total of five
\textsuperscript{12}C-control, five \textsuperscript{13}C-xylose, and four
\textsuperscript{13}C-cellulose microcosms. A density gradient (average density
1.69 g mL\textsuperscript{-1}) solution of 1.762 g cesium chloride (CsCl)
ml\textsuperscript{-1} in gradient buffer solution (pH 8.0 15 mM Tris-HCl, 15
mM EDTA, 15 mM KCl) was used to separate \textsuperscript{13}C-enriched and
\textsuperscript{12}C-nonenriched DNA. Each gradient was loaded with
approximately 5 $\mu$g of DNA and ultracentrifuged for 66 h at 55,000 rpm and
room temperature (RT). Fractions of $\sim$100 $\mu$L were collected from below
by displacing the DNA-CsCl-gradient buffer solution in the centrifugation tube
with water using a syringe pump at a flow rate of 3.3 $\mu$L
s\textsuperscript{-1} \cite{Manefield_2002} into Acroprep\textsuperscript{TM}
96 filter plate (Pall Life Sciences 5035). The refractive index of each
fraction was measured using a Reichart AR200 digital refractometer modified as
previously described \cite{Buckley_2007} to measure a volume of 5 $\mu$L. Then
buoyant density was calculated from the refractive index as previously
described \cite{Buckley_2007} (see also SI). The
collected DNA fractions were purified by repetitive washing of Acroprep filter
wells with TE. Finally, 50 $\mu$L TE was added to each fraction then
resuspended DNA was pipetted off the filter into a new microfuge tube. 


For every gradient, 20 fractions were chosen for sequencing between the density
range 1.67-1.75 g mL\textsuperscript{-1}. Barcoded 454 primers were designed
using 454-specific adapter B, 10 bp barcodes \cite{Hamady_2008}, a 2 bp linker
(5'-CA-3'), and 806R primer for reverse primer (BA806R); and 454-specific
adapter A, a 2 bp linker  (5'-TC-3'), and 515F primer for forward primer
(BA515F). Each fraction was PCR amplified using 0.25 $\mu$L
5 U $\mu$l\textsuperscript{-1} AmpliTaq Gold (Life Technologies, Grand Island,
NY; N8080243), 2.5 $\mu$L 10X Buffer II (100 mM Tris-HCl, pH 8.3, 500 mM KCl),
2.5 $\mu$L 25 mM MgCl$_{2}$, 4 $\mu$L 5 mM dNTP, 1.25 $\mu$L 10 mg
mL\textsuperscript{-1} BSA, 0.5 $\mu$L 10 $\mu$M BA515F, 1 $\mu$L 5 $\mu$M
BA806R, 3 $\mu$L H$_{2}$O, 10 $\mu$L 1:30 DNA template) in triplicate. Samples
were normalized either using Pico green quantification and manual calculation
or by SequalPrep\textsuperscript{TM} normalization plates (Invitrogen,
Carlsbad, CA; A10510), then pooled in equimolar concentrations. Pooled DNA was
gel extracted from a 1\% agarose gel using Wizard SV gel and PCR clean-up
system (Promega, Madison, WI; A9281) per manufacturer's protocol. Amplicons
were sequenced on Roche 454 FLX system using titanium chemistry at Selah
Genomics (formerly EnGenCore, Columbia, SC)  

