\textbf{Soil Collection and Preparation}



Soils were collected from an organic farm in Penn Yan, New York.  These soils are characterized as Honoeye/Lima, a silty clay loam on calcareous bedrock.  To get a field average, 10cm cores were collected (in duplicate) from six different sampling locations around the field using a slide hammer bulk density sampler (coordinates: (1) N 42° 40.288’ W 77° 02.438’, (2) N 42° 40.296’ W 77° 02.438’, (3) N 42° 40.309’ W 77° 02.445’, (4) N 42° 40.333’ W 77° 02.425’, (5) N 42° 40.340’ W 77° 02.420’, (6) N 42° 40.353’ W 77° 02.417’) on November 21, 2011.   Cores were all sieved through a 2mm sieve, homogenized by mixing, and stored at 4°C until use (within 1-2 week of collection).  Carbon and nitrogen content , as well as, water holding capacity were previously measured for these soils as blah blah blah.     



\textbf{Cellulose Production}



\textit{Gluconoacetobacter xylinus} was grown up on Heo and Son 0.1\% glucose agar plates (using 12^{C}-glucose) at 30°C without inositol (Heo and Son 2002).  Heo and Son liquid minimal media was made with 0.1\% glucose (one batch with 12^{C}- and another with 13^{C}-glucose).  To increase surface area for cellulose growth, 100mL of the media were added to individual 1L Erlenmeyer flasks.  Each aliquot of media was inoculated with three isolated colonies of \textit{Gluconoacetobacter xylinus} from the Heo and Son plate previously grown.  Flasks were kept static in the dark at 30°C for 2-3 weeks until thick cellulose pellicule had formed.  To wash culture from cellulose, two parts 1\% alconox was added to each flask and autoclaved.  Cellulose pellicules were rinsed with deionized (DI) water (1L) until no suds were produced (~10 rinses).  Each pellicule was put in 1L of DI water for overnight dialysis.  Pellicules were rinsed (3 washes) with DI water twice a day for 2-3 days to ensure no detergent or culture media remained associated with cellulose.  Harvested pellicules were dried overnight (60°C) and then cut into pieces and ground using ball grinder until desired size range (53um - 250um) was achieved (checked by sieve).  Size range was based on particulate organic matter to emulate how microbes may experience cellulose in the environment and for even distribution in microcosms.                    



\textbf{Microcosm}

A subset of soil was dried at 105°C overnight to determine soil moisture content. Microcosms (35 total) were created with the equivalent of ten grams dry soil weight of the sieved soil in a 250mL Erlenmeyer flask capped with a butyl rubber stopper to preventing drying of soils.  Microcosms were kept at 25°C for 2 weeks for bottle conditioning (better phrasing: pre-incubation?) and stoppers were removed for 10min every 3 days to prevent anoxia of soils. A complex carbon mixture (38\% cellulose, 23\% lignin, 20\% xylose, 3\% arabinose, 1\% galactose, 1\% glucose, 0.5\% mannose) was designed based on switch grass biomass composition.  This complex carbon mixture was added to microcosms along with an amino acid (in-house made replica of teknova Cat#C0705) and basal salt mixture (Murashige and Skoog, Sigma M5524) at 52.6mg^{-1}10g soil; representative of natural concentrations.  Cellulose (20mg) and lignin (12mg) were sprinkled over the surface of the soil in each microcosm as dry additions due to their insolubility to ensure same concentration in every flask. The remaining carbons, amino acids, and basal salts were added by pipetting evenly over soil surface as a liquid addition until 50\% water holding capacity is achieved.  Three parallel treatments were performed; (1) control, all \textsuperscript{12}C-carbons, (2) cellulose, all \textsuperscript{12}C-carbons except \textsuperscript{13}C-cellulose, (3) xylose, all \textsuperscript{12}C-carbons except \textsuperscript{13}C-xylose.  Replicate microcosms were harvested (stored at -80°C until nucleic acid processing) at days 1 (control and xylose only), 3, 7, 14, and 30.  Microcosms remained stoppered throughout incubation, except for every third day when flasks were flushed with fresh air for 2 minutes.  A subset of microcosm soil for each treatment and time point were isotopically analyzed at Cornell University Stable Isotope Laboratory to determine amount of \textsuperscript{13}C that remained at each time point.                       

50% water holding capacity measured. 



\textbf{Nucleic Acid Processing}

DNA was extracted from 0.25g soil using a modified Griffiths procotol (Griffiths 2000 Rapid Method for Coextraction of DNA and RNA from Natural Environments for Analysis of Ribosomal DNA- and rRNA-Based Microbial Community Composition, 10.1128/AEM.66.12.5488-5491.2000).  Soils were bead beat in 2mL lysis tubes containing 0.5g silica/zirconia beads (treated at 300°C for 4 hours to remove RNases), 0.5mL extraction buffer (240 mM Phosphate buffer
0.5\% N-lauryl sarcosine), and 0.5mL phenol-chloroform-isoamyl alcohol (25:24:1) for 1 min at 5.5 m s ^{-1}.  After beat beading, 85uL 5M NaCl and 60uL 10\% hexadecyltriammonium bromide (CTAB)/0.7M NaCl were added to lysis tube, vortexed, chilled for 1min on ice, and centrifuged at 16,000 x g for 5min at 4°C.  The aqueous layer was transferred to a new tube and reserved on ice.  To increase DNA recovery, a double extraction was performed by adding 85uL 5M NaCl and 0.5mL extraction buffer to same lysis tube, vortexed, and centrifuged same as before.  New aqueous layer was added to previously collected aqueous layer and washed with 0.5mL chloroform:isoamyl alcohol (24:1), vortexed, and centrifuged same as before.  Aqueous layer was transferred to a new tube and nucleic acids were precipitated using 2 volumes polyethylene glycol solution (30\% PEG 8000, 1.6M NaCl) on ice for 2hrs, followed by centrifugation at 16,000 x g, 4°C for 30min.  Supernatant was discarded and pellets were washed with 1mL ice cold 70\% EtOH; each wash being a vortex followed by centrifugation at 16,000 x g, 4°C for 10min.  Pellets were air dried, resuspended in 50uL TE, and replicate extractions pooled.  Nucleic acids were stored at -20°C.                   

To prepare for DNA-stable isotope probing (DNA-SIP), DNA was size selected (>4kb) using 1\% low melt agarose gel and /beta- agarase I enzyme extraction per manufacturers protocol (New England Biolab, M0392S)

\textbf{Post-Sequencing Analysis}
 