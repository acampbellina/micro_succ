\textbf{Soil Collection and Conditioning}



Soils were collected from an organic farm in Penn Yan, New York.  These soils are characterized as Honoeye/Lima, a silty clay loam on calcareous bedrock.  To get a field average, 10cm cores were collected (in duplicate) from six different sampling locations around the field using a slide hammer bulk density sampler (coordinates: (1) N 42° 40.288’ W 77° 02.438’, (2) N 42° 40.296’ W 77° 02.438’, (3) N 42° 40.309’ W 77° 02.445’, (4) N 42° 40.333’ W 77° 02.425’, (5) N 42° 40.340’ W 77° 02.420’, (6) N 42° 40.353’ W 77° 02.417’) on November 21, 2011.   Cores were all sieved through a 2mm sieve, homogenized by mixing, and stored at 4°C until use.  Carbon and nitrogen content were previously measured for these soils as blah blah blah.  Ten gdw equivalent soil was placed in 50 individual 250mL Erlenmeyer flasks capped with a butyl rubber stopper to preventing drying of soils.  Flasks were kept at 25°C for 2 weeks for pre-incubation and stoppers were removed for 10min every 3 days to prevent anoxia of soils.   



\textbf{Cellulose Production}



\textit{Gluconoacetobacter xylinus} was grown up on Heo and Son 0.1\% glucose agar plates (using 12C-glucose) at 30°C without inositol (Heo and Son 2002).  Heo and Son liquid minimal media was made with 0.1\% glucose (one batch with 12C- and another with 13C-glucose).  To increase surface area for cellulose growth, 100mL of the media was sterilely added to individual 1L Erlenmeyer flasks.  Each aliquot of media was inoculated with three isolated colonies of Gluconoacetobacter xylinus from Heo and Son plate.  Flasks were kept static in the dark at 30°C for 2-3 weeks until thick cellulose pellicule had formed.  



\textbf{Microcosm}

\textbf{Nucleic Acid Processing}

\textbf{Post-Sequencing Analysis}