\textbf{Soil Collection and Preparation}



Soils were collected from an organic farm in Penn Yan, New York.  These soils are characterized as Honoeye/Lima, a silty clay loam on calcareous bedrock.  To get a field average, 10cm cores were collected (in duplicate) from six different sampling locations around the field using a slide hammer bulk density sampler (coordinates: (1) N 42° 40.288’ W 77° 02.438’, (2) N 42° 40.296’ W 77° 02.438’, (3) N 42° 40.309’ W 77° 02.445’, (4) N 42° 40.333’ W 77° 02.425’, (5) N 42° 40.340’ W 77° 02.420’, (6) N 42° 40.353’ W 77° 02.417’) on November 21, 2011.   Cores were all sieved through a 2mm sieve, homogenized by mixing, and stored at 4°C until use (within 1-2 week of collection).  Carbon and nitrogen content , as well as, water holding capacity were previously measured for these soils as blah blah blah.     



\textbf{Cellulose Production}



\textit{Gluconoacetobacter xylinus} was grown up on Heo and Son 0.1\% glucose agar plates (using 12^{C}-glucose) at 30°C without inositol (Heo and Son 2002).  Heo and Son liquid minimal media was made with 0.1\% glucose (one batch with 12^{C}- and another with 13^{C}-glucose).  To increase surface area for cellulose growth, 100mL of the media were added to individual 1L Erlenmeyer flasks.  Each aliquot of media was inoculated with three isolated colonies of \textit{Gluconoacetobacter xylinus} from the Heo and Son plate previously grown.  Flasks were kept static in the dark at 30°C for 2-3 weeks until thick cellulose pellicule had formed.  To wash culture from cellulose, two parts 1\% alconox was added to each flask and autoclaved.  Cellulose pellicules were rinsed with deionized (DI) water (1L) until no suds were produced (~10 rinses).  Each pellicule was put in 1L of DI water for overnight dialysis.  Pellicules were rinsed (3 washes) with DI water twice a day for 2-3 days to ensure no detergent or culture media remained associated with cellulose.  Harvested pellicules were dried overnight (60°C) and then cut into pieces and ground using ball grinder until desired size range (53 \mu m - 250 \muu m) was achieved (checked by sieve).  Size range was based on particulate organic matter to emulate how microbes may experience cellulose in the environment and for even distribution in microcosms.                    



\textbf{Microcosm}

Ten grams dry weight equivalent soil was placed in 50 individual 250mL Erlenmeyer flasks capped with a butyl rubber stopper to preventing drying of soils.  Flasks were kept at 25°C for 2 weeks for pre-incubation and stoppers were removed for 10min every 3 days to prevent anoxia of soils.
50% water holding capacity measured. 
\textbf{Nucleic Acid Processing}

\textbf{Post-Sequencing Analysis}