\section{Methods}
Additional information on sample collection and analytical methods is provided in \href{https://www.authorea.com/users/3537/articles/8459/_show_article}{SI Materials and Methods}.


Soil cores were collected from an organically managed farm in Penn Yan, New York. Soils were pretreated by sieving (2 mm), homogenizing sieved soil, and incubating 10 g of dry soil weight in flasks for 2 weeks. Soils were amended with a carbon mixture (38\% cellulose, 23\% lignin, 20\% xylose, 3\% arabinose, 1\% galactose, 1\% glucose, 0.5\% mannose) along with an amino acid (in-house made replica of teknova Cat#C0705) and basal salt mixture (Murashige and Skoog, Sigma M5524) at 2 mg C g soil\textsuperscript{-1}; representative of natural concentrations \cite{Schneckenberger_2008}. Three parallel treatments were performed; (1) unlabeled control,(2)\textsuperscript{13}C-cellulose, (3)\textsuperscript{13}C-xylose (Sigma 
666378). Each treatment had Xtotalnumber of microcosms, with xnumber of replicates per time point. Other details relating to substrate addition can be found in Suppl M&M. Microcosms were sampled destructively (stored at -80{\textdegree}C until nucleic acid processing) at days 1 (control and xylose only), 3, 7, 14, and 30.



DNA was extracted using a modified Griffiths procotol \cite{Griffiths_2000}. To prepare nucleic acid extracts for isopycnic centrifugation as presviously described \cite{Buckley_2007}, DNA was size selected (\textgreater4kb) using 1\% low melt agarose gel and $\beta$- agarase I enzyme extraction per manufacturers protocol (New England Biolab, M0392S). Isopycnic gradients were established as described previously (35, 48) for five \textsuperscript{12}C-control, five \textsuperscript{13}C-xylose, and four \textsuperscript{13}C-cellulose microcosms; one per day harvested. A density gradient solution of 1.762 g cesium chloride (CsCl) ml\textsuperscript{-1} in gradient buffer solution (pH 8.0 15mM Tris-HCl, 15mM EDTA, 15mM KCl) was used to separate \textsuperscript{13}C-enriched and \textsuperscript{12}C-nonenriched DNA. Each gradient was loaded with approximately 5$\mu$g of DNA and utlracentrifuged for 66 h at 55,000 rpm and room temperature (RT). Fractions of $\sim$100 $\mu$L were collected into Acroprep\textsuperscript{TM} 96 filter plate (part no. 5035, Pall Life Sciences) by displacing the DNA-CsCl-gradient buffer solution in the centrifugation tube with water. The refractive index of each fraction was measured using a Reichart AR200 digital refractometer. Then buoyant density was calculated from the refractive index using the equation $\rho$=\textit{a}$\eta$-\textit{b}, where $\rho$ is the density of the CsCl (g ml\textsuperscript{-1}), $\eta$ is the measured refractive index, and \textit{a} and \textit{b} are coefficient values of 10.9276 and 13.593, respectively, for CsCl at 20{\textdegree}C \cite{9780408708036}. The collected DNA fractions were purified by repetitive washing with TE followed by centrifugation. Finally, 50$\mu$L TE was added to each fraction, incubated on filter for 5min at RT then resuspended DNA was pipetted off the filter into a new microfuge tube. For every gradient, 20 fractions were chosen for sequencing between the density range 1.67-1.75g/mL. 



Barcoded 454 primers were designed using 454-specific adapter B, 10bp barcodes \cite{Hamady_2008}, a 2bp linker (put those bp in here), and 806R primer for reverse primer (BA806R); and 454-specific adapter A, a 2bp linker, and 515F primer for forward primer (BA515F). Each fraction was PCR amplified using 0.25$\mu$L 5U/$\mu$l AmpliTaq Gold (N8080243, Life Technologies, Grand Island, NY), 2.5$\mu$L 10X Buffer II (100 mM Tris-HCl, pH 8.3, 500 mM KCl), 2.5$\mu$L 25mM MgCl_{2}, 4$\mu$L 5mM dNTP, 1.25$\mu$L 10mg/mL BSA, 0.5$\mu$L 10$\mu$M BA515F, 1$\mu$L 5$\mu$M BA806R,3$\mu$L H_{2}O, 10$\mu$L 1:30 DNA template) in triplicate. Samples were normalized either using Pico green quantification and manual calculation or by SequalPrep\textsuperscript{TM} normalization plates (A10510, Invitrogen, Carlsbad, CA), then pooled in equimolar concentrations.  Pool was gel extracted from a 1\% agarose gel using Wizard SV gel and PCR clean-up system (A9281, Promega, Madison, WI) per manufacturer's protocol.  Amplicons were sequenced on Roche 454 FLX system using titanium chemistry at Selah Genomics (formerly EnGenCore, Columbia, SC)  
