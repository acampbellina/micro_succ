\section{Methods}
%
All code to take raw SSU rRNA gene sequencing reads to final publication
figures and through all presented analyses is located at the following URL:\\
\url{https://github.com/chuckpr/CSIP_succession_data_analysis}.\\ DNA sequences
are deposited on MG-RAST (Accession XXXXXXX).

% Fakesubsubsection: Twelve soil cores
Twelve soil cores (5 cm diameter x 10 cm depth) were collected from six 
sampling locations within an organically managed agricultural field in Penn
Yan, New York. Soils were sieved (2 mm), homogenized, distributed into flasks
(10 g in each 250 ml flask, n = 36) and equilibrated for 2 weeks. We amended
soils with a mixture containing 2.9 mg C g$^{-1}$ soil dry weight (d.w.) and
brought soil to 50\% water holding capacity. By mass the amendment
contained 38\% cellulose, 23\% lignin, 20\% xylose, 3\% arabinose, 1\%
galactose, 1\% glucose, and 0.5\% mannose. 10.6\% amino acids (made in house
based on Teknova C9795 formulation) and 2.9\% Murashige Skoog basal salt
mixture which contains macro and micro-nutrients that are associated with plant
biomass (Sigma Aldrich M5524). This mixture approximates the molecular
composition of switchgrass biomass with hemicellulose replaced by its
constituent monomers \citep{Schneckenberger_2008}. We set up three parallel
treatments varying the isotopically labeled component in each treatment. The
treatments were (1) a control treatment with all unlabeled components, (2) a
treatment with $^{13}$C-cellulose instead of unlabeled cellulose (synthesized
as described in SI), and (3) a treatment with $^{13}$C-xylose (98 atom\%
$^{13}$C, Sigma Aldrich) instead of unlabeled xylose. Other details relating to
substrate addition can be found in SI. Microcosms were sampled destructively at
days~1 (control and xylose only),~3,~7,~14, and~30 and soils were stored at
-80 $^{\circ}$C until nucleic acid extraction. The abbreviation “13CXPS” refers
to the $^{13}$C-xylose treatment ($^{13}$C Xylose Plant Simulant), “13CCPS”
refers to the $^{13}$C-cellulose treatment, and “12CCPS” refers to the control
treatment.

We used DESeq2 (R package), an RNA-Seq differential expression statistical
framework \citep{love2014}, to identify OTUs that were enriched in high density
gradient fractions from $^{13}$C-treatments relative to corresponding gradient
fractions from control treatments (for review of RNA-Seq differential
expression statistics applied to microbiome OTU count data see
\citep{McMurdie2014}). We define "high density gradient fractions" as gradient
fractions whose density falls between 1.7125 and 1.755 g ml$^{-1}$. For each OTU,
we calculates logarithmic fold change (LFC) and corresponding standard error for
enrichment in high density fractions of $^{13}$C treatments relative to control.
Subsequently, a one-sided Wald test was used to assess the statistical significance
of LFC values with the null hypothesis that LFC was less than one standard deviation
above the mean of all LFC values. We independently filtered OTUs prior to multiple
comparison corrections on the basis of sparsity eliminating OTUs that failed to 
appear in at least 45\% of high density fractions for a given comparison. P-values
were adjusted for multiple comparisons using the Benjamini and Hochberg method
\citep{benjamini1995}. We selected a false discovery rate of 10\% to denote
statistical significance.



See SI for additional information on experimental and analytical methods.
