\section{Methods}
Additional information on sample collection and analytical methods is provided
in Supplemental Materials and Methods. 

Twelve soil cores (5 cm diameter x 10 cm depth) were collected from six 
sampling locations within an organically managed agricultural field in Penn
Yan, New York. Soils were sieved (2 mm), homogenized, distributed into flasks
(10 g in each 250 ml flask, n = 36) and equilibrated for 2 weeks. Soils were
amended with a mixture containing 5.3 mg C g$^{-1}$ soil dry weight (d.w.) and
brought to 50\% water holding capacity. The mixture contained 38\% cellulose,
23\% lignin, 20\% xylose, 3\% arabinose, 1\% galactose, 1\% glucose, and 0.5\%
mannose by mass, with the remaining 13.5\% mass composed of an amino acid
(in-house made replica of Teknova C0705) and macronutrient mixture (Murashige
and Skoog, Sigma Aldrich M5524).  This mixture approximates the molecular
composition of switchgrass biomass with hemicellulose replaced by its
constituent monomers \citep{Schneckenberger_2008}. Three parallel
treatments were performed which varied the isotopic composition of the mixture:
(1) unlabeled control, (2) $^{13}$C-cellulose (synthesized as described in
Supplemental Methods), (3) $^{13}$C-xylose (98 atom\% $^{13}$C, Sigma Aldrich).
A total of 12 microcosms were established per treatment. Other details relating
to substrate addition can be found in Supplemental Methods. Microcosms were
sampled destructively at days 1 (control and xylose only), 3, 7, 14, and 30 and
soils were stored at -80 $^{\circ}$C until nucleic acid extraction. \textbf{
The abbreviation “13CXPS” refers to the 13C-xylose treatment ($^{13}$C Xylose
Plant Simulant), “13CCPS” refers to the $^{13}$C treatment and “12CCPS” refers
to the unlabeled control.}

We used DESeq2 (R package), an RNA-Seq differential expression statistical
framework \citep{love2014}, to identify OTUs that were enriched in high
density gradient fractions from $^{13}$C-treatments relative to corresponding
density fractions from control treatments (for review of RNA-Seq differential
expression statistics applied to microbiome OTU count data see (30)). We define
"high density gradient fractions" as gradient fractions whose density falls
between 1.7125 and 1.755 g ml$^{-1}$. Briefly, DESeq2 includes several features that
enable robust estimates of standard error in addition to reliable ranking of
logarithmic fold change (LFC) in abundance (i.e. gamma-Poisson regression
coefficients) even with low count groups where LFC can often be noisy.
Further, statistical evaluation of LFC can be performed with selected
thresholds as opposed to the often default null hypothesis that differential
abundance for an OTU is exactly zero. This enables the most biologically
interesting OTUs to be selected for subsequent analyses. We calculated LFC
and corresponding standard errors for comparisons between $^{13}$C
treatments and control (high density fractions only) for each OTU.
Subsequently, a one-sided Wald test was used to statistically assess LFC
values (using corresponding standard errors). The user-defined null
hypothesis for the Wald test was that LFC was less than one standard
deviation above the mean of all LFC values. P-values were corrected for
multiple comparisons by using the Benjamini and Hochberg method
\citep{benjamini1995}. Independent filtering was performed on the basis of
sparsity prior to correcting P-values for multiple comparisons. The
sparsity value that yielded the most P-values less than 0.10 was selected
for independent filtering by sparsity. Briefly, OTUs were eliminated if
they failed to appear in at least 45\% of high density gradient fractions
for a given $^{13}$C/control treatment pair, these OTUs are unlikely to
have sufficient data to allow for the determination of statistical
significance.

See Supplemental Information for DNA extraction, PCR, DNA sequence quality
control, OTU ecological characteristic calculations, and SIP density gradient
fractionation methods.
