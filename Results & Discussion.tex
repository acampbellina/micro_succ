\section{Results and Discussion}


\textbf{Temporal microbial and C-cycling dynamics.} With the rapid advancement and declining costs of high throughput sequencing, it has become increasingly easy to investigate microbial communities. In this study, we couple stable-isotope probing with 454 pyrosequencing in order to better understand organic matter decomposition dynamics as a function of soil microbial community C utilization. Three series of parallel soil microcosms were amendeded with a complex C mixture containing either \textsuperscript{13}C-xylose, \textsuperscript{13}C-cellulose, or no isotope. At discrete time points over a period of 30 days, microcosms were harvested and the temporal dynamics and isotope assimilation of the microbial community were measured by sequencing 16S rRNA in the bulk microbial community and fractions from CsCl gradient fractionation (\href{https://www.authorea.com/users/3537/articles/8459/master/file/figures/20140708_ConceptualFig2/20140708_ConceptualFig2.pdf}{Fig. S1}). Overall, xylose degradation is immediately, within the first 7 days, while the bulk of cellulose degradation is observed after 14 days. 


The dynamics of \textsuperscript{13}C-cellulose and \textsuperscript{13}C-xylose assimilation varied dramatically for different microorganisms.  temporal changes in microbial community composition are consistent with C decomposition being accompanied by a microbial community succession. The dynamics of \textsuperscript{13}C-cellulose and \textsuperscript{13}C-xylose assimilation varied dramatically for different microorganisms. Isotope incorporation into DNA was revealed by analyzing variation in 16S rRNA amplicons across gradient fractions (n = 20) from control samples in relation to identical experimental samples that differed by a substitution of \textsuperscript{12}C-cellulose or \textsuperscript{12}C-xylose with their \textsuperscript{13}C equivalents (Figure 1). Isotope incorporation changes amplicon composition relative to control and this effect can be visualized in ordination by divergence of experimental samples from corresponding control points. This is demonstrated in the high-density fractions that are differentiating from the control along NMDS2. Differentiation of these fractions relative to control indicates the presence of \textsuperscript{13}C-assimilating OTUs. The differential separation of high density fractions in the \textsuperscript{13}C-xylose treatment compared to the \textsuperscript{13}C-cellulose treatment is indicative of a difference in the responders for each of the substrates (Fig 1A). There is an observable time signature of responders at days 1, 3, and 7 for the xylose treatment and days 14 and 30 for the cellulose treatment (Fig1B). This demonstrates that different microbial community members are responsible for the consumption of these two substrates; xylose is consumed quickly, whereas, cellulose decomposition takes longer. This supports the hypothesis of a microbial community succession during the decomposition process. Furthermore, this demonstrates the sensitivity of this technique by being able to detect \textsuperscript{13}C-label incorporation in samples with low C additions (2.18mgC g\textsuperscript{-1} soil).    

\textbf{Differential taxa C utilization.} Using fractions from within a density range of 1.7125-1.755 gmL\textsuperscript{-1}, relative abundances of taxa in the experimental treatments were compared to their respective relative abundances in the control treatment to calculate the log\textsubscript{2} fold change (Fig2). The log\textsubscript{2} fold change demonstrates the boom and bust of taxa with time. For OTUs passing a conservative threshold of \textit{p}-value = <0.10 for log\textsubscript{2} fold change (FigSx), we measured the density shift in the experimental treatment compared to the control (Fig Sx).  Those OTUs with a center of mass shift greater than zero were considered 'responders'.   

For the \textsuperscript{13}C-xylose treatment at day 1 OTUs within Firmicutes demonstrate the strongest response. This is not surprising as it has been demonstrated that Firmicutes maintain a metabolically-ready state \cite{Jenkins_2010,Griffiths_1998,Brookes_1987,De_Nobili_2001}. Additionally, Proteobacteria, Bacteroidetes, and Actinobacteria contain responder OTUs at day 1. Genomic analysis of fast growing bacteria, specifically Proteobacteria and Firmicutes, have a higher number of total transporters enabling them to import or export a broad range of compounds \cite{Barabote_2005}. The low affinity of these transporters facilitates fast growth in times of high nutrient conditions \cite{Trivedi_2013}. Day 3 exhibits a strong increase in Bacteroidetes response and the onset of Verrucomicrobia reponders. Notably, this pronounced response by Bacteroidetes is not captured by the bulk community abundances. Day 7 demonstrates an increased response from Proteo- and Actinobacterial OTUs. While there is a slight increase in their abundances in the bulk community analysis at day 7, it would be difficult to differentiate that change from natural variation or methodological noise. All OTUs have a decreasing log\textsubscript{2} fold change by days 14 and 30, with only a single Firmicutes OTU passing the 'responder' criteria.     

For the \textsuperscript{13}C-cellulose treatment only one Proteobacteria passes the 'responder' criteria at day 3 and two OTUs (Proteobacteria and Chloroflexi) at day 7. It is likely that the slow degradation of cellulose can be attributed to the energy-taxing process of synthesizing cellulolytic enzymes and exporting them, as cellulose is most often broken down externally (Schimel & Schaeffer 2012, Lynd et al 2002). As a result, microorganisms responsible for the synthesis of cellulases preferentially shuttle energy towards enzyme synthesis rather than biomass until cellulose hydrolysis begins (Schimel & Schaeffer 2012). This accounts for the delay in growth and ultimately the slow decomposition of cellulose (Perez et al 2002,Schimel & Schaeffer 2012).  

By day 14, responders are detected in Proteobacteria, Verrucomicrobia, Chloroflexi, Planctomycetes, and Actinobacteria. The degradation of cellulose by Verrucomicrobia is consistent with recalcitrant carbon degradation by Verrucomicrobia in soil, aquatic, and anoxic rice patty soils \cite{Fierer_2013,Herlemann_2013,10543821}. Fierer found Verrucomicrobia to be more than 50\% of their bulk community sequences which has strong implications of the importance of this taxa in soil carbon cycling \cite{Fierer_2013}. In the \textsuperscript{13}C-cellulose treatment the same responders were detected at day 30 as with earlier time points, with the exception of Actinobacteria and the addition of Bacteroidetes. The cellulose degrader trends are more readily observable in the bulk community abundances than discerned with xylose responders. This is likely due to the low abundance of these phlya, where changes in bulk community abundance are more pronounced and easier to detect. Comparatively, phlya of consistently high abundance mask response changes unless they present changes of grand proportions.          

Kernel density estimates (KDE) of the CsCl density shifts measured for responders in \textsuperscript{13}C-xylose were compared to those of \textsuperscript{13}C-cellulose responders (Fig 3A). An organism with 100\% \textsuperscript{13}C-labeling of DNA would exhibit a density shift of 0.04gmL\textsuperscript{-1}. Xylose utilizers have a smaller density shift (<0.02 gmL\textsuperscript{-1}) than cellulose utilizers (0.005-0.03 gmL\textsuperscript{-1}), with few exceptions. This suggests a greater substrate specificity among cellulose degraders than xylose degraders. Partial \textsuperscript{13}C-labeling (<0.04gmL\textsuperscript{-1} density shift) could be a result of various lifestyles ('trophic strategy' better word choice?) such as (1) assimilation of C from multiple substrates (both \textsuperscript{12}C and \textsuperscript{13}C in this instance) or (2) \textsuperscript{13}C-label dilution as it cascades through trophic levels via consumption of \textsuperscript{13}C-labeled organisms or waste products from organisms that are metabolizing the \textsuperscript{13}C-substrate.

Most xylose responders are found at higher rank abundances than cellulose responders which fall among the rare taxa in the tail of the RA curve (Fig 3B). This demonstrates that many taxa important to C-cycling are present in the rare biosphere and may be difficult or unable to detect in bulk community sequencing efforts. Responders of xylose or cellulose are wide spread across 7 phyla (Fig 4). There are very few OTUs that utilize both cellulose and xylose (should put in a number of how many OTUs utilize both over how many responders total there were), however, at the phyla level many phylum had responders for both xylose and cellulose. 

Here we see an increased abundance of Bacteroidetes accompanied with both high (xylose treatment response) and low (cellulose treatment response) C availability. The differential response of taxa within Bacteroidetes could be the cause of debate between the correlation of phylum level abundance associated with C availability \cite{Fierer_2007,Rui_2009,Sharp_2000,L_pez_Lozano_2013,Bastian_2009}. For Actinobacteria, genomic analysis has revealed the ability to utilize a diverse array of polysaccharides which has been attributed to their resilience to stressful soil conditions \cite{Trivedi_2013}. Despite this versatile ability, they have also been demonstrated to possess high affinity transporters for specific substrates \cite{Trivedi_2013}. In this study, Actinobacteria had responders for both substrates with a peak shift of ~0.036 gmL\textsuperscript{-1} suggesting a strong substrate specificity (Figures Sx and Sz - the substrate utilization charts). Albeit, there are no OTUs within Actinobacteria that responded to both xylose and cellulose. This information may suggest that while Actinobacteria exhibit an ability to utilize an array of carbon substrates, substrate specificity may be more clade specific for certain substrates and not widespread throughout the phylum (Figure 4). Additionally, we identified a previously undescribed clade of Chloroflexi that exhibit a high substrate specificity based on a density shift peak at between 0.03-0.04 gmL\textsuperscript{-1}.  We saw no other cellulose utilization in Chloroflexi outside of this clade although many members of this phylum has previously been demonstrated to utilize cellulose \cite{Goldfarb_2011,Cole_2013,Hug_2013}. These taxa and clade specific responses within each phylum serves as a strong argument towards moving away from extrapolating substrate utilization to phylum level responses.       

\textbf{Carbon substrate utilization is inconsistent within phylum.} More often than not we see ecological functionality assigned at the phylum level (refs). It has been proposed that the microbial community functionality responsible for soil C cycling appear at the level of phlya rather than species/genera \cite{Schimel_2012}. However, based on our evidence of clade or single taxa level responses for xylose and cellulose utilization, assigning phylum level functionality is not an accurate depiction of soil C utilization. Phylum level assignment conventions could in part be due to limitations in finer scale taxonomic identifications or to lack of sequencing resolution as seen here with many responders present in the rarer ranks (Figure 3B). 

Whole phylum responses were not detected for xylose or cellulose yet utilization of these substrates spanned many phylogenetically diverse groups. However, substrate utilization within each phylum was demonstrated at the clade or single taxa level. In a study that amended forest soils with single C substrates in the presence of 3-bromodeoxyuridine (BrdU), they determined that more than 500 taxa responded to labile C but occured predominantly in two phlya, Proteobacteria and Actinobacteria \cite{Goldfarb_2011}. On the other hand, the soils amended with more recalcitrant C such as cellulose and lignin were noted for spanning eight phyla, but were limited to a small number of taxa within each of those phyla that were responders \cite{Goldfarb_2011}. It has previously been suggested that all taxa within a phylum are unlikely to share ecological characteristics \cite{Fierer_2007}, and furthermore, within a species population \cite{Choudoir_2012,Preheim_2011,Hunt_2008}. Habitat traits of coastal Vibrio isolates were mapped onto microbial phylogeny revealing discrete ecological populations based on seasonal occurrence and particulate size fractionation \cite{Preheim_2011, Hunt_2008}. These data would suggest that portraying the response of a few OTUs or clades as a phylum level response would be overreaching.     

\textbf{Conclusions.} We have demonstrated how next generation sequencing-enabled SIP gives a taxa level resolution for substrate utilization. Using this technique, we are able to resolve discrete OTUs that would otherwise be missed using bulk community sequencing efforts. Additionally, this technique provides greater taxonomic resolution than previous techniques (cloning, TRFLP, ARISA) used to determine substrate utilizing community members. Traditionally, SIP is performed with a single time point in an attempt to minimize or eliminate cross-feeding of the substrate.  However, this cross-feeding is an exciting advantage of this system because it will enable us to track substrates of interest (i.e. cellulose) through many trophic cascades. Observed organisms will exhibit a range of \textsuperscript{13}C labeling, 0-100\%, with primary consumers being the most enriched and subsequent trophic levels being less enriched \cite{Morris_2002,McDonald_2005,Ziegler_2005}. This relationship of trophic level consumption to dilution of label will facilitate tracking C as it moves through various operational taxonomic units (OTUs) in the soil. While we are currently able to resolve highly responsive OTUs, there is still a need to resolve taxa that are partially responsive which we cannot differentiate from noise with confidence at this time. Yet, given the ability to resolve partially responsive taxa, the ecology would still be difficult to discern. For example, a generalist utilizing many substrates including \textsuperscript{12}C substrates and the \textsuperscript{13}C-labeled substrate may exhibit the same partial labeling that a specialist utilizing both the \textsuperscript{13}C-substrate and the same substrate (unlabeled) that is inherent in the soil. Additionally, partially labeled taxa could be further down the trophic cascade including predators or secondary consumers of waste products from primary consumer microbes that were highly labeled.   

Our study is consistent with carbon degradative succession that has previously been demonstrated \cite{Bastian_2009} (more refs). We demonstrate a rapid decrease in the labile carbon, xylose, confirmed by its \textsuperscript{13}C label incorporation into the microbial community DNA during the first 7 days of the experiment, after which, the label is not detectable in the DNA. Subsequently our data demonstrates a slow degradation of the more recalcitrant, polymeric carbon demonstrated by \textsuperscript{13}C-cellulose label incorporation into the microbial community DNA at 14 and 30 days. We did not observe the \textsuperscript{13}C-cellulose signal leave the DNA within the time limits, 30 days, of our experiment. This degradative succession is also confirmed by isotopic analysis of the soil from the microcosms (Table S1). NMDS demonstrates microbial succession and based on xylose and cellulose treatments separating away from the control, but in opposing directions indicating different microbial community members are responsible for degradation of the two C substrates. 

We did not observe consistent C utilization at the phylum level although both xylose and cellulose utilization were observed across 7 phyla each revealing a high diversity of bacteria able to utilize these substrates. The high taxonomic diversity may enable substrate metabolism under a broad range of environmental conditions \cite{Goldfarb_2011}. Other studies of microbial communities have observed a positive correlation with taxonomic or phylogenetic diversity and functional diversity \cite{Fierer_2012,Fierer_2013,Philippot_2010,Tringe_2005,Gilbert_2010,Bryant_2012}. A study on pre-agricultural prairie soils observed a decrease in functional capabilities of soil microbial communities with decreasing taxonomic diversity suggesting they do not exhibit a high degree of functional redundancy \cite{Fierer_2013}. The data presented here supports that specific functional attributes can be shared among diverse, yet distinct, taxa while closely related taxa may have very different physiologies \cite{Fierer_2012,Philippot_2010}. This information adds to the growing collection of data suggesting that community membership is important to biogeochemical processes. Furthermore, demonstrates a need to examine substrate utilization by discrete microbial taxa within whole community studies, versus culture isolation, to better understand how specific community members function within the whole. The sensitivity of SIP-NGS provides a means to elucidate substrate utilization by discrete microbial taxa with the hope that we can begin to construct a belowground C food web.          
 
 
 Degradative succession refers to the temporal changes in species or functional guilds that occurs during the sequential degradation of constituents of a nutrient resource \cite{townsend2003essentials,Bastian_2009}. The decomposition of a nutrient source is hypothesized to promote succession of active community members as compounds are sequentially degraded \cite{Biddanda_1988}. 


Analysis of the sequenced bulk community DNA demonstrates Proteobacteria (26-35\%), Actinobacteria (19-26\%), and Acidobacteria (12-21\%) as the most dominant phyla throughout the duration of the experiment. This is consistent with previous observations \cite{Goldfarb_2011,Fierer_2007,Rui_2009,Fierer_2012}. We found trends of Proteobacteria and Actinobacteria decreasing and Acidobacteria increasing as C availability declines (TableS1). This is congruent with findings in soils sampled from a wide range of ecosystems in the US \cite{Fierer_2007}. At days 1, 3, and 7 the bulk community was composed of $\sim$12-18\% Bacteriodetes and Firmicutes (combined).  At days 14 and 30, these phyla declined to a combined 7-9\% of the whole community accompanied with an increase in Planctomycetes, Verrucomicrobia, and Chloroflexi (2-3\% at day 1 to 5-7\% at day 30). There has been conflicting evidence about the correlation of Bacteriodetes abundance with C availability \cite{Fierer_2007,Rui_2009,Sharp_2000,L_pez_Lozano_2013,Bastian_2009}. Our bulk data demonstrates a positive correlation between labile C availability and Bacteroidetes abundance. Additionally, the abundance of Planctomycetes, Verrucomicrobia, and Chloroflexi at later stages of decomposition are in accord with findings in wheat straw degradation \cite{Bastian_2009}. The rank abundance (RA) of the community depicts transitions we observe in high ranking phyla abundance beginning at day 7 (Fig S2A). Despite the fluctuations we observe at the phylum level, the biological variability observed over time is low (FigS2B) demonstrating community stability.

