\section{Results}
% Fakesubsubsection:We tracked the flow of C from xylose
We tracked the flow of C from xylose and cellulose into microbial biomass over
time using DNA-SIP (Figure~\ref{fig:setup}). We added 3 mg C g$^{-1}$ dry
weight soil, comprising 25\% of total soil C, to each soil sample. The C
addition included 0.42 mg xylose-C and 0.88 mg cellulose-C g$^{-1}$ soil
dry weight (3.5\% and 7.3\% of total soil C, respectively). We harvested
soil samples at 1, 3, 7, 14 and 30 incubation days. The soil microbial
community metabolized the majority of xylose within one day. Sixty-five
percent of the $^{13}$C from $^{13}$C-xylose was respired by day 1 (Figure
SX). Twenty-nine percent of the remaining $^{13}$C persisted; 29\% of the
total $^{13}$C-xylose added remained in the soil at day 30 (Figure SX). In
contrast, $^{13}$C from $^{13}$C-cellulose declined at a constant rate of
approximately 18 μg C g$^{-1}$ d$^{-1}$. Forty percent of $^{13}$C added
as cellulose remained in the soil at day 30 (Figure SX). Microcosm
treatments are identified in figures by the following code: ``13CXPS''
refers to the amendment with $^{13}$C-xylose ($^{13}$\textbf{C}
\textbf{X}ylose \textbf{P}lant \textbf{S}imulant), ``13CCPS'' refers to
the $^{13}$C-cellulose amendment and ``12CCPS'' refers to the amendment
that only contained $^{12}$C (i.e. control).

\subsection{Soil microcosm microbial community changes with time}
% Fakesubsubsection:Changes in the soil microcosm microbial community structure
We assessed incorporation of $^{13}$C into microbial community DNA by comparing the
SSU rRNA gene sequence composition of SIP density gradient fractions from
either control or $^{13}$C-amended microcosm soil DNA. We set up three microcosm
series. All microcosms series received the same C mixture, but, in two
microcosm series a $^{13}$C-labeled substrate (i.e. $^{13}$C-xylose or
$^{13}$C-cellulose) was substituted for its $^{12}$C equivalent. In one
microcosm series, the control, all added C substrates were unlabeled. The
majority of the variance in SSU rRNA gene composition from control density
gradient fractions is represented by fraction density. DNA buoyant density is
correlated to G$+$C content \citep{Buckley_2007} and therefore control gradient
fractions SSU gene composition variation is strongly influenced by G$+$C
content. For the $^{13}$C-cellulose amendment, SSU rRNA gene composition in
gradient fractions deviate from control fractions at high density (> 1.72
g ml$^{-1}$) at days 14 and 30 (Figure~\ref{fig:ord}). For the $^{13}$C-xylose amendment,
SSU rRNA gene composition in density gradient fractions also deviate from
control in high density fractions but in contrast to the $^{13}$C-cellulose
amendment the SSU rRNA gene composition of density gradient fractions deviate
from control at days 1, 3, and 7 as opposed to days 14 an 30 (Figure~\ref{fig:ord}). SSU
rRNA gene composition in high density fractions differs between
$^{13}$C-cellulose amendments and $^{13}$C-xylose amendments indicating
microorganisms that incorporated $^{13}$C from xylose into DNA were different
from those that incorporated $^{13}$C from cellulose (Figure~\ref{fig:ord}). Further, the
SSU rRNA gene sequence composition of high density fractions from
$^{13}$C-cellulose amendments at days 14 and 30 is similar indicating similar
microorganisms have $^{13}$C labeled DNA at days 14 and 30. In contrast, the
SSU rRNA gene composition of the $^{13}$C-xylose amendment high density
gradient fractions varies between days 1, 3, and 7 indicating that different
microbes have $^{13}$C labeled DNA on these days. $^{13}$C-xylose amendment SSU
gene composition is similar to control across the whole density gradient on
days 14 and 30 (Figure~\ref{fig:ord}) indicating that $^{13}$C from $^{13}$C-xylose is no
longer detectable in DNA on days 14 and 30. 

\subsection{Temporal dynamics of microcosm microbial community composition}
% Fakesubsubsection:We monitored the soil microbial community
We monitored the soil microbial community over the course of the experiment by
sequencing SSU rRNA gene amplicons generated directly from soil DNA, without
density gradient fractionation. The SSU rRNA gene composition of the
non-fractionated soil DNA varied significantly over time (Figure~\ref{fig:bulk_ord},
P = 0.023, R$^{2}$ = 0.63, Adonis test \citet{Anderson2001a}), but did not
vary significantly with respect to amendment (Figure~\ref{fig:bulk_ord}, P = 0.23, R$^{2}$
= 0.21, Adonis test \citet{Anderson2001a}). The latter result demonstrates the substitution
of $^{13}$C-labeled substrates for unlabeled equivalents did not
significantly alter community composition. The variance in SSU rRNA gene
composition among non-fractionated soil DNA samples was significantly less
than the variance among gradient fractions (Figure~\ref{fig:bulk_ord}, P = 0.003,
“betadisper” function R Vegan package \citet{oksanen2007vegan} (15)) and
therefore the process of density gradient fractionation produced
significantly more variance in SSU rRNA gene composition than temporal
changes in the non-fractionated DNA SSU rRNA gene content.

% Fakesubsubsection:Twenty-nine OTUs
Twenty-nine OTUs changed in significantly in relative abundance with time ("BH”
adjusted P-value < 0.10 16)). When SSU rRNA gene abundances were combined at
the taxonomic rank of "Class", only Bacilli
(decreased), Flavobacteria (decreased), Gammaproteobacteria (decreased) and
Herpetosiphonales (increased) significantly changed in relative abundance with
time (P < 0.10, Figure~\ref{fig:time_class}). Of the 29 OTUs that changed in relative abundance
with time, 14 were found to have incorporated $^{13}$C into DNA (Figure~\ref{fig:time} and
below). Of the 14 OTUs that were found to have both incorporated $^{13}$C into
DNA and significantly changed in abundance with time, OTUs that incorporated
$^{13}$C from $^{13}$C-cellulose increased with time whereas those that
incorporated $^{13}$C from $^{13}$C-xylose decreased over time and OTUs that
responded to both substrates were found to either have increased or decreased
over time (Figure~\ref{fig:time}~and~\ref{fig:babund}).

\subsection{OTUs that assimilated $^{13}$C into DNA} \label{responders}
% Fakesubsubsection:Within the first 7 days of incubation approximately 63\% 
If an OTU exhibited strong evidence for $^{13}$C incorporation into DNA we
refer to that OTU as a "responder" (see Supplemental Note 1.7.4 for our
operational definition of "responder"). We evaluated XXXX and XXXX OTUs for
evidence of $^{13}$C incorporation into DNA from $^{13}$C-cellulose and
$^{13}$C-xylose, respectively. We detected a 41 OTUs that responded to
13-xylose and 55 OTUs that responded to $^{13}$C-cellulose (Figure~\ref{fig:l2fc}, 
Tables~{tab:cell}~and~\ref{tab:xyl}). Eight OTUs responded to both xylose and cellulose. The number of
xylose responders with $^{13}$C labeled DNA (i.e. "responders") peaked days
1 and 3 and declined with time, in contrast the number of cellulose responders
increased with time peaking at days 14 and 30 (Figure~\ref{fig:rspndr_count}). 

The phylogenetic types of 13C labeled OTUs changed with time (Figure~\ref{fig:l2fc}~and~\ref{fig:xyl_count}).
On day 1, Bacilli OTUs represented 84\% of xylose responders, and the
majority of these OTUs were closely related to cultivated representatives of
the genus \textit{Paenibacillus} (n = XX, Table~\ref{tab:xyl}). For example, "OTU.57"
(Table XX), annotated as \textit{Paenibacillus}, has a strong signal of isotope
incorporation from 13C-xylose at day 1, at its maximum relative abundance in
non-fractionated soil DNA. The relative abundance of "OTU.57" declines until
day 14 and evidence of 13C labeling is not detectable after day 1 (Figure
X). On day 3 \textit{Bacteroidetes} OTUs comprised 63\% of xylose responders
(Figure~\ref{fig:xyl_count}). These OTUs were closely related to cultivated representatives of
the \textit{Flavobacteriales} and \textit{Sphingobacteriales} (Table~\ref{tab:xyl}). For
example, "OTU.14", annotated as a Flavobacterium, has a strong signal 13C
labeling 13C-xylose at days 1 and 3 coinciding with its maximum relative
abundance in non-fractionated soil DNA. The relative abundance of "OTU.14" then
declines until day 14 and evidence of 13C labeling is not significant after day
3 (Figure X). Finally, on day 7, \textit{Actinobacteria} OTUs represented 53\%
of the xylose responders and these OTUs were closely related to cultivated
representatives of Micrococcales (Table~\ref{tab:xyl}). For example, "OTU.4", annotated as
\textit{Agromyces}, has signal of 13C labeling on days 1 through 7 with the
strongest evidence 13C labeling at day 7, its relative abundance in
non-fractionated soil increases until day 3 and then declines gradually until
day 30 and evidence of 13C labeling declines after day 7 (Figure X).
\textit{Proteobacteria} were also common among xylose responders at day 7 where
they comprised 40\% of xylose responsive OTUs. \textit{Proteobacteria} also
represented the majority (6 or 8) of OTUs that responded to both cellulose and
xylose dual responders. The majority (86\%) of xylose responders shared $>$
97\% SSU rRNA gene sequence identity to bacteria already cultivated in
isolation (Table~\ref{tab:xyl}). 

%Fakesubsubsection:Cellulose responders were
The phylogenetic types of cellulose responders did not change with time to the
same extent as xylose responders, and, also in contrast to xylose responders,
belonged to non-cultivated microbial clades. Both the relative abundance and
the total number of cellulose responders increased over time peaking at days 14
and 30 (Figures~\ref{fig:l2fc}, \ref{fig:rspndr_count}, and \ref{fig:babund}). The phylogenetic composition of cellulose
responders changed little between days 14 and 30 (Table~\ref{tab:cell}). Cellulose
responders belonged to the \textit{Proteobacteria} (46\%),
\textit{Verrucomicrobia} (16\%), \textit{Planctomycetes} (16\%),
\textit{Chloroflexi} (8\%), \textit{Bacteroidetes} (8\%),
\textit{Actinobacteria} (3\%), and \textit{Melainabacteria} (1 OTU) (Table~\ref{tab:cell}).
The majority (86\%) of cellulose responders in the \textit{Proteobacteria} were
closely related (> 97\% identity) to bacteria already cultivated in isolation,
including representatives of the genera: \textit{Cellvibrio}, \textit{Devosia},
\textit{Rhizobium}, and \textit{Sorangium}, which are known for their ability
to degrade cellulose (Table~\ref{tab:cell}). A relatively high number of cellulose
responding OTUs (13 OTUs) among proteobacterial OTUs belonged to the
\textit{Alphaproteobacteria}. Other proteobacterial cellulose responders
belonged to \textit{Beta-} (4 OTUs), \textit{Gamma-} (5 OTUs), and
\textit{Deltaproteobacteria} (6 OTUs). 

The majority (85\%) of cellulose responders outside of the
\textit{Proteobacteria} shared  $>$ 97\% SSU rRNA gene identity to bacteria
already cultivated in isolation. For example, most (70\%) of the
\textit{Verrucomicrobia} cellulose responders fell within a few unidentified
\textit{Spartobacteria} clades, and these had $<$ 85\% SSU rRNA gene identity
to any characterized isolate. The \textit{Spartobacteria} OTU "OTU.2192" was
characteristic of many cellulose responders (Figure X). In non-fractionated
soil DNA, "OTU.2192" gradually increased in relative abundance with time and
evidence for $^{13}$C labeling also increased gradually with the strongest evidence
at days 14 and 30 (Figure X). \textit{Choloflexi} cellulose responders
predominantly belonged to an unidentified clade within the
\textit{Herpetosiphonales} and these had < 89\% SSU rRNA gene identity to any
characterized isolate. Characteristic of other \textit{Chloroflexi} cellulose
responders "OTU.64" increased in relative abundance over 30 days and evidence
for $^{13}$C labeling peaked days 14 and 30 (Figure X). Cellulose responders found
within the \textit{Bacteroidetes} differed from the \textit{Bacteroidetes}
xylose responders falling instead within the \textit{Cytophagales} as opposed
to the \textit{Flavobacteria} or \textit{Sphingobacteriales}.
\textit{Bacteroidetes} cellulose responders included one OTU that had 100\% SSU
rRNA gene identity to species of \textit{Sporocytophaga}, a genus that includes
known cellulose degraders.

\subsection{Characteristics of cellulose and xylose responders}
% Fakesubsubsection:Cellulose responders tended
Cellulose responders tended to have lower relative abundance in
non-fractionated soil DNA, higher $^{13}$C:12C ratios in DNA, and lower estimated
rrn copy number than xylose responders. In the non-fractionated soil DNA,
cellulose responders had significantly lower relative abundance (7e$^{-4}$
± 2e$^{-3}$, s.d.) than xylose responders (2e$^{-3}$ ± 4e$^{-3}$, s.d.) (Figure~\ref{xyl_count},
P = 0.00028, Wilcoxon Rank Sum test). Xylose responders comprised 6 of the 10
most common OTUs observed in the non-fractionated soil DNA, and, of the 10
most abundant $^{13}$C substrate responders in the non-fractionated soil DNA
8 were xylose responders and 2 were cellulose responders. Cellulose responders
were mainly present at low abundance with XX\% of cellulose responders found at
relative abundance below XX in the non-fractionated soil DNA base on SSU rRNA
gene composition. However, $^{13}$C-xylose and $^{13}$C-cellulose responders
included OTUs at both high and low abundance (Figure~\ref{fig:shift}). Two
$^{13}$C-cellulose responders were not found in any bulk samples (``OTU.862''
and ``OTU.1312'', Table~\ref{tab:cell}).

% Fakesubsubsection:Cellulose responders exhibited a greater shift in BD
DNA buoyant density increases as its ratio of $^{13}$C to $^{12}$C increases.
An organism that only assimilates C into DNA from a $^{13}$C isotopically
labeled source, will have a greater DNA $^{13}$C to $^{12}$C ratio 
than an organism utilizing a mixture of isotopically labeled and unlabeled
C sources (see Supplemental~Note~1.8). Therefore, the extent of 13C
incorporation into DNA can be evaluated by the change in DNA buoyant density
(BD) upon 13C labeling. We do not know the absolute abundance of OTUs across
the density gradient as our SSU rRNA gene sequence counts are compositional in
nature so we cannot assess absolute OTU buoyant density shifts due to 13C
labeling. However, we can evaluate relative 13C incorporation by quantifying
and comparing the shift in the relative abundance profile for an OTU along
density gradient fractions for 13C amendments relative to its profile along the
control density fractions. Specifically, in this study we calculated each OTU's
relative abundance center of mass shift in each label/control DNA density
gradient pair (see supplemental methods for the detailed calculation). We refer
to this metric as $\hat{\Delta}BD$. This metric indicates relative differences
in DNA 13C:12C and can be used to compare DNA 13C:12C between groups of
responders. $\hat{\Delta}BD$ does not represent the true density shift for an
OTU because it is based on relative abundance and is therefore not directly
comparable to literature values for DNA density shifts due to isotopic
labeling. Cellulose responder $\hat{\Delta}BD$ (0.0163 ± 0.0094 g ml$^{-1}$, s.d.)
was significantly greater than that of xylose responders (0. 0097 ± 0.0094
g ml$^{-1}$, s.d.) (Figure S9, Figure~\ref{xyl_count}, P = 1.8610e$^{-6}$, Wilcoxon Rank Sum test). 

% Fakesubsubsection:We assessed phylogenetic
We assessed phylogenetic clustering of 13C-responsive OTUs with the Nearest
Taxon Index (NTI), the Net Relatedness Index (NRI) \citep{Webb2000}, and the consenTRAIT
metric (19). Briefly, positive NRI and NTI with corresponding low P-values
indicates deep phylogenetic clustering whereas negative NRI with high P-values
indicates taxa are overdispersed compared to the null model \citep{Evans2014a}.
The consenTRAIT metric is a measure of the average clade depth for a trait
in a phylogenetic tree. These metrics evaluate the degree to which the trait of
xylose or cellulose response is conserved phylogenetically. NRI values, which
indicate clustering in the overall phylogeny as a function of mean phylogenetic
distance, indicate that cellulose responders are clustered phylogenetically
(NRI: 4.49, P = 0.001) while xylose responders are overdispersed (NRI: -1.33,
P = 0.90). NTI values, which measure clustering at the tips of the phylogeny as
a function of nearest neighbor distance, show that both cellulose and xylose
responders are phylogenetically clustered (NTI: 1.43, P = 0.072; 2.69,
P = 0.001), respectively). The consenTRAIT clade depth for xylose and cellulose
responders was 0.012 and 0.028 SSU rRNA gene sequence dissimilarity,
respectively. As reference, average clade depth is 0.017 SSU rRNA gene sequence
dissimilarity for arabinose utilization (another five C sugar found in
hemicellulose) and was 0.013 and 0.034 SSU rRNA gene sequence dissimilarity for
Glucosidase and cellulase activity, respectively, as determined from genome
analyses of cultivated isolates \citep{Martiny2013} (Martiny et al., 2013;
Berlemont & Martiny, 2013)). These results indicate non-random phylugenetic
distribution of xylose and cellulose metabolism among soil microorganisms.





