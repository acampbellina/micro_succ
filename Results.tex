\section{Results}
% Fakesubsubsection:We tracked the flow of C from xylose
After adding the organic matter amendment to soil, we tracked the flow of
$^{13}$C from $^{13}$C-xylose or $^{13}$C-cellulose into microbial DNA over
time using DNA-SIP (Figure~\ref{fig:setup}). The amendment consisted of
compounds representative of plant biomass including cellulose, lignin, sugars
found in hemicellulose, amino acids, and inorganic nutrients (see Supplemental
Information (SI)). The amendment was added at~2.9 mg C g$^{-1}$ soil dry weight
(d.w.), and this comprised 19\% of the total C in the soil. The cellulose-C
(0.88 mg C g$^{-1}$ soil d.w.) and xylose-C (0.42 mg C g$^{-1}$ soil d.w.) in
the amendment comprised 6\% and 3\% of the total C in the soil, respectively.
The soil microbial community respired 65\% of the xylose within one day and
29\% of the added xylose remained in the soil at day~30 (Figure~\ref{fig:13C}).
In contrast, cellulose-C declined at a rate of approximately 18 $\mu$g
C d $^{-1}$ g $^{-1}$ soil d.w. and 40\% of added cellulose-C remained in the
soil at day~30 (Figure~\ref{fig:13C}).

\subsection{Types of $^{13}$C-labeled OTUs changed with time and substrate}
% Fakesubsubsection: We assessed assimilation of
We assessed assimilation of $^{13}$C into microbial DNA by comparing the SSU
rRNA gene sequence composition of SIP density gradient fractions between
$^{13}$C treatments and the unlabeled control (see Methods and SI). In the
gradient density fractions for the control treatment, fraction density
represented the majority of the variance in SSU rRNA gene composition
(Figure~\ref{fig:ord}). Genome G$+$C content correlates positively with DNA
buoyant density and influences SSU rRNA gene composition in gradient fractions
\citep{Buckley_2007}. For the $^{13}$C-cellulose treatment, the SSU rRNA gene
composition in gradient fractions deviated from control in high density
fractions ($>$~1.72 g mL$^{-1}$) on days~14 and~30 (Figure~\ref{fig:ord}). For
the $^{13}$C-xylose treatment, SSU rRNA gene composition in gradient
fractions also deviated from control in high density fractions, but it deviated
from control on days~1,~3,~and~7 (Figure~\ref{fig:ord}). The SSU rRNA gene
composition from the $^{13}$C-cellulose treatment and $^{13}$C-xylose treatment
high density gradient fractions differed indicating different microorganisms
assimilated C from xylose than cellulose (Figure~\ref{fig:ord}). Further, in
the $^{13}$C-cellulose treatment, the SSU rRNA gene sequence composition in
high density fractions was similar on days 14 and~30 indicating similar
microorganisms had $^{13}$C-labeled DNA in $^{13}$C-cellulose treatments at
days~14 and~30. In contrast, in the $^{13}$C-xylose treatment, the SSU rRNA
gene composition of high density fractions varied between days~1,~3,~and~7
indicating that different microbes had $^{13}$C-labeled DNA on each of these
days. In the $^{13}$C-xylose treatment, the SSU gene composition of high
density fractions was similar to control on days~14~and~30
(Figure~\ref{fig:ord}) indicating that $^{13}$C was no longer detectable in
bacterial DNA on these days for this treatment. 

\subsection{Temporal dynamics of OTU relative abundance in experimental soil}
% Fakesubsubsection:We monitored the soil microbial community
We monitored the experimental soil microbial community over the course of the
experiment by surveying SSU rRNA genes in non-fractionated DNA from the
experimental soil. The SSU rRNA gene composition of the non-fractionated DNA
changed with time (Figure~\ref{fig:bulk_ord}, P-value~$=$~0.023, R$^{2}$
$=$~0.63, Adonis test \citep{Anderson2001a}). In contrast, the non-fractionated
DNA SSU rRNA gene composition showed no statistical evidence for changing with
treatment (P-value~0.23, Adonis test) (Figure~\ref{fig:bulk_ord}). The latter
result demonstrates the substitution of $^{13}$C-labeled substrates for
unlabeled equivalents could not be shown to alter the soil microbial community
composition. Twenty-nine OTUs exhibited sufficient statistical evidence
(adjusted P-value $<$ 0.10, Wald test) to conclude they changed in relative
abundance in the non-fractionated DNA over the course of the experiment
(Figure~\ref{fig:time}). When SSU rRNA gene abundances were combined at the
taxonomic rank of "class", the classes that changed in abundance (adjusted
P-value~ $<$~0.10, Wald test) were the \textit{Bacilli} (decreased),
\textit{Flavobacteria} (decreased), \textit{Gammaproteobacteria} (decreased),
and \textit{Herpetosiphonales} (increased) (Figure~\ref{fig:time_class}). Of
the~29 OTUs that changed in relative abundance over time, 14 putatively
incorporated $^{13}$C into DNA (Figure~\ref{fig:time}). OTUs that likely
assimilated $^{13}$C from $^{13}$C-cellulose into DNA tended to increase in
relative abundance with time whereas OTUs that assimilated $^{13}$C from
$^{13}$C-xylose tended to decrease (Figure~\ref{fig:babund}). OTUs that
responded to both substrates did not exhibit a consistent relative abundance
response over time as a group (Figure~\ref{fig:time}~and~\ref{fig:babund}).

\subsection{Changes in the phylogenetic composition of $^{13}$C-labeled OTUs with time} \label{responders}
% Fakesubsubsection:If an OTU exhibited
If an OTU exhibited strong evidence for assimilating $^{13}$C into DNA, we
refer to that OTU as a "responder" (see Methods and SI for our operational
definition of "responder"). The SSU rRNA gene sequences produced in this study
were binned into 5,940 OTUs and we assessed evidence of $^{13}$C-labeling from
both $^{13}$C-cellulose and $^{13}$C-xylose for each OTU. Forty-one OTUs
responded to $^{13}$C-xylose,~55 OTUs responded to $^{13}$C-cellulose, and
8 OTUs responded to both xylose and cellulose (Figure~\ref{fig:l2fc},
Figure~\ref{fig:tiledtree}, Figure~\ref{fig:genspec}, Table~\ref{tab:xyl}, and
Table~\ref{tab:cell}). The number of xylose responders peaked at days~1 and~3
and declined with time. In contrast, the number of cellulose responders
increased with time peaking at days~14 and~30 (Figure~\ref{fig:rspndr_count}). 

% Fakesubsubsection: The phylogenetic composition
The phylogenetic composition of xylose responders changed with time
(Figure~\ref{fig:l2fc}~and~Figure~\ref{fig:xyl_count}) and 86\% of xylose responders
shared $>$ 97\% SSU rRNA gene sequence identity with bacteria cultured in
isolation (Table~\ref{tab:xyl}). On day 1, \textit{Bacilli} OTUs represented
84\% of xylose responders (Figure~\ref{fig:xyl_count}) and the majority of
these OTUs were closely related to cultured representatives of the genus
\textit{Paenibacillus} (Table~\ref{tab:xyl}, Figure~\ref{fig:tiledtree}). For
example, "OTU.57" (Table~\ref{tab:xyl}), annotated as \textit{Paenibacillus},
had a strong signal of $^{13}$C-labeling at day~1 coinciding with its
maximum relative abundance in non-fractionated DNA. The relative abundance
of ``OTU.57'' declined until day 14 and ``OTU.57'' did not appear to be
$^{13}$C-labeled after day~1 (Figure~\ref{fig:example}). On day~3,
\textit{Bacteroidetes} OTUs comprised 63\% of xylose responders
(Figure~\ref{fig:xyl_count}) and these OTUs were closely related to
cultured representatives of the \textit{Flavobacteriales} and
\textit{Sphingobacteriales} (Table~\ref{tab:xyl},
Figure~\ref{fig:tiledtree}). For example, ``OTU.14'', annotated as
a flavobacterium, had a strong signal for $^{13}$C-labeling in the
$^{13}$C-xylose treatment at days 1 and 3 coinciding with its maximum
relative abundance in non-fractionated DNA. The relative abundance of
``OTU.14'' then declined until day 14 and did not show evidence of
$^{13}$C-labeling beyond day~3 (Figure~\ref{fig:example}). Finally, on
day~7, \textit{Actinobacteria} OTUs represented 53\% of the xylose
responders (Figure~\ref{fig:xyl_count}) and these OTUs were closely
related to cultured representatives of \textit{Micrococcales}
(Table~\ref{tab:xyl}, Figure~\ref{fig:tiledtree}). For example, ``OTU.4'',
annotated as \textit{Agromyces}, had signal for $^{13}$C-labeling in the
$^{13}$C-xylose treatment on days~1,~3~and~7 with the strongest evidence
of $^{13}$C-labeling at day~7 and did not appear $^{13}$C-labeled at
days~14 and~30. The relative abundance of ``OTU.4'' in non-fractionated
DNA increased until day~3 and then declined until day~30
(Figure~\ref{fig:example}). \textit{Proteobacteria} were also common among
xylose responders at day~7 where they comprised 40\% of xylose responder
OTUs. Notably, \textit{Proteobacteria} represented the majority (6 of 8)
of OTUs that responded to both cellulose and xylose
(Figure~\ref{fig:genspec}). 

% Fakesubsubsection: The phylogenetic composition of cellulose responders were
The phylogenetic composition of cellulose responders did not change with time
to the same extent as the xylose responders. Also, in contrast to xylose
responders, cellulose responders often were not closely related ($<$ 97\% SSU
rRNA gene sequence identity) to cultured isolates. Both the relative abundance
and the number of cellulose responders increased over time peaking at days 14
and 30 (Figure~\ref{fig:l2fc}, Figure~\ref{fig:rspndr_count}, and Figure~\ref{fig:babund}).
Cellulose responders belonged to the \textit{Proteobacteria} (46\%),
\textit{Verrucomicrobia} (16\%), \textit{Planctomycetes} (16\%),
\textit{Chloroflexi} (8\%), \textit{Bacteroidetes} (8\%),
\textit{Actinobacteria} (3\%), and \textit{Melainabacteria} (1 OTU)
(Table~\ref{tab:cell}). 

% Fakesubsubsection:The majority (85\%) of cellulose
The majority (85\%) of cellulose responders outside of the
\textit{Proteobacteria} shared  $<$ 97\% SSU rRNA gene sequence identity to
bacteria cultured in isolation. For example, 70\% of the
\textit{Verrucomicrobia} cellulose responders fell within unidentified
\textit{Spartobacteria} clades (Figure~\ref{fig:tiledtree}), and these shared $<$
85\% SSU rRNA gene sequence identity to any characterized isolate. The
\textit{Spartobacteria} OTU ``OTU.2192'' exemplified many cellulose responders
(Table~\ref{tab:cell}, Figure~\ref{fig:example}).
``OTU.2192'' increased in non-fractionated DNA relative abundance with time and
evidence for $^{13}$C-labeling of ``OTU.2192'' in the $^{13}$C-cellulose
treatment increased over time with the strongest evidence at days~14
and~30 (Figure~\ref{fig:example}). Most \textit{Chloroflexi} cellulose
responders belonged to an unidentified clade within the
\textit{Herpetosiphonales} (Figure~\ref{fig:tiledtree}) and they shared
$<$ 89\% SSU rRNA gene sequence identity to any characterized isolate.
Characteristic of \textit{Chloroflexi} cellulose responders, "OTU.64"
increased in relative abundance over 30 days and evidence for
$^{13}$C-labeling of ``OTU.64'' in the $^{13}$C-cellulose treatment peaked
days 14 and~30 (Figure~\ref{fig:example}). \textit{Bacteroidetes}
cellulose responders fell within
the \textit{Cytophagales} in contrast with \textit{Bacteroidetes} xylose
responders that belonged instead to the \textit{Flavobacteriales} or
\textit{Sphingobacteriales} (Figure~\ref{fig:tiledtree}).
\textit{Bacteroidetes} cellulose responders included one OTU that shared
100\% SSU rRNA gene sequence identity to a \textit{Sporocytophaga}
species, a genus known to include cellulose degraders. The majority (86\%)
of cellulose responders in the \textit{Proteobacteria} were closely
related ($>$ 97\% identity) to bacteria cultured in isolation, including
representatives of the genera: \textit{Cellvibrio}, \textit{Devosia},
\textit{Rhizobium}, and \textit{Sorangium}, which are all known for their
ability to degrade cellulose (Table~\ref{tab:cell}). Proteobacterial
cellulose responders belonged to \textit{Alpha} (13~OTUs), \textit{Beta}
(4~OTUs), \textit{Gamma} (5~OTUs), and \textit{Delta-proteobacteria}
(6~OTUs).

\subsection{Characteristics of cellulose and xylose responders}
% Fakesubsubsection:Cellulose responders tended
Cellulose responders, relative to xylose responders, tended to have lower
relative abundance in non-fractionated DNA, demonstrated signal consistent with
higher atom \% $^{13}$C in labeled DNA, and had lower estimated \textit{rrn}
copy number (Figure~\ref{fig:shift}). In the non-fractionated DNA, cellulose
responders had lower relative abundance (1.2 x 10$^{-3}$ (s.d. 3.8
x 10$^{-3}$)) than xylose responders (3.5 x 10$^{-3}$ (s.d. 5.2 x 10$^{-3}$))
(Figure~\ref{fig:xyl_count}, P-value~$=$~1.12 x 10$^{-5}$, Wilcoxon Rank Sum
test). Six of the ten most common OTUs observed in the non-fractionated DNA
responded to xylose, and, seven of the ten most abundant responders to xylose
or cellulose in the non-fractionated DNA were xylose responders although
``OTU.6'' annotated as \textit{Cellvibrio} a cellulose responder at day 14 was
the responder found at highest relative abundance (approximately 3\% or SSU
rRNA genes at day~14, Figure~\ref{fig:example}).

% Fakesubsubsection:DNA buoyant density increases as the amount
DNA buoyant density (BD) increases in proportion to atom \% $^{13}$C.
Hence, the extent of $^{13}$C incorporation into DNA can be evaluated by
the difference in BD between $^{13}$C-labeled and unlabeled DNA. We
calculated for each OTU its mean BD weighted by relative abundance to
determine its ``center of mass'' within a given density gradient. We then
quantified for each OTU the difference in center of mass between control
gradients and gradients from $^{13}$C-xylose or $^{13}$C-cellulose treatments
(see SI for the detailed calculation, Figure~\ref{fig:c1}). We refer to the
change in center of mass position for an OTU in response to $^{13}$C-labeling
as $\Delta\hat{BD}$. $\Delta\hat{BD}$ can be used to compare relative
differences in $^{13}$C-labeling between OTUs. $\Delta\hat{BD}$ values,
however, are not comparable to the BD changes observed for DNA from pure
cultures both because they are  based on relative abundance in density gradient
fractions (and not DNA concentration) and because isolated strains grown in
uniform conditions generate uniformly labeled molecules while OTUs composed of
heterogeneous strains in complex environmental samples do not. Cellulose
responder $\Delta\hat{BD}$ (0.0163 g mL$^{-1}$ (s.d.~0.0094)) was greater than
that of xylose responders (0.0097 g mL$^{-1}$ (s.d.~0.0094))
(Figure~\ref{fig:shift}, P-value~$=$~1.8610 x 10$^{-6}$, Wilcoxon Rank Sum
test). 

% Fakesubsubsection:We predicted the rrn
We predicted the \textit{rrn} gene copy number for responders as described
\citep{Kembel_2012}. The ability to proliferate after rapid nutrient
influx correlates positively to a microorganism's \textit{rrn} copy number
\citep{Klappenbach_2000}. Cellulose responders possessed fewer estimated
\textit{rrn} copy numbers (2.7~(1.2~s.d.)) than xylose responders
(6.2~(3.4~s.d.)) ( P~=~1.878 x 10$^{-9}$, Wilcoxon Rank Sum test,
Figure~\ref{fig:shift} and Figure~\ref{fig:copy}). Furthermore, the
estimated \textit{rrn} gene copy number for xylose responders was
inversely related to the day of first response (P~=~2.02 x 10$^{-15}$,
Wilcoxon Rank Sum test, Figure~\ref{fig:copy},Figure~\ref{fig:shift}).

% Fakesubsubsection:We assessed phylogenetic
We assessed phylogenetic clustering of $^{13}$C-responsive OTUs with the
Nearest Taxon Index (NTI) and the Net Relatedness Index (NRI)
\citep{Webb2000}. We also quantified the average clade depth of cellulose and
xylose responders with the consenTRAIT metric \citep{Martiny2013}. Briefly, the
NRI and NTI evaluate phylogenetic clustering against a null model for the
distribution of a trait in a phylogeny. The NRI and NTI values are z-scores or
standard deviations from the mean and thus the greater the magnitude of the
NRI/NTI, the stronger the evidence for clustering (positive values) or
overdispersion (negative values). NRI assesses overall clustering whereas the
NTI assesses terminal clustering \citep{Evans2014a}. The consenTRAIT metric is
a measure of the average clade depth for a trait in a phylogenetic tree. NRI
values indicate that cellulose responders clustered overall and at the tips of
the phylogeny (NRI:~4.49, NTI:~1.43) while xylose responders clustered
terminally (NRI:~-1.33, NTI:~2.69). The consenTRAIT clade depth for xylose and
cellulose responders was~0.012 and~0.028 SSU rRNA gene sequence dissimilarity,
respectively. As reference, the average clade depth is approximately~0.017 SSU
rRNA gene sequence dissimilarity for arabinase (another five C sugar found in
hemicellulose) utilization as inferred from genomic analyses, and was~0.013
and~0.034 SSU rRNA gene sequence dissimilarity for glucosidase and cellulase
genomic potential, respectively \citep{Martiny2013,Berlemont2013}. These
results indicate xylose responders form terminal clusters dispersed throughout
the phylogeny while cellulose responders form deep clades of terminally
clustered OTUs.
