\section{Results}
% Fakesubsubsection:We observed C use dynamics by the soil microbial
We observed C use dynamics in an agricultural soil microbial community by
conducting a nucleic acid SIP experiment wherein xylose or cellulose carried
the isotopic label. We set up three soil microcosm series. Each microcosm was
amended with a C substrate mixture that included cellulose and xylose. The
C substrate mixture approximated the C composition of fresh plant biomass. The
same mixture was added to all microcosms, however, for each
series except the control, xylose or cellulose was substituted for its $^{13}$C
counterpart. Microcosm amendments are identified in figures by the following
code: "13CXPS" refers to the amendment with $^{13}$C-xylose ($^{13}$\textbf{C}
\textbf{X}ylose \textbf{P}lant \textbf{S}imulant), "13CCPS" refers to the
$^{13}$C-cellulose amendment and "12CCPS" refers to the amendment that only
contained $^{12}$C (i.e. control). 5.3 mg C substrate mixture per gram of soil
was added to each microcosm representing 18\% of the soil C. The mixture
included 0.42 mg xylose-C and 0.88 mg cellulose-C per gram soil.
Microcosms were harvested at days 1, 3, 7, 14 and 30 during a 30 day
incubation. $^{13}$C-xylose assimilation peaked immediately and tapered over
the 30 day incubation whereas $^{13}$C-cellulose assimilation peaked two weeks
after amendment additions(Figure~\ref{fig:ord}). See Supplemental~Note~XX for
sequencing and density fractionation statistics.

\subsection{Soil microcosm microbial community changes with time}
% Fakesubsubsection:Changes in the soil microcosm microbial community structure
Changes in the bulk soil microcosm microbial community structure and membership
correlated significantly with incubation time (Figure~\ref{fig:bulk_ord}B,
p-value 0.23, R$^{2}$ 0.63, Adonis test \citep{Anderson2001a}). The identity of
the $^{13}$C-labeled substrate added to the microcosms did not significantly
correlate with bulk soil community structure and membership (p-value 0.35).
Additionally, microcosm beta diversity was significantly less than gradient
fraction beta diversity (p-value 0.003, "betadisper" function R Vegan package
\citep{Anderson2006,oksanen2007vegan}, Figure~XX). Twenty-nine OTUs
significantly changed in relative abundance with time ("BH" adjusted p-value
$<$0.10, \citep{YBenjamini1995}) (Figure~\ref{fig:time}). Figuure~XX summarizes
the changes in relative abundance with time at the OTU level. If sequences were
grouped by taxonomic annotations at the class level, only four classes
significantly changed in abundance, \textit{Bacilli} (decreased),
\textit{Flavobacteria} (decreased), \textit{Gammaproteobacteria} (decreased)
and \textit{Herpetosiphonales} (increased) (Figure~\ref{fig:time_class}). Of
the 29 OTUs that changed significantly in relative abundance with time, 14 are
labeled substrate responders (see~\ref{responders}, Figure~\ref{fig:time}).
   
% Fakesubsubsection:Responder abundances summed at phylum level generally increased
Responder abundances summed at phylum level generally increased for
$^{13}$C-cellulose (Figure~\ref{fig:babund}) whereas $^{13}$C-xylose
responder abundances summed at the phylum level decreased over time for
\textit{Firmicutes}, \textit{Bacteroidetes}, \textit{Actinobacteria} and
\textit{Proteobacteria} although \textit{Proteobacteria} spiked at day 14
(Figure~\ref{fig:babund}). Bulk abundance trends are roughly consistent
with $^{13}$C assimilation.

\subsection{OTUs that assimilated $^{13}$C into DNA} \label{responders}
% Fakesubsubsection:Within the first 7 days of incubation approximately 63\% 
Within the first 7 days of incubation approximately 63\% of $^{13}$C-xylose was
respired and only an additional 6\% more was respired from day 7 to 30. At day
30, 30\% of the $^{13}$C from xylose remained. An average 16\% of the
$^{13}$C-cellulose added was respired within the first 7 days, 38\% by day 14,
and 60\% by day 30.   

% Fakesubsubsection:Isotope incorporation by an OTU
We refer to OTUs that putatively incorporated $^{13}$C into DNA originally from
an isotopically labeled substrate as  substrate ``responders'' (see
Supplemental Note~XX for "response" criteria). At day 1, 84\% of
$^{13}$C-xylose responsive OTUs belonged to \textit{Firmicutes}, 11\% to
\textit{Proteobacteria} and 5\% to \textit{Bacteroidetes}. \textit{Firmicutes}
responders decreased from 16 OTUs at day 1 to one OTU at day 3 while
\textit{Bacteroidetes} responders increased from one OTU at day
1 to 12 OTUs at day 3. The remaining day 3 responders are members of the
\textit{Proteobacteria} (26\%) and the \textit{Verrucomicrobia} (5\%). Day
7 responders were 53\% \textit{Actinobacteria}, 40\% \textit{Proteobacteria},
and 7\% \textit{Firmicutes}. The identities of $^{13}$C-xylose responders
changed with time. The numerically dominant $^{13}$C-xylose responder phylum
shifted from \textit{Firmicutes} to \textit{Bacteroidetes} and then to
\textit{Actinobacteria} across days 1, 3 and 7 (Figure~\ref{fig:l2fc},
Figure~\ref{fig:xyl_count}). 

% Fakesubsubsection:Only 2 and 5 OTUs were found to
Only 2 and 5 OTUs had incorporated $^{13}$C from
$^{13}$C-cellulose at days 3 and 7, respectively. At days 14 and 30
42 and 39 OTUs incorporated $^{13}$C from $^{13}$C-cellulose into
biomass. A \textit{Cellvibrio} and \textit{Sandaracinaceae} OTU assimilated
$^{13}$C from $^{13}$C-cellulose at day 3. Day 7 $^{13}$C-cellulose responders
included the same \textit{Cellvibrio} responder as day 3,
a \textit{Verrucomicrobia} OTU and three \textit{Chloroflexi} OTUs.  50\% of
Day 14 responders belong to \textit{Proteobacteria} (66\% Alpha-, 19\% Gamma-,
and 14\% Beta-) followed by 17\% \textit{Planctomycetes}, 14\%
\textit{Verrucomicrobia}, 10\% \textit{Chloroflexi}, 7\%
\textit{Actinobacteria} and 2\% cyanobacteria. \textit{Bacteroidetes} OTUs
began to incorporate $^{13}$C from cellulose at day
30 (13\% of day 30 responders). Other day 30 responding phyla included
\textit{Proteobacteria} (30\% of day 30 responders; 42\% Alpha-, 42\% Delta,
8\% Gamma-, and 8\% Beta-), \textit{Planctomycetes} (20\%),
\textit{Verrucomicrobia} (20\%), \textit{Chloroflexi} (13\%) and
cyanobacteria (3\%). \textit{Proteobacteria}, \textit{Verrucomicrobia}, and
\textit{Chloroflexi} had relatively high numbers of responders with strong
response across multiple time points (Figure~\ref{fig:l2fc}).

\subsection{Ecological strategies of $^{13}$C responders}
% Fakesubsubsection:$^{13}$C-xylose responders are generally more abundant members based
$^{13}$C-xylose responders are generally more abundant members based on
relative abundance in bulk DNA SSU rRNA gene content than $^{13}$C-cellulose
responders (Figure~\ref{fig:shift}, p-value 0.00028).  However, both abundant
and rare OTUs responded to $^{13}$C-xylose and $^{13}$C-cellulose
(Figure~\ref{fig:shift}). Two $^{13}$C-cellulose responders were not found in
any bulk samples ("OTU.862" and "OTU.1312", Table~\ref{tab:cell}). Of the 10
most abundant responders, 8 are $^{13}$C-xylose responders and 6 of these 8 are
consistently among the 10 most abundant OTUs in bulk samples.

% Fakesubsubsection:Cellulose responders exhibited a greater shift in BD
Cellulose responders exhibited a greater shift in buoyant density (BD) than
xylose responders in response to isotope incorporation (Figure~\ref{fig:shift},
p-value 1.8610x$^{-06}$). $^{13}$C-cellulose responders shifted on average
0.0163 g mL$^{-1}$ (sd 0.0094) whereas xylose responders shifted on average
0.0097 g mL$^{-1}$ (sd 0.0094). For reference, 100\% $^{13}$C DNA BD is 0.04
g mL$^{-1}$ greater than the BD of its $^{12}$C counterpart. DNA BD increases
as its ratio of $^{13}$C to $^{12}$C increases. An organism that only
assimilates C into DNA from a $^{13}$C isotopically labeled source, will have
a greater $^{13}$C:$^{12}$C ratio in its DNA than an organism utilizing
a mixture of isotopically labeled and unlabeled C sources (see supplemental
note XX). 

% Fakesubsubsection:$^{13}$C-xylose responder estimated \textit{rrn} gene copy
$^{13}$C-xylose responder estimated \textit{rrn} gene copy number was
inversely related time of first response (p-value 2.02x10$^{-15}$,
Figure~\ref{fig:copy}). OTUs that first respond at later time points have
fewer estimated \textit{rrn} copy number than OTUs that first respond
earlier (Figure~\ref{fig:copy}). 





