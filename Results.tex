\section{Results}
To observe C use dynamics by the soil microbial community, we conducted a
nucleic acid SIP experiment wherein xylose or cellulose carried the isotopic
label and assayed SSU rRNA gene content of CsCl gradient fractions using
high-throughput DNA sequencing technology. We set up three soil microcosm series. 
Microcosms in each were were amended with a C substrate mixture
that included cellulose and xylose. The C substrate mixture approximated
freshly degrading plant biomass. The same substrate mixture was added to microcosms
in each series, however, for each series except the control, one substrate
was substituted for its $^{13}$C counterpart. In one series cellulose was
$^{13}$C-labeled in another xylose was $^{13}$C-labeled and in the control
series no sustrates were $^{13}$C labeled. Microcosm amendments are shorthand identified
in the following figures by the following code: "13CXPS" refers to the
amendment with $^{13}$C-xylose (that is $^{13}$\textbf{C} \textbf{X}lose
\textbf{P}lant \textbf{S}imulant), "13CCPS" refers to the $^{13}$C-cellulose
amendment and "12CCPS" refers to the amendment that only contained $^{12}$C
substrates. Xylose or cellulose were chosen to carry the isotopic label to
contrast C assimilation for labile, soluble C (xylose) versus
insoluble, polymeric C (cellulose).  5.3 mg of C substrate mixture per gram
soil was added to each microcosm representing 18\% of the total soil C. The
mixture included 0.42 mg xylose-C and 0.88 mg cellulose-C g soil$^{-1}$.
Microcosms were harvested at days 1, 3, 7, 14 and 30  during a 30 day
incubation. $^{13}$C-xylose assimilation peaked immediately and tapered
over the 30 day incubation whereas $^{13}$C-cellulose assimilation peaked at
two weeks of (Figure~\ref{fig:ord}).
\subsection{Ordination of CsCl gradient fraction OTU profiles can be used to
observe fraction-level $^{13}$C assimilation dynamics and membership differences}
Each CsCl gradient fraction possesses a unique composition of 16S rRNA gene
phylogenetic types. Differences in the 16S rRNA gene composition of CsCl
fractions within a single CsCl gradient is driven by buoyant density. For
instance, lighter DNA is more abundant in fractions at lighter densities so
DNA with lower G+C will be found in greater abundance at the light end of the
CsCl gradient and vice versa.  Duplicate gradients receiving only $^{12}$C DNA
with the same bulk or non-fractionated phylogenetic composition would have the
same overall profile of SSU rRNA gene phylogenetic types across the density
gradient. As we fed microcosms identical C substrate mixtures save for the
identity of a $^{13}$C labeled substrate, DNA from all microcosms harvested at
a time point will be similar in bulk phylogenetic composition. Therefore,
differences between between gradients harvested at the same time are due to
$^{13}$C incorporation into bulk community DNA. $^{13}$C-DNA shifts from its
$^{12}$C position towards the heavy end of the density gradient. This causes
heavy fractions in gradients that received $^{13}$C-DNA to be different in
phylogenetic content than corresponding heavy fractions from gradients that
received $^{12}$C-DNA of the same bulk phylogenetic composition.

Ordination of CsCl gradient fraction phylogenetic profiles reveals differences
and similarities between gradients. It's clear that microcosms incorporated
$^{13}$C from both $^{13}$C-xylose and $^{13}$C-cellulose as gradients from
both $^{13}$C-xylose and $^{13}$C-cellulose microcosms differ from
corresponding control gradients (Figure~\ref{fig:ord}). These differences from
control gradients are focused in the heavy fractions (Figure~\ref{fig:ord}).
Also, $^{13}$C-DNA from $^{13}$C-xylose microcosms is different in phylogenetic
composition from $^{13}$C-cellulose microcosm $^{13}$C-DNA indicating that
xylose and cellulose were assimilated by different microbial community members
(Figure~\ref{fig:ord}). In general, analysis of SSU rRNA gene surveys has
greatly benefited from utilizing conventional methods for data exploration
in ecology such as ordination \citep{Lozupone_2008}.  SSU rRNA gene
phylogenetic profiles in CsCl gradient fractions have only recently been
surveyed with high-throughput DNA sequencing technology and subsequently
explored via ordination \citep{Angel_2013, Verastegui_2014}. Ordination of CsCl
gradient fraction phylogenetic profiles has reveled the relative influence of
buoyant density and soil type on gradient phylogenetic profile variance.
However, ordination has not demonstrated isotope incorporation.  Demonstrating
isotope incorporation requires careful comparisons between control and labeled
gradients over the same buoyant density range. By sequencing CsCl gradient
fractions from both control and labeled gradients across the full density
gradient with DNA harvested from microcosms at multiple time points, we can
observe where in the density gradient heavy isotope incorporation signal is
strongest and when heavy isotope incorporation begins (Figure~\ref{fig:ord}).
$^{13}$C incorporation from xylose and cellulose is most apparent at days 1/3/7
and days 14/30, respectively (Figure~\ref{fig:ord}). Moreover, labeled gradient
fraction phylogenetic profiles diverge from controls most dramatically at
relatively heavy buoyant densities (Figure~\ref{fig:ord}). Also apparent from
the ordination of CsCl gradient phylogenetic profiles is that OTUs responsive
to $^{13}$C-cellulose are generally different taxa than those responsive to
$^{13}$C-xylose and last, that $^{13}$C-xylose responders change in
phylogenetic type over incubation days 1, 3 and 7 (Figure~\ref{fig:ord}).

\subsection{$^{13}$C from cellulose was assimilated by canonical
cellulose-degrading and uncharacterized microbial lineages in many phyla
including Chloroflexi and Verrucomicrobia} 
Isotope incorporation by an OTU is revealed by enrichment of the OTU in heavy
CsCl gradient fractions containing $^{13}$C labeled DNA relative to heavy
fractions from control gradients containing no $^{13}$C labeled DNA. We refer
to OTUs that putatively incorporated $^{13}$C into DNA from an isotopically
labeled substrate as a substrate ``responder''. Only 2 and 5 OTUs were found to
have incorporated $^{13}$C from $^{13}$C-cellulose at days 3 and 7,
respectively. At days 14 and 30, however, 42 and 39 OTUs were found to
incorporate $^{13}$C from $^{13}$C-cellulose into biomass. An average 16\% of
the $^{13}$C-cellulose added was respired within the first 7 days, 38\% by day
14, and 60\% by day 30.  A \textit{Cellvibrio} and \textit{Sandaracinaceae} OTU
assimilated $^{13}$C from $^{13}$C-cellulose at day 3. Day 7 $^{13}$C-cellulose
responders included the same \textit{Cellvibrio} responder as day 3, a
\textit{Verrucomicrobia} OTU and three \textit{Chloroflexi} OTUs.  50\% of Day
14 responders belong to \textit{Proteobacteria} (66\% Alpha-, 19\% Gamma-, and
14\% Beta-) followed by 17\% \textit{Planctomycetes}, 14\%
\textit{Verrucomicrobia}, 10\% \textit{Chloroflexi}, 7\%
\textit{Actinobacteria} and 2\% cyanobacteria.  \textit{Bacteroidetes} OTUs
begin to incoporate $^{13}$C from cellulose at day 30 (13\% of day 30
responders). Other day 30 responding phyla include \textit{Proteobacteria}
(30\% of day 30 responders; 42\% Alpha-, 42\% Delta, 8\% Gamma-, and 8\%
Beta-), \textit{Planctomycetes} (20\%), \textit{Verrucomicrobia} (20\%),
\textit{Chloroflexi} (13\%) and cyanobacteria (3\%). \textit{Proteobacteria},
\textit{Verrucomicrobia}, and \textit{Chloroflexi} had relatively high numbers
of responders with heavy response across multiple time points
(Figure~\ref{fig:l2fc}).

\textit{Proteobacteria} represent 46\% of all $^{13}$C-cellulose responding
OTUs identified. \textit{Cellvibrio} accounted for 3\% of all proteobacterial
$^{13}$C-cellulose responding OTUs detected. \textit{Cellvibrio} was one of the
first identified cellulose degrading bacteria and was originally described by
Winogradsky in 1929 who named it for its cellulose degrading abilities
\citep{boone2001bergeys}. All $^{13}$C-cellulose responding
\textit{Proteobacteria} share high sequence identity with 16S rRNA genes from
sequenced cultured isolates (Table~\ref{tab:cell}) except for ``OTU.442'' (best
cultured isolate match 92\% sequence identity in the \textit{Chrondomyces}
genus, Table~\ref{tab:cell}) and ``OTU.663'' (best cultured isolate match
outside \textit{Proteobacteria} entirely, \textit{Clostridium} genus, 89\%
sequence identity, Table~\ref{tab:cell}). Some \textit{Proteobacteria}
responders share high sequence identity with type strains for genera known to
possess cellulose degraders including \textit{Rhizobium}, \textit{Devosia},
\textit{Stenotrophomonas} and \textit{Cellvibrio}. One \textit{Proteobacteria}
OTU shares high sequence identity with a \textit{Brevundimonas} cultured
isolate.  \textit{Brevundimonas} has not previously been identified as a
cellulose degrader, but has been shown to degrade cellouronic acid, an oxidized
form of cellulose \citep{Tavernier_2008}.

\textit{Verrucomicrobia}, a cosmopolitan soil phylum often found in high
abundance \citep{Fierer_2013}, are hypthothesized to degrade polysaccharides in
many environments \citep{Fierer_2013,Herlemann_2013,10543821}.
\textit{Verrucomicrobia} comprise 16\% of the total $^{13}$C-cellulose
responder OTUs detected. 40\% of \textit{Verrucomicrobia} $^{13}$C-cellulose
responders belong to the uncultured ``FukuN18'' family originally identified in
freshwater lakes \citep{Parveen_2013}.  The \textit{Verrucomicrobia} OTU with
the strongest \textit{Verrucomicrobial} response to $^{13}$C-cellulose shared
high sequence identity (97\%) with an isolate from Norway tundra soil
\citep{Jiang_2011} although growth on cellulose was not assessed for this
isolate. Only one other $^{13}$C-cellulose responding verrucomicrobium shared
high DNA sequence identity with a sequenced type strain, ``OTU.638''
(Table~\ref{tab:cell}) with \textit{Roseimicrobium gellanilyticum} (100\%
sequence identity).  \textit{Roseimicrobium gellanilyticum} grows on soluble
cellulose \citep{Otsuka_2012}. The remaining $^{13}$C-cellulose
\textit{Verrucomicrobia} responders did not share high sequence identity with
any cultured isolates (maximum sequence identity with any cultured isolate
93\%). 

\textit{Chloroflexi} are traditionally known for metabolically dynamic
lifestyles ranging from anoxygenic phototrophy to organohalide respiration
\citep{Hug_2013}. Recent studies have focused on \textit{Chloroflexi} roles in
C cycling \citep{Hug_2013, Goldfarb_2011,Cole_2013} and several
\textit{Chloroflexi} utilize cellulose \citep{Goldfarb_2011, Cole_2013,
Hug_2013}. Four closely related OTUs in an undescribed \textit{Chloroflexi}
lineage (closest matching cultured isolate for all four OTUs:
\textit{Herpetosiphon geysericola}, 89\% sequence identity,
Table~\ref{tab:cell}) responded to $^{13}$C-cellulose (Figure~\ref{fig:trees}).
One additional OTU also from a poorly characterized \textit{Chloroflexi}
lineage (closest cultured isolate match a proteobacterium at 78\% sequence
identity) responded to $^{13}$C-cellulose (Figure~\ref{fig:trees}).

Other notable $^{13}$C-cellulose responders include a \textit{Bacteroidetes}
OTU that shares high sequence identity (99\%) to \textit{Sporocytophaga
myxococcoides} a known cellulose degrader \citep{Vance_1980}, and three
\textit{Actinobacteria} OTUs that share high sequence identity (100\%) with
sequenced cultured isolates. One of the three \textit{Actinobacteria}
$^{13}$C-cellulose responders is in the \textit{Streptomyces}, a genus known to
possess cellulose degraders, while the other two share high sequence identity
to cultured isolates \textit{Allokutzneriz albata} \citep{Labeda_2008,
Tomita_1993} and \textit{Lentzea waywayandensis} \citep{LABEDA_1989,
Labeda_2001}; neither isolate decomposes cellulose in culture. Nine
\textit{Plantomycetes} OTUs responded to $^{13}$C-cellulose but none are within
described genera (closest cultured isolate match 91\% sequence identity,
Table~\ref{tab:cell}) (Figure~\ref{fig:trees}). Interestingly, one
$^{13}$C-cellulose responder is annotated as ``cyanobacteria''.
The cyanobacteria phylum annotation is misleading as the OTU is not closely
related to any oxygenic phototrophs (closest cultured isolate match
\textit{Vampirovibrio chlorellavorus}, 95\% sequence identity,
Table~\ref{tab:cell}). A sister clade to the oxygenic phototrophs classically
annotated as ``cyanobacteria'' in SSU rRNA gene reference databases but does
not possess any known phototrophs has recently been proposed to constitute its own
phylum, "Melainabacteria" \citet{Di_Rienzi_2013}. Although the phylogenetic
position of ``Melainabacteria'' is debated \citep{Soo_2014}. The catalog of
metabolic capabilities associated with cyanobacteria (or candidate phyla
previously annotated as cyanobacteria) are quickly expanding
\citep{Di_Rienzi_2013, Soo_2014}.  Our findings provide evidence for cellulose
degradation within a lineage closely related to but apart from oxygenic
phototrophs. Notably, polysaccharide degradation is suggested by an analysis of
a ``Melainabacteria'' genome \citep{Di_Rienzi_2013}. Although we highlight
$^{13}$C-cellulose responders that share high sequence identity with described
genera, most $^{13}$C-cellulose responders uncovered in this experiment are not
closely related to cultured isolates (Table~\ref{tab:cell}).

\subsection{Putative spore-formers in the Firmicutes assimilate $^{13}$C from
xylose within first day after soil amendment followed by Bacteroidetes and then
Actinobacteria OTUs} 
Within the first 7 days of incubation 63\% on average of $^{13}$C-xylose was
respired and only an additional 6\% more was respired from day 7 to 30. At
the end of the 30 day incubation 30\% of the $^{13}$C from added xylose
remained in the soils. The $^{13}$C remaining in the soil from $^{13}$C-xylose
addition was likely stabilized by assimilation into microbial biomass
and/or microbial conversion into other forms of organic matter. It is also
possible that some $^{13}$C-xylose remains unavailable to microbes due to
abiotic interactions in soil \citep{Kalbitz_2000}. 

At day 1, 84\% of $^{13}$C-xylose responsive OTUs belong to
\textit{Firmicutes}, 11\% to \textit{Proteobacteria} and 5\% to
\textit{Bacteroidetes}. At day 3, \textit{Firmicutes} responders decreased to
5\% (from 16 OTUs to 1) while \textit{Bacteroidetes} increased to 63\% (from 1
to 12 OTUs) of day 3 responders. The remaining day 3 responders are members of
the \textit{Proteobacteria} (26\%) and the \textit{Verrucomicrobia} (5\%). Day
7 responders were 53\% \textit{Actinobacteria}, 40\% \textit{Proteobacteria},
and 7\% \textit{Firmicutes}. The identities of $^{13}$C-xylose responders
change with time. The numerically dominant $^{13}$C-xylose responder phylum
shifts from \textit{Firmicutes} to \textit{Bacteroidetes} and then to
\textit{Actinobacteria} across days 1, 3 and 7 (Figure~\ref{fig:l2fc},
Figure~\ref{fig:xyl_count}). 

All of the $^{13}$C-xylose responders in the \textit{Firmicutes} phylum are
closely related (at least 99\% sequence identity) to cultured isolates from
genera that are known to form endospores (Table~\ref{tab:xyl}). Each
$^{13}$C-xylose responder is closely related to isolates annotated as members
of \textit{Bacillus}, \textit{Paenibacillus} or \textit{Lysinibacillus}.
\textit{Bacteroidetes} $^{13}$C-xylose responders are predominantly closely
related to \textit{Flavobacterium} species (5 of 8 total responders)
(Table~\ref{tab:xyl}.  Only one \textit{Bacteroidetes} $^{13}$C-xylose
responder is not closely related to a cultured isolate, ``OTU.183'' (closest
LTP BLAST hit, \textit{Chitinophaca sp.}, 89.5\% sequence identity,
Table~\ref{tab:xyl}). OTU.183 shares high sequence identity with environmental
clones derived from rhizosphere samples (accession AM158371, unpublished) and
the skin microbiome (accession JF219881, \citet{Kong_2012}). Other
\textit{Bacteroidetes} responders share high sequence identities with canonical
soil genera including \textit{Dyadobacer}, \textit{Solibius} and
\textit{Terrimonas}. Six of the 8 \textit{Actinobacteria} $^{13}$C-xylose
responders are in the \textit{Micrococcales} order. One $^{13}$C-xylose
responding \textit{Actinobacteria} OTU shares 100\% sequence identity with
\textit{Agromyces ramosus} (Table~\ref{tab:xyl}).  \textit{A. ramosus} is a
known predatory bacterium but is not dependent on a host for growth in culture
\citep{16346402}. It is not possible to determine the specific origin of
assimilated $^{13}$C in a DNA-SIP experiment. $^{13}$C can be passed down
through trophic levels although heavy isotope representation in C pools
targeted by cross-feeders and predators would be diluted with depth into the
trophic cascade.  It's possible, however, that the $^{13}$C labeled
\textit{Agromyces} OTU was assimilating $^{13}$C primarily by predation if the
\textit{Agromyces} OTU was selective enough with respect to its prey that it
primarily attacked $^{13}$C-xylose assimilating organisms and that those
$^{13}$C-xylose assimilating organisms utilized $^{13}$C-xylose as a sole
carbon source.

\subsection{Cellulose degrader DNA shifts further along the BD gradient upon
$^{13}$C incorporation than xylose degrader DNA} 
Cellulose responders exhibited a greater shift in BD than xylose responders in
response to isotope incorporation (Figure~\ref{fig:shift}, p-value
1.86e$^{-06}$). $^{13}$C-cellulose responders shifted on average 0.0163 g/mL
(sd 0.0094) whereas xylose responders shifted on average 0.0097 (sd
0.0094). For reference, 100\% $^{13}$C DNA shifts X.XX g/mL
relative to the BD of its $^{12}$C counterpart. DNA BD increases
as its ratio of $^{13}$C to $^{12}$C increases. An organism that only
assimilates C into DNA from a $^{13}$C isotopically labeled source, will have a
greater $^{13}$C:$^{12}$C ratio in its DNA than an organism utilizing a mixture
of isotopically labeled and unlabeled C sources. Upon labeling, DNA from an
organism that incorporates exclusively $^{13}$C will increase in buoyant density
more than DNA from an organism that does not exclusively utilize
isotopically labeled C. Therefore the magnitude DNA buoyant density shifts
indicate substrate specificity given our experimental design as only one
substrate was labeled in each amendment. We measured density shift
as the change in an OTU's density profile center of mass between corresponding
contol and labeled gradients. Density shifts, however, should not be evaluated
on an individual OTU basis as a small number of density shifts are observed for
each OTU and the variance of the density shift metric at the level of
individual OTUs is unknown. It is therefore more informative to compare density
shifts among substrate responder groups. Further, density shifts are based on
relative abundance profiles and would be theoretically muted in comparison to
density shifts based on absolute abundance profiles and should be interpreted
with this transformation in mind. It should also be noted that there was
overlap in observed density shifts between $^{13}$C-cellulose and
$^{13}$C-xylose responder groups suggesting that although cellulose degraders
are generally more substrate specific than xylose utilizers, some cellulose
degraders show less substrate specificity for cellulose than some xylose
utilizers for xylose (Figure~\ref{fig:shift}), and, each responder group
exhibits a range of substrate specificites (Figure~\ref{fig:shift}).

\subsection{Xylose responders at day 1 have more estimated rRNA operon copy
numbers per genome than xylose responders at days 3 and 7, and, Xylose
responders have more rRNA operon copy numbers
than cellulose responders.}
$^{13}$C-xylose responder rRNA operon genome copy number is inversely related
to time of first response (p-value 2.02e$^{-15}$, Figure~\ref{fig:copy}). OTUs
that first respond at later time points have fewer estimated rRNA operons per
genome than OTUs that first respond earlier (Figure~\ref{fig:copy}). rRNA
operon copy number estimation is a recent advance in microbiome science
\citep{Kembel_2012} while the relationship of rRNA operon copy number per genome
with ecological strategy is well established \citep{Klappenbach_2000}.
Microorganisms with a high number of rRNA operons per genome tend
to be fast growers specialized to take advantage of boom-bust environments
whereas microorganisms with low rRNA operon copy numbers per genome favor
slower growth under lower and more consistent nutrient input
\citep{Klappenbach_2000}. At the beginning of our incubation, OTUs with
estimated high rRNA operon copy numbers per genome or ``fast-growers''
assimilate xylose into biomass and with time slower growers (lower rRNA operon
number per genome) begin to respond to the xylose addition.  Further,
$^{13}$C-xylose responders have more estimated rRNA operon copy numbers per
genome than $^{13}$C-cellulose responders (p-value 1.878e$^{-09}$) suggesting
xylose respiring microbes are generally faster growers than cellulose
degraders.

\subsection{Xylose responders are more abundant in the soil community than cellulose
responders}
$^{13}$C-xylose responders are generally more abundant members based on
relative abundance in bulk DNA SSU rRNA gene content than $^{13}$C-cellulose
responders (Figure~\ref{fig:shift}, p-value 0.00028).  However, both abundant and
rare OTUs responded to $^{13}$C-xylose and $^{13}$C-cellulose
(Figure~\ref{fig:shift}). For instance, a \textit{Delftia} $^{13}$C-cellulose
responder is fairly abundant in the bulk samples (``OTU.5'',
Table~\ref{tab:cell}). OTU.5 was on average the
13th most abundant OTU in bulk samples. A $^{13}$C-xylose responder (``OTU.1040'',
Table~\ref{tab:xyl}) has a mean relative abundance in bulk samples of
3.57e$^{-05}$. Two $^{13}$C-cellulose responders wer not found
in any bulk samples ("OTU.862" and "OTU.1312", Table~\ref{tab:cell}). Of the 10 most abundant
responders 8 are $^{13}$C-xylose responders and 6 of these 8 are consistently
among the 10 most abundant OTUs in bulk samples.
\subsection{Variation in bulk soil DNA microbial community structure is
significantly less than variation in gradient fractions} 
Using a distance metric that incorporates relative abundance information
(weighted Unifrac metric, \citep{Lozupone_2005}) bulk sample beta diversity was
less than gradient fraction beta diversity (p-value 0.003). Time was
significantly correlated to bulk sample phylogenetic profile variation (p-value
0.23, R$^{2}$ 0.63, Figure~\ref{fig:bulk_ord}) but the contrast between only
$^{12}$C additions with additions that included isotopically labeled substrates
was not (p-value 0.35). When responder abundances were summed within phyla for
$^{13}$C-cellulose and $^{13}$C-xylose, summed resonder abundaces generally 
increased for $^{13}$C-cellulose and \textit{Firmicutes}, \textit{Bacteroidetes},
\textit{Actinobacteria} and \textit{Proteobacteria}, the numerically dominant
$^{13}$C-xylose responder phyla, decreased over time although 
\textit{Proteobacteria} spiked at day 14 (Figure~\ref{fig:babund}). Bulk abundance trends are
roughly consistent with $^{13}$C assimilation activity. 
