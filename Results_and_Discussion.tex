\section{Results and Discussion}
In this study, we couple nucleic acid SIP with next generation sequencing (SIP-NGS) to observe C use dynamics by the soil microbial community. A series of parallel soil microcosms all amended with a C substrate mixture were incubated for 30 days. The substrate mixture was identical for each bottle except in one series of bottles the cellulose was $^{13}$C-labeled in another the xylose was $^{13}$C-labeld and in the last no sustrattes were labeled. The C substrate mixture was designed to approximate freshly degrading plant biomass. Xylose or cellulose carried the isotopic label so we could examine C assimilation dynamics for labile, soluble C versus insoluble, polymeric C. 5.3 mg total mass of C substrate mixture per gram soil (including 0.42 mg xylose-C and 0.88 mg cellulose-C g soil$^{-1}$) was added to each microcosm representing 18\% of the total soil C. Microcosms were harvested at several time points during the incubation period and $^{13}$C assimilation was observed by sequencing 16S rRNA gene amplicons from bulk soil DNA and CsCl gradient fractions (\href{https://www.authorea.com/users/3537/articles/8459/master/file/figures/20140708_ConceptualFig2/20140708_ConceptualFig2.pdf}{Fig. S1}). Xylose degradation was observed immediately, while cellulose degradation was observed after two weeks.

\subsection{$^{13}$C from cellulose assimilated by canonical cellulose-degrading and uncharacterized microbial lineages in many phyla including \textit{Chloroflexi} and \textit{Verrucomicrobia}}
Only 2 and 5 OTUs were found to have incorporated $^{13}$C from labeled cellulose at days 3 and 7, respectively. At days 14 and 30, however, 42 and 39 OTUs were found to incorporate $^{13}$C from cellulose into biomass. An average 16\% of the $^{13}$C-cellulose added was respired within the first 7 days, 38\% by day 14, and 60\% by day 30. A \textit{Cellvibrio} and \textit{Sandaracinaceae} OTU assimilated $^{13}$C from cellulose at day 3. Day 7 responders included the same \textit{Cellvirio} responder at day 3, a \texit{Verrucomicrobia} OTU and three \textit{Chloroflexi} OTUs. 50\% of Day 14 responders belong to Proteobacteria (66\% Alpha-, 19\% Gamma-, and 14\% Beta-) followed by 17\% \textit{Planctomycetes}, 14\% \textit{Verrucomicrobia}, 10\% \textit{Chloroflexi}, 7\% \textit{Actinobacteria} and 2\% \textit{Cyanobacteria}. \textit{Bacteroidetes} OTUs begin to incoporate $^{13}$C from cellulose at dat 30 (13\% of day 30 responders). Other day 30 responding phyla include \textit{Proteobacteria} (30\% of day 30 responders; 42\% Alpha-, 42\% Delta, 8\% Gamma-, and 8\% Beta-), \textit{Planctomycetes} (20\%), \textit{Verrucomicrobia} (20\%), \textit{Chloroflexi} (13\%) and \textit{Cyanobacteria} (3\%). \textit{Proteobacteria}, \textit{Verrucomicrobia}, and \textit{Chloroflexi} had relatively high numbers of responders with heavy response across multiple time points (ref l2fc figure).

\textit{Proteobacteria} represent 46\% of all cellulose responding OTUs identified. \textit{Cellvibrio} accounted for 3\% of all Proteobacterial responding OTUs detected (\href{https://www.authorea.com/users/3537/articles/3612/master/file/figures/l2fc_fig1/l2fc_fig.pdf}{Figs. 2}, \href{https://authorea.com/users/3537/articles/8459/master/file/figures/l2fc_fig_pVal/l2fc_fig_pVal.png}{S4}, \href{https://authorea.com/users/3537/articles/8459/master/file/figures/cellulose_resp_profiles/cellulose_resp_profiles.png}{S6}). \textit{Cellvibrio} was one of the first identified cellulose degrading bacteria and was originally described by Winogradsky in 1929 who named it for its cellulose degrading abilities \cite{boone2001bergeys}. All $^{13}$C-cellulose responding \textit{Proteobacteria} share high sequence identity with 16S rRNA genes from sequenced type straints (Table XX) except for OTU.442 (best type strain match 92\% sequence identity in the \textit{Chrondomyces} genus) and OTU.663 (best type strain match outside \textit{Proteobacteria} entirely, \textit{Clostridium} genus, 89\% sequence identity). Some \textit{Proteobacteria} responders share high sequence identity with type strains for genera known to posess cellulose degraders including \textit{Rhizobium}, \textit{Devosia}, \textit{Stenotrophomonas} and \textit{Cellvibrio}. One \textit{Proteobacteria} OTU shares high sequence identity with the \textit{Brevundimonas} type strain. \textit{Brevundimonas} has not previously been identified as a cellulose degrader, but has been shown to degrade cellouronic acid, an oxidized form of cellulose \cite{Tavernier_2008}.

Verrucomicrobia, a phylum found to be ubiquitous and in high abundance in soil \cite{Fierer_2013}, have been noted for degradation of polysaccharides in soil, aquatic, and anoxic rice patty soils \cite{Fierer_2013,Herlemann_2013,10543821}. In this study, Verrucomicrobia comprise ~11\% of the total cellulose responder OTUs detected (\href{https://www.authorea.com/users/3537/articles/3612/master/file/figures/l2fc_fig1/l2fc_fig.pdf}{Figs. 2}, \href{https://authorea.com/users/3537/articles/8459/master/file/figures/l2fc_fig_pVal/l2fc_fig_pVal.png}{S4}, \href{https://authorea.com/users/3537/articles/8459/master/file/figures/cellulose_resp_profiles/cellulose_resp_profiles.png}{S6}) most of which belong to the uncultured FukuN18 clade originally identified in freshwater lakes \cite{Parveen_2013}. Yet the largest enrichment measured (l2fc = 3.7) during the whole time series for \textsuperscript{13}C-cellulose assimilation was by an uncultured Verrucomicrobia in the Verrucomicrobiaceae family on d14 (\href{https://authorea.com/users/3537/articles/3612/master/file/figures/bacteria_tree/bacteria_tree.png}{Fig. 4}). 

Chloroflexi, ubitiquous across many diverse environments, are traditionally known for their metabolically dynamic lifestyles ranging from anoxygenic phototropy to organohalide respiration \cite{Hug_2013}. Yet, only recently has focus shifted towards the metabolic functions of Chloroflexi in C cycling \cite{Hug_2013,Goldfarb_2011,Cole_2013}. In this study, we identified a previously undescribed clade within the Chloroflexi class (closest relative at a 96\% identity being Herpetosiphon) that exhibited a high substrate specificity based on an average BD shift 0.019 $\pm$ 0.002 g mL\textsuperscript{-1} (\href{https://authorea.com/users/3537/articles/8459/master/file/figures/cellulose_resp_profiles/cellulose_resp_profiles.png}{Fig. S6}). We observed no other cellulose utilization in Chloroflexi outside of this clade although many members of this phylum have previously been demonstrated or implicated in cellulose utilization \cite{Goldfarb_2011,Cole_2013,Hug_2013}. 
One of the most interesting findings is a single Cyanobacterial responder OTU, which exhibit the third greatest enrichment measured (l2fc = 3.35) in response to \textsuperscript{13}C-cellulose assimilation. This OTU falls into the recently described candidate phylum Melainabacteria \cite{Di_Rienzi_2013}, although its phylogenetic position is debated \cite{Soo_2014}. More importantly, the catalog of metabolic capabilities associated with Cyanobacteria are quickly expanding \cite{Di_Rienzi_2013, Soo_2014}. Our findings provide evidence of cellulose degradation for Melainabacteria, supporting hypothesized polysaccharide degradation suggested by genomic analysis \cite{Di_Rienzi_2013}.     



\subsection{Putative spore-formers in the \textit{Firmicutes} assimilate $^{13}$C from xylose within first day after soil amendment}
Within the first 7 days of incubation an average 63\% of \textsuperscript{13}C-xylose was respired and only an additional 6\% more was respired between days 7 and 30 (\href{https://authorea.com/users/3537/articles/8459/master/file/figures/Percent_respired_13C/Percent_respired_13C.png}{Table S1}). At the end of the 30 day experiment 30\% of the original \textsuperscript{13}C from xylose remained in the soils. The \textsuperscript{13}C remaining in the soil from \textsuperscript{13}C-xylose addition has likely been stabilized by assimilation into microbial biomass and/or microbial conversion into other forms of organic matter, though it is possible that some \textsuperscript{13}C-xylose remains unavailable to microbes due to abiotic interactions in soil \cite{Kalbitz_2000}. Of the 60 total xylose responders 53 were responsive within the first 7 days and only 7 responders detected for days 14 and 30 (\href{https://authorea.com/users/3537/articles/8459/master/file/figures/resp_table/resp_table.png}{Table S1}). 

At day 1 (d1), 57\% of xylose responsive OTUs belong to Firmicutes (Paenibacillaceae, Planococcaceae, and Bacillaceae) and the remaining 43\% of responders were comprised of 19\% Bacteroidetes (Flavobacteriaceae), 14\% Proteobacteria (Enterobacteriaceae, Comamonadaceae, and uncultured Gammaproteobacteria), and 10\% Actinobacteria (Micrococcaceae and Microbacteriaceae) (\href{https://www.authorea.com/users/3537/articles/3612/master/file/figures/l2fc_fig1/l2fc_fig.pdf}{Fig. 2}). Of the xylose responders detected at day 3 (d3), Firmicutes responders decreased to 8\% (from 12 OTUs to 1) and an increase in Bacteroidetes (67\%) and Proteobacteria (25\%) from d1. Cellvibrio, a canonical cellulose degrader, was notably one of the three Proteobacterial xylose responders at d3 (discussed later). Day 7 (d7) responders were 50\% Actinobacteria, 35\% Proteobacteria, and 5\% of each Bacteroidetes, Firmicutes, and Planctomycetes. A substantial amount (75\%) of xylose responders for day 7 had not previously been identified as responders at earlier time points which attests to the wide number of taxa able to use xylose. Each of the 10 Actinobacteria responders at day 7 belonged to a different family making it the phylum with the most wide-spread use of xylose. However, it should be noted that they were confined within two Actinobacterial Orders; Frankiales and Micrococcales. 

We observe dynamic changes in \textsuperscript{13}C-xylose assimilation with time; dominant xylose responders shift from Firmicutes (d1) to Bacteroidetes (d3) then finally Actinobacteria (d7). At any given time soils harbor microorganisms at varying degrees of dormancy depending on nutrient availability \cite{Jones_2010}. The sudden addition of our complex C mixture would most certainly prompt dormant and non-dormant microbes back into metabolic activity, with those exhibiting higher rRNA operon copy numbers responding the fastest. The responders identified at d1 for xylose utilization have all been noted for exhibiting some form of dormancy strategy \cite{Jones_2010, Mulyukin_2009, Darcy_2011, Sachidanandham_2008, Finkel_2006, Rittershaus_2013, Tada_2013, Lay_2013}, though the only spore-forming responders at d1 are Firmicutes. Additionally, d1 responders exhibit 6-14 rRNA operon copies with the exception of the Betaproteobacteria Comamonadaceae and the Actinobacterial OTUs which exhibit 1-2 copies according to representative taxa in the rrnDB v. 3.1.227 \cite{18948294,11125085}. With the exception of the single Firmicutes (Paenibacillaceae), responders on d3 possess 3-6 rRNA operon copies, less than responders on d1. Similar to d3, 85\% of d7 responders exhibit between 1-5 rRNA operon copies with the remaining 15\% (Flavobacteriaceae, Enterobacteriaceae, and Paenibacillaceae) containing between 6-14 rRNA operon copies. Paenibacillus (100\% identity) was the only OTU to be identified at every time point (up to day 30) as a xylose responder. This result suggests that large numbers of cells from Paenibacillus sporulated after \textsuperscript{13}C-labeling of their DNA and that these spores remained throughout the experiment. It was the the second most enriched xylose responder, log\text$_{2}$fold change (l2fc) of 3.5, measured in the time series second only to a Gammaproteobacteria (Xanthomonadaceae; l2fc = 3.7).

\subsection{Cellulose degrader DNA exhibits greater bouyant density shifts upon $^{13}$C incorporation from $^{13}$C-cellulose than xylose degraders} 

\textbf{Temporal dynamics of C-assimilation in soil.}  
The dynamics of $^{13}$C-xylose and $^{13}$C-cellulose assimilation varied dramatically. Isotope incorporation increases the bouyant density (BD) of DNA and labeled DNA is enriched in 'heavy' fractions of the density gradient. Isotope incorporation by an OTU is revealed by enrichment of the OTU in heavy CsCl gradient fractions containing $^{13}$C labeled DNA relative to heavy fractions from control gradients (no $^{13}$C labeled DNA). Variation in 16S rRNA gene amplicon pool composition in fractions of $^{13}$C-labeled samples and their corresponding controls is readily observed in 'heavy' gradient fractions (\href{https://www.authorea.com/users/3537/articles/3612/master/file/figures/ordination_all1/ordination_all1.png}{partitioning along axis 2, Fig. 1}). The amplicon pool composition of 'heavy' fractions of $^{13}$C-xylose and $^{13}$C-cellulose samples vary from corresponding controls and from each other, indicating that the substrates were assimilated by different members of the  microbial community(\href{https://www.authorea.com/users/3537/articles/3612/master/file/figures/ordination_all1/ordination_all1.png}{Fig. 1A}).

The \textsuperscript{13}C-incorporation reveals temporal dynamics of C degradation demonstrated by \textsuperscript{13}C-xylose incorporation at days 1, 3, and 7 and \textsuperscript{13}C-cellulose incorporation at days 14 and 30 (\href{https://www.authorea.com/users/3537/articles/3612/master/file/figures/ordination_all1/ordination_all1.png}{Fig. 1B}), as expected \cite{Amelung_2008}. The microbial community changed significantly (pval) with time in the bulk community supporting the temporal dynamics observed in the gradient fraction amplicons (\href{https://authorea.com/users/3537/articles/8459/master/file/figures/bulk_ordination/bulk_ordination.png}{Fig. S2}). Although within a single time point, the bulk community demonstrated no significant difference between treatments. 

'Heavy' fraction amplicon pools from samples that received \textsuperscript{13}C-xylose diverged from corresponding controls on days 1 through 7 (\href{https://www.authorea.com/users/3537/articles/3612/master/file/figures/ordination_all1/ordination_all1.png}{Fig. 1}). Furthermore, amplicon pool composition varied across these days (\href{https://www.authorea.com/users/3537/articles/3612/master/file/figures/ordination_all1/ordination_all1.png}{Fig. 1B}) indicating dynamic changes in \textsuperscript{13}C-xylose assimilation with time. At days 14 and 30 heavy fractions from \textsuperscript{13}C-xylose labeled samples are no longer differentiated from corresponding controls indicating that \textsuperscript{13}C is no longer detectable in DNA. The decline in \textsuperscript{13}C-labelling of DNA is likely due to isotopic dilution resulting from assimilation of unlabeled C and/or due to cell turnover resulting from mortality. 

\textsuperscript{13}C-cellulose incorporation isn't detected until day 14 and amplicon composition is consistent for both days 14 and 30 (\href{https://www.authorea.com/users/3537/articles/3612/master/file/figures/ordination_all1/ordination_all1.png}{Fig. 1}). The consistency of amplicon composition for cellulose degradation over time compared to xylose suggests a wider array of microorganisms utilize xylose, whereas, cellulose utilization occurs in a select few. This is consistent with long standing notions that more microorganisms are capable of utilizing simple carbohydrates than complex C substrates. 



\textbf{Differential C utilization by taxa.} Individual OTUs that assimilated \textsuperscript{13}C-substrates were identified using the DESeq framework \cite{Anders_Huber_2010} to analyze differential representation in 'heavy' fractions (\href{https://www.authorea.com/users/3537/articles/3612/master/file/figures/l2fc_fig1/l2fc_fig.pdf}{Fig. 2}). There were 43 and 35 unique OTUs that significantly (false discovery rate corrected \textit{P}-values \textless 0.10, \href{https://authorea.com/users/3537/articles/8459/_show_article}{SI}) assimilated \textsuperscript{13}C-xylose and \textsuperscript{13}C-cellulose, respectively; herein called 'responders' (\href{https://www.authorea.com/users/3537/articles/8459/master/file/figures/OTU_screening_schematic/OTU_screening_schematic.pdf}{Figs. S3}, \href{https://www.authorea.com/users/3537/articles/8459/master/file/figures/l2fc_fig_pVal/l2fc_fig_pVal.png}{S4},\href{https://authorea.com/users/3537/articles/8459/master/file/figures/manhattan/manhattan.png}{S5}).






There were 6 shared responders among all unique responders identified in both the xylose and cellulose treatments (n = 72); Stenotrophomonas, Planctomyces, two Rhizobiaceae, Comamonadaceae, and Cellvibrio. Of these, Stenotrophomonas and Comamonadaceae are the only taxa that are among the top ten l2fc responses measured in both treatments. On the other hand, the only shared responder that is not among the top ten responders for either the cellulose or xylose treatment is Rhizobiaceae. Two of the shared responders corresponded in time between the two treatments (\href{https://authorea.com/users/3537/articles/8459/master/file/figures/resp_table/resp_table.png}{Table S1}); Cellvibrio (d3) and Planctomyces (d14).  

\textit{Responder Characteristics}.  We found xylose responders were from higher rank abundances than cellulose responders, however, cellulose responders exhibited a greater shift in BD (i.e. assimilated more \textsuperscript{13}C) than xylose responders in response to isotope incorporation (\href{https://authorea.com/users/3537/articles/3612/master/file/figures/shift_and_rabund2/shift_and_rabund2.png}{Fig. 3}). 

The kernel density estimate (KDE) of BD shifts resulting from \textsuperscript{13}C-assimilation reveal that cellulose responders exhibit a significantly (\textit{p}\textless 0.01) greater BD shift than xylose responders (\href{https://authorea.com/users/3537/articles/3612/master/file/figures/shift_and_rabund2/shift_and_rabund2.png}{Fig. 3A}). A density profile for each responder is generated for the experimental and control treatment at each of the sampling time points using relative abundances from sequence libraries (\href{https://authorea.com/users/3537/articles/8459/master/file/figures/xylose_resp_profiles/xylose_resp_profiles.png}{Figs. S5}\href{https://authorea.com/users/3537/articles/8459/master/file/figures/cellulose_resp_profiles/cellulose_resp_profiles.png}{, S6}). The difference in center of mass for each set of density profiles (control and experimental) is measured (supp. MM) and each KDE curve represents the collection of density shifts calculated for all responders in the \textsuperscript{13}C-cellulose or \textsuperscript{13}C-xylose treatment (\href{https://authorea.com/users/3537/articles/3612/master/file/figures/shift_and_rabund2/shift_and_rabund2.png}{Fig. 3A}). We observe xylose utilizers having a smaller density shift (0.008 $\pm$ 0.008 g mL\textsuperscript{-1}) than cellulose utilizers (0.015 $\pm$ 0.009 g mL\textsuperscript{-1}), with few exceptions. 

Most xylose responders are found at higher rank abundances than cellulose responders (0.01 \textless \textit{p} \textless 0.05), which fall among the rarer taxa in the tail of the RA curve (Fig 3B). This demonstrates that many taxa important to cellulose cycling are present in the rarer fraction of the overall microbial community. Yet, the transitions in abundances of responders is difficult to discern in the bulk community abundances (\href{https://authorea.com/users/3537/articles/8459/master/file/figures/xylose_resp_profiles/xylose_resp_profiles.png}{Figs. S5} \href{https://authorea.com/users/3537/articles/8459/master/file/figures/cellulose_resp_profiles/cellulose_resp_profiles.png}{, S6}) or may not be detected with bulk community sequencing efforts. For example, the increase in Bacteroidetes in the xylose treatment at d3 is not observed in the bulk community abundances. Other instances may result in subtle changes in bulk community abundance that would be difficult to differentiate from natural variation or methodological noise.


\textbf{Patterns of carbon use vary dramatically within phylum.} Dynamic patterns of \textsuperscript{13}C-assimilation from xylose and cellulose occur at discrete, fine-scale taxonomic units (\href{https://authorea.com/users/3537/articles/3612/master/file/figures/bacteria_tree/bacteria_tree.png}{Fig. 4}). Responders for xylose and cellulose are widespread across 6 and 7 phyla, respectively (\href{https://authorea.com/users/3537/articles/3612/master/file/figures/bacteria_tree/bacteria_tree.png}{Fig. 4}). There are 5 phyla containing responders for both treatments; of all the responder OTUS detected within those phyla for either xylose or cellulose, there are only six OTUs that respond to both xylose and cellulose (discussed previously). This result suggests that phyla do not represent coherent ecological units with respect to the soil C-cycle, that is, taxa within phyla exhibit differences in substrate use, level of substrate specialization, and dynamics of incorporation. 

In this study, we have identified Actinobacteria responders for both substrates (\href{https://authorea.com/users/3537/articles/8459/master/file/figures/xylose_resp_profiles/xylose_resp_profiles.png}{Figs. S5} \href{https://authorea.com/users/3537/articles/8459/master/file/figures/cellulose_resp_profiles/cellulose_resp_profiles.png}{, S6}). Although there were no shared Actinobacteria OTUs that responded to both xylose (Microbacteriaceae, Micrococcaceae, Cellulomonadaceae, Nakamurellaceae, Promicromonosporaceae, and Geodermatophilaceae) and cellulose (Streptomycetaceae and Pseudonocardiaceae). This information may suggest that while Actinobacteria exhibit an ability to utilize an array of carbon substrates, substrate use may be more clade specific and not widespread throughout the phylum (\href{https://authorea.com/users/3537/articles/3612/master/file/figures/bacteria_tree/bacteria_tree.png}{Fig. 4}). Similarly, Bacteroidetes responders were identified for both substrates, yet, at a finer taxonomic resolution there is a clear differential response for xylose (Flavobacteriaceae and Chitinophagaceae) and cellulose (Cytophagaceae). 


Whole phylum responses were not detected for xylose or cellulose yet utilization of these substrates spanned many phylogenetically diverse groups. Within each phylum we observed substrate utilization at the clade or single taxa level with each exhibiting a unique pattern of \textsuperscript{13}C-assimilation over time (\href{https://authorea.com/users/3537/articles/3612/master/file/figures/bacteria_tree/bacteria_tree.png}{Fig. 4, heatmap}). It has previously been suggested that all taxa within a phylum are unlikely to share ecological characteristics \cite{Fierer_2007}, and furthermore, within a species population \cite{Choudoir_2012,Preheim_2011,Hunt_2008}. Habitat traits of coastal Vibrio isolates were mapped onto microbial phylogeny revealing discrete ecological populations based on seasonal occurrence and particulate size fractionation \cite{Preheim_2011,Hunt_2008}. Yet, it has been proposed that the microbial community functionality responsible for soil C cycling appear at the level of phlya rather than species/genera \cite{Schimel_2012}. The traditional phylum level assignment conventions could in part be due to limitations in finer scale taxonomic identifications or methodological limitations (\textit{i.e.} sequencing depth). Our data in concert with others \cite{Goldfarb_2011,Fierer_2007,Choudoir_2012,Preheim_2011,Hunt_2008} would suggest that assigning substrate utilization of a few OTUs or clades as a phylum level response is not accurate.

\textbf{Conclusions.} We have demonstrated how next generation sequencing-enabled SIP gives an OTU level resolution for substrate utilization. Using this technique, we are able to resolve discrete OTUs that would otherwise be missed using bulk community sequencing efforts. Additionally, this technique provides greater taxonomic resolution than previous techniques (cloning, TRFLP, ARISA) used to determine substrate utilizing community members. While we are currently able to resolve highly responsive OTUs, there is still a need to resolve taxa that are partially responsive which we cannot differentiate from noise with confidence at this time. Although, if we could identify partially responsive taxa, their contributions to the C-cycle would still be difficult to discern. For example, a generalist utilizing many substrates including \textsuperscript{12}C substrates and the \textsuperscript{13}C-labeled substrate may exhibit the same partial labeling that a specialist utilizing both the \textsuperscript{13}C-substrate and the same substrate (unlabeled) that is inherent in the soil. Additionally, partially labeled taxa could be further down the trophic cascade including predators or secondary consumers of waste products from primary consumer microbes that were highly labeled.   

OTUs that assimilate xylose and those that assimilate cellulose are largely mutually exclusive. Those OTUs that assimilate xylose are labeled within 1-7 days, while those that assimilate cellulose are labeled primarily after 2-4 weeks. The xylose responders demonstrate a smaller change in BD than the cellulose responders suggesting that xylose responders assimilate multiple C sources (labeled and unlabeled) consistent with a generalist response, while cellulose responders are more heavily labeled suggesting that cellulose is their main source of C, a response more consistent with a specialist lifestyle. Xylose responders include many taxa, such as spore-fomers, known for the ability to respond rapidly to an influx of new nutrients while cellulose responders include many OTUs that are common uncultivated soil organisms. Finally, xylose responders are more abundant in the community while cellulose responders are, on average, more rare as indicated by their rank abundance within the soil community. These results indicate that different bacteria in soil have distinct physiological and ecological responses which govern their interactions with soil C pools. 

We did not observe consistent C utilization at the phylum level although both xylose and cellulose utilization were observed across 7 phyla each revealing a high diversity of bacteria able to utilize these substrates. The high taxonomic diversity may enable substrate metabolism under a broad range of environmental conditions \cite{Goldfarb_2011}. Other studies of microbial communities have observed a positive correlation with taxonomic or phylogenetic diversity and functional diversity \cite{Fierer_2012,Fierer_2013,Philippot_2010,Tringe_2005,Gilbert_2010,Bryant_2012}. The data presented here supports that specific functional attributes can be shared among diverse, yet distinct, taxa while closely related taxa may have very different physiologies \cite{Fierer_2012,Philippot_2010}. This information adds to the growing collection of data suggesting that community membership is important to biogeochemical processes. Furthermore, it highlights a need to examine substrate utilization by discrete microbial taxa within a whole community context to better understand how specific community members function within the whole. 

The sensitivity of SIP-NGS provides a means to elucidate substrate utilization by discrete microbial taxa with the hope that we can begin to construct a belowground C food web. We obtained enough information to conclusively determine isotope incorporation for 61\% of the more than 6,000 OTUs detected. For those OTUs with enough information (n = 3,825), approximately 2\% (n = 72) significantly assimilated \textsuperscript{13}C from either xylose or cellulose. In the future deeper sequencing will enable us to increase coverage and assess C use by more community members. Using the informations we gain from SIP-NGS, we can expand our knowledge of specific C-cycling OTUs by taking a targeted metagenomic approach in the nucleic acid pools of 'heavy' fractions. Furthermore, we can now expand our knowledge of soil C use dynamics to a wide array of C substrates and increase our grasp on specific community member contributions. Illuminating these microbial contributions associated with decomposition in soil are important because as environments change, there are measurable and functional changes in soil C \cite{Grandy_2008} which could cumulatively have large impacts at a global scale.