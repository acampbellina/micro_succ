\section{Results and Discussion}

In this study, we couple DNA-SIP with high-throughput sequencing in order to better understand microbial community C use dynamics in soil. A series of parallel soil microcosms amendeded with an identical C mixture, in which the only difference is the identity of the \textsuperscript{13}C-labeled substrate, were incubated for 30 days. The C mixture was designed to approximate freshly degrading plant biomass and either xylose or cellulose were isotopically labeled to examine the dynamics of C assimilation for labile soluble C and insoluble polymeric C. A total of 5.3 mg C g soil\textsuperscript{-1} (including 0.42 mg xylose-C g soil\textsuperscript{-1} and 0.88 mg cellulose-C g soil\textsuperscript{-1}) was added to each microcosm and this represented 18\% of the total C present in the soils. Microcosms were harvested at discrete time points during the incubation period and the temporal and isotope assimilation dynamics of the microbial community were measured by sequencing 16S rRNA gene amplicons in the bulk microbial community and fractions from CsCl gradient fractionation (\href{https://www.authorea.com/users/3537/articles/8459/master/file/figures/20140708_ConceptualFig2/20140708_ConceptualFig2.pdf}{Fig. S1}). Xylose degradation was observed immediately within the first 7 days, while cellulose degradation is observed after 14 days.

\textbf{Temporal dynamics of C-assimilation in soil.}  
The dynamics of \textsuperscript{13}C-xylose and \textsuperscript{13}C-cellulose assimilation varied dramatically within the microbial community. Isotope incorporation into DNA was revealed by analyzing variation in 16S amplicons across gradient fractions (n = 20) from control samples in relation to identical experimental samples that differed by a single substitution of \textsuperscript{12}C-xylose or \textsuperscript{12}C-cellulose with their \textsuperscript{13}C equivalents (\href{https://www.authorea.com/users/3537/articles/8459/master/file/figures/20140708_ConceptualFig2/20140708_ConceptualFig2.pdf}{Fig. S1}). Isotope incorporation increases the bouyant density (BD) of DNA and this causes the relative abundance of an OTU to increase in amplicon pools from 'heavy' fractions of the density gradient. As a result, isotopic incorporation into DNA will cause variation in amplicon pool composition in 'heavy' fractions containing isotopically-labeled DNA relative to corresponding control fractions (\href{https://www.authorea.com/users/3537/articles/3612/master/file/figures/ordination_all1/ordination_all1.png}{axis 1, Fig. 1}). Primary variation of amplicon composition in gradient fractions is attributed to varying bouyant densities of genomes due to G+C content (\href{https://www.authorea.com/users/3537/articles/3612/master/file/figures/ordination_all1/ordination_all1.png}{Fig. 1}). 

Variation in amplicon pool composition between fractions of \textsuperscript{13}C-labeled samples and their corresponding controls is readily observed in 'heavy' gradient fractions (\href{https://www.authorea.com/users/3537/articles/3612/master/file/figures/ordination_all1/ordination_all1.png}{partitioning along axis 2, Fig. 1}). The amplicon pool composition of 'heavy' fractions of \textsuperscript{13}C-xylose and \textsuperscript{13}C-cellulose samples vary dramatically from corresponding controls and from each other, indicating that there are distinct microbial community responses for each of these substrates (\href{https://www.authorea.com/users/3537/articles/3612/master/file/figures/ordination_all1/ordination_all1.png}{Fig. 1A}). Had the isotope incorportation from \textsuperscript{13}C-xylose and \textsuperscript{13}C-cellulose occured in the same community members, the divergence of the high-buoyant density fractions of these two treatments relative to control would have coincided in the ordination space.  

The \textsuperscript{13}C-incorporation reveals temporal dynamics of C degradation demonstrated by \textsuperscript{13}C-xylose incorporation at days 1, 3, and 7 and \textsuperscript{13}C-cellulose incorporation at days 14 and 30 (\href{https://www.authorea.com/users/3537/articles/3612/master/file/figures/ordination_all1/ordination_all1.png}{Fig. 1B}), as expected \cite{Amelung_2008}. The microbial community changed significantly (pval) with time in the bulk community supporting the temporal dynamics observed in the gradient fraction amplicons (\href{https://authorea.com/users/3537/articles/8459/master/file/figures/bulk_ordination/bulk_ordination.png}{Fig. S2}). Although within a single time point, the bulk community demonstrated no significant difference between treatments. 

'Heavy' fraction amplicon pools from samples that received \textsuperscript{13}C-xylose diverged from corresponding controls on days 1 through 7 (\href{https://www.authorea.com/users/3537/articles/3612/master/file/figures/ordination_all1/ordination_all1.png}{Fig. 1}). Furthermore, amplicon pool composition varied across these days (\href{https://www.authorea.com/users/3537/articles/3612/master/file/figures/ordination_all1/ordination_all1.png}{Fig. 1B}) indicating dynamic changes in \textsuperscript{13}C-xylose assimilation with time. At day 14 and 30 heavy fractions from \textsuperscript{13}C-xylose labeled samples are no longer differentiated from corresponding controls indicating that \textsuperscript{13}C is no longer detectable in DNA. The decline in \textsuperscript{13}C-labelling of DNA is likely due to isotopic dilution resulting from assimilation of unlabeled C and/or due to cell turnover resulting from mortality. 

\textsuperscript{13}C-cellulose incorporation isn't detected until day 14 and amplicon composition is consistent for both days 14 and 30 (\href{https://www.authorea.com/users/3537/articles/3612/master/file/figures/ordination_all1/ordination_all1.png}{Fig. 1}). The consistency of amplicon composition for cellulose degradation over time compared to xylose suggests a wider array of microorganisms utilize xylose, whereas, cellulose utilization occurs in a select few. This is consistent with long standing notions that more microorganisms are capable of utilizing simple carbohydrates than complex C substrates. 

Overall patterns of C degradation observed in this study demonstrate different microbial community members are responsible for the consumption of these two substrates; xylose is consumed quickly, whereas, cellulose decomposition takes longer. This suggests a pattern of microbial community transition accompanying the decomposition process. This is consistent with Engelking \textit{et al.}\cite{Engelking_2007} who observed as much as 75\% of labile C respired or converted into microbial biomass in the first 5 days of decomposition, whereas, cellulose degraders take longer to respond with less than 42\% of cellulose metabolized over the first 5 days of incubation \cite{Hu_1997}. 

\textbf{Differential C utilization by taxa.} Individual OTUs that assimilated \textsuperscript{13}C-substrates were identified using the DESeq framework \cite{Anders_Huber_2010} to analyze differential representation in heavy fractions (\href{https://www.authorea.com/users/3537/articles/3612/master/file/figures/l2fc_fig1/l2fc_fig.pdf}{Fig. 2}). There were 43 and 35 unique OTUs that significantly (false discovery rate corrected \textit{P}-values \textless 0.10) assimilated \textsuperscript{13}C-xylose and \textsuperscript{13}C-cellulose, respectively; herein called 'responders' (\href{https://www.authorea.com/users/3537/articles/8459/master/file/figures/OTU_screening_schematic/OTU_screening_schematic.pdf}{Figs. S2}, \href{https://www.authorea.com/users/3537/articles/8459/master/file/figures/l2fc_fig_pVal/l2fc_fig_pVal.png}{& S3}).

\textit{Xylose}. Within the first 7 days of incubation an average 63\% of \textsuperscript{13}C-xylose was respired and only an additional 6\% more was respired between days 7 and 30. At the end of the 30 day experiment 30\% of the original \textsuperscript{13}C from xylose remained in the soils. The \textsuperscript{13}C remaining in the soil from \textsuperscript{13}C-xylose addition has likely been stabilized by assimilation into microbial biomass and/or microbial conversion into other forms of organic matter, though it is possible that some \textsuperscript{13}C-xylose remains unavailable to microbes due to abiotic interactions in soil \cite{Kalbitz_2000}. Of the 60 total xylose responders 53 were responsive within the first 7 days and only 7 responders detected for days 14 and 30 (\href{https://authorea.com/users/3537/articles/8459/master/file/figures/resp_table/resp_table.png}{Table S1}). 

At day 1 (d1), 57\% of xylose responsive OTUs belong to Firmicutes (Paenibacillaceae, Planococcaceae, and Bacillaceae) and the remaining 43\% of responders were comprised of 19\% Bacteroidetes (Flavobacteriaceae), 14\% Proteobacteria (Enterobacteriaceae, Comamonadaceae, and uncultured Gammaproteobacteria), and 10\% Actinobacteria (Micrococcaceae and Microbacteriaceae) (\href{https://www.authorea.com/users/3537/articles/3612/master/file/figures/l2fc_fig1/l2fc_fig.pdf}{Fig. 2}). Of the xylose responders detected at day 3 (d3), Firmicutes responders decreased to 8\% (from 12 OTUs to 1) and an increase in Bacteroidetes (67\%) and Proteobacteria (25\%) from d1. Day 7 (d7) responders were 50\% Actinobacteria, 35\% Proteobacteria, and 5\% of each Bacteroidetes, Firmicutes, and Planctomycetes. A substantial amount (75\%) of xylose responders for day 7 had not previously been identified as responders at earlier time points which attests to the wide number of taxa able to use xylose. Each of the 10 Actinobacteria responders, the most dominant response group (50\%) at day 7, belonged to a different family making it the phylum with the most wide-spread use of xylose. However, it should be noted that they were confined within two Actinobacterial Orders; Frankiales and Micrococcales. 

We observe dynamic changes in \textsuperscript{13}C-xylose assimilation with time; dominant xylose responders shift from Firmicutes (d1) to Bacteroidetes (d3) then finally Actinobacteria (d7). At any given time soils harbor microorganisms at varying degrees of dormancy depending on nutrient availability \cite{Jones_2010}. The sudden addition of our complex C mixture would most certainly prompt dormant and non-dormant microbes back into metabolic activity, with those exhibiting higher rRNA operon copy numbers responding the fastest. The responders identified at d1 for xylose utilization have all been noted for exhibiting some form of dormancy strategy \cite{Jones_2010, Mulyukin_2009, Darcy_2011, Sachidanandham_2008, Finkel_2006, Rittershaus_2013, Tada_2013, Lay_2013}, though the only spore-forming responders at d1 are Firmicutes. Additionally, d1 responders exhibit 6-14 rRNA operon copies with the exception of the Betaproteobacteria Comamonadaceae and the Actinobacterial OTUs which exhibit 1-2 copies according to representative taxa in the rrnDB v. 3.1.227 \cite{18948294,11125085}. With the exception of the single Firmicutes (Paenibacillaceae), responders on d3 possess 3-6 rRNA operon copies, less than responders on d1. Similar to d3, 85\% of d7 responders exhibit between 1-5 rRNA operon copies with the remaining 15\% (Flavobacteriaceae, Enterobacteriaceae, and Paenibacillaceae) containing between 6-14 rRNA operon copies. Paenibacillus (100\% identity) was the only OTU to be identified at every time point (up to day 30) as a xylose responder. This result suggests that large numbers of cells from Paenibacillus sporulated after \textsuperscript{13}C-labeling of their DNA and that these spores remained throughout the experiment. It was the the second most enriched xylose responder, log\textsubscript{2}fold change (l2fc) of 3.5, measured in the time series second only to a Gammaproteobacteria (Xanthomonadaceae; l2fc = 3.7).


\textit{Cellulose}. In contrast with xylose responders, there were only three \textsuperscript{13}C-cellulose responders detected within the first 7 days of incubation and 46 for days 14 and 30. An average of 16\% of the \textsuperscript{13}C-cellulose added was respired within the first 7 days, 38\% by day 14 (d14), and 60\% by day 30 (d30). The earliest responders detected for \textsuperscript{13}C-cellulose assimilation were a canonical cellulose degrader, Cellvibrio, in Proteobacteria and a novel clade in Chloroflexi (\href{https://www.authorea.com/users/3537/articles/3612/master/file/figures/l2fc_fig1/l2fc_fig.pdf}{Figs. 2}, \href{https://authorea.com/users/3537/articles/8459/master/file/figures/l2fc_fig_pVal/l2fc_fig_pVal.png}{& S4}). At day 14, 55\% of the responders belong to Proteobacteria (65\% Alpha-, 23\% Gamma-, and 12\% Beta-), 13\% Chloroflexi, 13\% Planctomycetes, 10\% Verrucomicrobia, 6\% Actinobacteria, and 3\% Cyanobacteria.

All cellulose responders for day 30 (n = 15) had been identified as responders at earlier time points in this study except two; a Deltaproteobacteria (Sandaracinaceae family) and a Bacteroidetes (Cytophagaceae family). While there are known cellulose degraders in the Bacteroidetes Cytophagaceae family, there are currently no known cellulose degraders in the Sandaracinaceae family although its sister family Polyangiaceae has known cellulose degraders (\cite{Reichenbach_2006}, Bergey's ISBN:978-0-387-24145-6). Throughout the time series, cellulose responders with the greatest enrichment were Verrucomicrobia (Verrucomicrobiaceae), Chloroflexi, Cyanobacteria, Proteobacteria (Cellvibrio, Brevundimonas, Stenotrophomonas, Devosia), and Planctomycetes (Planctomycetaceae) (\href{https://authorea.com/users/3537/articles/3612/master/file/figures/bacteria_tree/bacteria_tree.png}{Fig. 4}).      

Verrucomicrobia, a phylum found to be ubiquitous and in high abundance in soil \cite{Fierer_2013}, have been noted for degradation of polysaccharides in soil, aquatic, and anoxic rice patty soils \cite{Fierer_2013,Herlemann_2013,10543821}. In this study, Verrucomicrobia comprise ~11\% of the total cellulose responder OTUs detected (\href{https://www.authorea.com/users/3537/articles/3612/master/file/figures/l2fc_fig1/l2fc_fig.pdf}{Figs. 2}, \href{https://authorea.com/users/3537/articles/8459/master/file/figures/l2fc_fig_pVal/l2fc_fig_pVal.png}{S4}, \href{https://authorea.com/users/3537/articles/8459/master/file/figures/cellulose_resp_profiles/cellulose_resp_profiles.png}{S6}) most of which belong to the uncultured FukuN18 clade originally identified in freshwater lakes \cite{Parveen_2013}. Yet the largest enrichment measured (l2fc = 3.7) during the whole time series for \textsuperscript{13}C-cellulose assimilation was by an uncultured Verrucomicrobia in the Verrucomicrobiaceae family on d14 (\href{https://authorea.com/users/3537/articles/3612/master/file/figures/bacteria_tree/bacteria_tree.png}{Fig. 4}). 

Chloroflexi, ubitiquous across many diverse environments, are traditionally known for their metabolically dynamic lifestyles ranging from anoxygenic phototropy to organohalide respiration \cite{Hug_2013}. Yet, only recently have we shifted focus towards the metabolic functions of Chloroflexi in C cycling \cite{Hug_2013,Goldfarb_2011,Cole_2013}. In this study, we identified a previously undescribed clade within the Chloroflexi class (closest relative at a 96\% identity being Herpetosiphon) that exhibited a high substrate specificity based on a BD shift peak between 0.03-0.04 gmL\textsuperscript{-1} (\href{https://authorea.com/users/3537/articles/8459/master/file/figures/cellulose_resp_profiles/cellulose_resp_profiles.png}{Fig. S6}). We observed no other cellulose utilization in Chloroflexi outside of this clade although many members of this phylum have previously been demonstrated or implicated in cellulose utilization \cite{Goldfarb_2011,Cole_2013,Hug_2013}. 

One of the most interesting findings is a single Cyanobacterial responder OTU, which exhibit the third greatest enrichment measured (l2fc = 3.35) in response to \textsuperscript{13}C-cellulose assimilation. This OTU falls into the recently described candidate phylum Melainabacteria \cite{Di_Rienzi_2013}, although its phylogenetic position is debated \cite{Soo_2014}. More importantly, the catalog of metabolic capabilities associated with Cyanobacteria are quickly expanding \cite{Di_Rienzi_2013, Soo_2014}. Our findings provide evidence of cellulose degradation for Melainabacteria, supporting hypothesized polysaccharide degradation suggested by genomic analysis \cite{Di_Rienzi_2013}.     

Proteobacteria represent 47\% of all cellulose responders identified.  Of those, Cellvibrio accounted for 5\% of all Proteobacterial responders detected (\href{https://www.authorea.com/users/3537/articles/3612/master/file/figures/l2fc_fig1/l2fc_fig.pdf}{Figs. 2}, \href{https://authorea.com/users/3537/articles/8459/master/file/figures/l2fc_fig_pVal/l2fc_fig_pVal.png}{S4}, \href{https://authorea.com/users/3537/articles/8459/master/file/figures/cellulose_resp_profiles/cellulose_resp_profiles.png}{S6}). Cellvibrio was one of the first identified cellulose degrading bacteria and was originally described by Winogradsky in 1929 who named it for its cellulose degrading abilities(Bergey's ISBN 9780387241449). Details about max density shift (need to talk to Chuck because this data is not easy to acquire). Other prominent cellulose degrading Proteobacteria, Stenotrophomonas and Devosia have previous been demonstrated in degrading cellulose \cite{Trujillo_Cabrera_2012, Verastegui_2014}. Brevundimonas has not previously been identified as a cellulose degrader, but has been show to degrade cellouronic acid, an oxidized form of cellulose \cite{Tavernier_2008}.

There were 6 shared responders among all responders identified in both the xylose and cellulose treatments (n = 72); Stenotrophomonas, Planctomyces, two Rhizobiaceae, Comamonadaceae, and Cellvibrio. Of these, Stenotrophomonas and Comamonadaceae are the only taxa that are among the top ten l2fc responses measured in both treatments. On the other hand, the only shared responder that is not among the top ten responders for either the cellulose or xylose treatment is Rhizobiaceae. Two of the shared responders corresponded in time between the two treatments (supplemental table); Cellvibrio (d3) and Planctomyces (d14).  

\textit{Responder Characteristics}.  Overall, we found xylose responders were from higher rank abundances than cellulose responders, however, cellulose responders exhibited a greater change in BD (i.e. assimilated more \textsuperscript{13}C) than xylose responders in response to isotope incorporation (\href{https://authorea.com/users/3537/articles/3612/master/file/figures/shift_and_rabund2/shift_and_rabund2.png}{Fig. 3}). 

The kernel density estimate (KDE) of BD shifts resulting from \textsuperscript{13}C-assimilation reveal that cellulose responders exhibit a significantly (wilcox rank sum; p\textless...) greater BD shift than xylose responders (\href{https://authorea.com/users/3537/articles/3612/master/file/figures/shift_and_rabund2/shift_and_rabund2.png}{Fig. 3A}). A density profile for each responder is generated for the experimental and control treatment at each of the sampling time points using relative abundances from sequence libraries (\href{https://authorea.com/users/3537/articles/8459/master/file/figures/xylose_resp_profiles/xylose_resp_profiles.png}{Figs. S5}\href{https://authorea.com/users/3537/articles/8459/master/file/figures/cellulose_resp_profiles/cellulose_resp_profiles.png}{, S6}). The difference in center of mass for each set of density profiles (control and experimental) is measured (supp. MM) and each KDE curve represents the collection of density shifts calculated for all responders in the \textsuperscript{13}C-cellulose or \textsuperscript{13}C-xylose treatment (\href{https://authorea.com/users/3537/articles/3612/master/file/figures/shift_and_rabund2/shift_and_rabund2.png}{Fig. 3A}). We observe xylose utilizers having a smaller density shift (\textless0.02 g mL\textsuperscript{-1}) than cellulose utilizers (0.005-0.03 g mL\textsuperscript{-1}), with few exceptions. 

Most xylose responders are found at higher rank abundances than cellulose responders, which fall among the rarer taxa in the tail of the RA curve (Fig 3B). This demonstrates that many taxa important to cellulose cycling are present in the rarer fraction of the overall microbial community and may be difficult or unable to detect in bulk community sequencing efforts. Additionally, the transitions in abundances of responders is difficult to discern in the bulk community abundances (\href{https://authorea.com/users/3537/articles/8459/master/file/figures/xylose_resp_profiles/xylose_resp_profiles.png}{Figs. S5} \href{https://authorea.com/users/3537/articles/8459/master/file/figures/cellulose_resp_profiles/cellulose_resp_profiles.png}{, S6}). For example, the increase in Bacteroidetes in the xylose treatment at d3 is not observed in the bulk community abundances. In another instance, the increased response from Proteo- and Actinobacterial OTUs at d7 is also observed in the bulk community analysis as a marginal increase, yet it would be difficult to differentiate that change from natural variation or methodological noise.



\textbf{Patterns of carbon use vary dramatically within phylum.} Dynamic patterns of \textsuperscript{13}C-assimilation from xylose and cellulose occur at discrete, fine-scale taxonomic units (\href{https://authorea.com/users/3537/articles/3612/master/file/figures/bacteria_tree/bacteria_tree.png}{Fig. 4}). Responders for xylose and cellulose are widespread across 6 and 7 phyla, respectively (\href{https://authorea.com/users/3537/articles/3612/master/file/figures/bacteria_tree/bacteria_tree.png}{Fig. 4}). There are 5 phyla containing responders for both treatments although within a phylum we observe utilization of both substrates by only six OTUs (discussed previously). This result suggests that phyla do not represent coherent ecological units with respect to the soil C-cycle, that is, taxa within phyla exhibit differences in substrate use, level of substrate specialization, and dynamics of incorporation. 

In this study, we have identified Actinobacteria responders for both substrates with a peak shift of ~0.036 g mL\textsuperscript{-1} for cellulose and ~x g mL\textsuperscript{-1} for xylose, suggesting a strong substrate specificity (\href{https://authorea.com/users/3537/articles/8459/master/file/figures/xylose_resp_profiles/xylose_resp_profiles.png}{Figs. S5} \href{https://authorea.com/users/3537/articles/8459/master/file/figures/cellulose_resp_profiles/cellulose_resp_profiles.png}{, S6}). Although there were no shared Actinobacteria OTUs that responded to both xylose (Microbacteriaceae, Micrococcaceae, Cellulomonadaceae, Nakamurellaceae, Promicromonosporaceae, and Geodermatophilaceae) and cellulose (Streptomycetaceae and Pseudonocardiaceae). This information may suggest that while Actinobacteria exhibit an ability to utilize an array of carbon substrates, substrate use may be more clade specific and not widespread throughout the phylum (\href{https://authorea.com/users/3537/articles/3612/master/file/figures/bacteria_tree/bacteria_tree.png}{Fig. 4}). Similarly, Bacteroidetes responders were identified for both substrates, yet, at a finer taxonomic resolution there is a clear differential response for xylose (Flavobacteriaceae and Chitinophagaceae) and cellulose (Cytophagaceae). Disagreeing correlations of phylum level abundance associated with C availability has been the source of debate for the role of Bacteroidetes in the degradation process \cite{Fierer_2007,Rui_2009,Sharp_2000,L_pez_Lozano_2013,Bastian_2009}. Our results would suggest that both perspectives are correct at a phylum level, but highlights the need of greater resolution to capture the subtleties of substrate utilization.   

Whole phylum responses were not detected for xylose or cellulose yet utilization of these substrates spanned many phylogenetically diverse groups. Within each phylum we observed substrate utilization at the clade or single taxa level with each exhibiting a unique pattern of \textsuperscript{13}C-assimilation over time (\href{https://authorea.com/users/3537/articles/3612/master/file/figures/bacteria_tree/bacteria_tree.png}{Fig. 4, heatmap}). In a study that amended forest soils with single C substrates in the presence of 3-bromodeoxyuridine (BrdU), they determined that more than 500 taxa responded to labile C but occured predominantly in two phlya, Proteobacteria and Actinobacteria \cite{Goldfarb_2011}. On the other hand, the soils amended with polymeric C such as cellulose and lignin were noted for spanning eight phyla, but were limited to a small number of taxa within each of those phyla that were responders \cite{Goldfarb_2011}. It has previously been suggested that all taxa within a phylum are unlikely to share ecological characteristics \cite{Fierer_2007}, and furthermore, within a species population \cite{Choudoir_2012,Preheim_2011,Hunt_2008}. Habitat traits of coastal Vibrio isolates were mapped onto microbial phylogeny revealing discrete ecological populations based on seasonal occurrence and particulate size fractionation \cite{Preheim_2011,Hunt_2008}. Yet, it has been proposed that the microbial community functionality responsible for soil C cycling appear at the level of phlya rather than species/genera \cite{Schimel_2012}. The traditional phylum level assignment conventions could in part be due to limitations in finer scale taxonomic identifications or methodological limitations (ie sequencing depth), but our data in concert with others \cite{Goldfarb_2011,Fierer_2007,Choudoir_2012,Preheim_2011,Hunt_2008} would suggest that portraying the response of a few OTUs or clades as a phylum level response would be overreaching serves as a strong argument towards moving away from extrapolating substrate utilization to phylum level responses.

\textbf{Conclusions.} We have demonstrated how next generation sequencing-enabled SIP gives an OTU level resolution for substrate utilization. Using this technique, we are able to resolve discrete OTUs that would otherwise be missed using bulk community sequencing efforts. Additionally, this technique provides greater taxonomic resolution than previous techniques (cloning, TRFLP, ARISA) used to determine substrate utilizing community members. Traditionally, SIP is performed with a single time point in an attempt to minimize or eliminate cross-feeding of the substrate. While we are currently able to resolve highly responsive OTUs, there is still a need to resolve taxa that are partially responsive which we cannot differentiate from noise with confidence at this time. Although, if we could identify partially responsive taxa, their contributions to the C-cycle would still be difficult to discern. For example, a generalist utilizing many substrates including \textsuperscript{12}C substrates and the \textsuperscript{13}C-labeled substrate may exhibit the same partial labeling that a specialist utilizing both the \textsuperscript{13}C-substrate and the same substrate (unlabeled) that is inherent in the soil. Additionally, partially labeled taxa could be further down the trophic cascade including predators or secondary consumers of waste products from primary consumer microbes that were highly labeled.   

OTUs that assimilate xylose and those that assimilate cellulose are largely mutually exclusive. Those OTUs that assimilate xylose are labeled within 1-3 days, while those that assimilate cellulose are labeled primarily after 2-4 weeks. The xylose responders demonstrate a smaller change in DNA BD than the cellulose responders suggesting that xylose responders assimilate multiple C sources (labeled and unlabeled) consistent with a generalist response, while cellulose responders are more heavily labeled suggesting that cellulose is their main source of C, a response more consistent with a specialist lifestyle. Xylose responders include many taxa, such as spore-fomers, known for the ability to respond rapidly to an influx of new nutrients while cellulose responders include many OTU that are common uncultivated soil organisms. Finally, xylose responders are more abundant in the community while cellulose responders are, on average, more rare as indicated by their rank abundance within the soil community. These results indicate that different bacteria in soil have distinct physiological and ecological responses which govern their interactions with soil carbon pools. 

We did not observe consistent C utilization at the phylum level although both xylose and cellulose utilization were observed across 7 phyla each revealing a high diversity of bacteria able to utilize these substrates. The high taxonomic diversity may enable substrate metabolism under a broad range of environmental conditions \cite{Goldfarb_2011}. Other studies of microbial communities have observed a positive correlation with taxonomic or phylogenetic diversity and functional diversity \cite{Fierer_2012,Fierer_2013,Philippot_2010,Tringe_2005,Gilbert_2010,Bryant_2012}. The data presented here supports that specific functional attributes can be shared among diverse, yet distinct, taxa while closely related taxa may have very different physiologies \cite{Fierer_2012,Philippot_2010}. This information adds to the growing collection of data suggesting that community membership is important to biogeochemical processes. Furthermore, it highlights a need to examine substrate utilization by discrete microbial taxa within a whole community context to better understand how specific community members function within the whole. The sensitivity of SIP-NGS provides a means to elucidate substrate utilization by discrete microbial taxa with the hope that we can begin to construct a belowground C food web.          
