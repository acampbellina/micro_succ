\section{Significance} 
We have little understanding about belowground soil microbial carbon cycle dynamics and food webs. Next generation sequencing enabled stable isotope probing is a novel approach that allows us to identify discrete microbial taxa utilizing a given 13C-labeled substrate amidst  
The terrestrial biosphere contains a large fraction of global C and nearly 70\% of the organic C in these systems is found in soils (Chapin et al. 2002). When organic C from plants reaches soil it is degraded through the activity of fungi and by both aerobic and anaerobic bacteria. This C may be rapidly returned to the atmosphere as CO2 or may remain in the soil as humic substances that may persist for 20 to 2000 years (Yanagita 1990). The majority of C is respired and on an annual basis soil respiration produces 10 times more CO2 than anthropogenic emissions (Chapin et al. 2002). Global changes in atmospheric CO2, temperature, and ecosystem N inputs, are expected to impact primary production and C inputs to soils (van Groenigen et al. 2006), but it remains difficult to predict the response of soil processes to anthropogenic change (Davidson et al. 2006). Changes in soil temperature are well known to have strong impacts on soil respiration (Waksman and Gerretsen 1931) but the effects of temperature and litter quality interact across different soils, confounding precise estimates of soil processes. In a regional study, temperature and substrate quality could explain only 53\% of the variation in Q10 values (Fierer et al. 2006). Likewise, the impact of N enrichment on the storage of C in organic carbon pools is difficult to estimate with soil systems, demonstrating a range of different responses (Waldrop et al. 2004, McLauchlan 2006, van Groenigen et al. 2006). Our difficulty in predicting how soil processes will respond to environmental change suggests a need for a greater understanding of the biotic mechanisms that govern the soil C-cycle (Bradford et al. 2008).