\section{Significance} We have a limited understanding of soil carbon (C)
cycling yet soil contains a large fraction of the global C pool. Microorganisms
mediate most soil C cycling but have proven difficult to study due to the
complexity of soil C biochemistry and the wide range of soil microorganisms
participating in C reactions. We demonstrate C use dynamics by soil microbial
taxa. Furthermore, we identified microorganisms involved in cellulose
decomposition that were previously uncharacterized physiologically -- cellulose
is the most globally abundant biopolymer. Our results expand knowledge of soil
functional guild diversity and activity which reveal soil structure-function
relationships. This study is a departure from typical nucleic acid SIP studies
that focus on listing the identities of heavy isotope labeled organisms. Our
approach enables DNA-SIP to identify $^{13}$C labeled microorganisms with
greater resolution producing a better sampling of functional guilds. This not
only allows us to connect function to genetic identity but also allows us to
assess functional guild diversity and uncover ecological strategies. Further,
we demonstrate how substrate specificity can be assessed from DNA-SIP data. 
