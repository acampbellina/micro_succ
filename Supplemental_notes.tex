\section{Supplemental Notes}
\subsection{Phylogenetic affiliation of $^{13}$C-cellulose and
    $^{13}$C-xylose responsive microorganisms}
% Fakesubsubsection:\textit{Verrucomicrobia} are ubiquitous in soil
\textit{Verrucomicrobia}, cosmopolitan soil microbes
\citep{Bergmann_2011}, can comprise up to 23\% of 16S rRNA gene sequences in
high-throughput DNA sequencing surveys of SSU rRNA genes in soil
\citep{Bergmann_2011} and can account for up to 9.8\% of
soil 16S rRNA \citep{Buckley_2001}. Many \textit{Verrucomicrobia} were first
isolated in the last decade \cite{Wertz_2011} but only one of the 15 most
abundant verrucomicrobial phylotypes in a global soil sample collection shared
greater than 93\% sequence identity with a cultured isolate
\citep{Bergmann_2011}. Genomic analyses and physiological profiling of
\textit{Verrucomicrobia} isolates revealed \textit{Verrucomicrobia} are capable
of methanotrophy, diazotrophy, and cellulose degradation \citep{Otsuka_2012,
Wertz_2011}, yet the function of soil \textit{Verrucomicrobia} in global C-cycling
remains unknown. Only two of the ten putative cellulose degrading
\textit{Verrucomicrobia} identified in this experiment share at least 95\%
sequence identity with an isolate ("OTU.83" and "OTU.627",
Table~\ref{tab:cell}). Seven of ten $^{13}$C-cellulose responding
verrucomicrobial OTUs were classified as \textit{Spartobacteria} which are
a numerically dominant family of \textit{Verrucomicrobia} in SSU rRNA gene
surveys of 181 globally distributed soil samples \citep{Bergmann_2011}. Given
their ubiquity and abundance in soil as well as their demonstrated
incorporation of $^{13}$C from $^{13}$C-cellulose, \textit{Verrucomicrobia}
lineages, particularly \textit{Spartobacteria}, may be important contributors
to cellulose decomposition on a global scale.

% Fakesubsubsection:Soil \textit{Chloroflexi} have been found to assimilate
Cellulose degrading soil \textit{Chloroflexi} have previously been identified
in DNA-SIP studies \citep{Schellenberger_2010}. The cellulose degrading
\textit{Chloroflexi} in this study are only distantly related to isolates
\ref{tab:cell}. Chloroflexi are among the six most abundant soil phyla commonly
recovered soil microbial diversity surveys \citep{Janssen2006}.
Chloroflexi are typically not as as abundant as \textit{Verrucomicrobia} but
are roughly as abundant as \textit{Bacteroidetes} and \textit{Planctomycetes}
\citep{Janssen2006}.  Four of five $^{13}$C-cellulose responsive
\textit{Chloroflexi} identified in this study are annotated as belonging to the
\textit{Herpetosiphon} although they share less than 95\% sequence identity with their closest cultured relative in the
\textit{Herpetosiphon} genus (\textit{H. geysericola}). \textit{H. geysericola}
is a predatory bacterium shown to prey upon \textit{Aerobacter} in culture and
can also digest cellulose \citep{Lewin1970}. In our study, "Herpetosiphon"
$^{13}$C-cellulose responders did not show a delayed response to
$^{13}$C-cellulose as compared to other responders but nonetheless could have
become labeled by feeding on primary $^{13}$C-cellulose degraders. The prey
specificity of predatory bacteria is not well established especially \textit{in
situ}. $^{13}$C-labeling would be positively correlated with prey specificity.
If the predator specifically preyed upon one population then it could take on
the same labeling percent as that population given enough generations. Preying on multiple types would
produce a mixed and dilute labeling signature if some of the prey
were not isotopically labeled.

% Fakesubsubsection:We also observed $^{13}$C-incorporation
We also observed $^{13}$C-incorporation from cellulose by
\textit{Proteobacteria}, \textit{Planctomycetes} and \textit{Bacteroidetes}. 
Strains in \textit{Proteobacteria}, \textit{Planctomycetes} and \textit{Bacteroidetes} have all been
previously implicated in cellulose degradation. \textit{Planctomycetes} is the
least studied of the three phyla and only one \textit{Planctomycetes} isolate can
grow on cellulose. None of the seven \textit{Planctomycetes} cellulose degraders
identified in this experiment are closely related to isolates.
\textit{Acidobacteria} did not pass or operational criteria for assessing
$^{13}$C incorporation from cellulose into DNA in our microcosms. 
\textit{Acidobacteria} have been found to degrade cellulose in culture CITE and
are a numerically significant soil phylum CITE. \textit{Acidobacteria} have
been shown to dominate at low nutrient availability (CITE: cederlund 2014),
which may explain why they were not active upon nutrient additions. 
The \textit{Acidobacteria} in our microcosms were mainly annotated as belonging
to candidate orders in the Silva taxonomic nomenclature. The highest relative
abundance for any \textit{Acidobacteria} order in the bulk samples was 0.20
(order "DA023") and the highest relative abundance of the Acidobacteria phylum
was 0.23.

