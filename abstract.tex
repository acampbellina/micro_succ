\section{Abstract} 
We have limited knowledge of the contributions of specific microbial taxa to carbon (C) cycling in soil systems; which traditionally has depended heavily on the need for enrichment and isolation. Nucleic acid stable isotope probing (SIP) facilitates the tracking of carbon flow through microbial communities. We employed an approach in which a C mixture simulating soil organic matter was added to soil microcosms. In this approach, constituents of the added C mixture are systematically replaced with their 13C-labeled equivalents enabling the elucidation of organisms utilizing the C13 substrate. Initial treatments included C mixtures that were either unlabeled or incorporated a 13C-xylose or 13C-cellulose label.  Using CsCl gradient fractionation, incorporation of C13 into DNA was measured over a period of 30 days. The 16S rRNA gene sequences from 20 CsCl gradient fractions for each treatment and time point were characterized by 454 pyrosequencing and classified into Operational Taxonomic Units (OTU).  OTUs that assimilated xylose and cellulose were identified using Edge PCA analysis and patterns of isotope incorporation were measured over time. Incorporation of C13 from xylose into OTUs was observed at days 3, 7, and 14, while notable incorporation of C13 from cellulose was observed only after day 14. At the end of 30 days 29\% and 40\% of the C13-label remained in soil for xylose and cellulose, respectively.  Using this information we can describe specific patterns of C-assimilation by discrete OTUs as a function of substrate, time, and level of isotope incorporation. For example, an OTU most similar to Cellvibrio was observed to incorporate C13 from cellulose at 30 days obtaining a maximal buoyant density shift of 0.03g ml-1 relative to unlabeled controls. In contrast, C13 from xylose was incorporated by an OTU most similar to Arthrobacter within 1 day, obtaining a maximal buoyant density shift of 0.02 g ml-1 relative to unlabeled controls.  These observations demonstrate that the Arthrobacter-like population grew rapidly and assimilated C from multiple substrates while the Cellvibrio-like population grew slowly and consumed primarily cellulose.  The composition of the microbial community changed over time in response to C amendment consistent with the microbial community succession hypothesis of decomposition. 