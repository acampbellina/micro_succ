\subsection{Soil microcosm microbial community changes with time}
Changes in the soil microcosm microbial community structure and membership
correlated with incubation time (Figure~\ref{fig:bulk_ord}B, p-value 0.23,
R$^{2}$ 0.63, Adonis test \citet{Anderson2001a}). Labeled C-substrate (no
$^{13}$C, xylose with the $^{13}$C label or cellulose with the
$^{13}$C label) did not significantly correlate with soil microcosm community
structure and membership (p-value 0.35). Additionally, bulk sample beta
diversity was significantly less than gradient fraction beta diversity (p-value
0.003, \citet{Anderson2006}). Twenty-nine OTUs significantly changed in
abundance with time (adjusted p-value $<$ 0.10, \citet{YBenjamini1995}). OTUs
that significantly increased in abundance with time included OTUs in the
\textit{Verrucomicrobia}, \textit{Proteobacteria}, \textit{Planctomycetes},
\textit{Cyanobacteria}, \textit{Chloroflexi} and \textit{Acidobacteria}. OTUs
that sifnificantly decreased in abundance included OTUs in the
\textit{Proteobacteria}, \textit{Firmicutes}, \textit{Bacteroidetes} and
\textit{Actinobacteria} (Figure~XX).  \textit{Proteobacteria} was the only
phylum that had OTUs that significantly increased and OTUs that significantly
decreased in abundance with time. If sequences were grouped by taxonomic
annotations at the class level, only four classes significantly changed in
abundance, \textit{Bacilli} (decreased), \textit{Flavobacteria} (decreased),
\textit{Gammaproteobacteria} (decreased) and \textit{Herpetosiphonales}
(increased) (Figure~XX). 
