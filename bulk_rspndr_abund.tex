\subsection{Xylose responders are more abundant in the soil community than cellulose
responders}
$^{13}$C-xylose responders are generally more abundant members based on
relative abundance in bulk DNA SSU rRNA gene content than $^{13}$C-cellulose
responders (Figure~\ref{fig:shift}, p-value 0.00028).  However, both abundant and
rare OTUs responded to $^{13}$C-xylose and $^{13}$C-cellulose
(Figure~\ref{fig:shift}). For instance, a \textit{Delftia} $^{13}$C-cellulose
responder is fairly abundant in the bulk samples (``OTU.5'',
Table~\ref{tab:cell}). OTU.5 was on average the
13th most abundant OTU in bulk samples. A $^{13}$C-xylose responder (``OTU.1040'',
Table~\ref{tab:xyl}) has a mean relative abundance in bulk samples of
3.57e$^{-05}$. Two $^{13}$C-cellulose responders wer not found
in any bulk samples ("OTU.862" and "OTU.1312", Table~\ref{tab:cell}). Of the 10 most abundant
responders 8 are $^{13}$C-xylose responders and 6 of these 8 are consistently
among the 10 most abundant OTUs in bulk samples.

Responder abundances summed at phylum level generally 
increased for $^{13}$C-cellulose (Figure~XX) whereas $^{13}$C-xylose responder abundances 
summed at the phylum level decreased over time for \textit{Firmicutes}, \textit{Bacteroidetes},
\textit{Actinobacteria} and \textit{Proteobacteria} although 
\textit{Proteobacteria} spiked at day 14 (Figure~\ref{fig:babund}). Bulk abundance trends are
roughly consistent with $^{13}$C assimilation activity. 