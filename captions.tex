\subsection{Figure 1}
NMDS analysis from weighted unifrac distances of 454 sequence data from SIP fractions of each treatment over time. Twenty fractions from a CsCl gradient fractionation for each treatment at each time point were sequenced (Fig. S1). Each point on the NMDS represents the bacterial composition based on 16S sequencing for a single fraction where the size of the point is representative of the density of that fraction and the colors represent the treatments (A) or days (B). 
\subsection{Figure 2}
Log$_{2}$ fold change of $^{13}$C-responders in cellulose
treatment (top) and xylose treatment (bottom).  Log$_{2}$ fold change is based
on the relative abundance in the experimental treatment compared to the control
within the density range 1.7125-1.755 g ml$^{-1}$. Taxa are
colored by phylum. 'Counts' is a histogram of log$_{2}$ fold change values.    
\subsection{Figure 3}
$^{13}$C-responder characteristics based on density shift (A) and rank (B).
Kernel density estimation of $^{13}$C-responder's density shift in cellulose
treatment (blue) and xylose treatment (green) demonstrates degree of labeling
for responders for each respective substrate. $^{13}$C-responders in rank
abundance are labeled by substrate (cellulose, blue; xylose, green). Ticks at top indicate
location of $^{13}$C-cylose responders in bulk community. Ticks at bottom indicate location of
$^{13}$C-cellulose responders in bulk community. OTU rank was assessed from day 1 bulk samples.\subsection{Figure 4}
Maximum log$_{2}$ fold change in response to labeled substrate incorporation for each substrate and all OTUs that pass sparsity filtering in both substrate analyses. Points colored by response. Line has slope of 1 with intercept at the origin. OTUs falling in the top-right quadrant responded to both substrates while those in the top-left and bottom-right responded to $^{13}$C-xylose and $^{13}$C-cellulose, respectively.