\subsection{Figure 1}
NMDS analysis of SIP gradient fraction SSU rRNA gene sequence composition reveals
differences in the sequence composition of gradient fractions is correlated to
fraction density, isotopic labeling, and time. SSU rRNA gene compositon was
profiled for fractions for each density gradient. $^{13}$C-labeling of DNA is
apparent because the SSU rRNA gene composition of gradient fractions from
$^{13}$C and control treatments differ at high density. Each point on the NMDS
plot represents one gradient fraction. SSU rRNA gene composition differences
between gradient fractions were quantified by the weighted Unifrac metric. The
size of each point is positively correlated with density and colors indicate
the treatment (A) or day (B).
\subsection{Figure 2}
OTU enrichment in $^{13}$C-treatment heavy density fractions relative to
control expressed as LFC (see Methods) for the $^{13}$C-cellulose treatment
(top) and $^{13}$C-xylose treatment (bottom). High LFC indicate the OTU
incorporated $^{13}$C into DNA (each point represents an OTU LFC for the given
treatment relative to control at the day indicated). Different colors
represent different phyla and different panels represent different days. The
final column shows the frequency distribution of LFC values in each row. Within
each panel, shaded areas are used to indicate LFC plus or minus one standard
deviation (dark shading) or two standard devations (light shading) about the
mean of all LFC values.
    
    
\subsection{Figure 3}
Phylogenetic relationships of OTUs passing independent filtering when quantifying OTU enrichment in heavy gradient fractions relative to control (see Methods). Only those phyla that contain responders are shown. Colored dots are used to identify xylose responders (green) and cellulse responders (blue). The heatmaps indicate enrichment in high denstiy fractions relative to control (represented as LFC) for each OTU in response to both $^{13}$C-cellulose (``13CCPS'', leftmost heatmap) and $^{13}$C-xylose (``13CXPS'', rightmost heatmap) with values for different days in each heatmap column. Greater enrichment (represented as LFC) in heavy density fractions provide evidence of $^{13}$C-labeled DNA.  \subsection{Figure 4}
Xylose reponders in the \textit{Actinobacteria}, \textit{Bacteroidetes},
\textit{Firmicutes} exhibit distinct temporal dynamics of $^{13}$C-labeling.
The left column shows counts of $^{13}$C-xylose responders in the
\textit{Actinobacteria, Bacteroidetes, Firmicutes} and \textit{Proteobacteria}
at days 1, 3, 7 and 30. The right panel shows enrichment in high density
gradient fractions (expressed as fold change, not logarithmic) for responders
(large points) as well as a boxplot for the distribution of fold change values
(small dots are outliers, i.e. beyond 1.5 times the interquartile range (IR).
Whiskers extend to 1.5 times the IR, and the box extends one IR about the
median (solid line)). Each day in the right column shows all responders (i.e.
OTUs that responded to xylose at any point in time). Greater enrichment in high
density fractions of the $^{13}$C-xylose treatment relative to control indicates
DNA is $^{13}$C-labeled.
    
\subsection{Figure 5}
Characteristics of xylose responders (green) and cellulose responders (blue)
based on estimated \textit{rrn} copy number (A), $\Delta\hat{BD}$ (B), and
relative abundance in non-fractionated DNA (C). The estimated \textit{rrn} copy
number of all responders is shown versus time (A). Kernel density histogram of
$\Delta\hat{BD}$ values shows cellulose responders had generally higher
$\Delta\hat{BD}$ than xylose responders indicating potentially greater $^{13}$C
incorporation per unit DNA (B). The final panel indicates the rank relative
abundance of all OTUs observed in the non-fractionated DNA (C) where rank was
determined at day 1 (bold line) but relative abundance for each OTU is
indicated for all days by colored lines (see legend). Xylose responders (green
ticks) have higher relative abundance than xylose responders (ticks are based
on day 1 relative abundance).
