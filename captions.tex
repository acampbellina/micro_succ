\subsection{Figure 1}
NMDS ordination of SSU rRNA gene sequence composition in
gradient fractions shows that variation between fractions is
correlated with fraction density, isotopic labeling, and
time. Dissimilarity in SSU rRNA gene sequence composition was quantified using the 
weighted UniFrac metric. SSU rRNA gene sequencess were surveyed in twenty
gradient fractions at each sampling point for each treatment (Figure~S1).
$^{13}$C-labeling of DNA is apparent because the SSU rRNA gene sequence composition of
gradient fractions from $^{13}$C and control treatments differ at high density.
Each point on the NMDS plot represents one gradient fraction.  SSU rRNA gene sequence
composition differences between gradient fractions were quantified by the
weighted Unifrac metric. The size of each point is positively correlated with
density and colors indicate the treatment (A) or day (B).
\subsection{Figure 2}
Enrichment of OTUs in either $^{13}$C-cellulose (13CCPS, upper panels) or
$^{13}$C-xylose (13CXPS, bottom panels) treatments relative to control,
expressed as LFC (see Methods). Each point indicates the LFC for a single OTU.
High enrichment values indicate an OTU is likely $^{13}$C-labeled. Different
colors represent different phyla and different panels represent different days.
The final column shows the frequency distribution of LFC values in each row.
Within each panel, shaded areas are used to indicate one standard deviation
(dark shading) or two standard deviations (light shading) about the mean of all
LFC values.
\subsection{Figure 3}
Phylogenetic position of cellulose responders and xylose responders in the
context of all OTUs that passed sparsity independent filtering criteria (see
Methods). Only those phyla that contain responders are shown.  Colored dots are
used to identify xylose responders (green) and cellulose responders (blue). The
heatmaps indicate enrichment in high density fractions relative to control
(represented as LFC) for each OTU in response to both $^{13}$C-cellulose
(13CCPS, leftmost heatmap) and $^{13}$C-xylose (13CXPS, rightmost heatmap) with
values for different days in each heatmap column. High enrichment values
(represented as LFC) provide evidence of $^{13}$C-labeled DNA.  

\subsection{Figure 4}
Xylose reponders in the \textit{Actinobacteria}, \textit{Bacteroidetes},
\textit{Firmicutes} exhibit distinct temporal dynamics of $^{13}$C-labeling.
The left column shows counts of $^{13}$C-xylose responders in the
\textit{Actinobacteria, Bacteroidetes, Firmicutes} and \textit{Proteobacteria}
at days 1, 3, 7 and 30. The right panel shows OTU enrichment in high density
gradient fractions (gray points, expressed as fold change) for responders
as well as a boxplot for the distribution of fold change values
(The box extends one interquartile range, whiskers extend 1.5 times
the IR, and small dots are outliers (i.e. beyond 1.5 times the IR)).
Each day in the right column shows all responders (i.e.
OTUs that responded to xylose at any point in time). High enrichment values 
indicates OTU DNA is likely $^{13}$C-labeled.
    
\subsection{Figure 5}
Characteristics of xylose responders (green) and cellulose responders (blue)
based on estimated \textit{rrn} copy number (A), $\Delta\hat{BD}$ (B), and
relative abundance in non-fractionated DNA (C). The estimated \textit{rrn} copy
number of all responders is shown versus time (A). Kernel density histogram of
$\Delta\hat{BD}$ values shows cellulose responders had higher average
$\Delta\hat{BD}$ than xylose responders indicating higher average atom \%
$^{13}$C in OTU DNA (B). The final panel indicates the rank relative abundance
of all OTUs observed in the non-fractionated DNA (C) where rank was determined
at day 1 (bold line) and relative abundance for each OTU is indicated for all
days by colored lines (see legend). Xylose responders (green ticks) have higher
relative abundance in non-fractionated DNA than cellulose responders (green ticks). 
All ticks are based on day 1 relative abundance.
