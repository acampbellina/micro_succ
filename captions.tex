\subsection{Figure 1}
NMDS analysis from weighted unifrac distances of 454 sequence data from SIP fractions of each treatment over time. Twenty fractions from a CsCl gradient fractionation for each treatment at each time point were sequenced (Fig. S1). Each point on the NMDS represents the bacterial composition based on 16S sequencing for a single fraction where the size of the point is representative of the density of that fraction and the colors represent the treatments (A) or days (B). 
\subsection{Figure 2}
Log$_{2}$ fold change of $^{13}$C-responders in cellulose
treatment (top) and xylose treatment (bottom).  Log$_{2}$ fold change is based
on the relative abundance in the experimental treatment compared to the control
within the density range 1.7125-1.755 g ml$^{-1}$. Taxa are
colored by phylum. The last column shows the distribution of all fold change values in each row. The darker gray band represents one standard deviation and the lighter band represents two standard deviations about the mean of all fold change values for both treatments.  

    
    \subsection{Figure 3}
Phylum specific trees. Heatmap indicates fold change between heavy fractions of control gradients versus labeled gradients. Dots indicate the position of "responders" to $^{13}$C-xylose (green) or $^{13}$C-cellulose (blue).\subsection{Figure 4}
Left column shows counts of $^{13}$C-xylose responders in the \textit{Actinobacteria, Bacteroidetes, Firmicutes} and \textit{Proteobacteria} at days 1, 3, 7 and 30. Right panel shows OTU fold enrichment in heavy gradient fractions for $^{13}$C-xylose amendment DNA relative to corresponding control fractions. 
    \subsection{Figure 5}
$^{13}$C-responder characteristics based on density shift (A) and rank (B).
Kernel density estimation of $^{13}$C-responder's density shift in cellulose
treatment (blue) and xylose treatment (green) demonstrates degree of labeling
for responders for each respective substrate. $^{13}$C-responders in rank
abundance are labeled by substrate (cellulose, blue; xylose, green). Ticks at top indicate
location of $^{13}$C-cylose responders in bulk community. Ticks at bottom indicate location of
$^{13}$C-cellulose responders in bulk community. OTU rank was assessed from day 1 bulk samples.