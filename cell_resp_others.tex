Other notable $^{13}$C-cellulose responders include a \textit{Bacteroidetes}
OTU that shares high sequence identity (99\%) to \textit{Sporocytophaga
myxococcoides} a known cellulose degrader \citep{Vance_1980}, and three
\textit{Actinobacteria} OTUs that share high sequence identity (100\%) with
sequenced cultured isolates. One of the three \textit{Actinobacteria}
$^{13}$C-cellulose responders is in the \textit{Streptomyces}, a genus known to
possess cellulose degraders, while the other two share high sequence identity
to cultured isolates \textit{Allokutzneriz albata} \citep{Labeda_2008,
Tomita_1993} and \textit{Lentzea waywayandensis} \citep{LABEDA_1989,
Labeda_2001}; neither isolate decomposes cellulose in culture. Nine
\textit{Plantomycetes} OTUs responded to $^{13}$C-cellulose but none are within
described genera (closest cultured isolate match 91\% sequence identity,
Table~\ref{tab:cell}) (Figure~\ref{fig:trees}). Interestingly, one
$^{13}$C-cellulose responder is annotated as ``cyanobacteria''.
The cyanobacteria phylum annotation is misleading as the OTU is not closely
related to any oxygenic phototrophs (closest cultured isolate match
\textit{Vampirovibrio chlorellavorus}, 95\% sequence identity,
Table~\ref{tab:cell}). A sister clade to the oxygenic phototrophs classically
annotated as ``cyanobacteria'' in SSU rRNA gene reference databases but does
not possess any known phototrophs has recently been proposed to constitute its own
phylum, "Melainabacteria" \citet{Di_Rienzi_2013}. Although the phylogenetic
position of ``Melainabacteria'' is debated \citep{Soo_2014}. The catalog of
metabolic capabilities associated with cyanobacteria (or candidate phyla
previously annotated as cyanobacteria) are quickly expanding
\citep{Di_Rienzi_2013, Soo_2014}.  Our findings provide evidence for cellulose
degradation within a lineage closely related to but apart from oxygenic
phototrophs. Notably, polysaccharide degradation is suggested by an analysis of
a ``Melainabacteria'' genome \citep{Di_Rienzi_2013}. Although we highlight
$^{13}$C-cellulose responders that share high sequence identity with described
genera, most $^{13}$C-cellulose responders uncovered in this experiment are not
closely related to cultured isolates (Table~\ref{tab:cell}).