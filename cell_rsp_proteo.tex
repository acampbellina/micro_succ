\par\textit{Proteobacteria} represent 46\% of all $^{13}$C-cellulose responding
OTUs identified. \textit{Cellvibrio} accounted for 3\% of all proteobacterial
$^{13}$C-cellulose responding OTUs detected. \textit{Cellvibrio} was one of the
first identified cellulose degrading bacteria and was originally described by
Winogradsky in 1929 who named it for its cellulose degrading abilities
\citep{boone2001bergeys}. All $^{13}$C-cellulose responding
\textit{Proteobacteria} share high sequence identity with 16S rRNA genes from
sequenced cultured isolates (Table~\ref{tab:cell}) except for ``OTU.442'' (best
cultured isolate match 92\% sequence identity in the \textit{Chrondomyces}
genus, Table~\ref{tab:cell}) and ``OTU.663'' (best cultured isolate match
outside \textit{Proteobacteria} entirely, \textit{Clostridium} genus, 89\%
sequence identity, Table~\ref{tab:cell}). Some \textit{Proteobacteria}
responders share high sequence identity with type strains for genera known to
possess cellulose degraders including \textit{Rhizobium}, \textit{Devosia},
\textit{Stenotrophomonas} and \textit{Cellvibrio}. One \textit{Proteobacteria}
OTU shares high sequence identity with a \textit{Brevundimonas} cultured
isolate.  \textit{Brevundimonas} has not previously been identified as a
cellulose degrader, but has been shown to degrade cellouronic acid, an oxidized
form of cellulose \citep{Tavernier_2008}.
