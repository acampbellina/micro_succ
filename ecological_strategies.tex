\subsection{Ecological strategies of xylose and cellulose degrading OTUs in soil}
HR-SIP allows us to assess C substrate specifity and temporal dynamics of
$^{13}$C-incorporation. C substrate specificity is assed by measureing the 
BD shift of OTU DNA upon $^{13}$C incorporation. OTUs that incorporate more
$^{13}$C per unit DNA have greater specify for the labeled substrate than
OTUs that incorporate less $^{13}$C per unit DNA. $^{13}$C-cellulose 
incorporating OTUs as a group displayed greater substrate specificity than
$^{13}$C-xylose incorporating OTUs. This suggests that polymeric C-degraders
tend to be specialists tuned to particular C-substrates such as cellulose
or lignin whereas labile C-degraders are generalists able to assimilate
C from many different labile sources. Although we observed a succession of 
$^{13}$C-xylose responders (Figure~\ref{fig:l2fc} and \ref{fig:xyl_count}), 
there was no discernible difference in substrate specificity between 
$^{13}$C-xylose responders that first responded at days 1, 3 or 7. There was,
however, a strong positive relationship between $^{13}$C-xylose responders that
first responded at days 1, 3 or 7 in \textit{rrn} copy number. $^{13}$C-xylose 
responders that first responded at day 1 had higher estimated \textit{rrn} copy 
number than responders that first responded at day 3 which had higher \textit{rrn}
copy number than responders that first responded at day 1. Therefore, OTUs that 
grow faster assimilate C from xylose faster intuitively. However, fast growers
are replaced with respect to C assimilation from xylose with slower growers as
xylose diminishes. There was a succession of xylose degraders with time from
fast growing spore-formers to \textit{Bacteroidetes} types and finally 
\textit{Actinobacteria} in out microcosms. The succession hypothesis of decomposition
groups ecological units by substrate CITE, however, our results suggest there is 
a succession of microbial activity for even a single substrate. Hence, in soil, there
is an ecological hierarchy coarsely defined by the ability to assimilate
C from labile or polymeric sources but then within the labile substrate 
degraders there are ecological subunits tuned to specific substrate concentrations.

