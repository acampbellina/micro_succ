NMDS analysis of SIP gradient fraction SSU rRNA gene sequence composition reveals
sequence composition is a function of fraction density, isotopic labeling, and
time. SSU rRNA gene compositon was profiled in twenty gradient fractions at each
sampling point for each treatment.  $^{13}$C-labeling of DNA is apparent because the SSU rRNA gene
composition of gradient fractions from $^{13}$C and control treatments differ
at high density. Each point on the NMDS plot represents one gradient fraction.
SSU rRNA gene composition differences between gradient fractions were
quantified by the weighted Unifrac metric. The size of each point is positively
correlated with density and colors indicate the treatment (A) or day (B).
