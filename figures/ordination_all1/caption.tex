NMDS ordination of SSU rRNA gene sequence composition in
gradient fractions shows that it is
a function of many factors including fraction density, isotopic labeling, and
time. Dissimilarity in SSU rRNA gene sequence composition was quantified using the 
weighted UniFrac metric. SSU rRNA gene sequencess were surveyed in twenty
gradient fractions at each sampling point for each treatment (Figure~S1).
$^{13}$C-labeling of DNA is apparent because the SSU rRNA gene sequence composition of
gradient fractions from $^{13}$C and control treatments differ at high density.
Each point on the NMDS plot represents one gradient fraction.  SSU rRNA gene sequence
composition differences between gradient fractions were quantified by the
weighted Unifrac metric. The size of each point is positively correlated with
density and colors indicate the treatment (A) or day (B).
