\section{Introduction}
 
There are 2,300 Pg of carbon (C) stored in soils worldwide, excluding plant biomass, which accounts for \sim 80\% of the global terrestrial C pool \cite{Amundson_2001,Mendelsohn_2001,IPCC2007Synth,elsen_Ayres_Wall_Bardgett_2011,Lal_2008,BATJES_1996,Lal_2008}. Current climate change models concur on atmospheric and ocean C predictions but not terrestrial \cite{Friedlingstein_2006}.  The disagreeable predictive power between models for terrestrial ecosystems reflects how little we know about belowground C cycling dynamics. It is estimated that 80-90\% of the C cycling in soil is mediated by microorganisms \cite{ColemanCrossley_1996,Nannipieri_2003}. Understanding microbial processing of nutrients in soils presents a special challenge due to the hetergeneous nature of soil ecosystems and our limitations in methodologies. Soils consist of an overwhelming biological, chemical, and physical complexity which affects microbial community composition, diversity, and structure (refs).  Confounding factors such as physical protection/aggregation, moisture content, pH, temperature, frequency and type of land disturbance, soil history, mineralogy, N quality and availability, and litter quality have all been shown to affect the ability of the soil microbial community to access and metabolize C substrates \cite{Schlesinger_1977,dgett_Wall_Hattenschwiler_2010,Sollins_Homann_Caldwell_1996,Torn_Vitousek_Trumbore_2005,TRUMBORE_2006,Schimel_2012}. Furthermore, rates of metabolism are often measured without knowing the identity of the microbial species specifically involved \cite{ndi_Pietramellara_Renella_2003} resulting in uncertainty in importance of community diversity in maintaining ecosystem functioning \cite{Allison_2008,ndi_Pietramellara_Renella_2003,Schimel_2012}. 

The first step in teasing out belowground C cycling dynamics is to identify microbial groups responsible for the measured process and understand the relationship between genetic diversity, community structure, and function \cite{O_Donnell_2002}. Stable-isotope probing (SIP) provides a unique opportunity to link microbial identity to activity \cite{Chen_Murrell_2010}. SIP has been utilized to expanded our knowledge of a myriad of important biogeochemical processes \cite{Chen_Murrell_2010}, yet, there remain limitations. Frequently utilized SIP-coupled molecular applications such as TRFLP, DGGE, and cloning provide insufficient resolution of taxon identity and depth of coverage and, to our knowledge, are usually conducted under the narrow scope of single substrate additions with few exceptions \cite{Lueders_2003,Chauhan_2009}. SIP studies use single substrate experimental designs to minimize isotope signal dilution, however, it detracts from how microbes may experience that substrate naturally, calling into question its environmental and biological relevance.

Degradative succession refers to the temporal changes in species or functional guilds that occurs during the sequential degradation of constituents of a nutrient resource \cite{townsend2003essentials,Bastian_2009}.  The decomposition of a nutrient source is hypothesized to promote succession of active community members as compounds are sequentially degraded \cite{Biddanda_1988}.  Differences in the degradation of labile C sources (ie. glucose, xylose, or sucrose) and complex C polymers (ie. cellulose or lignin) have been detected \cite{Engelking_2007,Anderson_1973,Stotzky_1961,Alden_2001,Furukawa_1996,Fontaine_2003,Blagodatskaya_2007,Jenkins_2010,Rui_2009,Fierer_2010}. A classic example of plant litter degradative succession is characterized by a series of stages in which sugar fungi dominate in stage one, followed by cellulolytic fungi in stage two, and lignin degrading fungi in the final stage \cite{Gessner_2010}.  This demonstrates not only the succession of detritivores but also the sequential degradation of litter constituents starting with consumption of the most labile C sources followed by degradation of more complex and polymeric C sources. These single substrate studies suggest that if a complex mixture of labile and polymeric C were added to soil two waves of degradation could be observed; labile C degradation early on and subsequent polymeric C degradation.  We propose this temporal cascade from labile C degraders preceeding the polymeric C degraders occurs in natural microbial communities, called herein microbial community succession.  


Goldfarb then functional guild literature (fierer) then propose that phyla do not serve as a single functional unit as many clades within a phyla may respond differently.    
\cite{Fierer_2012} - demonstrates that taxonomic and phylogenetic diversity does not confer functional diversity.  
It has been suggested that "Therefore, the abun- dant soil microbial taxa do not appear to exhibit a high degree of functional redundancy (or equiv- alence), "\cite{Fierer_2013} implying that community membership matters.  Another reason why we should study the specific microbial taxa consuming different C substrates.  

The aim of this study is to track the path of C added to soil as a complex C mixture to provide insight into these dynamic systems.  A previous study has shown that 13C labeled plant residues enable tracking of C through microbial pathways (Evershed et al 2006). Utilizing this technique with single 13C labeled substrates added as a complex C mixture allows us to test how different C substrates cascade through discrete taxa within the soil microbial community. Powerful techniques such as nucleic acid stable isotope probing (SIP) coupled with next generation sequencing can then be used to parse out and identify these 13C labeled portions of the microbial community to reveal the community members that are responsible for the transformation of the labeled C. Using these methods in compilation with microcosm incubations provides a means of looking into microbial community functions while minimizing other environmental factors affecting the fate of C. Ultimately we are able to identify specific organisms or functional guilds that are responsible for the cycling of specific C substrates.  


"Laboratory incubations of soil provide a controlled environment for understanding the mechanisms underlying changes in SOM chemistry during decomposition and their relationship to biological processes and C storage." Gandy, fierer, et al Geoderma 2009 

"There has been considera- ble debate in the field of ecology as to how the taxonomic diversity of communities relates to the observed functional, or trait-level, diversity (16); we found a strong positive correlation be- tween the functional and taxonomic diversity of soil microbial communities"\cite{Fierer_2013}
"Therefore, the abun- dant soil microbial taxa do not appear to exhibit a high degree of functional redundancy (or equiv- alence), as is often assumed (19); reductions in tax- onomic diversity were associated with decreases in the breadth of functional traits contained with- in these soil communities."\cite{Fierer_2013}

"2) there is a common decomposition sequence independent of plant inputs or other ecosystem properties; and 3) molecular decomposition sequences, although consistent, are not uni- form and can be altered, resulting in measurable and functional changes in soil C."\cite{Grandy_2008}

"We determinedthat simple substrateswere degradedby the same groups of organisms in both soils, and at similar rates, but pine litter was degraded by different microbial groups in the two soils, and at different rates. Thus as substrate complexity increased, the functional group responsible for its degradation became more distinct between the two soils."\cite{Waldrop_2004}



-------
" In addition, the analysis by phylogenetic and ecological groups, suggest that the communities respond in a similar trajectory of initial colonization first by heterotropic generalists and later specialists. "\cite{L_pez_Lozano_2013}
"Resource availability is also likely to be a funda- mental driver of microbial succession, but the limiting resources and environmental factors regulating succession will be more complex given the far greater physiological diversity contained within microbial communities and the breadth of environments in which succession can occur. During endoge- nous heterotrophic succession, labile substrates will be consumed first, supporting copiotrophic microbial taxa that are later replaced by more oligotrophic taxa that metabolize the remaining, more recalcitrant, organic C pools in the later stages of succession (Rui et al., 2009)."\cite{Fierer_2010}

"In correspondence with the dynamics of fatty acids, the bac- terial community showed a distinct succession during the course of residue decomposition"\cite{Rui_2009}
"dynamics of community structure seemed to be related to changes in the availability of carbon resources occurring during degradation"\cite{Bastian_2009}
