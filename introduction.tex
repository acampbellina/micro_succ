\section{Introduction} 
Importance of Carbon 

Globally, there are 2,300 Pg of carbon (C) stored in soils, excluding plant biomass, which accounts for ~80\% of the global terrestrial C pool (\cite{Amundson_2001}, IPCC 2000, IPCC 2007, Nielson et al 2011, http://rstb.royalsocietypublishing.org/content/363/1492/815.full - rattan lal 2008, \cite{BATJES_1996}). It is estimated that 80-90\% of the C cycling in soil is mediated by microorganisms (Coleman & Crossley 1996, Nannipieri & Badalucco 2003), however, understanding microbial processing of nutrients in soils presents a special challenge due to the overwhelming complexity of the soil ecosystem and our limitations in methodologies.  By nature, soils are challenging systems to study due to their heterogeneity and the presence of complex biological, chemical, and physical interactions.  These intricacies affect microbial community composition, diversity, and structure (refs) with confounding factors such as physical protection/aggregation, moisture content of the soil, pH, temperature, frequency and type of land disturbance, soil history, mineralogy, litter quality, and N quality and availability that dictate the fate of C (Gessner et al 2010, Sollins et al 1996, \cite{Torn_Vitousek_Trumbore_2005},  \cite{TRUMBORE_2006}). These factors have all been shown to affect the ability of the soil microbial community to access and metabolize C substrates (Schlesinger 1977). Furthermore, rates of metabolism are often measured without knowing the identity of the microbial species specifically involved in the cycling of the measured process (\cite{ndi_Pietramellara_Renella_2003}).  Therefore, the first step in teasing out this central problem is to identify microbial groups responsible for the measured process and understand the relations between genetic diversity, community structure, and function (O’Donnell et al 2001).  The importance of community diversity in maintaining ecosystem functioning remains uncertain (Allison & Martiny 2008, \cite{ndi_Pietramellara_Renella_2003})


The aim of this proposal is to track the path of C added to soil as a complex C mixture that chemically simulates switchgrass biomass, to provide insight into these complex systems.  A previous study has shown that 13C labeled plant residues enable tracking of C through microbial pathways (Evershed et al 2006).  Utilizing this technique with single 13C labeled substrates added as a complex C mixture will allow us to test how different C containing components move through soil trophic cascades. Powerful techniques such as nucleic acid stable isotope probing (SIP) coupled with 454 pyrosequencing can then be used to separate out and identify these 13C labeled portions of the microbial community to reveal the community members that are responsible for the transformation of C. Coupling “plant simulant” additions to microcosm incubations with nucleic acid stable isotope probing (SIP) may provide a means of looking into microbial community functions while minimizing other factors affecting the fate of C.  This will allow a researcher to sift out the specific organisms or functional guilds that are responsible for the degradation and cycling of that specific C substrate.  Furthermore, the processing of the C substrate can be traced over time whether it is assimilated into microbial biomass, respired as CO2, or incorporated into recalcitrant soil organic matter.


Degradative succession refers to the temporal changes in species or functional guilds that occurs during the sequential degradation of constituents of a nutrient resource (Townsend et al 2003).  The decomposition of a nutrient source is hypothesized to promote succession of active community members as compounds are sequentially degraded (Biddanda & Pomeroy 1988).  A classic example of plant litter degradative succession is characterized by a series of stages in which sugar fungi dominate in stage one, followed by cellulolytic fungi in stage two, and lignin degrading fungi in the final stage (Gessner et al 2010).  This demonstrates not only the succession of detritivores but also the sequential degradation of litter constituents starting with consumption of the most labile C sources followed by degradation of more complex and polymeric C sources. 

Plant residues represent the largest proportion of C entering soil systems and are composed primarily of cellulose making it the most abundant biopolymer globally (Paul & Clark 1989, Berg & Laskowski 2005, Klemm et al 2005, Coleman & Crossley 1996, Nannipieri & Badalucco 2003). Much work has been done to study cellulose degradation by microorganisms (Beguin & Aubert 1994, Lynd et al 2002, Haicher et al 2007, Eichorst & Kuske 2012).  With the advent of culture independent techniques, we have discovered that there are far more cellulose degraders than our limited culture collection represents.  Some studies have used labeled leaf litter to identify soil decomposers, however, this makes it difficult to differentiate which organisms are responsible for the actual degradation of the cellulose (Evershed et al 2006). Other cellulose degradation studies add cellulose as a single substrate addition to a cultured cellulose degrader or to soil (Haichar et al 2007, Li et al 2009, Eichorst & Kuske 2012).  This method may present false positives, as an organism that might not normally consume a given C substrate may opportunistically do so under the circumstances that it is the only C source available. Additionally, organisms may behave quite differently when given a large bolus of a nutrient source versus what they would actually encounter in nature.  

Differences in the degradation of labile C sources (ie. glucose, xylose, or sucrose) and complex C polymers (ie. cellulose or lignin) have been detected (Engelking et al 2007, Anderson and Domsch 1973, Stotzky and Norman 1961, Alden et al 2001, Nannipieri et al 2003).  Nearly complete degradation of sucrose within five days has been observed, while degradation of cellulose takes between 15-25 days (Engelking et al 2007).  Respiration from glucose-treated soils showed an initial increase in respiration at 2-6 hours followed by a second increase between 6-10 hours (Anderson and Domsch 1973, Stotzky and Norman 1961, Alden et al 2001, Nannipieri et al 2003).

It is likely that the slow degradation of cellulose can be attributed to the energy-taxing process of synthesizing cellulolytic enzymes and exporting them, as cellulose is broken down externally (Schimel & Schaeffer 2012, Lynd et al 2002).  As a result, microorganisms responsible for the synthesis of cellulases preferentially shuttle energy towards enzyme synthesis rather than biomass until cellulose hydrolysis begins (Schimel & Schaeffer 2012).  This accounts for the delay in growth and ultimately the slow decomposition of cellulose (Perez et al 2002, Schimel & Schaeffer 2012). 

These single substrate studies suggest that if a complex mixture of labile and polymer C were added to soil two waves of degradation could be observed; labile C degradation early on and subsequent polymer C degradation.  
Stable-isotope probing provides a unique opportunity to link microbial identity to activity (Chen & Murrell 2011).  

Since its development, the technique has been utilized for identifying key microorganisms and functional genes in many important biogeochemical processes (Chen & Murrell 2011).  However, only two studies have employed SIP to better understand microbial food webs (Lueders et al 2004b, Chauhan et al 2009). SIP studies have expanded our knowledge of important biogeochemical processes by linking microbial identity to function.  Yet, there remain limitations of experimental design, poor resolution using fingerprinting techniques, or sampling full diversity due to low throughput methods such as seen with clone libraries.   

By coupling SIP with 454 pyrosequencing previous studies have charted the diversity of organisms or genes with a directed focus towards microbes active in specific processes (Bell et al 2011, Lee et al 2011, Chen & Murrell 2010).  My goal is to expand the utility of coupled SIP-454 pyrosequencing to serve as a tool that will increase our ability to look at biogeochemical cycling.  Specifically, I aim to develop this methodology as a means to chart active communities over time in order to better understand what is happening to C once it enters an ecosystem. The variety of available isotopically labeled substrates and the sequencing depth obtained from pyrosequencing offers great potential for utilizing this technique to examine local cycling of a given substrate.  

In order to map the microbial food web, a 13C-carbon substrate is added to the soil and tracked in heavy fractions of extracted nucleic acids over time.  Traditionally, SIP is performed with a single time point in an attempt to minimize or eliminate cross-feeding of the substrate, as cross-feeding could cause false positives in the data set.  However, in a temporal food web mapping study, this cross-feeding is an exciting advantage of this system because it will enable us to track substrates of interest (i.e. cellulose) through many trophic cascades.  Observed organisms will exhibit a range of 13C labeling, 0-100%, with primary consumers being the most enriched and subsequent trophic levels being less enriched (Morris 2002, McDonald 2005, Ziegler 2005).  This relationship of trophic level consumption to dilution of label will facilitate tracking C as it moves through various operational taxonomic units (OTUs) in the soil.  



What we know 


What we don't know


Method limitations


How we propose to better this issue

With the rapid advancement and declining costs of high throughput sequencing, it has become increasingly easier to probe microbial communities.  In this study, we couple stable-isotope probing with 454 pyrosequencing in order to better understand organic matter decomposition dynamics as a function of soil microbial community C utilization. Overall, changes in microbial community composition over time is consistent with C decomposition being accompanied by a microbial community succession. 
