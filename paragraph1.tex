\section{Introduction}
\dropcap{W}e have only a rudimentary understanding of carbon flow through soil microbial communities. This deficiency is driven by the staggering complexity of soil microbial food webs and the opacity of these biological systems to current methods for describing microbial metabolism in the environment. Relating community composition to overall soil processes, such as nitrification and denitrification, which are mediated by defined functional groups has been a useful approach. However, carbon-cycling processes have proven more recalcitrant to study due to the wide range of organisms participating in these reactions and our inability to discern diagnostic functional genetic markers.

Excluding plant biomass, there are 2,300 Pg of carbon (C) stored in soils worldwide which accounts for $\sim$80\% of the global terrestrial C pool \cite{Amundson_2001,BATJES_1996}. When organic C from plants reaches soil it is degraded by fungi, archaea, and bacteria. This C is rapidly returned to the atmosphere as CO$_{2}$ or remains in the soil as humic substances that can persist up to 2000 years \cite{yanagita1990natural}. The majority of plant biomass C in soil is respired and produces 10 times more CO$_{2}$ than anthropogenic emissions on an annual basis \cite{chapin2002principles}. Global changes in atmospheric CO$_{2}$, temperature, and ecosystem nitrogen inputs, are expected to impact primary production and C inputs to soils \cite{Groenigen_2006} but it remains difficult to predict the response of soil processes to anthropogenic change \cite{DAVIDSON_2006}. Current climate change models concur on atmospheric and ocean C predictions but not terrestrial \cite{Friedlingstein_2006}. These contrasting terrestrial ecosystem model predictions reflect how little is known about soil C cycling dynamics and it has been suggested that incosistencies in terrestial modeling could be improved by elucidating the relationship between dissolved organic carbon and microbial communities in soils \cite{Neff_2001}. 

An estimated 80-90\% of C cycling in soil is mediated by microorganisms \cite{ColemanCrossley_1996,Nannipieri_2003}. Understanding microbial processing of nutrients in soils presents a special challenge due to the hetergeneous nature of soil ecosystems and methods limitations. Soils are biologically, chemically, and physically complex which affects microbial community composition, diversity, and structure \cite{Nannipieri_2003}. Confounding factors such as physical protection/aggregation, moisture content, pH, temperature, frequency and type of land disturbance, soil history, mineralogy, N quality and availability, and litter quality have all been shown to affect the ability of the soil microbial community to access and metabolize C substrates \cite{Sollins_Homann_Caldwell_1996,Kalbitz_2000}. Further, rates of metabolism are often measured without knowing the identity of the microbial species involved \cite{ndi_Pietramellara_Renella_2003} leaving the importance of community membership towards maintaining ecosystem functions unknown \cite{Allison_2008,ndi_Pietramellara_Renella_2003,Schimel_2012}. Litter bag experiments have shown that the community composition of soils can have quantitative and qualitative impacts on the breakdown of plant materials \cite{Schimel_1995}. Reciprocal exchange of litter type and microbial inocula under controlled environmental conditions reveals that differences in community composition can account for 85\% of the variation in litter carbon mineralization \cite{Strickland_2009}. In addition, assembled communities of cellulose degraders reveal that the composition of the community has significant impacts on the rate of cellulose degradation \cite{Wohl_2004}. 
