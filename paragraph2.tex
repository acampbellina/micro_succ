Excluding plant biomass, there are 2,300 Pg of carbon (C) stored in soils worldwide which accounts for $\sim$80\% of the global terrestrial C pool \cite{Amundson_2001,BATJES_1996}. When organic C from plants reaches soil it is degraded by fungi, archaea, and bacteria. This C is rapidly returned to the atmosphere as CO$_{2}$ or remains in the soil as humic substances that can persist up to 2000 years \cite{yanagita1990natural}. The majority of plant biomass C in soil is respired and produces 10 times more CO$_{2}$ than anthropogenic emissions on an annual basis \cite{chapin2002principles}. Global changes in atmospheric CO$_{2}$, temperature, and ecosystem nitrogen inputs, are expected to impact primary production and C inputs to soils \cite{Groenigen_2006} but it remains difficult to predict the response of soil processes to anthropogenic change \cite{DAVIDSON_2006}. Current climate change models concur on atmospheric and ocean C predictions but not terrestrial \cite{Friedlingstein_2006}. These contrasting terrestrial ecosystem model predictions reflect how little is known about soil C cycling dynamics and it has been suggested that incosistencies in terrestial modeling could be improved by elucidating the relationship between dissolved organic carbon and microbial communities in soils \cite{Neff_2001}. 