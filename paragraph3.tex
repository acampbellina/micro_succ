An estimated 80-90\% of C cycling in soil is mediated by microorganisms \cite{ColemanCrossley_1996,Nannipieri_2003}. Understanding microbial processing of nutrients in soils presents a special challenge due to the hetergeneous nature of soil ecosystems and methods limitations. Soils are biologically, chemically, and physically complex which affects microbial community composition, diversity, and structure \cite{Nannipieri_2003}. Confounding factors such as physical protection/aggregation, moisture content, pH, temperature, frequency and type of land disturbance, soil history, mineralogy, N quality and availability, and litter quality have all been shown to affect the ability of the soil microbial community to access and metabolize C substrates \cite{Sollins_Homann_Caldwell_1996,Kalbitz_2000}. Further, rates of metabolism are often measured without knowing the identity of the microbial species involved \cite{ndi_Pietramellara_Renella_2003} leaving the importance of community membership towards maintaining ecosystem functions unknown \cite{Allison_2008,ndi_Pietramellara_Renella_2003,Schimel_2012}. Litter bag experiments have shown that the community composition of soils can have quantitative and qualitative impacts on the breakdown of plant materials \cite{Schimel_1995}. Reciprocal exchange of litter type and microbial inocula under controlled environmental conditions reveals that differences in community composition can account for 85\% of the variation in litter carbon mineralization \cite{Strickland_2009}. In addition, assembled communities of cellulose degraders reveal that the composition of the community has significant impacts on the rate of cellulose degradation \cite{Wohl_2004}. 
