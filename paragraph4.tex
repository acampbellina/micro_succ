An important step in understanding soil C cycling dynamics is to identify individual contributions of discrete microorganisms and to investigate the relationship between genetic diversity, community structure, and function \cite{O_Donnell_2002}. The vast majority of microorganisms continue to resist cultivation in the laboratory, and even when cultivation is achieved, the traits expressed by a microorganism in culture may not be representative of those expressed when in its natural habitat. Stable-isotope probing (SIP) provides a unique opportunity to link microbial identity to activity and has been utilized to expand our knowledge of a myriad of important biogeochemical processes \cite{Chen_Murrell_2010}. The most successful applications of this technique have identified organisms which mediate processes performed by a narrow set of functional guilds such as methanogens \cite{Lu_2005}. The technique has been less applicable to the study of soil C cycling because of limitations in resolving power as a result of simultaneous labeling of many different organisms in the community. Additionally, molecular applications such as TRFLP, DGGE, and cloning that are frequently used in conjunction with SIP provide insufficient resolution of taxon identity and depth of coverage. We have developed an approach that employs a complex mixture of substrates added to soil at a low concentration relative to soil organic matter pools along with massively parallel DNA sequencing. This greatly expands the ability of nucleic acid SIP to explore complex patterns of C-cycling in microbial communities with increased resolution.