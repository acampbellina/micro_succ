A temporal cascade occurs in natural microbial communities during the plant biomass degradation in which labile C degradation preceeds polymeric C \cite{Hu_1997,Rui_2009}. The aim of this study is to track the temporal dynamics of C assimilation through discrete individuals of the soil microbial community to provide greater insight into soil C-cycling. Our experimental approach employs the addition of a soil organic matter (SOM) simulant (a complex mixture of model carbon sources and inorganic nutrients common to plant biomass), where a single C constituent is substituted for its \textsuperscript{13}C-labeled equivalent, to soil. Parallel incubations of soils amended with this complex C mixture allows us to test how different C substrates cascade through discrete taxa within the soil microbial community. In this study we use \textsuperscript{13}C-xylose and \textsuperscript{13}C-cellulose as a proxy for labile and polymeric C, respectively. Using a novel approach we couple nucleic acid stable isotope probing with next generation sequencing (SIP-NGS) to elucidating soil microbial community members responsible for specific C transformations. Amplicon sequencing of 16S rRNA gene fragments from many gradient fractions and multiple gradients make it possible to track C assimilation by hundreds of different taxa. Ultimately we identify discrete microorganisms responsible for the cycling of specific C substrates. 