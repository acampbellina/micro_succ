\section{Results}
We ran a temporal series of parallel microcosms and measured the changes in the microbial community as result of the addition of a complex carbon mixture and 454 pyrosequenced the bulk microbial community and fractions from SIP gradient fractionation (Fig. S1). Bulk community analysis demonstrates a community dominated by (insert taxa here) at days 1, 3, and 7 followed by a transistion to a (insert taxa here) dominated community at days 14 and 30. Most abundant OTUs at earlier time points were most closely related to members of Bacteriodetes, Actinomycetes, and Proteobacteria. Abundant OTUs of later time points were most closely related to members of Chloroflexi, Proteobacteria, and Verrucomicrobia. Twenty fractions from a SIP gradient fractionation for each treatment at each time point were sequenced. Using PCoA analysis from weighted unifrac distances the relationship between all the fractions from all treatments and time points are projected into the ordination space (Fig X). Each point on the PCoA represents the microbial community from a single fraction where the size of the point is representative of the density of that fraction (the smaller the point, the less dense), the shape represents X, and the colors (color1, color 2, etc) represents the days (1,3,7,14, and 30 - respectively). PCoA1 demonstrates a separation based on density (large to small points). PCoA2 demonstrates fractions of greater density (1.73-1.74...figure out exact densities) separating away from the control at days 1,3, and 7 for the xylose treatment and days 14 and 30 for the cellulose treatment. 13C-labeled organisms are expected be to found in the denser fractions (large points, Fig x). Using biplots/Edge PCA, potentially 13C-responsive OTUs were resolved.  Targeting potentially responsive OTUs, their relative abundance in all fractions can be traced for an experimental treatment and compared to its relative abundance in the control fractions from the same time point (Fig Y).       
          

 