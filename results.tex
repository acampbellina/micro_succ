Bulk community analysis demonstrates a community dominated by (insert taxa here) at days 1, 3, and 7 followed by a transistion to a (insert taxa here) dominated community at days 14 and 30. Most abundant OTUs at earlier time points were mostly closely related to members of Bacteriodetes, Actinomycetes, and Proteobacteria.  Abundant OTUs of later time points were most closely related to members of Chloroflexi, Proteobacteria, and Verrucomicrobia.  Twenty fractions from a SIP gradient fractionation for each treatment at each time point were sequenced.  Using PCoA analysis from weighted unifrac distances the relationship between all the fractions from all treatments and time points are projected into the ordination space (Fig X).  Each point on the PCoA represents the microbial community from a single fraction (where the size of the point is representative of the density of that fraction; the smaller the point, the less dense). PCoA1 demonstrates a separation based on denisty (large to small points).  PCoA2 demonstrates fractions of greater density (1.73-1.74...figure out exact densities) separating away from the control at days 1,3, and 7 for the xylose treatment and days 14 and 30 for the cellulose treatment.  13C-labeled organisms are expected be to found in the denser fractions (large points, Fig x).  Using biplots/Edge PCA, we are able to resolve discrete OTUs that are responsible for the separation in ordination space.  
  
  indicating a difference in microbial community composition. 
         

PCoA demonstrates microbial succession and based on xylose and cellulose falling out separately, it indicates different microbial community members are responsible for C degradation.  Using X ordination we can tease out members that cause the greatest shift in treatment versus control.  We then generate C utilization charts to demonstration discrete OTUs in control versus treatment.  Bulk community shifts over time.  