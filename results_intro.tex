\section{Results}
To observe C use dynamics by the soil microbial community, we conducted a
nucleic acid SIP experiment wherein xylose or cellulose carried the isotopic
label, and, we assayed SSU rRNA gene content of CsCl gradient fractions using
high-throughput DNA sequencing technology. We set up three soil microcosm series. 
Microcosms in each series were were amended with a C substrate mixture
that included cellulose and xylose. The C substrate mixture approximated
freshly degrading plant biomass. The same substrate mixture was added to microcosms
in each series, however, for each series except the control, one substrate
was substituted for its $^{13}$C counterpart. In one series cellulose was
$^{13}$C-labeled in another xylose was $^{13}$C-labeled and in the control
series no sustrates were $^{13}$C labeled. Microcosm amendments are shorthand identified
in the following figures by the following code: "13CXPS" refers to the
amendment with $^{13}$C-xylose (that is $^{13}$\textbf{C} \textbf{X}lose
\textbf{P}lant \textbf{S}imulant), "13CCPS" refers to the $^{13}$C-cellulose
amendment and "12CCPS" refers to the amendment that only contained $^{12}$C
substrates. Xylose or cellulose were chosen to carry the isotopic label to
contrast C assimilation for labile, soluble C (xylose) versus
insoluble, polymeric C (cellulose).  5.3 mg of C substrate mixture per gram
soil was added to each microcosm representing 18\% of the total soil C. The
mixture included 0.42 mg xylose-C and 0.88 mg cellulose-C g soil$^{-1}$.
Microcosms were harvested at days 1, 3, 7, 14 and 30  during a 30 day
incubation. $^{13}$C-xylose assimilation peaked immediately and tapered
over the 30 day incubation whereas $^{13}$C-cellulose assimilation peaked at
two weeks of (Figure~\ref{fig:ord}).

We sequenced SSU rRNA gene smplicons from a total of 277 CsCl gradient fractions from 14 
CsCl gradients and 12 bulk microcosm DNA samples. The SSU rRNA gene data set contained 1,376,008 
total sequences. The average number of sequences per sample was 3,816 (sd 3,629) and 265 samples 
had over 1,000 sequences. We sequenced SSU rRNA gene amplicons from an average of 19.8 fractions
per CsCl gradient (sd 0.57). The average density between fractions was  0.0040 g mL$^{-1}$ 
The sequencing effort recoverd a total of 5,940 OTUs. 2,943 of the total
5,940 OTUs were observed in bulk samples. We observed 33 unique phylum and 340 unique genus
annotations.