\subsection{Putative spore-formers in the Firmicutes assimilate $^{13}$C from
xylose within first day after soil amendment followed by Bacteroidetes and then
Actinobacteria OTUs} 
Within the first 7 days of incubation 63\% on average of $^{13}$C-xylose was
respired and only an additional 6\% more was respired from day 7 to 30. At
the end of the 30 day incubation 30\% of the $^{13}$C from added xylose
remained in the soils. The $^{13}$C remaining in the soil from $^{13}$C-xylose
addition was likely stabilized by assimilation into microbial biomass
and/or microbial conversion into other forms of organic matter. It is also
possible that some $^{13}$C-xylose remains unavailable to microbes due to
abiotic interactions in soil \citep{Kalbitz_2000}. 

At day 1, 84\% of $^{13}$C-xylose responsive OTUs belong to
\textit{Firmicutes}, 11\% to \textit{Proteobacteria} and 5\% to
\textit{Bacteroidetes}. At day 3, \textit{Firmicutes} responders decreased to
5\% (from 16 OTUs to 1) while \textit{Bacteroidetes} increased to 63\% (from 1
to 12 OTUs) of day 3 responders. The remaining day 3 responders are members of
the \textit{Proteobacteria} (26\%) and the \textit{Verrucomicrobia} (5\%). Day
7 responders were 53\% \textit{Actinobacteria}, 40\% \textit{Proteobacteria},
and 7\% \textit{Firmicutes}. The identities of $^{13}$C-xylose responders
change with time. The numerically dominant $^{13}$C-xylose responder phylum
shifts from \textit{Firmicutes} to \textit{Bacteroidetes} and then to
\textit{Actinobacteria} across days 1, 3 and 7 (Figure~\ref{fig:l2fc},
Figure~\ref{fig:xyl_count}). 