All of the $^{13}$C-xylose responders in the \textit{Firmicutes} phylum are
closely related (at least 99\% sequence identity) to cultured isolates from
genera that are known to form endospores (Table~\ref{tab:xyl}). Each
$^{13}$C-xylose responder is closely related to isolates annotated as members
of \textit{Bacillus}, \textit{Paenibacillus} or \textit{Lysinibacillus}.
\textit{Bacteroidetes} $^{13}$C-xylose responders are predominantly closely
related to \textit{Flavobacterium} species (5 of 8 total responders)
(Table~\ref{tab:xyl}.  Only one \textit{Bacteroidetes} $^{13}$C-xylose
responder is not closely related to a cultured isolate, ``OTU.183'' (closest
LTP BLAST hit, \textit{Chitinophaca sp.}, 89.5\% sequence identity,
Table~\ref{tab:xyl}). OTU.183 shares high sequence identity with environmental
clones derived from rhizosphere samples (accession AM158371, unpublished) and
the skin microbiome (accession JF219881, \citet{Kong_2012}). Other
\textit{Bacteroidetes} responders share high sequence identities with canonical
soil genera including \textit{Dyadobacer}, \textit{Solibius} and
\textit{Terrimonas}. Six of the 8 \textit{Actinobacteria} $^{13}$C-xylose
responders are in the \textit{Micrococcales} order. One $^{13}$C-xylose
responding \textit{Actinobacteria} OTU shares 100\% sequence identity with
\textit{Agromyces ramosus} (Table~\ref{tab:xyl}).  \textit{A. ramosus} is a
known predatory bacterium but is not dependent on a host for growth in culture
\citep{16346402}. It is not possible to determine the specific origin of
assimilated $^{13}$C in a DNA-SIP experiment. $^{13}$C can be passed down
through trophic levels although heavy isotope representation in C pools
targeted by cross-feeders and predators would be diluted with depth into the
trophic cascade.  It's possible, however, that the $^{13}$C labeled
\textit{Agromyces} OTU was assimilating $^{13}$C primarily by predation if the
\textit{Agromyces} OTU was selective enough with respect to its prey that it
primarily attacked $^{13}$C-xylose assimilating organisms and that those
$^{13}$C-xylose assimilating organisms utilized $^{13}$C-xylose as a sole
carbon source.