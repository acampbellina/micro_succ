\subsection{Variation in bulk soil DNA microbial community structure is
significantly less than variation in gradient fractions} 
Using a distance metric that incorporates relative abundance information
(weighted Unifrac metric, \citep{Lozupone_2005}) bulk sample beta diversity was
less than gradient fraction beta diversity (p-value 0.003). Time was
significantly correlated to bulk sample phylogenetic profile variation (p-value
0.23, R$^{2}$ 0.63, Figure~\ref{fig:bulk_ord}) but the contrast between only
$^{12}$C additions with additions that included isotopically labeled substrates
was not (p-value 0.35). When responder abundances were summed within phyla for
$^{13}$C-cellulose and $^{13}$C-xylose, summed resonder abundaces generally 
increased for $^{13}$C-cellulose and \textit{Firmicutes}, \textit{Bacteroidetes},
\textit{Actinobacteria} and \textit{Proteobacteria}, the numerically dominant
$^{13}$C-xylose responder phyla, decreased over time although 
\textit{Proteobacteria} spiked at day 14 (Figure~\ref{fig:babund}). Bulk abundance trends are
roughly consistent with $^{13}$C assimilation activity. 