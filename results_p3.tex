\subsection{Ordination of CsCl gradient fraction OTU profiles can be used to
observe fraction-level $^{13}$C assimilation dynamics and membership differences}
Each CsCl gradient fraction possesses a unique composition of SSU rRNA gene
phylogenetic types. DNA buoyant density (BD) drives differences in CsCl
gradient fraction SSU rRNA gene composition. For
instance, lighter DNA is more abundant in fractions at lighter densities so
DNA with lower G+C will be found in greater abundance at the light end of the
CsCl gradient and vice versa.  Duplicate gradients receiving only $^{12}$C DNA
with the same bulk or non-fractionated SSU rRNA gene phylogenetic composition 
would have the same overall profile of SSU rRNA gene phylogenetic types across 
the density gradient. We fed microcosms identical C substrate mixtures save 
for the identity of a $^{13}$C labeled substrate, and by design, DNA from all 
microcosms harvested at a time point will be similar in bulk phylogenetic 
composition. Therefore, SSU rRNA gene profile differences between between gradients 
harvested at the same time are due to $^{13}$C incorporation into bulk community DNA. 
$^{13}$C-DNA shifts from its $^{12}$C position towards the heavy end of the density gradient. This causes heavy fractions in gradients that received $^{13}$C-DNA to be different in
phylogenetic content than corresponding heavy fractions from gradients that
received $^{12}$C-DNA of the same bulk phylogenetic composition.