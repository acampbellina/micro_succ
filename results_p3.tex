\subsection{Ordination of CsCl gradient fraction OTU profiles can be used to
observe fraction-level $^{13}$C assimilation dynamics and membership differences}
Each CsCl gradient fraction possesses a unique composition of SSU rRNA gene
phylogenetic types. DNA buoyant density (BD) drives differences in CsCl
gradient fraction SSU rRNA gene composition. For
instance, lighter DNA is more abundant in fractions at lighter densities so
DNA with lower G+C will be found in greater abundance at the light end of the
CsCl gradient and vice versa.  Duplicate gradients receiving only $^{12}$C DNA
with the same bulk or non-fractionated SSU rRNA gene phylogenetic composition 
would have the same overall profile of SSU rRNA gene phylogenetic types across 
the density gradient. We fed microcosms identical C substrate mixtures save 
for the identity of a $^{13}$C labeled substrate, and by design, DNA from all 
microcosms harvested at a time point will be similar in bulk phylogenetic 
composition. Therefore, SSU rRNA gene profile differences between between gradients 
harvested at the same time are due to $^{13}$C incorporation into bulk community DNA. 
$^{13}$C-DNA shifts from its $^{12}$C position towards the heavy end of the density gradient. This causes heavy fractions in gradients that received $^{13}$C-DNA to be different in
phylogenetic content than corresponding heavy fractions from gradients that
received $^{12}$C-DNA of the same bulk phylogenetic composition.o

Ordination of CsCl gradient fraction phylogenetic profiles reveals differences
and similarities between gradients. It's clear that microcosms incorporated
$^{13}$C from both $^{13}$C-xylose and $^{13}$C-cellulose as gradients from
both $^{13}$C-xylose and $^{13}$C-cellulose microcosms differ from
corresponding control gradients (Figure~\ref{fig:ord}). These differences from
control gradients are focused in the heavy fractions (Figure~\ref{fig:ord}).
Analysis of SSU rRNA gene surveys has
greatly benefited from utilizing conventional methods for data exploration
in ecology such as ordination \citep{Lozupone_2008}.  SSU rRNA gene
phylogenetic profiles in CsCl gradient fractions have only recently been
surveyed with high-throughput DNA sequencing technology and subsequently
explored via ordination \citep{Angel_2013, Verastegui_2014}. Ordination of CsCl
gradient fraction phylogenetic profiles has reveled the relative influence of
buoyant density and soil type on gradient phylogenetic profile variance, 
however, ordination has not demonstrated isotope incorporation.  Demonstrating
isotope incorporation requires careful comparisons between control and labeled
gradients over the same buoyant density range. By sequencing CsCl gradient
fractions from both control and labeled gradients across the full density
gradient with DNA harvested from microcosms at multiple time points, we can
observe where in the density gradient $^{13}$C isotope incorporation signal is
strongest and when $^{13}$C isotope incorporation begins (Figure~\ref{fig:ord}).
$^{13}$C incorporation from xylose and cellulose is most apparent at days 1/3/7
and days 14/30, respectively (Figure~\ref{fig:ord}). Moreover, labeled gradient
fraction phylogenetic profiles diverge from controls most dramatically at
relatively heavy buoyant densities (Figure~\ref{fig:ord}). Also, $^{13}$C-DNA from $^{13}$C-xylose microcosms is different in phylogenetic composition from $^{13}$C-cellulose microcosm $^{13}$C-DNA indicating that xylose and cellulose were assimilated by different microbial community members (Figure~\ref{fig:ord}). Lastly, ordination indicates
organisms that assimilated $^{13}$C from $^{13}$C-xylose changed in phylogenetic type over incubation days 1, 3 and 7 (Figure~\ref{fig:ord}).