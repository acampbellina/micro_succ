Ordination of CsCl gradient fraction phylogenetic profiles reveals differences
and similarities between gradients. It's clear that microcosms incorporated
$^{13}$C from both $^{13}$C-xylose and $^{13}$C-cellulose as gradients from
both $^{13}$C-xylose and $^{13}$C-cellulose microcosms differ from
corresponding control gradients (Figure~\ref{fig:ord}). These differences from
control gradients are focused in the heavy fractions (Figure~\ref{fig:ord}).
Analysis of SSU rRNA gene surveys has
greatly benefited from utilizing conventional methods for data exploration
in ecology such as ordination \citep{Lozupone_2008}.  SSU rRNA gene
phylogenetic profiles in CsCl gradient fractions have only recently been
surveyed with high-throughput DNA sequencing technology and subsequently
explored via ordination \citep{Angel_2013, Verastegui_2014}. Ordination of CsCl
gradient fraction phylogenetic profiles has reveled the relative influence of
buoyant density and soil type on gradient phylogenetic profile variance, 
however, ordination has not demonstrated isotope incorporation.  Demonstrating
isotope incorporation requires careful comparisons between control and labeled
gradients over the same buoyant density range. By sequencing CsCl gradient
fractions from both control and labeled gradients across the full density
gradient with DNA harvested from microcosms at multiple time points, we can
observe where in the density gradient $^{13}$C isotope incorporation signal is
strongest and when $^{13}$C isotope incorporation begins (Figure~\ref{fig:ord}).
$^{13}$C incorporation from xylose and cellulose is most apparent at days 1/3/7
and days 14/30, respectively (Figure~\ref{fig:ord}). Moreover, labeled gradient
fraction phylogenetic profiles diverge from controls most dramatically at
relatively heavy buoyant densities (Figure~\ref{fig:ord}). Also, $^{13}$C-DNA from $^{13}$C-xylose microcosms is different in phylogenetic composition from $^{13}$C-cellulose microcosm $^{13}$C-DNA indicating that xylose and cellulose were assimilated by different microbial community members (Figure~\ref{fig:ord}). Lastly, ordination indicates
organisms that assimilated $^{13}$C from $^{13}$C-xylose changed in phylogenetic type over incubation days 1, 3 and 7 (Figure~\ref{fig:ord}).