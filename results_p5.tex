\subsection{$^{13}$C from cellulose was assimilated by canonical
cellulose-degrading and uncharacterized microbial lineages in many phyla
including Chloroflexi and Verrucomicrobia} 
Isotope incorporation by an OTU is revealed by enrichment of the OTU in heavy
CsCl gradient fractions containing $^{13}$C labeled DNA relative to heavy
fractions from control gradients containing no $^{13}$C labeled DNA. We refer
to OTUs that putatively incorporated $^{13}$C into DNA from an isotopically
labeled substrate as a substrate ``responder''. Only 2 and 5 OTUs were found to
have incorporated $^{13}$C from $^{13}$C-cellulose at days 3 and 7,
respectively. At days 14 and 30, however, 42 and 39 OTUs were found to
incorporate $^{13}$C from $^{13}$C-cellulose into biomass. An average 16\% of
the $^{13}$C-cellulose added was respired within the first 7 days, 38\% by day
14, and 60\% by day 30.  A \textit{Cellvibrio} and \textit{Sandaracinaceae} OTU
assimilated $^{13}$C from $^{13}$C-cellulose at day 3. Day 7 $^{13}$C-cellulose
responders included the same \textit{Cellvibrio} responder as day 3, a
\textit{Verrucomicrobia} OTU and three \textit{Chloroflexi} OTUs.  50\% of Day
14 responders belong to \textit{Proteobacteria} (66\% Alpha-, 19\% Gamma-, and
14\% Beta-) followed by 17\% \textit{Planctomycetes}, 14\%
\textit{Verrucomicrobia}, 10\% \textit{Chloroflexi}, 7\%
\textit{Actinobacteria} and 2\% cyanobacteria.  \textit{Bacteroidetes} OTUs
begin to incoporate $^{13}$C from cellulose at day 30 (13\% of day 30
responders). Other day 30 responding phyla include \textit{Proteobacteria}
(30\% of day 30 responders; 42\% Alpha-, 42\% Delta, 8\% Gamma-, and 8\%
Beta-), \textit{Planctomycetes} (20\%), \textit{Verrucomicrobia} (20\%),
\textit{Chloroflexi} (13\%) and cyanobacteria (3\%). \textit{Proteobacteria},
\textit{Verrucomicrobia}, and \textit{Chloroflexi} had relatively high numbers
of responders with strong response across multiple time points
(Figure~\ref{fig:l2fc}).