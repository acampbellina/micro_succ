\textit{Verrucomicrobia}, a cosmopolitan soil phylum often found in high
abundance \citep{Fierer_2013}, are hypthothesized to degrade polysaccharides in
many environments \citep{Fierer_2013,Herlemann_2013,10543821}.
\textit{Verrucomicrobia} comprise 16\% of the total $^{13}$C-cellulose
responder OTUs detected. 40\% of \textit{Verrucomicrobia} $^{13}$C-cellulose
responders belong to the uncultured ``FukuN18'' family originally identified in
freshwater lakes \citep{Parveen_2013}.  The \textit{Verrucomicrobia} OTU with
the strongest \textit{Verrucomicrobial} response to $^{13}$C-cellulose shared
high sequence identity (97\%) with an isolate from Norway tundra soil
\citep{Jiang_2011} although growth on cellulose was not assessed for this
isolate. Only one other $^{13}$C-cellulose responding verrucomicrobium shared
high DNA sequence identity with a sequenced type strain, ``OTU.638''
(Table~\ref{tab:cell}) with \textit{Roseimicrobium gellanilyticum} (100\%
sequence identity).  \textit{Roseimicrobium gellanilyticum} grows on soluble
cellulose \citep{Otsuka_2012}. The remaining $^{13}$C-cellulose
\textit{Verrucomicrobia} responders did not share high sequence identity with
any cultured isolates (maximum sequence identity with any cultured isolate
93\%).