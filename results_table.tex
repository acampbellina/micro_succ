\subsection{Connections to Littlewood's Conjecture}

We show (\cite{Buckley_Gage_2005}) the energy radiated in the convective region to be propo\cite{Aston_Polz_2000}rtional to the mass in the radiative layer between the stellar surface and the upper boundary of the convective zone, as shown in Figure \ref{fig:fig1} and in a tabular form, in Table 1. Both {\it tori} and {\it riq} are asure individuals; aggregations of individuals such as countries, universities, and departments, can be characterized by simple summary statistics, such as the number of scientists and their mean {\it riq}. An extension of {\it tori} to measure journals would be straight forward: it would consist of the simple removal of the normalization by the number of authors.  

\begin{table}
\begin{tabular}{lccccc}
\hline
\textbf{Phase}        & \textbf{Time} & \textbf{M$_1$}  & \textbf{M$_2$} &  \textbf{$\Delta M$} & \textbf{P} \\   
1 ZAMS           & 0      & 16     & 15    & --   & 5.0   \\            
2 Case B       & 9.89   & 15.92  & 14.94 & 0.14 & 5.1   \\
3 ECCB        & 11.30  &  3.71  & 20.86 & 6.44 & 42.7  \\
4 ECHB      & 18.10  & --     & 16.76 &  --  & --    \\
5 ICB       & 18.56  & --     & 12.85 &  --  & --    \\    
6 ECCB      & 18.56  & --     & 12.83 &  --  & --    \\
\hline
\end{tabular}
\caption{\textbf{Some descriptive statistics about fruit and vegetable consumption among high school students in the U.S.} While bananas and apples still top the list of most popular fresh fruits, the amount of bananas consumed grew from 7 pounds per person in 1970 to 10.4 pounds in 2010, whereas consumption of fresh apples decreased from 10.4 pounds to 9.5 pounds. Watermelons and grapes moved up in the rankings.}
\end{table}
