\subsection{Estimated rrn gene copy number in substrate responder groups}
$^{13}$C-xylose responder estimated \textit{rrn} gene copy number is inversely
related to time of first response (p-value 2.02x10$^{-15}$,
Figure~\ref{fig:copy}). OTUs that first respond at later time points have fewer
estimated \textit{rrn} copy number than OTUs that first respond earlier
(Figure~\ref{fig:copy}). \textit{rrn} copy number estimation is a recent
advance in microbiome science \citep{Kembel_2012} while the relationship of
\textit{rrn} copy number per genome with ecological strategy is well
established \citep{Klappenbach_2000}.  Microorganisms with a high \textit{rrn} 
copy number tend to be fast growers specialized to take advantage of boom-bust
environments whereas microorganisms with low \textit{rrn} copy number favor
slower growth under lower and more consistent nutrient input
\citep{Klappenbach_2000}. At the beginning of our incubation, OTUs with
estimated high \textit{rrn} copy number or ``fast-growers''
assimilate xylose into biomass and with time slower growers (lower \textit{rrn} 
copy number) begin to incorporate $^{13}$C from xylose.  Further,
$^{13}$C-xylose responders have more estimated rRNA operon copy numbers per
genome than $^{13}$C-cellulose responders (p-value 1.878x10$^{-09}$) suggesting
xylose respiring microbes are generally faster growers than cellulose
degraders.
