\subsection{Xylose responders at day 1 have more estimated rRNA operon copy
numbers per genome than xylose responders at days 3 and 7, and, Xylose
responders have more rRNA operon copy numbers
than cellulose responders.}
$^{13}$C-xylose responder rRNA operon genome copy number is inversely related
to time of first response (p-value 2.02e$^{-15}$, Figure~\ref{fig:copy}). OTUs
that first respond at later time points have fewer estimated rRNA operons per
genome than OTUs that first respond earlier (Figure~\ref{fig:copy}). rRNA
operon copy number estimation is a recent advance in microbiome science
\citep{Kembel_2012} while the relationship of rRNA operon copy number per genome
with ecological strategy is well established \citep{Klappenbach_2000}.
Microorganisms with a high number of rRNA operons per genome tend
to be fast growers specialized to take advantage of boom-bust environments
whereas microorganisms with low rRNA operon copy numbers per genome favor
slower growth under lower and more consistent nutrient input
\citep{Klappenbach_2000}. At the beginning of our incubation, OTUs with
estimated high rRNA operon copy numbers per genome or ``fast-growers''
assimilate xylose into biomass and with time slower growers (lower rRNA operon
number per genome) begin to respond to the xylose addition.  Further,
$^{13}$C-xylose responders have more estimated rRNA operon copy numbers per
genome than $^{13}$C-cellulose responders (p-value 1.878e$^{-09}$) suggesting
xylose respiring microbes are generally faster growers than cellulose
degraders.