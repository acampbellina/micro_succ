\subsection{Figure S1}
We added a carbon mixture that contained inorganic
salts and amino acids (not shown here) to each soil microcosm where the
only difference between treatments was the $^{13}$C-labeled isotope (in red). At days
1, 3, 7, 14, and 30 replicate microcosms were destructively harvested for
downstream molecular applications. DNA from each treatment and time (n = 14)
was subjected to CsCl density gradient centrifugation and density gradients
were fractionated (orange tubes wherein each arrow represents a fraction from
the density gradient). SSU rRNA genes from each gradient fraction were PCR
amplified and sequenced. In addition, SSU rRNA genes were also PCR amplified
and sequenced from non-fractionated DNA to represent the soil microbial
community.

\subsection{Figure S2}
The metabolization of $^{13}$C-xylose and $^{13}$C-cellulose is indicated by
the percentage of the added $^{13}$C that remains in soil over time. Soils
were pooled (three samples per time point per treatment) prior to measuring
$^{13}$C-content.
\subsection{Figure S3}
NMDS analysis of SSU rRNA gene composition in non-fractionated DNA (colored points) indicates
that isotopic labelling does not alter overall microbial community composition,
microbial community composition in the soil microcosms changes over time, and
variance in non-fractionated DNA is smaller than variance in fractionated DNA (black points).
SSU rRNA gene sequences were determined for non-fractionated DNA from the
unlabeled control, $^{13}$C-xylose, and $^{13}$C-cellulose treatments over time (colors
indicate time, different symbols used for different treatments). Distance in SSU
rRNA gene composition was quantified with the UniFrac metric. The
leftmost panel indicates NMDS of data from both non-fractionated and
fractionated samples. The rightmost panel indicates NMDS of data only from
non-fractionated DNA. Statistical analysis is presented in main text. Samples not
represented in the ordination did not sequence successfully and constitute
missing data (e.g. ``12CCPS'' day 7).
\subsection{Figure S4}
Change in non-fractionated DNA relative abundance versus time (expressed
as LFC) for OTUs that changed significantly over time (P-value $<$ 0.10, Wald test).
Each panel shows one phylum (labeled on the right). The taxonomic class is
indicated on the left. OTUs that responded to just xylose are shown in 
green, just cellulose in blue, and both xylose and cellulose in red.  

\subsection{Figure S5}
Relative abundance in non-fractionated DNA versus time for classes that changed significantly.
Samples from different treatments are labeled with different colors as
indicated in the scale. Statistical analysis is presented in main text.

\subsection{Figure S6}
Change in relative abundance in non-fractionated DNA over time for xylose
responders (13CXPS) and cellulose responders (13CCPS). Each panel represents
a responders to the indicated substrate (i.e. cellulose (13CCPS) or xylose (13CXPS)) 
within the indicated phylum except for the lower right panel which shows all reponders to both
xylose and celluose. The abbreviations Proteo., Verruco., and Plancto.,
correspond to \textit{Proteobacteria}, \textit{Verrucomicrobia}, and \textit{Planctomycetes},
respectively.

\subsection{Figure S7}
Maximum enrichment at any point in time in high density fractions
of $^{13}$C-treatments relative to control (expressed as LFC) shown
for $^{13}$C-cellulose versus $^{13}$C-xylose treatments. Each point
represents an OTU. Blue points are cellulose responders, green xylose
responders, red are responders to both xylose and cellulose, and gray points
are OTUs that did not respond to either substrate. Line indicates a slope of
one. 

\subsection{Figure S8}
Counts of xylose responders and cellulose responders over time. \subsection{Figure S9}
Raw data from individual responders highlighted in the main text (see
Results). The left column shows OTU relative abundance in density gradient 
fractions for the indicated treatment pair at each sampling point. Time is
indicated by the line color (see legend). Gradient profiles are shaded
to represent the different treatments where orange represents ``control'',
blue ``$^{13}$C-cellulose'', and green ``$^{13}$C-xylose.'' The right column
shows the relative abundance of each OTU in non-fractionated DNA. Enrichment in the
high density fractions of $^{13}$C-treatments indicates an OTU likely
has $^{13}$C-labeled DNA. 

\subsection{Figure S10}
Estimated \textit{rrn} copy number for xylose and cellulose responders. The leftmost panel contrasts estimated \textit{rrn} copy number for cellulose (13CCPS) and xylose (13CXPS) responders. The right panel shows estimated \textit{rrn} copy number versus time of first response for xylose responders. Colors denote the phylum of the OTUs (see legend).  \subsection{Figure S11}
Density profile for a single cellulose responder in the $^{13}$C-cellulose
treatment (blue) and control (orange). Vertical lines show center of mass for
each density profile and the arrow denotes the magnitude and direction of
$\Delta\hat{BD}$. Right panel shows relative abundance values in the high
density fractions (The boxplot line is the median value. The box spans one
interquartile range (IR) about the median, whiskers extend 1.5 times the IR.   

\subsection{Figure S12}
Density profile for a single non-responder OTU. The $^{13}$C-cellulose
treatment is in blue and the control treatment is in orange. The vertical line
shows where high density fractions begin as defined in our analysis. The right
panel shows relative abundance values in the high density fractions for each
gradient (The boxplot line is the median value. The box spans one
interquartile range (IR) about the median, whiskers extend 1.5 times the IR.  

