\documentclass{article}
\usepackage{amsfonts,amsmath,amssymb}
\usepackage{graphicx}
\usepackage{color}
\usepackage{url}
\usepackage{lineno}
\usepackage{xr}
\externaldocument{full_paper}

\usepackage[numbers,sort&compress]{natbib}

\usepackage{geometry}

\usepackage{multirow, array, booktabs, longtable, threeparttablex}
\newcolumntype{P}[1]{>{\raggedright\arraybackslash}p{#1}}

\usepackage{caption}

\usepackage{hyperref}
\hypersetup{
    colorlinks,
    citecolor=black,
    filecolor=black,
    linkcolor=black,
    urlcolor=black
}

\newcommand{\beginsupplement}{%
\setcounter{table}{0} \renewcommand{\thetable}{S\arabic{table}}%
\setcounter{figure}{0} \renewcommand{\thefigure}{S\arabic{figure}}% 
}

\bibliographystyle{pnas}

\title{Supplemental Information}
\date{\vspace{-5ex}}

\begin{document}

\linenumbers[1222]

\maketitle

\tableofcontents

%\listoffigures 

%\listoftables

\section{Supplemental Discussion}
\subsection{Supplemental note 1 -- Phylogenetic affiliation of cellulose responders}
% Fakesubsubsection: Verrucomicrobia comprise
\textit{Verrucomicrobia} represented 16\% of the cellulose responders.
\textit{Verrucomicrobia} are cosmopolitan soil microorganisms \citep{Bergmann_2011}
that can make up to 23\% of SSU rRNA gene sequences in soils
\citep{Bergmann_2011} and 9.8\% of soil SSU rRNA \citep{Buckley_2001}. Genomic
analyses and laboratory experiments show that various isolates within the
\textit{Verrucomicrobia} are capable of methanotrophy, diazotrophy, and
cellulose degradation \citep{Wertz_2011,Otsuka_2012}. Moreover,
\textit{Verrucomicrobia} have been hypothesized to degrade polysaccharides in
many environments \citep{Fierer_2013,10543821,Herlemann_2013}. However, the
role of soil \textit{Verrucomicrobia} in global C-cycling remains unknown. The
majority of verrucomicrobial cellulose responders belonged to two clades that
fell within the \textit{Spartobacteria} (Figure~\ref{fig:tiledtree}).
\textit{Spartobacteria} outnumbered all other \textit{Verrucomicrobia}
phylotypes in SSU rRNA gene surveys of 181 globally distributed soil samples
\citep{Bergmann_2011}. Given their ubiquity and abundance
in soil as well as their demonstrated incorporation of $^{13}$C from
$^{13}$C-cellulose, \textit{Verrucomicrobia} lineages, particularly
\textit{Spartobacteria}, may be important contributors to global cellulose
turnover. 

% Fakesubsubsection: Other notable cellulose responders
Other notable cellulose responders include OTUs in the \textit{Planctomycetes}
and \textit{Chloroflexi} both of which have previously been shown to
assimilate $^{13}$C from $^{13}$C-cellulose added to soil
\citep{Schellenberger_2010}. \textit{Planctomycetes} are common in soil
\citep{Janssen2006}, comprising 4 to 7\% of bacterial cells in many soils
\citep{Zarda_1997,Chatzinotas_1998} and 7\% $\pm$ 5\% of SSU rRNA
\citep{buckley_2003}. Although soil \textit{Planctomycetes} are widespread,
their activities in soil remain uncharacterized. \textit{Plantomycetes}
represented 16\% of cellulose responders and shared $<$ 92\% SSU rRNA gene
sequence identity to their most closely related cultured isolates.
\textit{Chloroflexi} are known for metabolically diverse lifestyles ranging
from anoxygenic phototrophy to organohalide respiration \citep{Hug_2013} and
are among the six most abundant bacterial phyla in soil \citep{Janssen2006}.
Recent studies have focused on \textit{Chloroflexi} roles in C cycling
\citep{Hug_2013,Goldfarb_2011,Cole_2013} and several \textit{Chloroflexi}
isolates use cellulose \citep{Hug_2013,Goldfarb_2011,Cole_2013}. Four of the
five \textit{Chloroflexi} cellulose responders belong to a single clade within
the \textit{Herpetosiphonales} (Figure~\ref{fig:tiledtree}). 

% Fakesubsubsection: Finally, a single
Finally, a single cellulose responder belonged to the \textit{Melainabacteria}
phylum (95\% shared SSU rRNA gene sequence identity with \textit{Vampirovibrio
chlorellavorus}). The phylogenetic position of \textit{Melainabacteria} is
debated but \textit{Melainabacteria} have been proposed to be
a non-phototrophic sister phylum to \textit{Cyanobacteria}. An analysis of
a \textit{Melainabacteria} genome \citep{Di_Rienzi_2013} suggests the
genomic capacity to degrade polysaccharides though \textit{Vampirovibrio
chlorellavorus} is an obligate predator of green alga \citep{gromov_1972}.

\subsection{Supplemental note 2 -- Implications for soil-C models}
% Fakesubsubsection: Biogeochemical processes
Biogeochemical processes mediated by a broad array of taxa are assumed
insensitive to community change relative to
processes mediated by a narrow suite of microorganisms
\citep{Schimel_1995,McGuire2010}. In addition, the diversity of
a functionally defined group engaged in a specific C transformation is
expected to correlate positively with C lability \citep{McGuire2010}.
However, the diversity of labile C and structural C decomposers in soil
has not been quantified directly. We found comparable numbers of OTUs
responded to $^{13}$C-cellulose and $^{13}$C-xylose (63 and~49,
respectively). Cellulose responders were phylogenetically clustered
suggesting that the ability to degrade cellulose is phylogenetically
conserved. The clade depth of cellulose responders, 0.028 SSU rRNA gene
sequence dissimilarity, is on the same order as that observed for
glycoside hydrolases which are diagnostic enzymes for cellulose
degradation \citep{Berlemont2013}. Xylose responders clustered in terminal
branches indicating groups of closely related taxa metabolized xylose but
xylose responders also clustered phylogenetically with respect to time of
response (Figure~\ref{fig:tiledtree}, Figure~\ref{fig:xyl_count}).
For example, xylose responders on day~1 are dominated by members of
\textit{Paenibacillus}. Thus, microorganisms that degraded labile C and
structural C were both limited in diversity. Although the genes for xylose
metabolism are likely widespread in the soil community, it's possible only
a limited diversity of organisms had the ecological characteristics
required to degrade xylose under experimental conditions. Therefore it's
possible that only a limited number of taxa actually participate in the
metabolism of labile C-sources under a given set of conditions, and hence
changes in community composition may alter the dynamics of structural
\textit{and} labile C-transformations in soil.

% Fakesubsubsection: Broadly, we observed
Broadly, we observed labile C use by fast growing generalists and structural
C use by slow growing specialists. These results agree with the MIMICS model
which simulates leaf litter decomposition by modeling microbial decomposers
as two functionally defined groups, copiotrophs or oligotrophs
\citep{wieder_2014a}. Including these functional types improved predictions
of C storage in response to environmental change. We identified
microbial lineages engaged in labile and structural C decomposition that
can be defined as copiotrophs or oligotrophs, respectively. We highlight two
additional considerations for soil-C process models based on our results.
First, soil-C may travel through multiple trophic levels within the bacterial
community where each C transfer represents an opportunity for C stabilization
in association with soil minerals or C loss by respiration. And second,
although labile C consumption is generally considered to be a broad process in
terms of microbial participants, we observed that only a small number of
related OTUs conclusively consumed xylose-C (see SI for additional discussion) and
that fast growth, as opposed the ability to use xylose, may constrain the diversity
of microorganisms that process labile-C \textit{in situ} which may often be
pulse delivered and transient. The diversity of microbial participants in a
biogeochemical process is thought to determine how robust process rates are to
changes in community composition. Our understanding of soil C dynamics will
likely improve as we develop a more granular understanding of the ecological
diversity of microorganisms that mediate C transformations in soil.

\subsection{Supplemental note 3 -- Evidence for trophic C exchange}
% Fakesubsubsection: Responders did not necessarily
Responders did not necessarily assimilate $^{13}$C directly
from $^{13}$C-xylose or $^{13}$C-cellulose but, in many ways, knowledge of
secondary C degradation and/or microbial biomass turnover may be more
interesting with respect to the soil C-cycle than knowledge of primary
degradation. The response to xylose suggests xylose-C moved through different
trophic levels within the soil bacterial food web. The \textit{Bacilli}
degraded xylose first (65\% of the xylose-C had been respired by day~1)
representing 84\% of day~1 xylose responders. \textit{Bacilli} also comprised
about 6\% of SSU rRNA genes present in non-fractionated DNA on day~1. However,
few \textit{Bacilli} remained $^{13}$C-labeled by day~3 and their abundance
declined reaching about 2\% of soil SSU rRNA genes by day~30. Members of the
\textit{Bacillus} \citep{Cleveland2007} and \textit{Paenibacillus} in
particular \citep{Verastegui_2014} have been previously implicated as labile
C decomposers. The decline in relative abundance of \textit{Bacilli} could be
attributed to mortality and/or sporulation coupled to mother cell lysis.
\textit{Bacteroidetes} OTUs appeared $^{13}$C-labeled at day~3 concomitant with
the decline in relative abundance and loss of $^{13}$C-label for
\textit{Bacilli}. Finally, \textit{Actinobacteria} appeared $^{13}$C-labeled at
day~7 as \textit{Bacteroidetes} xylose responders declined in relative
abundance and became unlabeled. Hence, it seems reasonable to propose that
\textit{Bacteroidetes} and \textit{Actinobacteria} xylose responders became
labeled via the consumption of $^{13}$C derived from $^{13}$C-labeled microbial
biomass as opposed to primary degradation of $^{13}$C-xylose. 

% Fakesubsubsection: Trophic transfer could be due
The inferred physiology of \textit{Actinobacteria} and \textit{Bacteroidetes}
xylose responders provides further evidence for C transfer by
saprotrophy and/or predation. Most of the \textit{Actinobacteria} xylose
responders that appeared $^{13}$C-labeled at day~7 were members of the
\textit{Micrococcales} (Figure~\ref{fig:tiledtree}) and the most abundant
$^{13}$C-labeled \textit{Micrococcales} OTU at day~7 (“OTU.4”,
Table~\ref{tab:xyl}) is annotated as belonging in the \textit{Agromyces}.
\textit{Agromyces} are facultative predators that feed on the gram-positive
\textit{Micrococcus luteus} in culture \citep{16346402}. Additionally, certain types
of \textit{Bacteroidetes} can assimilate $^{13}$C from $^{13}$C-labeled
\textit{Escherichia coli} added to soil \citep{Lueders2006}.
Alternatively, it is possible that \textit{Bacilli},
\textit{Bacteroidetes}, and \textit{Actinobacteria} are adapted to use
xylose at different concentrations and that the observed activity dynamics
resulted from changes in xylose concentration over time and/or that
\textit{Actinobacteria} and \textit{Bacteroidetes} xylose responders
consumed waste products generated by primary xylose metabolism (e.g.
organic acids produced during xylose metabolism). These latter two
hypotheses cannot explain the sequential loss of $^{13}$C-label in combination
with the abundance dynamics in non-fractionated DNA, however.
If trophic transfer caused the activity dynamics, at least three different
ecological groups exchanged C in~7 days. Models of the soil C cycle often
exclude trophic interactions between soil bacteria (e.g.
\citep{Moore1988}), yet when soil C models do account for predators and/or
saprophytes, trophic interactions are predicted to have significant
effects on the fate of soil C \citep{Kaiser2014a}. 

\subsection{Supplemental Note 4 -- Major C components of plant biomass}
% Fakesubsubsection: The degradative succession
Most plant C is comprised of cellulose (30-50\%) followed by hemicellulose
(20-40\%), and lignin (15-25\%) \citep{Lynd2002}. Hemicellulose, being the most
soluble, degrades in the early stages of decomposition. Xylans are often an
abundant component of hemicellulose, and xylose is often the most abundant
sugar in hemicellulose, comprising as much as 60-90\% of xylan in some plants
(e.g  switchgrass \citep{Bunnell2013}). 

\section{Supplemental Methods} 
\subsection{Soil Collection and Preparation} 
% Fakesubsubsection: We collected soils from
We collected soils from an organic farm in Penn Yan, New York. Soils were
Honoeye/Lima, a silty clay loam on calcareous bedrock. The agricultural
field site has been described previously \citep{Berthrong_2013}. To get a field average,
cores (5 cm diameter x 10 cm depth) were collected in duplicate from six
different sampling locations around the field using a slide hammer bulk density
sampler (coordinates: (1) N 42$^{\circ}$ 40.288’ W 77$^{\circ}$ 02.438’, (2)
N 42° 40.296’ W 77$^{\circ}$ 02.438’, (3) N 42$^{\circ}$ 40.309’ W 77$^{\circ}$
02.445’, (4) N 42$^{\circ}$ 40.333’ W 77$^{\circ}$ 02.425’, (5) N 42$^{\circ}$
   40.340’ W 77$^{\circ}$ 02.420’, (6) N 42$^{\circ}$ 40.353’ W 77$^{\circ}$
      02.417’) on November 21,
2011. Soil cores were sieved (2mm), homogenized by mixing, and stored at
      4 $^{\circ}$C until pre-incubation (within 1-2 week of collection).
        Carbon (C) and nitrogen (N) content were previously measured for these
        soils \citep{Berthrong_2013}. Reported soil C values for the organic
        field were
12.15 ($\pm$ s.d. 0.78) mg C g$^{-1}$ dry soil and 1.16 ($\pm$ s.d. 0.13) mg
   N g$^{-1}$ dry soil \citep{Berthrong_2013}. 

\subsection{Cellulose production}
% Fakesubsubsection: Bacterial cellulose was produced by
Bacterial cellulose was produced by \textit{Gluconoacetobacter xylinus} grown
in Heo and Son \citep{Heo_2002} minimal media (HS medium) made with 0.1\%
glucose and without inositol. This cellulose while tractable for use in the lab
will not possess the structural complexity of lignocellulosic plant biomass and thus
we cannot account for the effects plant biomass structural complexity on which 
microorganisms utilize cellulose in soil .For the production of $^{13}$C-cellulose,
$^{13}$C$_{6}$-D-glucose, 99 atom \% $^{13}$C (Cambridge Isotope) was used. Cellulose
was produced in 1L Erlenmeyer flasks containing 100 mL HS medium inoculated
with three colonies of \textit{Gluconoacetobacter xylinus} grown on HS agar
plates. Flasks were incubated statically in the dark at 30$^{\circ}$C for 2-3
weeks. Cellulose pellicules were decanted, rinsed with deionized water,
suspended in two volumes of 1\% Alconox, and then autoclaved. Cellulose
pellicules were purified by dialysis for 12 hr in 1 L deionized water and
dialysis was repeated~10 times. Harvested pellicules were dried overnight
(60$^{\circ}$C), cut into pieces, and ground using a 5100 Mixer/Mill (SPEX
SamplePrep, Metuchen, NJ), and dry sieved to 53 $\mu$m - 250 $\mu$m. The particulate
size range was selected to be representative of particulate organic matter in
soils (3).

% Fakesubsubsection: The purity of ground cellulose
The purity of ground cellulose was checked by biological assay, Benedict's
reducing sugars assay, Bradford assay, and isotopic analysis. \textit{E. coli}
is not able to use cellulose as a C source but is capable of growth on
a variety of nutrients available in the HS medium. The biological assay
consisted of \textit{E. coli} inoculated into minimal M9 media which lacked
a C source and was supplemented with either: (1) 0.01\% glucose, (2) 2.5 mg
purified, ground cellulose, (3) 25 mg purified, ground cellulose, (4) 25 mg
purified, ground cellulose and 0.01\% glucose. Growth in media was checked by
spectrometer (OD$_{450}$). No measurable growth was observed with either~2 mg
or~25 mg cellulose, indicating absence of contaminating nutrients that can
support growth of \textit{E. coli}. In addition, the presence of 25 mg
cellulose did not inhibit the growth of \textit{E.coli} cultures provided with
glucose (relative to control), indicating the absence of compounds in the
purified cellulose that could inhibit microbial growth.

% Fakesubsubsection: Purified cellulose was also assayed
Purified cellulose was also assayed for residual proteins and sugars using
Bradford and Benedict's assays, respectively. Bradford assay was performed as
in \citep{Bradford_1976}. Ground, purified cellulose contained 6.92 $\mu$g protein mg
cellulose$^{-1} (\textit{i.e.}$ 99.31\% purity). Reducing sugars were not
detected in cellulose using Benedict's reducing sugar assay
\citep{benedict1909reagent} tested at 10 mg cellulose ml$^{-1}$. Finally,
$^{13}$C-cellulose had an average 96\% $\pm$
5 (s.d.) degree of $^{13}$C labeling as determined by isotopic analysis
  (UCDavis Stable Isotope Facility).           

\subsection{Soil microcosms}
% Fakesubsubsection: Microcosms were created
Microcosms were created by adding 10 g d.w. sieved soil to a 250 mL
Erlenmeyer flask capped with a butyl rubber stopper. The headspace was flushed
with air every 3 days which was sufficient to prevent anoxia (data not shown).
Microcosms were pre-incubated at room temperature for 2 weeks until the soil
respiration rate (determined by GCMS measurement of headspace CO$_{2}$) had
stabilized. Sieving causes a transient increase in soil respiration rate
presumably due to the liberation of fresh labile soil organic matter
\citep{Datta_2014}. Pre-incubation ensures that this labile organic matter is
consumed and/or stabilized prior to the beginning of the experiment.
Respiration rate (CO$_{2}$) stabilized after 10 days (data not shown). 

% Fakesubsubsection: Three parallel treatments
Three parallel treatments were established. Each treatment received the same
amendment, where the only difference was the isotopically labeled component in
the amendment. The treatments included an unlabeled control treatment and
treatments that substituted either $^{13}$C-cellulose (synthesized as described
above) or $^{13}$C$_{5}$-D-xylose (98 atom \% $^{13}$C (Isotec)) for their
unlabeled equivalents. The molecular composition of the amendment was designed
to approximate switchgrass biomass with hemicellulose replaced by its
constituent monomers \citep{Yan_2010,David_2010}. The amendment was added
at~5.3 mg g$^{-1}$ d.w. soil which is representative of natural concentrations
in soil during early phases of decomposition \citep{Schneckenberger_2008}. The
amendment contained by mass: 38\% cellulose, 23\% lignin, 20\% xylose, 3\%
arabinose, 1\% galactose, 1\% glucose, and 0.5\% mannose, 10.6\% amino acids
(Teknova C0705), and 2.9\% Murashige Skoog basal salt mixture which contains
macro- and micro-nutrients that are associated with plant biomass (Sigma
Aldrich M5524). The amendment had a C:N ratio of
10. Cellulose (2 mg cellulose g$^{-1}$ d.w. soil) and lignin (1.2 mg lignin g-1
d.w. soil) were uniformly distributed over the soil surface as a powder and
the remaining constituents were added in solution in a volume of
0.12 ml g$^{-1}$ d.w. soil. The volume of liquid was determined in relation to
soil moisture to achieve 50\% water holding capacity. Water holding capacity
of 50\% was chosen, in relation to the texture for this soil, to achieve
approximately 70\% water filled pore space, which is the optimal water content
for respiration \citep{Linn_1984}. A total of 12 microcosms were established
for the $^{13}$C-xylose treatment and 10 for the $^{13}$C-cellulose treatment.
Microcosms were sampled destructively on days 1, 3, 7, 14, and 30 and soils
were frozen at -80 $^{\circ}$C. The cellulose treatment was not sampled on day
1 because it was not expected that significant cellulose metabolism would have
occurred within this time. The abbreviation “13CXPS” refers to the
$^{13}$C-xylose treatment ($^{13}$C Xylose Plant Simulant), “13CCPS” refers to
the $^{13}$C-cellulose treatment and “12CCPS” refers to the unlabeled control.
A subset of soil from each sample was reserved for isotopic analysis at the
Cornell University Stable Isotope Laboratory to determine the mass of $^{13}$C
remaining in soil.

\subsection{Nucleic acid extraction}
% Fakesubsubsection: Nucleic acids were extracted
Nucleic acids were extracted from 0.25 g soil using a modified Griffiths
protocol \citep{Griffiths_2000}. Cells were lysed by bead beating for
1 min at 5.5 m s$^{-1}$ in 2 mL lysis tubes containing 0.5 g of 0.1 mm diameter
  silica/zirconia beads (treated at 300 $^{\circ}$°C for 4 hours to remove RNAses), 0.5 mL
  extraction buffer (240 mM Phosphate buffer 0.5\% N-lauryl sarcosine), and 0.5
  mL phenol-chloroform-isoamyl alcohol (25:24:1) for 1 min at
5.5 m s$^{-1}$. After lysis, 85 $\mu$L 5 M NaCl and 60 $\mu$L 10\% hexadecyltriammonium
  bromide (CTAB)/0.7 M NaCl were added to lysis tube, vortexed, chilled for
  1 min on ice, and centrifuged at 16,000 x g for 5 min at 4°C. The aqueous
  layer was transferred to a new tube and reserved on ice. To increase DNA
  recovery, the pellet was back extracted with 85 $\mu$L 5 M NaCl and
0.5 mL extraction buffer. The aqueous extract was washed with 0.5 mL
chloroform:isoamyl alcohol (24:1). Nucleic acids were precipitated by addition
of 2 volumes polyethylene glycol solution (30\% PEG 8000, 1.6 M NaCl) on ice
for 2 hrs, followed by centrifugation at 16,000 x g, 4°C for 30 min. The
supernatant was discarded and pellets were washed with 1 mL ice cold 70\% EtOH.
Pellets were air dried, resuspended in 50 $\mu$L TE and stored at -20
$^{\circ}$C. To prepare nucleic acid extracts for isopycnic centrifugation as
previously described \citep{Buckley_2007}, DNA was size selected ($>$ 4kb)
using 1\% low melt agarose gel and $\beta$-agarase I enzyme extraction per
manufacturers protocol (New England Biolab, M0392S).  Final resuspension of DNA
pellet was in 50 $\mu$L TE.   

\subsection{Isopycnic centrifugation and fractionation} 
% Fakesubsubsection: We fractionated DNA on density gradients
We fractionated DNA on density gradients for $^{13}$C-xylose treatments (days
1, 3, 7, 14, 30), $^{13}$C-cellulose treatments (days 3, 7, 14, 30), and
control treatments (days 1, 3 ,7, 14, 30). A total of 5 $\mu$g DNA was added to
each~4.7 mL CsCl density gradient.  Density gradient were composed of 1.69
g mL$^{-1}$ CsCl ml$^{-1}$ in gradient buffer solution (pH 8.0 15 mM
Tris-HCl,~15 mM EDTA, 15 mM KCl). Centrifugation was performed at 55,000 rpm~20
$^{\circ}$C for 66 hr using a TLA-110 rotor in a Bechman Coulter Optima MAX-E
ultracentrifuge. Fractions of $\sim$100 $\mu$L were collected from below by
displacing the DNA-CsCl-gradient buffer solution in the centrifugation tube
with water using a syringe pump at a flow rate of 3.3 $\mu$L s$^{-1}$
\citep{Manefield_2002}. Fractions were collected in Acroprep 96 filter plates (part no. 5035, Pall Life
Sciences). The refractive index of each fraction was measured using a Reichart
AR200 digital refractometer modified as previously described to measure
a volume of 5 $\mu$L \citep{Buckley_2007}. Buoyant density was calculated from
the refractive index as previously described \citep{Buckley_2007} using the
equation $\rho$=\textit{a}$\eta$-\textit{b}, where $\rho$ is the density of the
CsCl (g ml$^{-1}$), $\eta$ is the measured refractive index, and \textit{a} and
\textit{b} are coefficient values of 10.9276 and 13.593, respectively, for CsCl
at 20 $^{\circ}$C \citep{9780408708036}. The refractive index (Ri) was
corrected to account for the Ri of the gradient buffer using the equation: 
$Ri_{corrected} = Ri_{observed} - (Ri_{buffer} - 1.3333)$. A total of 35
gractions were collected from each gradient and the average density between
fractions was~0.0040 g mL$^{-1}$. The DNA was desalted by washing with TE (5X
200 $\mu$L) in the Acroprep filter wells. DNA was resuspeneded in~50 $\mu$L TE. 

\subsection{DNA Sequencing}
\subsubsection{PCR amplification of SSU rRNA genes} 
% Fakesubsubsection: SSU rRNA genes
SSU rRNA genes were amplified from gradient fractions (n = 20 per
gradient) and from non-fractionated DNA from soil. Barcoded primers
consisted of: 454-specific adapter B, a 10 bp barcode (Reference
90), a 2 bp linker (5’-CA-3’), and 806R primer for reverse primer
(BA806R); and 454-specific adapter A, a 2 bp linker (5’-TC-3’), and 515F
primer for forward primer (BA515F). Each PCR contained 1.25 U μl-1
AmpliTaq Gold (Life Technologies, Grand Island, NY; N8080243), 1X Buffer
II (100 mM Tris-HCl, 500 mM KCl, pH
8.3), 2.5 mM MgCl2, 200 μM of each dNTP, 0.5 mg ml-1 BSA, 0.2 μM BA515F, 0.2 μM
BA806R, and 10 μL of 1:30 DNA template in 25 μl total volume). The PCR
conditions were 95 $^{\circ}$C for 5min followed by 22 cycles of 95$^{\circ}$C
for 10 s, 53 $^{\circ}$C for 30 s, and 72 $^{\circ}$C for 30 s, followed by
a final elongation at 72 $^{\circ}$C for 5 min. Amplification products were
checked by 1\% agarose gel. Reactions were performed in triplicate and pooled.
Amplified DNA was gel purified (1\% low melt agarose) using the Wizard SV gel
and PCR clean-up system (Promega, Madison, WI; A9281) per manufacturer’s
protocol. Samples were normalized by SequalPrep normalization plates
(Invitrogen, Carlsbad, CA; A10510) or based on PicoGreen DNA quantification and
pooled in equimolar concentration. Amplicons were sequenced on Roche 454 FLX
system using titanium chemistry at Selah Genomics (Columbia, SC).

\subsubsection{DNA sequence quality control}
% Fakesubsubsection: SSU rRNA gene sequences
SSU rRNA gene sequences were initially screened by maximum expected errors
at a specific read length threshold \citep{edgar2013}. Reads that had more
than~0.5 expected errors at a length of 250 nt were discarded. The
remaining reads were aligned to the Silva Reference Alignment as provided
in the Mothur software package using the Mothur NAST aligner
\citep{DeSantis2005,schloss2009}. Reads that did not align to the expected
region of the SSU rRNA gene were discarded. After expected error and
alignment based quality control. The remaining quality controlled reads
were annotated using the “UClust” taxonomic annotation framework in
\citep{caporaso2010,edgar2010}. We used 97\% cluster seeds from the Silva
SSU rRNA database (release 111Ref) \citep{quast2013} as reference for
taxonomic annotation (provided on the QIIME website) \citep{quast2013}.
Quality control screening filtered out 344,472 of 1,720,480 raw sequencing
reads leaving 1,376,008 reads for downstream analyses. Reads annotated as
"Chlorloplast", "Eukaryota", "Archaea", "Unassigned" or "mitochondria" were
culled from the dataset. 

\subsubsection{OTU binning} 
% Fakesubsubsection: Sequences were distributed
Sequences were distributed into OTUs with a centroid based clustering
algorithm (i.e. UPARSE \citep{edgar2013}). The centroid selection also
included robust chimera screening \citep{edgar2013}. OTU centroids were
established at a threshold of 97\% sequence identity and non-centroid
sequences were mapped back to centroids. Reads that could not be mapped to
an OTU centroid at greater than or equal to 97\% sequence identity were
discarded. 

\subsubsection{Phylogenetic reconstruction}
% Fakesubsubsection: For phylogenetic
We used SSU-Align \citep{nawrocki2009,nawrocki2013} to align SSU rRNA gene
sequences. Columns in the alignment that were aligned with poor confidence
($<$ 95\% of characters had posterior probability $>$ 95\%) were not considered
when building the phylogenetic tree leaving a multiple sequence alignment
of~216 columns. Additionally, the alignment was trimmed to coordinates such
that all sequences in the alignment began and ended at the same positions.
FastTree \citep{price2010} was used with default parameters to build the
phylogeny. 

\subsubsection{Ordination and statistical analysis of differences in SSU rRNA
gene composition} 
% Fakesubsubsection: NMDS
NMDS ordination was performed on weighted Unifrac \citep{lozupone2005}
distances between samples. The Phyloseq \citep{mcmurdie2013} wrapper for Vegan
\citep{oksanen2015} (both R packages) was used to compute sample values along
NMDS axes. The 'adonis' function in Vegan was used to perform Adonis tests
(default parameters) \citep{Anderson2001a}.


\subsection{OTU characteristics}
\subsubsection{Identifying $^{13}$C responders} 
% Fakesubsubsection: Figures S11 and S12
Figures~S11 and~S12 demonstrate raw data for responder and non-responder OTUs,
respectively. Responders increased in relative abundance in the high density fractions
due to $^{13}$C-labeling of their DNA. As our data is compositional, often OTUs
had consistent \textit{relative} abundance across the density gradients indicating the
OTU DNA concentration across the gradient mirrored that of the total DNA concentration. If OTU
DNA is centered outside the main distribution of DNA due to G$+$C content
and/or $^{13}$C-labeling its relative abundance increases near the center of the OTU DNA
concentration profile. Thus, we identified responders by finding OTUs
enriched in high density fractions of $^{13}$C-treatment gradients relative to
control. This technique accounts for the variation in OTU base abundance and
the variation in OTU G$+$C content (and therefore natural buoyant density)
because relative abundances in gradient fractions from $^{13}$C-treatments are
always compared to those in corresponding gradient fractions from control
gradients.  

We used DESeq2 (R package), an RNA-Seq differential expression statistical
framework \citep{love2014}, to identify OTUs that were enriched in high density
gradient fractions from $^{13}$C-treatments relative to corresponding gradient
fractions from control treatments (for review of RNA-Seq differential
expression statistics applied to microbiome OTU count data see
\citep{McMurdie2014}). We define ``high density gradient fractions'' as gradient
fractions whose density falls between 1.7125 and 1.755 g ml$^{-1}$.
Briefly, DESeq2 includes several features that enable robust estimates of standard error
in addition to reliable ranking of logarithmic fold change (LFC) (i.e.
gamma-Poisson regression coefficients) in OTU relative abundance even with low
count OTUs where LFC can often be noisy. Further, statistical evaluation of LFC
can be performed with user-selected thresholds, as opposed to the typical null
hypothesis that LFC is exactly zero, enabling the most biologically interesting
OTUs to be identified for subsequent analyses. For each OTU, we calculated LFC
and corresponding standard errors for enrichment in high density gradient
fractions of $^{13}$C treatments relative to control. Subsequently, a one-sided
Wald test was used to statistically assess LFC values. The user-defined null
hypothesis was that LFC was less than one standard deviation above the mean of
all LFC values. P-values were corrected for multiple comparisons using the
Benjamini and Hochberg method \citep{benjamini1995}. We independently filtered
OTUs on the basis of sparsity prior to correcting P-values for multiple
comparisons. The sparsity value that yielded the most adjusted P-values less
than 0.10 was selected for independent filtering by sparsity. Briefly, OTUs
were eliminated if they failed to appear in at least 45\% of high density
gradient fractions for a given $^{13}$C/control treatment pair. These sparse
OTUs are unlikely to have sufficient data to allow for the determination of
statistical significance. We selected a false discovery rate of 10\% to denote
statistical significance.

\subsubsection{Estimating rrn copy number}
% Fakesubsubsection: We estimated the rrn
We estimated the \textit{rrn} copy number for each OTU as described
\citep{Kembel_2012} (i.e. we used the code and reference information provided
by the authors \citep{Kembel_2012} directly). In brief, OTU centroid sequences
were inserted into a reference SSU rRNA gene phylogeny \citep{matsen2010} from
organisms of known \textit{rrn} copy number. The \textit{rrn} copy number was
then inferred from the phylogenetic placement in the reference phylogeny. 

\subsubsection{NRI, NTI, and consenTRAIT}
% Fakesubsubsection: NRI and NTI were calculated using
NRI and NTI were calculated using the ``picante'' R package
\citep{kembel2010}. We used the ``independentswap'' null model for
phylogenetic distribution. The consenTRAIT clade depth for xylose and cellulose
responders was calculated using R code from the original publication describing
the metric \citep{Martiny2013} which employs the R ``adephylo'' package
\citep{jombart2010}.

\subsubsection{Buoyant density shift estimates}
% Fakesubsubsection: Upon labeling, DNA from
DNA buoyant density (BD) increases with atom \% $^{13}$C. Therefore, the
magnitude of $\Delta\hat{BD}$ indicates the degreee of isotopic labeling
for an OTU. We measured $\Delta\hat{BD}$ as the change in an OTU's density
profile center of mass between corresponding control and labeled gradients
(Figure~S11). Because all gradients did not span the same density range and
gradient fractions cannot be taken at specific density positions, we limited
our $\Delta\hat{BD}$ analysis to the density range for where all density gradients
overlapped. Within this density range we linearly interpolated~20
evenly spaced relative abundance values. The center of mass for an OTU along
the density gradient was then the density weighted average where weights were
the linearly interpolated relative abundance values. $\Delta\hat{BD}$ values are based on
relative abundance profiles and would be distorted in comparison to
$\Delta\hat{BD}$ based on absolute DNA concentration profiles and should be
interpreted with this transformation in mind.

\subsubsection{Finding cultured relatives of OTUs}
% Fakesubsubsection
OTU centroids were compared (BLAST \citep{altschul1990,camacho2009}) to
sequences in ``The All-Species Living Tree'' project (LTP). The LTP is
a collection of SSU rRNA gene sequences for classified species of Archaea
and Bacteria \citep{yarza2008}. We used LTP version~115 for analyses in
this paper.

\subsubsection{OTU changes in relative abundance with time}
% Fakesubsubsection: We identified OTUs that changed in
We identified OTUs that changed in relative abundance over time using
DESeq2 \citep{love2014}. Specifically, we used day treated as an ordered
factor as the regressor with LFC of the relative abundance in
non-fractionated DNA as the outcome in the general linear model. We used
the default DESeq2 base mean independent filtering and disabled the Cook's
cutoff outlier detection. The null model was that abundance did not change
with time and we assessed significance at a false discovery rate of 10\%. 

\subsection{Sequencing and density fractionation statistics}\label{seq_stats}
% Fakesubsubsection: Microcosm DNA was density
Microcosm DNA was density fractionated on CsCl density gradients. We sequenced
SSU rRNA gene amplicons from a total of 277 CsCl gradient fractions from 14
CsCl gradients and 12 bulk microcosm DNA samples. The SSU rRNA gene data set
contained 1,102,685 total sequences. The average number of sequences per sample
was 3,816 (sd 3,629) and 265 samples had over 1,000 sequences. We sequenced SSU
rRNA gene amplicons from an average of 19.8 fractions per CsCl gradient (sd
0.57). The average density between fractions was  0.0040 g mL$^{-1}$ The
sequencing effort recovered a total of 5,940 OTUs. 2,943 of the total 5,940
OTUs were observed in bulk samples. We observed 33 unique phylum and 340 unique
genus annotations.

% Fakesubsubseciton: bib
\bibliography{bibliography/biblio}

\end{document}
